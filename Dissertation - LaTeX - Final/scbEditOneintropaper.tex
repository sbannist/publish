\linespread{1.9}
\chapter{General introduction}
\linespread{1.6}

	This introductory chapter outlines and connects the two primary chapters in this dissertation.
	
	Chapter 2 covers the main hypotheses and evidence that are central to the dissertation's claims: the English Industrial Revolution, whatever else it was in terms of changed societies and institutions, was primarily an energy revolution in the strong sense that without the energy revolution, there could not be an industrial revolution leading to modern economic growth.
	
	Further, this energy revolution was in fact two separate but related energy revolutions. The first---wood to coal---provided scale for domestic and industrial heating needs not provided by the prior main source---wood. Deforestation as populations grew depleted that source. The infrastructure required for this first revolution---capital investments---enabled the energy inputs required for the second revolution---muscle--power to steam--power.
	
	And using basic economic principles can provide a clear picture of the incentives that inventors and entrepreneurs faced that pulled them into the revolutions.
	
	Using Sung China as a natural experiment to test these hypotheses shows that the incentives and outcomes are more general than the English experiment, but that England had unique price differences that caused it to succeed.
	
	Chapter 3 then explores the institutional implications of these revolutions. It focuses on what is, arguably, the most important institution arising from the English Industrial Revolution---industrial capitalism.
	
	As in Chapter 2, Sung China is used to test the hypothesis that we can use basic economic principles to argue that the tendency toward industrial capitalism is more general than the English experience given the right economic conditions.


\linespread{1.0}
\section{Chapter 2 -- ``Energy and institutions: What really happened in the English
Industrial Revolution? What did not happen in China?''}
\linespread{1.6}
\vspace{.21in}

	This chapter is the core of the project to understand the link between economic growth and energy. By analysing two very long series---English gross domestic product and English energy consumption---with statistical tests, the chapter demonstrates that there is essentially no difference between these series. The original methodological strategy was to perform a cointegration analysis to test this hypothesis. However, presented with the graphical evidence and very high correlation coefficient, it was judged unnecessary to present those results in this chapter.

	This evidence suggests that at least a plurality of economic and other historians who attribute the English Industrial Revolution (EIR) to one or more aspects of culture or institutions might additionally consider this very physical energy--growth channel as an important cause. The institutional and cultural changes were certainly large, but the chapter questions if they would have happened without the great surge in output, incomes, and wealth that can only be explained by learning to consume a virtually unconstrained amount of energy in the production process. So that is the chapter's major claim; the chapter then explores both macroeconomic and microeconomic theories to support the case.
	
	After developing the English data and descriptive history, the chapter then suggests that after accounting for a background of increasing aggregate demand, it is useful to apply basic microeconomic principles to explain what would cause inventors, innovators, and entrepreneurs to invest in overcoming the great technical difficulties required to remove the supply--side constraint on growth in living standards before the EIR.
	
	This same framework is then applied to the case of Sung China that experienced a period of economic growth, including living standards, that is remarkable in history. Some historians go so far as to call this episode an industrial revolution. The chapter develops partial support for that position.

	Section \ref{sec:2.1} of Chapter 2 starting on page \pageref{sec:2.1} is structured as a literature review and has several sections. The first section reviews the data and descriptive sources used to analyze English energy and macroeconomic performance over the period 1300 to 1873 common era (CE). Included are some who place the energy story very high in the list of possible explanations of the EIR. W. Fred Cottrell in particular takes a very thermodynamic--economic approach in his discussion of the transition to ``high--intensity energy converters'' from ``low--intensity energy converters'' as the primary mechanism causing the EIR. Kenneth Pomeranz believes it is English geographical luck that accounts for a large part of the causation of the EIR. Robert Allen takes a similar approach.

	The second introductory section reviews the institutional literature. This is the major alternative explanation to the more physical explanations from the sources discussed in section one. The sources include Douglass North, David Landes, Jack Goldstone, Max Weber, and Daron Acemoglu.
	
	The third introductory section reviews the literature on Chinese energy data focusing on the period of significant economic growth that occurred during the Sung dynasty. The sources include Robert Hartwell, William McNeil, Mark Elvin, and Robert Allen.
	
	The fourth introductory section reviews Chinese institutions. This discussion is mainly about the Ming dynasty although Robert Hartwell discusses Sung dynasty institutions. The sources include Kenneth Pomeranz, R. Bin Wong, and Peer Vries.
	
	The fifth introductory section reviews the literature of Chinese science and invention. This becomes important when attempting to understand some arguments that China had an insufficient tradition of invention and innovation to develop the technologies required to produce an industrial revolution. These sources claim China had a very rich tradition of innovation and invention probably sufficient to accomplish an industrial revolution. The authors include Joseph Needham and John Hobson.
	
	The sixth introductory section reviews economic growth theory. The sources include Roy Harrod, Evsey Domar, Robert Solow, Trevor Swan, Paul Romer, and Robert Ayers.
	
	Section \ref{sec:2.2} on page \pageref{sec:2.2} develops and analyzes English data, econometrics, and economics. The topics covered include discussions of the sources and methods for the data, an analysis of modern economic growth, a discussion of energy revolutions, formal econometric analyses, and the economic analyses.
	
	In this section, the chapter applies structural change econometric methods to the data series and deduces four energy--gross domestic product (GDP) eras covering the historical period. Each has different aggregate demand and supply characteristics. Table \ref{tbl:EnergyGdp} is a brief summary of each:

%%% want ds

\linespread{1.0}
\begin{table}[h!]

\center
\begin{tabular}{ll}
%\hline
Era&AD/AS regime\\
\hline \hline
1300 -- 1500&European Marriage Pattern, Black Death, \\&wages/family income increasing\\
1500 -- 1600&Positive demand shock, high wages\\
1600 -- 1750&Energy supply constraint\\
1750 -- 1873&Positive supply shock, large income effect,\\&``virtuous'' macro feedback cycle\\
\hline
\end{tabular}
\caption{Energy/GDP eras}
\label{tbl:EnergyGdp}
\end{table}
\linespread{1.9}

% want ds 
\vspace*{-10.5pt}

The important conclusion is that until 1750 with brief exceptions, economic growth for a growing population was largely constrained by a lack of energy supplies. The structural change analysis show that this constraint started lifting in about 1600 and then accelerated in the mid--eighteenth century. Using these data, the chapter claims that the energy revolution that became the EIR started 150 years earlier than the common starting point many historians claim. The story is consistent with a Malthusian story of temporary growth spurts in population that were eventually constrained by supply (in this story energy supply). Chapter 2 discusses the Malthusian constraint and its removal.
	
	Section \ref{sec:2.3} on page \pageref{sec:2.3} develops and analyzes Chinese data and institutions. The topics include discussions of the sources and methods for the data, a discussion of regional and global population and gross domestic product dynamics, and a discussion comparing Chinese and English institutions.
	
	Section \ref{sec:2.4} on page \pageref{sec:2.4} develops the beginnings of a theoretical framework of industrial revolutions.
	
	Section \ref{sec:2.5} on page \pageref{sec:2.5} concludes.

\linespread{1.0}	
\section{Chapter 3 --- ``The rise of industrial capitalism. What happens next?''}
\linespread{1.6}
\vspace{.21in}

	Chapter 3 analyzes the rise of industrial capitalism and its links with industrial revolutions. This is done both for England and China. England is normally thought of as the birthplace of industrial capitalism; the chapter attempts to identify traces of embryonic capitalism in the Chinese economic history starting with the Sung dynasty (960--1126 CE). If there are common institutional elements that can be linked to industrial revolutions, then it will improve our understanding of how both industrial revolutions and industrial capitalism happen.
	
	The introduction for Chapter 3 is structured as a literature review. Topics include some definitions and then reviews of the sources for the English transition to industrial capitalism, and sources for discussion of a Chinese transition to industrial capitalism.
	
	Section \ref{sec:3.2} on page \pageref{sec:3.2} discusses the rise of English industrial capitalism. Topics include a discussion of the data including sources and methods, an analysis of global population trends, a discussion of Jan de Vries' survey of early modern capitalism to help understand common approaches to explaining the event, a review of industrial revolutions, and how they give rise to demand for the large capital investments of support the two English energy revolutions that were fundamental to the EIR. The section closes with a discussion of the two primary roles that capital played in the EIR.
	
	Section \ref{sec:3.3} on page \pageref{sec:3.3} discusses evidence for shoots of (embryonic) Chinese industrial capitalism including data and a discussion of the three eras for which we have evidence---the Sung, Ming, and Qing dynasties.
	
	Section \ref{sec:3.4} on page \pageref{sec:3.4} concludes.
