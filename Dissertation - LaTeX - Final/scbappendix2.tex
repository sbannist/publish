%\addcontentsline{toc}{chapter}{\numberline {B.} TECHNICAL APPENDIX}
\chapter{Technical Appendix} \label{sec:techAppendix}
%\fixchapterheading


\section{Importance of energy for growth and development}

\linespread{1.2}
\begin{table}[h!]
\begin{center}
\begin{tabular}{lr}
%\hline
Period&Pearson Correlation Coefficient: \\
&energy and GDP\\
\hline \hline
England 1300-1873&0.998\\
World 1980-2008& 0.993\\
\hline
\end{tabular}

\end{center}
\caption{Energy/GDP correlations -- the case for energy revolutions}\label{tbl:cor}

\end{table}
\linespread{1.9}
%\newpage
\section{Cross-country history of energy consumption}

\linespread{1.0}
\begin{table}[h!]
	\centering
	\begin{tabular}{lrrrr}
%	\hline
	Year&England&China&Netherlands&India\\
	\hline \hline
	1650$^a$&&&0.63&  \\
	1820&0.61&&&\\
	1840$^a$ &&&0.33& \\
	1870&2.21&\\
	1970$^a$ &&&8.07&0.33 \\
	1973&&0.48&&\\
	1998$^b$&6.56&1.18\\
	2008$^b$&5.99&2.56&9.86&  \\
	\hline
	\end{tabular}
	\caption{Per--capita primary energy consumption, annual tonnes of oil \\equivalent. \textit{Source:} Angus Maddison, $^a$de Zeeuw, $^b$US DOE EIA}
	\label{tab:capitaEnergy}

	\end{table}
\linespread{1.9}
\newpage

	\section{Theory of industrial revolutions}	
	With the price spread between coal and wood used for such an essential economic input as energy for heating moving dramatically in coal's favor, the basic economic mechanism of input--price substitution should work. It does explain the transition. To formalize this, we can write:

\linespread{1.2}
		\begin{equation}
		\label{eq:mrp1}
		\frac{\text{Marginal Product}_{\text{wood Joule}}}{\text{Price}_{\text{wood Joule}}} \ll \frac{\text{Marginal Product}_{\text{coal Joule}}}{\text{Price}_{\text{coal Joule}}},
		\end{equation}

\vspace*{21.5pt}
or if one prefers a non-neoclassical writing:

		\begin{equation}
		\label{eq:mrpA}
		\frac{\text{Average Product}_{\text{wood Joule}}}{\text{Price}_{\text{wood Joule}}} \ll \frac{\text{Average Product}_{\text{coal Joule}}}{\text{Price}_{\text{coal Joule}}}.
		\end{equation}
\linespread{1.9}

Either writing leads to the same theoretical conclusion: assuming no qualitative difference in the two inputs in terms of work being done (a Joule is a Joule) with the data showing the right--hand--side coal ratio being significantly greater than the wood ratio, we would expect entrepreneurs to substitute away from wood to coal. This is the first phase of an industrial revolution.

Equation \ref{eq:mrpB} is a variation on production theory that will be familiar to those who remember their Econ 101. A major topic of mainstream production theory is how entrepreneurs maximize profits given the derived demand curves of the various input choices. 

%This equation is a variation on that theme:\footnote{We can proceed either with a neo-classical factor substitution argument, or a more general classical view of normal prices of production. Either approach will react to the enormous productivity-enhancing energy supply shock that was the Industrial Revolution. A more challenging story to tell is one which identifies the sources of aggregate demand that supported expansion of English production. Here, I simply stipulate that aggregate demand existed.}

		\begin{equation}
		\label{eq:mrpB}
		\frac{\text{Average Product}_{\text{labor Joule}}}{\text{Price}_{\text{labor Joule}}} \ll \frac{\text{Average Product}_{\text{steam Joule}}}{\text{Price}_{\text{steam Joule}}}
		\end{equation}

\vspace*{21.5pt}
Instead of using different substitutable inputs such as labor and capital, we apply the theory to the different sources of energy since that is essentially the only non--substitutable input, as in you must have Joules from whatever source to do any economic transformation. If we take the numerators in Equation \ref{eq:mrpB} to be equal, abstracting again from the difficulties in invention that were eventually solved then because of the much lower price of English coal--Joules than wages for labor--Joules, the relentless (in the face of rising wages) pressure will be for the inventors to invent and the entrepreneurs to commercialize steam--power, thus creating the machine age and completing the EIR.
