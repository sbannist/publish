% Complete in less than 350 words.

\setcounter{page}{4}

%This dissertation is described here in less than 350 words using no
%footnotes, diagrams, references or outside anything.

	This dissertation contains two related chapters and this introduction. The common themes they explore are the unresolved questions surrounding the English Industrial Revolution (EIR). The questions include what happened, why did ``it'' happen first in England, why did it happen then in history, and what are the consequences? The story is a history of economic growth from a specific point of view---energy consumption for an economy; the framework can be used to illuminate our economic present and possible economic futures.

	Economic and other historians have been grappling with these puzzles for a long time; their answers fall along a continuum from New Institutional Economics (some mix of institutions and perhaps culture) to almost pure chance. Institutional explanations are at least a plurality; this work makes the case that these explanations are not sufficient in the sense of not being primarily causal or sufficiently explanatory in the EIR's history. The work further explores that at least the major institutional changes are endogenous to the revolutionary economic changes.

	The major claim is that the EIR was primarily an energy consumption revolution, the English having had the correct economic incentives and historical path to learn how to use steam power to replace muscle power. The contribution is the attempt to apply economic principles to the data and history and measure their explanatory power.

	The work identifies two energy revolutions explaining the EIR. The first, converting from wood to coal for industrial and domestic heating purposes, probably happened several times in history at other places in addition to England. In addition to this first--phase energy revolution in England, chapter two documents an added noteworthy instance, that of the iron and steel industry in Sung China (960--1126 CE). The second revolution, converting from muscle power to steam power happened first in England before engulfing the world.

	To support the claims the work employs several methods including empirical analyses, microeconomic theory, macroeconomic theory, and descriptive narratives from many sources. The general method is to apply basic economic principles to the available data and narratives.

	Among the insights the work proposes a hypothesis of industrial revolutions that can be tested beyond the cases included in this work. This work uses basic microeconomics, macroeconomics, and relevant empirical data (as the data permit) to test the cases of China and England.

	To support the revolutionary growth on the supply side, the work makes the case that there was sufficient consumer demand to drive the efforts of the entrepreneurs and inventors.
	
	Once the theoretical framework for industrial revolutions is explored then the work turns to the question of how did these momentous economic events affect the growth of industrial capitalism since it is one of the more important institutions that is associated with the EIR.
