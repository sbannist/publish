\iffalse
\documentclass[11pt]{article}
\title{Energy and institutions: What \textit{really} happened in the English Industrial Revolution? What did not happen in China?}
\author{Stephen C. Bannister\\
	Department of Economics\\
	University of Utah\\
	Salt Lake City, Utah 84112\\
	USA\\
	\href{mailto:steve.bannister@econ.utah.edu}{steve.bannister@econ.utah.edu}\\
	}

%\date{Drafts May 2012,}
\date{}

\usepackage[latin1]{inputenc}
%\usepackage[english]{babel}
\usepackage{amsmath}
\usepackage{amsfonts}
\usepackage{txfonts}
\usepackage{amssymb}
\usepackage{pgfpages}
\usepackage{booktabs}
\usepackage{longtable}

\usepackage{chngpage}
%\usepackage{pdfpages}
\usepackage{graphicx}
\usepackage[lofdepth,lotdepth,position=bottom]{subfig}
\usepackage{caption}
\usepackage{float}
%\usepackage{draftwatermark}

%\newtheorem{mydef}{Definition}[section]
%\numberwithin{equation}{section}


\usepackage{verbatim}
%\usepackage{underscore}
\linespread{1.9}	% remove for single, 1.3 for 1.5 and 1.6 for 2.0. use this setting for print editing

\usepackage{glossaries}

\graphicspath{{C:/Users/Steve/Documents/GitHub/publish/diss1/images/}}

%\textwidth{7.5in}
\addtolength{\textwidth}{1.0in} 
\addtolength{\oddsidemargin}{-0.5in} 
\addtolength{\evensidemargin}{-0.5in} 
\addtolength{\textheight}{1.25in}
\addtolength{\topmargin}{-0.75in}

\usepackage{tocloft}

\usepackage{natbib}
\bibpunct{(}{)}{;}{a}{,}{,}

\usepackage{hyperref}


\makeglossaries

%\loadglsentries{glossary.tex}

%\setcounter{secnumdepth}{4}%to number paragraphs so can ref them?

\begin{document}

%\SetWatermarkLightness{0.93}
%\SetWatermarkScale{1}

	\maketitle
%	\nocite{*}
%	\bibliographystyle{E:/LaTeX-Portable/MikTex-Portable/bibtex/bst/base/IEEEanot}

%	\bibliographystyle{E:/LaTeX-Portable/MikTex-Portable/bibtex/bst/base/plain-annote}
%	\bibliographystyle{plain}


%\newcommand{\listequationsname}{List of Equations}
%\newlistof{myequations}{equ}{\listequationsname}
%\newlistof{myequations}{equ}{\listequationsname}
%\newcommand{\myequations}[1]{%
\addcontentsline{equ}{myequations}{\protect\numberline{\theequation}#1}\par}
\fi

\chapter{Energy and institutions: What \textit{really} happened in the English Industrial Revolution? What did not happen in China?}

\section{Introduction}
%\begin{abstract}
	
	England during the period leading up to and spanning the first Industrial Revolution collectively learned how to consume a virtually unconstrained quantity of fossil (mainly coal) energy. Led by the period's effective aggregate demand growth this resulted directly in productivity growth that then led to modern economic growth in living standards for the first time in recorded history. 
		
	Studying the event empirically we can use recent long--period series estimates of levels of English energy consumption, gross domestic product, and population to test the hypothesis that this was primarily an \textit{energy} revolution with important but mostly proximate institutional and cultural support.
	
	Then a natural experiment is run using Ming and Qing China using limited data and important institutional comparisons that would not preclude China from completing an industrial revolution. In order to explain the English success and the Chinese failure a theoretical framework for industrial revolutions is explored.
	
	The outcome should provide insights into economic development for growth economists by highlighting the importance of energy transitions for growth of economic systems. Additionally, the analytic framework developed can be applied across time and geography adding insights to ongoing development puzzles.% and to the realistic chances of curbing ecologically damaging mineral (fossil) energy consumption for ecological economists and others interested in that critical topic.
%	\end{abstract}



\subsection{English energy data}

	As early as 1734 observers of the economic panorama, later including economic and other historians, have commented on the role of energy inputs in economic activity and its social outcomes. These comments are not always explicitly related to energy but their implications often are. Jean Theophilus Desaguliers \cite{desaguliers_course_1734} was a member of the Royal Society and ``natural philosopher'' (physicist and engineer) and observes that using human labor to pump water from coal mines was not profitable. He recommends ``fire engines'' (steam engines) to solve that problem. This is a clear call to substitute coal as a cheaper energy input for more expensive human and animal energy inputs to pump water from flooding coal mines.

	Friedrich Engels \cite{engels_condition_1892} while writing of 1844 England asserts that the invention of the steam engine and machines for spinning and weaving cotton gives the impetus to the Industrial Revolution and changes the entire social structure of middle-class society. William Stanley Jevons \cite{jevons_coal_1965} frets that England will lose it's economic dominance when the coal supply runs out as perhaps an early version of today's ``peak oil'' concerns. Later, Edwin Eckel \cite{eckel_coal_1921} reports coal reserve estimates for several major economies and claims that World War I is significantly about resources including coal. Frederick Soddy who is a 1921 Nobel Laureate in chemistry writes widely on economics rooted in principles of physics and thermodynamics \cite{soddy_matter_1911, soddy_cartesian_1921, soddy_money_1931, soddy_wealth_1933, soddy_role_1934}, presaging Herman Daly and Nicholas Georgescu-Roegen.

	In John Nef's two-volume history of the coal industry in Britain \cite{nef_rise_1932} he demonstrates a strong sense of the importance of energy consumption primarily from coal in the growth of the British economy through an extended period from the sixteenth century on. He also describes in depth how the coal industry influences and encourages the rise of industrial capitalism.

	French historian Paul Mantoux \cite{mantoux_industrial_1961} writes in the early twentieth century of the machine industry transition in England during the eighteenth century with deep analyses of the key industries especially wool and cotton textiles.

	Later in the twentieth century W. Fred Cottrell \cite{cottrell_energy_1955} writes about energy sources from the neolithic through nuclear energy. Cottrell uses an unusual syntax in describing this history: low-intensity energy converters for humans and animals and high-intensity energy converters for machines. Peculiarly he never as far as I could find uses the word ``capital'' just high-intensity energy converter. He thus focuses clearly on the distinction between low-capacity muscle-powered work and high-capacity machine-powered work an essential distinction made later in discussing industrial revolutions. He also discusses the impact each of the energy sources makes on society.

	The Italian economic historian Carlo Cipolla \cite{cipolla_sources_1961, cipolla_economic_1962, cipolla_guns_1966, cipolla_before_1983} writes widely of energy revolutions including neolithic agriculture, the early modern European sea dominance, and the Industrial Revolution. Cipolla is an early chronicler of the roles various technologies played in these revolutions in a sense presaging Joel Mokyr \cite{mokyr_lever_1992}.

	Phyllis Deane in writing of the English Industrial Revolution notes ``The most important achievement of the industrial revolution was that it [i.e. coal] converted the British economy from a wood-and-water basis to a coal-and-iron basis'' \cite[p.~129]{deane_first_1979}. Deane's comment is representative of energy-aware observers but misses the full significance of the energy source revolution that became the English Industrial Revolution. I plan to extend such thoughts into a more comprehensive story of this history.

	E. A. Wrigley \cite{wrigley_continuity_1988, wrigley_energy_2010} writes extensively about England's transformation from an ``advanced organic'' society mainly engaged in agriculture to an ``industrial inorganic society'' engaged primarily in non-agricultural production in centralized factories. Wrigley interweaves the social impacts into this story very notably how it influenced the transition away from Malthusian demographic dynamics to a post-Malthusian dynamic. The Industrial Revolution eventually changed the sign of the correlation between increased living standards and fertility rates from positive to negative. This is a sign change that holds profound implications for our economic future.

	What the paper calls an energy revolution Italian economic historian Paolo Malanima \cite{malanima_path_2010} calls a transformation of the energy system. His time frame is the same as John Nef's and mine---from the sixteenth century through the nineteenth century. Malanima sketches out formally the essential features of this transition that become the focus for England and China in this paper. These include population growth, rising energy costs, and  substitutions for heat and muscle power energy sources across Europe. He does this at a macroeconomic level. A focus on England allows us to explain in depth the energy foundations of the first Industrial Revolution, examine why they happened in endogenously in England, and describe both the microeconomic incentives behind the revolution hinted at by Desaguliers and its macroeconomic phases.

	The twenty-first century has seen some very important work among historians relating energy inputs and growth. Kenneth Pomeranz is a Sinologist who like William McNeill is a ``world'' historian but unlike McNeill \cite{mcneill_pursuit_1982} focuses on explaining the ``great divergence'' between China and England starting around 1800 \cite{pomeranz_great_2001,pomeranz_beyond_2002}. Pomeranz explains why the English did the Industrial Revolution first compared to anyone else especially compared to China by invoking the English advantages in coal, colonies, and cotton. Coal removed the energy constraint faced by all growing economies from depending on wood for heat and steam. The English colonies provided both input resources such as cotton and (colonial) consumer markets for absorbing the increased capacity as production constraints dissolved in the face of steam-powered factories. This is a classic case of Adam Smith's vent-for-surplus theory \cite{smith_inquiry_1977} that Pomeranz invokes along with armed mercantilism as instrumental to the England's successful industrialization. But very clearly he returns many times to the central fact: England was geographically and geologically lucky to have cheaply accessible coal supplies. The English Industrial Revolution was foremost an energy revolution.

	Economic historian Robert Allen \cite{allen_british_2009} intensified the explanation of the English Industrial Revolution as an English energy revolution. Allen's approach is data-intensive; in particular he presents wage and energy cost series for England, China, and other important economies in the early and late modern eras. This allows him to construct a comparative wage-to-energy-price ratio for these areas in a critical proto-industrial era that not only answers the ``why England and not China'' question surrounding the Industrial Revolution but allows one to begin formalizing a theory of Industrial Revolutions or even more generally a new approach to growth theory as discussed below. 

	Allen's analysis bolsters the ``energy revolution as primary'' approach that the paper explores; he summarizes his view strikingly: ``... there was only one route to the twentieth century -- and it traversed northern Britain'' \cite[p.~275]{allen_british_2009}. His view is that expensive English wages and cheap coal energy from Newcastle though a historical accident were the uniquely English causes for the Industrial Revolution and modern economic growth. As an essentialist Allen views the primary or ultimate cause of the English Industrial Revolution to be English labor and coal price differentials compared to other historians who might invoke several proximate causes.

	While the scholars and observers cited above place energy consumption at the center of their explanations for the English Industrial Revolution and modern economic growth they seldom do so explicitly. The most explicit are W. Fred Cottrell \cite{cottrell_energy_1955}, Robert Allen \cite{allen_british_2009}, E. A. Wrigley \cite{wrigley_continuity_1988,wrigley_energy_2010}, and Vaclav Smil \cite{smil_energy_1994,smil_energy_2008} not mentioned above but a scientist and scholar with a very broad understanding of energy's role in society. The others cited represent a group of scholars who at least hint at the primary role energy plays in the \textit{sui generis} English experience.

	In a more general vein Nicholas Georgescu-Roegen \cite{georgescu-roegen_entropy_1971} focuses on the thermodynamic foundations of economic systems and helps found the field of ecological economics. This seemingly stark description of our normal daily activities holds an important truth: all economic activities indeed all activities require energy inputs. We can impute from this that limited energy inputs will limit economic outputs. Following his thinking I sometimes think that the only non-substitutable input is energy (as in Joules); energy sources can be substituted but you must have Joules for life and economic activity. Energy source substitution becomes fundamental to a story of industrial revolutions. Timothy Garrett \cite{garrett_are_2009,garrett_modes_2012,garrett_long-run_2015} advances a modern treatment of this energy-based thermodynamic work including its impact on long-range climate forecasts.

\subsection{New institutionalists}

	Arrayed against this countably small group of major scholars is a large literature on the role of culture and institutions in explaining why England succeeded in its industrial revolution before anyone else was able to do so. I will review the very high points of this literature and then turn to a review of relevant Chinese literature as representing a ``natural experiment'' to compare with England. 

	This paper highlights the role of energy consumption and revolutions it its use as being at the center of the English Industrial Revolution and more generally on industrial revolutions and economic development and growth. While this necessarily displaces culture or institutions as prime causes of these events the purpose of this paper is to develop evidence and theory to make the different focus justifiable.

	We first must include Max Weber \cite{weber_religion_1964,weber_protestant_2002} as representative of the institutional literature on the English Industrial Revolutions. Weber is clearly an early eurocentric scholar invoking European Protestantism as a motivating force for capitalism and the events that flowed from it.

	Douglass North is an economic historian instrumental in founding both New Economic History (Cliometrics) and New Institutional Economics and works on the broad issues of economic growth and development. He takes a very historical approach by describing market expansion from tribal local exchange dominated by informal rules to long-distance trade that require new institutions to deal with the problems of agency (not having physical control of the goods) and contract (providing transport protection and enforcement of contracts). 

	North \cite{north_rise_1973,north_institutions_1990} focuses on the idea that economies require ``efficient organization'' to grow that is a self-admittedly neo-classical approach. Efficiency entails developing sufficient institutional arrangements to create individual incentives to inventors and producers. The most important institution is property rights. The West necessarily developed these institutions as conditions for its rise. He discusses both extensive growth defined as overall growth because of increases in the traditional factors of production (land, labor, capital) and intensive or per-capita growth that for him is true economic growth. Intensive growth is in turn caused by either per-capita increases in factor inputs or increased productivity through economies of scale, education, capital improvements via technology embedding, and by reducing market imperfections. He answers the puzzle of why given the straightforward prescription above every economy has not developed economically. And of course it is because they are not efficiently organized, lacking required institutions including most importantly property rights. North also comments on population growth as being important to economic growth; this important insight helps explain the basic motivation for inventors and entrepreneurs to invent and produce---population growth leads to increasing consumer demands that are the source of all production and input demands.

	Contrasted with North, the major historian David Landes \cite{landes_unbound_1969,landes_wealth_1999} writes widely on Western culture as primal in the Industrial Revolution. Landes like scholars discusses the role of energy and the technologies that enable its use but returns to culture as the reason for the rise of the West. A more recent approach to this theme are books by Deirdre McCloskey \cite{mccloskey_bourgeois_2007,mccloskey_bourgeois_2010} discussing the primacy of Western values, ethics, and culture in the comparative rise of the West; McCloskey does talk about the importance of coal but in a glancing discussion.

	Another economic historians who emphasize cultural roots as the explanation for the rise of the West is Jack Goldstone. Goldstone is a member of the ``California School'' of economic history and writes widely \cite{goldstone_cultural_1987,goldstone_rise_2000,goldstone_why_2008} on the West's cultural primacy allowing its comparative rise. In particular \cite{goldstone_rise_2000} he develops the concept of ``Efflorescence'' or the asymmetric rise of economic activity among nations due to institutional differences. To illustrate he invokes the difference between North and South Korea since their partition and radical institutional divergence.

	Daron Acemoglu's work represents a modern quantitative version of institutionally--driven growth; in particular he studies the role of the state \cite{acemoglu_chapter_2005}, growth theory \cite{acemoglu_introduction_2012}, and institutions as causing growth \cite{acemoglu_politics_2005}). Acemoglu often attributes growth differences to the presence or absence of Western--style property rights.

	The defining point of view for this group is that certainly there was something that happened to the energy system yet the causes of the English  Industrial Revolution and subsequent rise of the West were cultural and institutional. In this paper an there is an appeal to something even more fundamental and this is used to develop the view that while institutions are important they arise in response to underlying economic changes. Therefore we must study those to truly be able to answer North's puzzle of ``why not everyone?''

	The ``culture and institutions \textit{are} growth and development'' group's view was not the first institutional approach to the question. Karl Marx \cite{marx_contribution_1904} and Thorstein Veblen \cite{veblen_theory_2009} among other original institutionalists view institutional development as endogenous to the major economic developments. This is a point of view I have come to share and will develop in this paper.

\subsection{Chinese energy data}
	Now turning attention to China as an important ``natural experiment'' comparison to England in order to test the hypotheses about growth and industrial revolutions. If in say 1400 a group of growth economists at a conference were sitting at the bar and speculating on what country was likely to accomplish the first industrial revolution China almost surely would have been in the lead. Large markets, one-quarter of global population, more than one-quarter of global GDP, and important inventions are among several important drivers legitimizing China as the leader in the gathering race toward industrialization. Some had never heard of England---it was a small even backwater and backwards economy somewhere near the Eurasian land mass. Yet three centuries later England was accelerating along its path of becoming the leading economy in the world. And by 1800 was clearly diverging from China and in the global economic lead.

	The Chinese ``energy'' story is not nearly as well-developed as the English possibly because China did not experience a complete industrial revolution and thus did not generate all the questions related to that event that England did; nonetheless there are modern scholars who have important contributions. The cultural and institutional story surrounding China has ossified for many years attempting to explaining the puzzle the economists in 1400 were discussing of why China was not first. This Eurocentric attitude is best summarized by Marx as the ``Asiatic mode of production''  where Marx (and Engels) describe Asia as consumed by despotic rulers expropriating surplus from the economy, monopolizing land ownership, controlling irrigation systems, preventing trade and technological development, and in many other ways thus preventing modern economic development. This widely-held story may be too simplistic and is increasingly challenged by modern scholars.

	Economic historian and Sinologist (and student of John Nef) Robert Hartwell lays the foundations for understanding the iron and coal revolution during the Northern Sung dynasty (A.D. 960-1126) ruling China from Kaifeng in northern China \cite{hartwell_revolution_1962, hartwell_markets_1966,hartwell_cycle_1967, hartwell_demographic_1982}. Mark Elvin \cite{elvin_pattern_1973}, William McNeill \cite{mcneill_pursuit_1982}, Fredrick Mote \cite{mote_imperial_1999}, and Eric Jones \cite{jones_growth_1988, jones_recurrent_1996} all make the key points: first, China during the Northern Sung blossomed economically including a significant period of intensive growth (growth in living standards); second, a significant part of the economic growth involved the rise of a large coal-fed iron and steel industry. Tim Wright \cite{tim_wright_economic_2007} provides a survey that places the historical China work in context and empasizes the importance of Hartwell's contribution.

	Robert Allen \cite{allen_british_2009} provides comparative wage and energy cost data for China that plays prominently in my theses. While Chinese data is sparse compared to English data Allen publishes labor wage time series and energy price series that include China. Using this work a story is developed that Sung China had an energy revolution and a first--phase industrial revolution. These terms are defined later. China and England are very comparable in the theoretic structure developed below. However, China did not complete its industrial revolution and thus further theoretic structures are applied to describe the English success and test those against the Chinese failure.

	Given that China failed at its industrial revolution attempt (though presumably no one except our conference--attending economists knew what an industrial revolution was) and that the preponderance of Western scholarship claims that the failure must be culturally or institutionally caused, a review is needed of the recent scholarship debunking this point-of-view.

\subsection{Chinese institutions}

	Kenneth Pomeranz \cite{pomeranz_great_2001} reviews China's institutional capabilities and comes to the conclusion that eighteenth-century England and regions of eighteenth--century China (as well as other global regions) were not significantly different from an institutional point-of-view. Among the areas Pomeranz investigates: dubious claims of English/Western European productivity advantages; a demographic-marital system that did not produce superior fertility control or life expectancy; a capital stock that was not larger and did not embody decisively superior technology; land and labor markets that were possibly less ``Smithian'' than elsewhere specifically including China; and China's pattern of family labor use that responded to shifting opportunities and price signals as well as Europe's input factors did.

	His conclusion on institutional differences is striking: ``Far from being unique the most developed parts of western Europe seem to have shared crucial economic features--commercialization, commodification of goods, land, and labor, market--driven growth, and adjustment by households of both fertility and labor allocation to economic trends--with other densely populated core areas in Eurasia'' \cite[p.~107]{pomeranz_great_2001}. Chinese and English institutions were, then, functionally similar enough that they should not prevent similar economic outcomes and indeed they did not. By functional similarity is mean supporting similar outcomes in important areas of economic performance.

	Pomeranz makes a further striking observation: ``Furthermore, there is no reason to think that these patterns of development were leading `naturally' to an industrial breakthrough anywhere. Instead, all these core areas were experiencing modest per-capita growth, mostly through increased division of labor, within a context of basic technological and ecological constraints that markets alone could not solve'' \cite[p.~107]{pomeranz_great_2001}. Existing institutions anywhere were not sufficient to produce an industrial revolution. These observations help motivate the research question: what really happened in the English Industrial Revolution?

	Pomeranz's radical claims have generated both academic support and refutation. See Philip Huang \cite{huang_review_2002} for support and Peter Perdue \cite{peter_c_perdue_china_2009} and Ricardo Duchesne \cite{duchesne_rise_2004, duchesne_uniqueness_2011} for refutation. Duchesne further voices full--throated support for Western exceptionalism. 

	Pomeranz is not alone in observing the lack of functional institutional differences between China and England. R. Bin Wong \cite{wong_china_1997} provides a broad institutional comparison between China and England and comes to the same conclusion: functionally unremarkable institutional differences.

	Peer Vries \cite{vries_via_2003} attempts to straddle the arguments by claiming it was (must be?) culture but acknowledging that he cannot explain the fundamental reasons why people reacted differently; this puzzle further motivates the research. Kenneth Pomeranz replies to Vries' 2003 book in a most useful way since the book is written in Mandarin that Pomeranz reads. 

	Pomeranz \cite{pomeranz_via_2004} notes the following as areas of agreement between Vries and the ``California School'' of Chinese historical (including the relevant eighteenth century) revisionists:
\begin{itemize}
\item[--]The Qing state did not interfere with most economic transactions.
\item[--]Confucianism was no obstacle to economic development.
\item[--]Some (if not all) Chinese markets were remarkably well integrated.
\item[--]Even in the late eighteenth century Chinese agriculture had much higher land productivity than Britain.
\item[--]Differences in agricultural labor productivity were minimal.
\item[--]Differences in per--capita incomes (living standards?) were probably small.
\end{itemize}

	What of Pomeranz's last point? He uses ``per--capita income'' and Robert Allen \cite{allen_british_2009} successfully demonstrates that incomes at least as represented by real silver wages were significantly different between eighteenth century China and England. Living standards could have been relatively the same if for example the Chinese cost-of-living was relatively lower than English cost-of-living; Allen \cite{allen_great_2001, allen_wages_2007, allen_agricultural_2009} provides support for this difference as well.

	Pomeranz further notes areas of less agreement with Vries:
\begin{itemize}
\item[--]That differences in English and Chinese technical ability cannot have been very great before Britain's technological take-off.
\item[--]Less proletarianism in China (fewer potential wage labor or factory workers).
\item[--]Less emphasis on the comparative inability of China to relieve resource shortages.
\item[--]The importance of British mercantilism and state activism.
\end{itemize}

	Pomeranz views Vries' book as as representing a narrowing of the differences between the two great schools: culture versus geography. If true this current research could advance the role of geography and basic economic forces in a more hospitable climate.

\subsection{Chinese science and invention}

	There is a literature claiming China was not able to invent necessary industrialization technologies for cultural/institutional reasons. Several scholars refute this. Joseph Needham \cite{needham_science_1954} who started a still-ongoing project in eight volumes documenting the great Chinese scientific and technical achievements. Accepting this leads one to a useful question: why did they not commercialize their relevant technologies as the British did?

	John Hobson provides more direct and recent refutation of the literature that for various reasons Chinese science and technology were sufficiently deficient that the Chinese could not have had an industrial revolution. Hobson \cite{hobson_eastern_2004} makes two strong claims: first, each major developmental turning point of the ``oriental West'' was informed by assimilating Eastern inventions including ideas, technologies, and institutions that diffused from the more advanced East through oriental globalisation between 500 and 1800 CE; second, Europe after 1453 became imperialist and appropriated many Eastern resources including land, labor, and markets. This timing coincided with the Ottoman seizure of Constantinople and Pope Pius II resurrecting calls for a ``great Crusade'' to save Christendom from the Islamic threat.

	Specifically Hobson recounts that as early as 31 CE Chinese water-mills propelled the bellows in iron blast furnaces; significantly the Chinese water bellows used a piston-rod and driving belt that bore a ``remarkable'' resemblance to the mechanics in John Wilkinsons's precursor to James Watts' steam engine. A device very similar to Wilkinson's was described in Chinese print form in 1313 CE and Hobson suggests it was one of the Chinese technologies assimilated by the Europeans in this case the defining technology of the English Industrial Revolution \cite[p.~225]{hobson_eastern_2004}.

	Hobson additionally claims the Chinese preceded the English in replacing charcoal with coal to produce iron in the eleventh century, originated the blast furnace in the second century BCE, and in the fifth century CE developed the process to produce steel by fusing wrought and cast iron. 

	Hobson claims these inventions made their way West and became the key technologies of the English Industrial Revolution \cite[227]{hobson_eastern_2004}.

	It thus appears the Chinese were on the path to develop the technologies required to produce an industrial revolution; they did not and the question remains: why not? This research attempts to shed additional light on an answer.

\subsection{Growth theory}

	Concluding this introduction there is a brief review the major economic growth theories. While not the focus of this work, clearly there was no need for growth theory before the the English Industrial Revolution because there was no persistent growth in living standards. Most countries had similar living standards---close to subsistence. After the event living standards diverged widely; the goal of growth theory is to explain this divergence in an attempt to provide policy prescriptions for economies that have not converged toward the living standards bar set by advanced economies. This work will suggest extensions for growth theory.

	The first macroeconomic growth model many economists encounter is the Harrod--Domar model named for Roy F. Harrod \cite{harrod_essay_1939} and Evsey Domar who developed it independently. This model like most growth models is specified as a production function. For Harrod--Domar output is a function of (exogenous) capital stock---higher stock produces more output.

	The next significant growth model is Solow-Swan named for Robert Solow \cite{solow_technical_1957} and Trevor Swan. Solow--Swan extended Harrod--Domar by adding labor and productivity to the aggregate production function; productivity is assumed to be labor-augmenting technology or knowledge. This is still an exogenously-driven model.

	Paul Romer \cite{romer_origins_1994} developed a modern growth model that is the foundation for much subsequent work and contains the key feature of endogeneity---growth rates are determined by factors internal to the model and incorporates a constant marginal product of capital rather than a diminishing one as found in older theories.

	The striking fact is that none of these models explicitly incorporate energy as an input. Given what the energy-aware observers cited above say that seems like a major oversight. These models may indeed pick up energy inputs indirectly because the mainstream models always have capital stock as an input. A, perhaps the primary, purpose of capital stock is to apply energy inputs to the production process. However capital, being a stock that is depleted---used up---at a much lower rate than direct inputs such as energy is therefore not in the correct units we need to specify a model. This needs to be kept in mind in thinking about modeling output using an aggregated production function.

	There is a small but significant thread of research that does incorporate energy inputs. Perhaps the most provocative for mainstream models is the work of Robert Ayers \cite{warr_rexs:_2006, ayres_underestimated_2013}. Ayers specifies a production function using solely an energy input and takes the model to U.S. GDP data between 1900 and 1998. The model residuals vary depending on the time frame from about zero to twelve percent. This is a striking result in the context of the empirical fit of other growth models. For example the canonical fit that Robert Solow did on his labor and capital input model resulted in a residual term of about 88 percent. Ayres' empirical results suggest we need a different approach to growth modeling. So this paper investigates using energy inputs as the primary way of modeling the English Industrial Revolution and for other comparisons.

\section{English data, econometrics, and economics}
\subsection{A first look at the data}
	This section describes the three data series numerically and graphically.
\subsubsection{Sources and methods}
	The primary data used to model the English Industrial Revolution are gross domestic product (GDP), population (for per-capita measures), and energy consumption. Since the estimated energy consumption series starts in 1300 CE and as it may be useful for both the model and theory to incorporate that entire series population and gross domestic product series were composed starting in the same year. The time series stop at 1873 CE because that is the date based on econometric structural change analyses developed later in this paper that England's reign as the premier industrialized economy starts to decline.

Table \ref{tbl:dataSources} describes the sources for the data series.

\begin{table}[h!]
\caption{Data Sources}
\label{tbl:dataSources}
\begin{tabular}{lrll}
Data series&Year range&Geography&Source\\
\hline \hline
Energy consumption&1300--1873&England/Wales&Roger Fouquet (2008)\\
\hline
Gross domestic product&1300--1700&England&Graeme Snooks (1994)\\
&1741--1873&England/Wales&Lawrence Officer (2009)\\
\hline
Population&1300--1540&England&Graeme Snooks (1994)\\
&1541--1800&England&B. R. Mitchell (1988)\\
&1801--1873&England/Wales&B. R. Mitchell (1988)\\
\hline
\end{tabular}
\end{table}

	Roger Fouquet provides an invaluable time series of English energy consumption and related data. Fouquet's book \cite{fouquet_heat_2008} and papers with Peter Pearson \cite{fouquet_thousand_1998} are a remarkable accomplishment; these data are a major contribution to this work. Professor Fouquet gave me his data files and permission to use them. 

	Fouquet's methods for constructing the data series depend on the source of the energy and type of primary records. Overall he estimates energy consumption by energy services (essentially end use) in categories of domestic heating, industrial heating, industrial power, passenger transportation, freight transportation, and lighting.

	He uses actual data when possible and models data as necessary with a variety of techniques. He describes the methods in the data appendix to the book \cite{fouquet_heat_2008} and they include formal modeling, interpolation, extrapolation, and assumptions. His energy sources include wood, coal, food for horses (power and transport) and humans, wind and water power, and steam power (almost exclusively using coal). He does include electricity but its general use is beyond the study time frame so does not apply. Importantly for the work here there is no indication that any of his estimates use GDP and thus the energy consumption series and GDP are methodologically independent.

	The GDP estimates are composed from two sources. The period from 1300 to 1700 uses data from Graeme Snooks \cite{snooks_was_1994}. Snooks is a sometimes--controversial English historian in the sense he estimates higher growth rates over a longer period than other estimates say from Angus Maddison as an example. His data are useful because they matches the studies' geographic coverage needs in its time frame. In any case the GDP sources going that far back are rare. 

	In general the GDP and population data are benchmarks often decadal and sometimes longer. For econometric purposes interpolation among the benchmarks is useful. The interpolation method is called Stineman as described and implemented by Bjornsson and Grothendieck \cite{bjornsson_stinepack:_2012} in a R package named stinepack. All descriptive, modeling, and graphical work is done using the statistical analysis software R authored by the R Core Team \cite{r_core_team_r:_2014}.

	For GDP estimates from 1700 through 1873 the study uses data from Lawrence Officer \cite{officer_what_2009}.

	The population estimates are composed from three sources. Before 1801 the estimates are for England proper. After 1800 England became Great Britain and the population estimates are for a greater area.

	From 1300 to 1540 the study uses data from Snooks \cite{snooks_was_1994}. From 1541 to 1801 the series uses data from B. R. Mitchell \cite{mitchell_british_1988}. After 1801 the series uses a Mitchell \cite{mitchell_british_1988} data series for England and Wales. This geographic discontinuity did not significantly affect the splicing of the data as far as the results are concerned.

Figure \ref{fig:overall levels} presents the three historical series. 

\begin{figure}[h!]
\center
\caption{Author/time-span series of energy consumption, GDP, and population}
\label{fig:overall levels}
\includegraphics[width=0.9\textwidth]{C:/Users/Steve/Documents/GitHub/publish/diss2/images/overallLevels}
\end{figure}

	In this display we can see that both energy consumption and GDP have very similar shapes (the graphs are scaled to the same vertical distance so despite difference in units we can visually compare shapes) implying just at a visual level that they may be statistically cointegrated. this will be further discussed later. And we can see the levels increased most dramatically after 1700 and certainly after 1800.

	The population graph's shape is less steep in the later periods implying the increase in living standards we already know happened based on many sources. The Black Death's (1348--1353) effect on the population level and its relatively long recovery period show nicely on this graph.

\subsubsection{Modern economic growth}
	Simon Kuznets defined modern economic growth as sustained and high rates of growth of per--capita product and population \cite{kuznets_modern_1966}. Figures \ref{fig:ggdp} and \ref{fig:gdpLog} indicate that England experienced high rates of growth of per-capita product in (possibly) two eras from 1500 to 1600 that was not sustained and after 1750 that was mostly sustained. Clearly after about 1820 England had a high and sustained rate of growth in per--capita product here measured as gross domestic product. The annual rate after 1800 was 2.4 percent per-year total growth and 1.1 percent per--capita growth as seen in table \ref{tbl:growthByCentury}. Figure \ref{fig:popLog} shows the log of population growth that supports the Kuznets definition and mirrors GDP growth with a lag.

\linespread{1.0}
		\begin{figure}[h!]
		\caption{English real gross domestic product, \\
		levels and per--capita }
		\label{fig:ggdp}		
		\centerline{
		\mbox{\includegraphics[width=0.55\textwidth]{C:/Users/Steve/Documents/GitHub/publish/diss2/images/ggdp}}
		\mbox{\includegraphics[width=0.55\textwidth]{C:/Users/Steve/Documents/GitHub/publish/diss2/images/ggdppop}}
		}
		\end{figure}

		\begin{figure}[h!]
		\caption{English real gross domestic product, \\
		log levels and log per--capita}
		\label{fig:gdpLog}		
		\centerline{
		\mbox{\includegraphics[width=0.55\textwidth]{C:/Users/Steve/Documents/GitHub/publish/diss2/images/gdpLog}}
		\mbox{\includegraphics[width=0.55\textwidth]{C:/Users/Steve/Documents/GitHub/publish/diss2/images/gdpPopLog}}
		}
		\end{figure}
\linespread{1.9}		
		
	Examining the log levels and log per--capita transformations in Figure \ref{fig:gdpLog} note the interesting periods of growth rate changes. For example GDP growth rates plummet during the period of the Black Death, rise significantly after 1500, then go almost flat during the seventeenth century before recovering into high growth rates after about 1750. The flattening can be explained by what paleo-climatologists define as the ``Little Ice Age.'' During this era average temperatures fell by about two or three degrees centigrade enough to shrink agricultural output and by some accounts caused population declines of about thirty percent due to higher mortality (famine) and lower fertility rates. See Jean Grove \cite{grove_little_2003} and Geoffrey Parker \cite{parker_global_2014} in a masterful historical account of the ``long'' seventeenth century.

	Further comments appear below on the rise after 1500 in the population discussion although the significant per capita growth is somewhat of a surprise perhaps a continuation of the growth spurt in the middle ages and possibly some artifact in Snooks' GDP data.

	To see the magnitude of the growth rates by century and compounded annually refer to Table \ref{tbl:growthByCentury}. This table uses the same data as the graphs but does quantify the rates and the biggest surprise (certainly to our fifteenth-century economists) is the growth in living standards of over 100 percent between 1800 and 1873 and its annualized rate of 1.1 percent---a rate probably never attained or approached in prior eras. Of course this was possible because of the comparatively huge growth rate in total output (and its driver energy consumption) not completely matched by population growth. Note that we should discount the sixteenth century numbers due to possible artifacts in the Snooks data.


\linespread{1.0}
\begin{table}[h!]
\caption{Growth rates by century}
\label{tbl:growthByCentury}
\begin{tabular}{lrrrrrrrr}
&					&	1300 &	1400	&	1500 &	1600 &	1700 & 1801  &	\\
Year range	&	1300	&	- 1400	&	- 1500	&	- 1600	&	- 1700	&	- 1801	&	- 1873&Total	\\
\hline \hline
GDP Million\\ 2005 GBP	&	3115	&	815	&	994	&	6,031	&	8,361	&	18,110	&	102,811&	\\
Century-over-century\\rate of growth&&-0.738&0.220&5.066&0.386&1.166&4.677&32.008\\
Compounded annual \\rate of growth&&-0.013&0.002&0.018&0.003&0.008&0.024&0.006\\
\hline
Energy consumption&1.7	&	1	&	1.3	&	2.2	&	3.6	&	11.6	&	66.1&	\\
Century-over-century\\rate of growth&&-0.412&0.300&0.692&0.636&2.222&4.698&37.882\\
Compounded annual \\rate of growth&&-0.005&0.0026&0.005&0.005&0.012&0.024&0.006\\
\hline
Per--capita GDP\\2005 GBP&542&  329&  421& 1,484& 1,663& 1,999& 4,392\\
Century-over-century\\rate of growth&&-0.393& 0.282&2.521&0.121&0.202&1.198& 7.108\\
Compounded annual \\rate of growth&&-0.005&0.002&0.013&0.001&0.002& 0.011&0.004\\
\hline
\end{tabular}
\end{table}
\linespread{1.9}

	Turning to the population data Figure \ref{fig:popLog} provides a log levels picture. Note the similar patterns to the other series; a dip in growth rates due to the Black Death, the acceleration in the sixteenth century, a deceleration in the seventeenth century, probably a lagged reaction due to Little Ice Age fertility decreases, and the acceleration starting in the mid-eighteenth century. The vertical red lines indicate statistical structural breaks dating probable significant changes in the growth rates.

\begin{figure}[h!]
\center
\caption{Log of population, with structural breaks}
\label{fig:popLog}
\includegraphics[width=0.9\textwidth]{C:/Users/Steve/Documents/GitHub/publish/diss2/images/popLog}
\end{figure}

	Examining these data patterns and the timing of their changes in growth rates along with the energy--consumption series discussed later suggests theoretical macroeconomic interpretations described next.

\newpage
\subsubsection{An energy revolution}
	This paper's central assertion is that the EIR was primarily an energy revolution on the supply--side. More generally, this was a demand--side consumer goods consumption revolution supported by a supply--side energy source revolution. To begin support for that hypothesis first review the data:

	Figure \ref{fig:energyLog} presents the log transformation of energy consumption over the study period; the vertical lines are formally determined structural breaks.\footnote{The structural breaks use an F-test methodology on the time series as implemented in the $R$ package struccchange \cite{zeileis_testing_2003}} The log presentation enhances rate-of-change and potential structural differences in the series. We can observe four significantly different periods or regimes. The first is from 1300 to 1500 a period dominated by the Black Death epidemic; energy consumption clearly drops then recovers. The second is from 1500 to roughly 1600 as determined by the structural breaks. The third is the period from 1600 to roughly 1750; note that the rate-of-change of energy growth in this period is approximately the same as in the prior period; this rate of change similarity is confirmed by the presentation in Table \ref{tbl:growthByCentury}. The final period is from 1750 through 1873; clearly the energy consumption rate-of-change accelerates as confirmed by the structural breaks in Figure \ref{fig:energyLog} and table \ref{tbl:growthByCentury}.

	Based on the structural changes and based on the hypothesis that the EIR was an energy revolution one could propose that the revolution happened as two main eras: one starting in the mid-to-late sixteenth century \footnote{This validates John U. Nef's hypothesis of an early start to the British Industrial Revolution \cite{nef_rise_1932}} and one starting after 1750. Under this hypothesis the first revolution would have set the stage for the second. The second revolution required energy infrastructure built for the first.

\begin{figure}[h!]
\center
\caption{Log of energy consumption, with structural breaks}
\label{fig:energyLog}
\includegraphics[width=0.55\textwidth]{C:/Users/Steve/Documents/GitHub/publish/diss2/images/energyLog1.png}
\end{figure}

\clearpage

	If we were to overlay the energy levels or logs charts with the GDP levels or logs charts the similarities would be informative; perhaps a more productive view is figure \ref{fig:energyVsGdpLD}. This figure shows levels of energy consumption through the study period and has a standardized series of GDP for the same period. By standardized is meant matched in levels at the first period; the series' evolutions thus show differences in growth rates through continuous time. Again we see four distinct regimes. The most notable features are the periods from 1500 to 1600 when growth in GDP clearly leads energy growth and after 1750 (especially after 1800) when energy growth leads GDP growth.

		\begin{figure}[h!]
		\caption{Energy consumption vs. standarized GDP,\\
		levels and differences }
		\label{fig:energyVsGdpLD}		
		\centerline{
		\mbox{\includegraphics[width=0.55\textwidth]{C:/Users/Steve/Documents/GitHub/publish/diss2/images/energyVsGdp}}
		\mbox{\includegraphics[width=0.55\textwidth]{C:/Users/Steve/Documents/GitHub/publish/diss2/images/energyVsGdpDiff}}
		}
		\end{figure}
		
	The dynamics of GDP growth and energy consumption growth can be seen more clearly by taking the differences shown in the right panel.

	The Black Death and its aftermath affected the relatively flat net economic performance from 1300 to 1500 but set the stage for a growth boom in the period 1500 to 1600; this is subject to the caveat already mentioned regarding Snooks' GDP data but nonetheless there was a substantial pick up in growth rates during that period. We can see this by looking at energy consumption graphs that show smaller growth rates than GDP but still very significant growth. Table \ref{tbl:growthByCentury} also clearly shows this comparison. In the period 1600 to 1750 growth in both GDP and energy consumptions flattened and then boomed again during the period 1750 to 1873.

	Observing the panels in Figure \ref{fig:energyVsGdpLD} suggests a very close correlation between energy consumption and GDP; the major divergence in these series is in the fourth period that has been identified after 1800 when data accuracy for GDP is probably the best in the sample. Even so this divergence is not large.  More formal tests of the correlations will appear in the next section.

\subsection{Econometric and economic analyses}
	To formalize the observations in the previous section correlations, paired t--tests, Granger--causality tests, and formal structural--break tests are used.

It is perhaps methodologically instructive to briefly discuss what is not covered in this paper. The original intent was to do a cointegrated vector error correction model (VECM) of energy and GDP. This methodology approaches equilibrium in a useful way for long-run macroeconomic models in the following sense: the only equilibrium a VECM assumes is a statistical one; this is sharply different than normal economic modeling that presumes some mean--reversion---a long run dynamic of stationarity. When one looks at any of the long--run macroeconomic series they clearly are not stationary. The are either exponential or super--exponential. 

The results of cointegration tests on energy consumption and GDP series are that they are cointegrated of order about 2.5--clearly in the super-exponential range. Why then not model with this specification? Simply any of the graphs displaying energy consumption and GDP indicate a very high degree of correlation. And a very wise statistician teaches that you only need to do what is econometrically sufficient to make your point. So we  proceed with that thought in mind.

Next some simple analytic statistics are presented to support the hypothesis that the EIR was at its root an energy revolution responding to a positive aggregate demand shock.

\subsubsection{Econometric analysis}

Starting simply a Pearson's correlation coefficient and a paired t-test of energy consumption and GDP yield the results in table \ref{tbl:fitTest}: 

\linespread{1.2}
\begin{table}[h!]
\caption{Energy and GDP fit tests}
\label{tbl:fitTest}
\begin{center}
\begin{tabular}{lrr}
\hline\hline
Test&Statistic&p-value\tabularnewline
\multicolumn{1}{c}{}\tabularnewline
\hline
Pearson's correlation&$0.998$&\tabularnewline
\hline
Paired t-test&$5.592$&4.991e-07\tabularnewline
\hline
Chi-square&2864&0.0004998\tabularnewline
\hline
\end{tabular}
\end{center}
\end{table}
\linespread{1.9}

%These simple results suggest that the two series are statistically very similar; a more formal co-integration test could be expected to be positive, and is presented as Appendix B in section \ref{app:Appendix B}. However, for the purposes of this paper, a scatterplot of the series 

	These simple results suggest that the two series are statistically very similar; in fact at that level of correlation one could think about claiming that these two series are the same---the result of a common data--generating process. A more formal co--integration test could be expected to be positive and will be  presented in a future version. For the purposes of this paper a scatterplot of the series 
is shown in figure \ref{fig:scatterplot}. The solid green line is a linear fit; the solid red line is a \textit{lowess} (non-parametric and non-linear) fit.

\begin{figure}[h!]
\center
\caption{Scatterplot of energy consumption vs. GDP}
\label{fig:scatterplot}
\includegraphics[width=0.9\textwidth]{C:/Users/Steve/Documents/GitHub/publish/diss2/images/scatterplot.png}
\end{figure}

	Clearly there is a very high correlation between the two series. For current purposes more formal modelling is not needed. Overall statistically these two series are very close to being the same that is they share a common data generating process. In a strong sense this is a validation of the thermodynamic view of economic production and growth at least in the long run.

	From an economic point of view this graph suggests a Leontief fixed-factors production function that would also be consistent with a Sraffian production interpretation.

	However this overall view does hide important dynamics that the data contain. By examining these more subtle results next the stage is set for telling a history of the EIR. The study uses a Granger causality test to do so. \cite{granger_investigating_1969}. 

	Using the Granger bi--variate test to examine changing dynamics provides the results in Table \ref{tbl:grangerEnergyGdp}; the eras tested were suggested by the statistics above and in total.

\linespread{1.2}
\begin{table}[h!]
\caption{Granger tests of energy-GDP dynamics}
\label{tbl:grangerEnergyGdp}
\begin{tabular}{lrrl}
Era&Energy $\sim$ GDP Pr($>$F)& GDP $\sim$ Energy Pr($>$F)&AD/AS regime\\
\hline \hline
1300--1500&0.0106&0.0003&EMP, Black Death, \\&&&wages/family income increasing\\
1500--1600&0.1939&0.6126&Positive demand shock\\
1600--1750&0.3529&0.5185&Energy supply constraint\\
1750--1873&0.0024&0.1100&Positive supply shock,\\&&&``virtuous'' macro feedback cycle\\
\hline
1300--1873& 0.0002& 0.0361&Total study period\\
\hline
\end{tabular}
\end{table}
\linespread{1.9}

	During the first energy/GDP era Granger causality between energy and GDP runs both ways at significant levels; while not ignoring these results we should not over--interpret what was happening given the huge aggregate demand and aggregate supply shocks of the Black Death. It is significant for later eras that the Black Death caused wages to rise and the European Marriage Pattern (EMP) \cite{hajnal_european_1965} increased family incomes entering the early modern period.

	During the second energy/GDP era of 1500 to 1600 causality from GDP growth to energy consumption is weakly significant; energy Granger--causing GDP growth is not at all significant. However there is narrative evidence that this was an important proto-industrial period when home manufacture for markets became important; this is the ``Industrious Revolution'' of Jan de Vries \cite{de_vries_industrial_1994}. There is further evidence that the English state supported an early version of Import Substitution Industrialization to replace imports and to increase exports \cite{thirsk_economic_1978}. These events support the idea that demand must have been growing in domestic consumption markets, for military goods demand from the government, and eventually for exports.

	These events occur in a backdrop of global population growth during a century of benign agricultural climate; croplands expanded, food was plentiful, real wages likely grew, nuptiality and fertility increased, and England participated in this bounty. The positive effect on agricultural productivity of the Columbian Exchange from transplanting highly efficient new--world potato and maize crops to Europe was in play. Alfred Crosby \cite{crosby_columbian_1972} provides the seminal account of this important event. The transfer increased productivity both extensively (the new crops could be grown on previously unproductive land) and intensively (more output both per hectare and per labor hour). Population growth is positive even though the era continues to be dominated by Malthusian population dynamics.

	In the third energy/GDP era of 1600 to 1750 neither direction of causality is significant. This will turn out to have important implications as the paper builds the history for the EIR.

	In the fourth energy/GDP era of 1750 to 1873 we again see both directions of causality significant with GDP Granger--causing energy consumption being the stronger.

	Notably over the entire study period GDP Granger-causes energy consumption more significantly than energy Granger--causes GDP but causality is significant in both directions.

	Finally, structural breaks in the series are examined; these are usually correlated with significant changes in underlying economic dynamics and will figure into the story of the EIR.

	Figure \ref{fig:structural} juxtaposes frames with logs of energy consumption, gross domestic product, and population, each with formal structural break lines noted. The point here is to note the correspondence of the structural breaks again suggesting the same underlying data generating process but with causality--implying lags in the population dynamics.

\linespread{1.0}
\begin{figure}[h!]
		\caption{Structural break comparison}
		\label{fig:structural}		
		\centerline{
		\mbox{\includegraphics[width=0.33\textwidth]{C:/Users/Steve/Documents/GitHub/publish/diss2/images/energyLog1}}
		\mbox{\includegraphics[width=0.33\textwidth]{C:/Users/Steve/Documents/GitHub/publish/diss2/images/gbpgdplog}}
		\mbox{\includegraphics[width=0.33\textwidth]{C:/Users/Steve/Documents/GitHub/publish/diss2/images/popLog}}
		}
\end{figure}
\linespread{1.9}

\subsubsection{Economic analysis} \label{sec:EIRstory}

	Now it is possible to compose a story of the EIR as supported by the data presented above. The eras refer to Table \ref{tbl:grangerEnergyGdp}.

	Energy/GDP era one because of the Black Death disaster saw both negative demand and supply shocks but set the stage for the subsequent EIR eras through long--term effects on wages, incomes, and effective aggregate demand. More broadly the five centuries prior to era one comprise the Medieval Warming Epoch (or Period) supporting higher agricultural output and population levels with both supporting increased effective aggregate demand through expanded incomes. See Figure \ref{fig:temps}.

\begin{figure}[h!]
\center
\caption{Late Holocene temperatures. \textit{source:} NASA and IPCC composite}
\label{fig:temps}
\includegraphics[width=0.9\textwidth]{C:/Users/Steve/Documents/GitHub/publish/diss2/images/2000_Year_Temperature_Comparison.png}
\end{figure}

	In energy/GDP era two wages rose due to the negative labor supply shock of era one. Aggregate demand had positive shocks as a result both of rising wages and of rising family incomes due to delayed marriages and women in the labor force---the EMP outcomes---and favorable agricultural conditions.  Expanded household production (Jan de Vries 1994) and explicit import substitution policies starting with Henry VIII and continuing through Edward VI and Elizabeth I, supported increased aggregate demand \cite{thirsk_economic_1978}. See table \ref{tbl:monarchs} for reigns. Aggregate supply expanded as can be seen by the stronger growth of energy consumption. Refer to table \ref{tbl:growthByCentury} or figure \ref{fig:energyLog}. This era provided the positive demand shocks and increasing supply constraints that caused the EIR. It started here.

	John Nef amplifies this view. He tells the story of era two as the ``age of timber.'' While his time frames are a bit offset he says ``\ldots no less appropriate to call the sixteenth and seventeenth centuries an age of timber'' \cite[p.~191]{nef_rise_1932}. Nef tells a very rich story of rising use of timber for industrial and home heating use, for construction, and the beginnings of a timber crisis. Dates for era two are 1500--1600 that Nef's dates overlap by going into era three.

	Rates of growth in energy/GDP era three for both GDP and energy consumptions stagnated. This still puzzles scholars including Braudel and Hobsbawm, but there are several potential stories that can be sketched here. Return to figure \ref{fig:temps} and notice that a decline in mean temperatures occurred in the early modern era. This era is called the Little Ice Age and is believed to have been a global phenomenon. This would have opposite effects from the Medieval Warming Epoch such that the climate conditions should reduce agricultural output and population levels and cause a negative aggregate demand shock due to reduced income levels. In a sense this is also a negative energy supply shock featuring shrinking growing space and time due to less effective insolation.  

	Scholarly discussion of both the Medieval Warming Epoch and the Little Ice Age seems concentrated among paleoclimatologists; yet they often refer to the effects on the economy sometimes citing contemporaneous accounts. Jean Grove provides a survey in \underline{The Little Ice Age} \cite{grove_little_2003}. Hubert Lamb is often cited as an early researcher.\footnote{See for example \cite{lamb_aspects_1980}. Lamb describes failed grain harvests in Scotland and the disappearance of the cod schools in the Atlantic. These examples are typical though not the focus in the climatology literature. They do provide a plausible economic explanation for the stagnation in GDP and the lagged stagnation in population growth.}

	A related story that fits the data and the history is that this era was one of a negative energy supply shock due to deforestation and growth in the whole economic system thus slowed. This era was the transition between primarily wood--supplied energy to primarily coal--supplied energy for both industrial and home heating needs. As London grew because of internal growth, exports, and world trade domination wood became scarcer and more expensive driving demand for coal for heating from the north east. You can see this pattern during the 1600 to 1750 era three in the following figure \ref{fig:woodCoal}.

\begin{figure}[h!]
\center
\caption{Coal and wood energy sources\\\textit{Source:} Pearson \& Fouquet}
\label{fig:woodCoal}
\includegraphics[width=0.9\textwidth]{C:/Users/Steve/Documents/GitHub/publish/diss2/images/woodCoal.png}
\end{figure}

	Notably this is also the era Nef calls the ``first energy crisis'' \cite{nef_early_1977}. According to Nef during the period 1550 to 1700 increased heating and building demand for wood and reduced woodlands due to agricultural demands caused wood prices to rise dramatically.

	We can hypothesize that this series of events provided the economic pressure to cause the first phase of the energy revolution---the transition from wood to coal for heating needs.

	A further potential explanation appeals to political events mainly the large number of wars during the period. The contemporary anecdotes were that war was economically stimulative \cite{thirsk_economic_1978}. 

	As research for this paper progressed reviews of further work by Jan de Vries showed he refuted any climatic explanation. In discussing the 1600 to 1750 era de Vries indeed says the climate evidence is not consistent with population evidence; the current work shows that population lags GDP and GDP was plausibly affected by climate change suggesting a more consistent data set. Separately note that energy/GDP era three has the same year boundaries as de Vries \cite{de_vries_economy_1976}. De Vries also has an extensive empirical look at Dutch temperatures and various measures of agricultural output. In the end he comes to few conclusions except that time--series data are essential \cite{de_vries_measuring_1980}.

	This demand for heating coal arising from the first energy crisis and the fortuitous geology of the English coal mines created the path necessary to support energy/GDP era four when the second phase of EIR accelerated into modern economic growth via a virtuous mutually reinforcing growth cycle between GDP and energy consumption. 

	The geology story is that the coal mines were water--infused and as they were mined deeper more water had to be pumped out. This provided an economically feasible site for the seminal but very inefficient Newcomen steam engines to pump the water. The coal was essentially free to power the engines. Human or horse power were too expensive. And as the steam engines gained efficiency they began to be applied to the products of industrial capitalism. That is the story of energy/GDP era four that becomes the age of steam. %I turn next to telling that story in more detail; again it is an economic story, supported by the data.

	A list of inventions that depended on and drove demand for steam power is impressive. Here is a broad list of Industrial Revolution--era inventions from many sources including Joel Mokyr (1992). See Table \ref{tbl:EIRinventions}. Many though not all of these inventions are steam--driven. Some such as Arkwright's water spinning frame were originally water-powered; these inventions switched to steam power as that technology matured. Others such as the sewing machine were eventually converted to electricity a dominant power source of what some call the second industrial revolution. Electricity is still largely produced by steam ``engines'' (generators). 

	Of course John Hobson \cite{hobson_eastern_2004} would claim Asiatic origins for many of these inventions; thus the puzzle of ``why not China?'' remains or perhaps the question arises why did the Chinese not commercialize the labor-saving inventions they were at least on the path to develop? To answer this it is useful to compose a narrative of China's failed industrial revolution next then begin work on a theory of industrial revolutions.

\linespread{1.2}
\begin{table}[h!]
\caption{Industrial Revolution inventions (partial list)}
\label{tbl:EIRinventions}
\begin{tabular}{rl}
Year&Inventor/invention\\
\hline \hline
1712 & Thomas Newcomen patents the atmospheric steam engine\\
1733 & John Kay invents the flying shuttle\\
1764 & James Hargreaves invents the spinning jenny\\
1768 & Richard Arkwright patents the spinning frame\\
1769 & James Watt invents an improved steam engine\\
1775 & Jacques Perrier invents a steamship\\
1779 & Samuel Crompton invents the spinning mule\\
1783 & Benjamin Hanks patents the self-winding clock\\
&Englishmen, Henry Cort invents the steel roller for steel production\\
1784 & Andrew Meikle invents the threshing machine\\
1785 & Edmund Cartwright invents the power loom\\
1786 & John Fitch invents a steamboat\\
1794 & Eli Whitney patents the cotton gin\\
&Welshmen, Philip Vaughan invents ball bearings\\
1797 & Wittemore patents a carding machine\\
&A British inventor, Henry Maudslay invents the first metal or precision lathe\\
1799 & Alessandro Volta invents the battery\\
&Louis Robert invents the Fourdrinier Machine for sheet paper making\\
1800 & Frenchmen, J.M. Jacquard invents the Jacquard Loom\\
&Count Alessandro Volta invents the battery\\
1804 & Richard Trevithick, an English mining engineer, developed the first steam-powered locomotive\\
1809 & Humphry Davy invents the first electric light -- the first arc lamp\\
1814 & George Stephenson designs the first steam locomotive\\
&Joseph Nic�phore Ni�pce was the first person to take a photograph\\
1825 & William Sturgeon invented the electromagnet\\
1829 & American, W.A. Burt invents a typewriter\\
1830 & Frenchmen, Barthelemy Thimonnier invents a sewing machine\\
1831 & American, Cyrus McCormick invents the first commercially successful reaper\\
&Michael Faraday invents a electric dynamo\\
1834 & Henry Blair patents a corn planter, he is the second black person to receive a U.S. patent\\
&Jacob Perkins invents an early refrigerator type device -- an ether ice machine\\
1835 & Englishmen, Henry Talbot invents calotype photography\\
&Englishmen, Francis Pettit Smith invents the propeller\\
&Charles Babbage invents a mechanical calculator\\
1836 & Francis Pettit Smith and John Ericcson co-invent the propeller\\
&Samuel Colt invented the first revolver\\
1837 & Samuel Morse invents the telegraph\\
\hline
\end{tabular}
\end{table}
\linespread{1.9}


\iffalse

1839 � American, Charles Goodyear invents rubber vulcanization; Frenchmen, Louis Daguerre and J.N. Niepce co-invent Daguerreotype photography; Kirkpatrick Macmillan invents a bicycle; Welshmen, Sir William Robert Grove conceives of the first hydrogen fuel cell

1843 � Alexander Bain of Scotland, invents the facsimile

1845 � American, Elias Howe invents a sewing machine; Robert William Thomson patents the first vulcanized rubber pneumatic tire

1850 � Joel Houghton was granted the first patent for a dishwasher

1851 � Isaac Singer invents a sewing machine

1852 � Henri Giffard builds an airship powered by the first aircraft engine � an unsuccessful design

1853 � George Cayley invents a manned glider

1854 � John Tyndall demonstrates the principles of fiber optics

1855 � Isaac Singer patents the sewing machine motor; Georges Audemars invents rayon

1858 � Hamilton Smith patents the rotary washing machine 
&Jean Lenoir invents an internal combustion engine

1862 � Richard Gatling patents the machine gun
& Alexander Parkes invents the first man-made plastic

1866 � Alfred Nobel invents dynamite 
&Englishmen Robert Whitehead invents a torpedo

1867 � Christopher Scholes invents the first practical and modern typewriter

1868 � Robert Mushet invents tungsten steel
&J P Knight invents traffic lights

1873 � Joseph Glidden invents barbed wire

1874 � American, C. Goodyear, Jr. invents the shoe welt stitcher

1876 � Alexander Graham Bell patents the telephone; Nicolaus August Otto invents the first practical four-stroke internal combustion engine; Melville Bissell patents the carpet sweeper

1877 � Thomas Edison invents the cylinder phonograph or tin foil phonograph; Eadweard Muybridge invents the first moving pictures

1881 � Alexander Graham Bell invents the first crude metal detector; David Houston patents the roll film for cameras; Edward Leveaux patents the automatic player piano

1884 � George Eastman patents paper-strip photographic film; Frenchmen, H. de Chardonnet invents rayon; James Ritty invents the first working, mechanical cash register; Charles Parson patents the steam turbine

1885 � Harim Maxim invents the machine gun; Karl Benz invents the first practical automobile to be powered by an internal-combustion engine; Gottlieb Daimler invents the first gas-engined motorcycle

1886 � Josephine Cochrane invents the dishwasher; Gottlieb Daimler builds the world�s first four-wheeled motor vehicle

1888 � John Boyd Dunlop patents a commercially successful pneumatic tire; Nikola Tesla invents the AC motor and transformer

1891 � Jesse W. Reno invents the escalator

1892 � Rudolf Diesel invents the diesel-fueled internal combustion engine

1895 � Lumiere Brothers invent a portable motion-picture camera, film processing unit and projector called the Cinematographe. Lumiere Brothers using their Cinematographe are the first to present a projected motion picture to an audience of more that one person

1898 � Edwin Prescott patents the roller coaster; Rudolf Diesel receives patent #608,845 for an �internal combustion engine� the Diesel engine

1899 � John Thurman patents the motor-driven vacuum cleaner

1900 � The zeppelin invented by Count Ferdinand von Zeppelin
\fi

\linespread{1.2}
\begin{table}[h!]
\caption{Early modern English monarchs}
\label{tbl:monarchs}
\center
\begin{tabular}{lll}
Monarch&Reign&House\\
\hline
Henry VIII&1509-1547&Tudor\\
Edward VI&1547-1553&Tudor\\
Mary I&1553-1558&Tudor\\
Elizabeth I&1558-1603&Tudor\\
James I&1603-1625&Stuart\\
Charles I&1625-1649&Stuart\\
Oliver Cromwell&1653-1658&Commonwealth\\
Richard Cromwell&1658-1659&Commonwealth\\
Charles II&1660-1685&Stuart\\
James II&1685-1688&Stuart\\
Mary II&1689-1694&Stuart\\
William III&1689-1702&Stuart\\
Anne&1702-1707&Stuart\\
\hline
\end{tabular}
\end{table}
\linespread{1.9}

\clearpage

\section{Chinese comparative data and institutions}
	It is time to focus on those key facts about China and its paradoxical failure to participate in the growth miracle emerging from the English Industrial Revolution. Recalling our group of fifteenth century conference--goers we remember the claim that they would have bet the ranch on China having the first industrial revolution while most had never heard of England. In this sense this story could be tagged as ``The empire that did not bark'' in the spirit of Arthur Conan Doyle \cite{bentley_murder_1941}.
	
	As it turns out the cleverest among them knew that China had already had an industrial revolution; more precisely they knew that they had a partial industrial revolution---identified as a first phase revolution---and being good growth economists knew that it positioned China for the second phase. These terms  are defined later. For now note first that the data for China are not nearly as rich as for England but after a preamble to set the comparative context between China and England let us examine the Chinese data. 

\subsection{Preamble to Chinese growth}	
	Given that recent scholarship suggests that eighteenth--century per--capita incomes in England and similar parts of China were roughly comparable and had both grown somewhat since the sixteenth century \cite{pomeranz_great_2001} why did English output then accelerate into the first continually sustained period of per--capita growth ever experienced---modern economic growth---and Chinese output relatively stagnate? 
	
	Since China is a highly integrated society sharing world population dominance with India by all the known rules explaining economic dynamics up to that time as summarized by the Reverend Thomas Malthus (1973) it should have dominated the world economy. And it did. From Angus Maddison's data (Angus Maddison 2007) China and India had roughly 50 percent of both world population and gross domestic product (GDP) at the beginning of the sixteenth century while England accounted for 1 percent of population and 1 percent of GDP. Yet England's growth so dominated the eighteenth and nineteenth century that in $1900$ England's share of world GDP was 9 percent while her population was only 3 percent of the world total. China and India's combined share of GDP in $1900$ had fallen to 20 percent while their combined population was still 44 percent of the world total. %\footnote{\citet{maddison_maddison_2010}}
	
	Many scholars search for and discern some combination of social, cultural, and institutional factors to explain the phenomenon of the Industrial Revolution. Yet the magnitude of the post eighteenth-century growth trajectory differences imply a level of English exceptionalism in those factors that begins to strain credulity. Are we to believe that over a very few generations English ``growth enabling'' institutions somehow grew sufficiently superior to Chinese institutions to account for the growth differences? This class of explanation is even more problematic in that it at least implicitly assumes that some one or some group understood what institutions were needed for this sui generis event and had the powers to form them.
	
	A further mystery is the ``Needham question'' that arises from the fact, as Joseph Needham \cite{needham_science_1954} documented in the eight volumes of \underline{Science and Civilisation in China}, that China had great scientific and technological discoveries but lost the ``race'' to both the Scientific and Industrial Revolutions. Needham seems to support the idea of functionally sufficient Chinese institutions of the very kind needed to supply the inventions required to participate in the revolutions. A later scholar John Hobson \cite{hobson_eastern_2004} explicitly makes this claim.
	
	In the long sweep of history England had a relatively brief period of per--capita growth dominance. By no later than 1875 the growth revolution was quickly spreading to North Western Europe, North America, and Meiji Japan. If England's lead in growth was uniquely determined by a specific set of exceptional institutions is there evidence that such usually long--gestation changes in culture, institutions, and society itself were so quickly transmitted to other cultures?
	
	And if transmitted institutional exceptionalism accounts for the rapid spread of growth why was it transmitted relatively narrowly until the second--half of the twentieth century? Why didn't China immediately converge? Is the relevant effect in fact that societies and their institutions oppose fundamental economic changes that in turn cause societal changes until the economic forces becoming overwhelming? Was China ``not barking'' because there was nothing to bark at because the dog saw nothing but the long familiar non--threatening agrarian empire? This explanation is certainly consistent with a story of China not enjoying English--style exceptionalism. Or is it rather a story that there were no Chinese economic forces that at the macroeconomic level would have driven Chinese entrepreneurs to English style energy innovation. For English exceptionalism claims see Max Weber \cite{weber_protestant_2002}, David Landes \cite{landes_unbound_1969, landes_wealth_1999}, and Deirdre McCloskey \cite{mccloskey_bourgeois_2007, mccloskey_bourgeois_2010}.
	
	This paper explores the counter--question: what underlying \textit{economic} reasons might account for this remarkable series of events and non-events? Above it is argued that what England discovered and transmitted to the world was an energy revolution in economic activity. Why did China fail to follow that revolutionary path until the twentieth century? Do basic \textit{economic} explanations provide a more satisfactory story for this ``great divergence?''

	A related question is one of primary or ultimate causality rather than monocausality. Institutionalists claim that superior institutions were the primary cause of the Industrial Revolution. One can show evidence and claim that superior economic dynamics were the primary cause while fully acknowledging the proximate supporting and surrounding institutional and cultural fabric as a necessary condition.	

\subsection{A first look at data and institutions}
	In this section the Chinese data in a global context and the institutional background is reviewed.
	
\subsubsection{Sources and methods}
	The Chinese data is not nearly as rich as the English data; nonetheless Angus Maddison \cite{maddison_world_2007}, Vaclav Smil \cite{smil_energy_1994, smil_energy_2008}, J. W. de Zeeuw \cite{j._w._de_zeeuw_peat_1978}, Robert Hartwell \cite{hartwell_revolution_1962, hartwell_markets_1966,hartwell_cycle_1967, hartwell_demographic_1982}, and the U. S. Energy Information Administration \cite{u.s._energy_information_administration_energy_????} provide interesting clues.
	
		Again for context, to support the thinking of our fifteenth century conference attendees, and to understand the scale of the divergence we can begin by examining world population, gross domestic product, and the resultant per--capita GDP through the current historical period covering the crucial pre--industrial and Industrial Revolution periods while showing the current levels for context. The initial data is from \cite{maddison_world_2007}. Maddison measures GDP in 1990 International Geary-Khamis Dollars that describe purchasing power parity (PPP) adjusted output. Maddison's data set whatever its challenges is widely cited and is where many comparative scholars start. This study also starts with it. 
		
\subsubsection{Regional population and GDP dynamics}		

		The top two panels in Figure \ref{fig:poplevel1900} show that both world population and GDP levels for years 1500 through 1900 CE underwent unprecedented growth; the bottom two proportion panels demonstrate that much of the growth was in Europe and the western offshoots. It is clear that China and India dominated both world population and GDP until about $1700$. These are the data that our conference group would have been relying on. However when world GDP started a period of super--exponential growth the proportion charts show that Western Europe and the United States dominated GDP growth and had population growth above the world rate.

\linespread{1.2}
		\begin{figure}[h!]
		\caption{Population and GDP levels from 1500 to 1900; population and GDP proportions from 0 to 2008. \textit{Source:} Data from \cite{maddison_world_2007}, graphs by author.}
		\label{fig:poplevel1900}

		\includegraphics[width=0.5\textwidth]{C:/Users/Steve/Documents/GitHub/publish/1310utahSoc/images/maddisonregpoplevels1900.png}
		\includegraphics[width=0.5\textwidth]{C:/Users/Steve/Documents/GitHub/publish/1310utahSoc/images/maddisonreggdplevels1900.png}\\
		\includegraphics[width=0.5\textwidth]{C:/Users/Steve/Documents/GitHub/publish/1310utahSoc/images/maddisonregpoppct.png}
		\includegraphics[width=0.5\textwidth]{C:/Users/Steve/Documents/GitHub/publish/1310utahSoc/images/maddisonreggdppct.png}

		\end{figure}	
\linespread{1.9}
		
		The pattern of faster population growth rate in both Chinese and English proto-industrial periods remains an open demographic question \cite[p.~22]{pomeranz_great_2001} though on this chart the English growth is hard to see.  \footnote{One theory (Alfred Crosby \cite{crosby_columbian_1972} and others) asserts that the post--``Columbian Exchange'' arrival in Europe and China of American crops like maize and potatoes increase agricultural productivity per land unit by 3 or 4 times, enabling a rise in otherwise Malthusian constrained subsistence population levels.} 
		
		To abstract from that next examine per--capita GDP growth. Figure \ref{fig:capita} shows per--capita GDP by regional and national groupings of interest from 1 through 1900 CE, using the underlying Maddison \cite{maddison_world_2007}. Two facts stand out. First, China maintains a relatively constant level of per--capita GDP throughout the period. The Chinese did not become absolutely poorer; however China did not share in the great average output growth of the Western nations. Second, the grouping denoted the EU-11, \footnote{The EU-11 grouping includes Austria, Belgium, Denmark, Finland, France, Germany, Italy, the Netherlands, Norway, Sweden, and Switzerland.} led by England is increasing in per--capita GDP starting in $1500$ with rapid increases after $1800$. The Western Offshoots show a similar growth pattern of per--capita GDP. The sustained productivity growth arising during the Industrial Revolution led to sustained standard-of-living increases. This sui generis episode of modern economic growth stands in stark contrast to China and the rest of the world. \footnote{The Western Offshoots are statistically dominated by the United States but also include Canada, Australia, and New Zealand.}

\linespread{1.2}
		\begin{figure}[h!]
		\centering

		\caption{Comparative World Per--Capita GDP. \textit{Source:} Data from \cite{maddison_world_2007}, graphs by author.}
		\label{fig:capita}

		\includegraphics[width=0.8\textwidth]{C:/Users/Steve/Documents/GitHub/publish/1310utahSoc/images/ggdpcapitadodge.png}
		\end{figure}
\linespread{1.9}
		
		The lack of a growth pattern in Chinese per--capita GDP leads to a fascinating question: How much is our perception of this fact coloured by our twenty-first century point-of-view? More formally what would our expectations for the rate of growth of per--capita GDP have been as an astute economic observer in eighteenth--century China? Or, for that matter in England?
		
		The evidence is that the classical economists had no expectations for any prolonged positive growth in GDP per--capita because they had never observed that phenomenon. Thomas Malthus clearly represents the then widespread point-of-view that expectations were for subsistence GDP meaning essentially zero--growth per--capita levels forever. Thus our fascination with what actually happened and our dramatically different modern expectations.
		
%\begin{comment}		
		
		The next several charts illuminate these dramatic changes. Figures \ref{fig:1500pop}, \ref{fig:1820pop}, and \ref{fig:1900pop} trace the evolution of global population shares from CE $1500$ through $1900$ grouped by major regions. We see China undergoing a population explosion and collapse between $1500$ and $1900$ CE with a peak share of $37\%$ of world population in $1820$. England is on a steady growth march starting at $1$ percent share in $1500$ and ending at $3$ percent in 1900. We can discern the proto--industrial population growth in both economies prior to $1820$ and only England continues growth after that that.

\linespread{1.2}
		\begin{figure}[h!]
		\centering
		\caption{World population shares, 1500 CE}
		\label{fig:1500pop}
		\includegraphics[width=0.8\textwidth]{C:/Users/Steve/Documents/GitHub/publish/1310utahSoc/images/1500pop.png}

		\end{figure}
			
		\begin{figure}[h!]
		\centering
		\caption{World population shares, 1820 CE}
		\label{fig:1820pop}
		\includegraphics[width=0.8\textwidth]{C:/Users/Steve/Documents/GitHub/publish/1310utahSoc/images/1820pop.png}
		\end{figure}
		
		\begin{figure}[h!]
		\centering
		\caption{World population shares, 1900 CE}
		\label{fig:1900pop}
		\end{figure}
		\includegraphics[width=0.8\textwidth]{C:/Users/Steve/Documents/GitHub/publish/1310utahSoc/images/1900pop.png}
\linespread{1.9}		

		Figures \ref{fig:1500gdp}, \ref{fig:1820gdp}, and \ref{fig:1900gdp} trace the path of global GDP shares from $1500$ through $1900$ CE grouped by major regions. We see China's gobal GDP share staying roughly in line with its populations share so peaking in $1820$ at the end of the world proto--industrial era.
				
		\begin{figure}[h!]
		\centering
		\caption{World GDP shares, 1500 CE}
		\label{fig:1500gdp}
		\includegraphics[width=0.8\textwidth]{C:/Users/Steve/Documents/GitHub/publish/1310utahSoc/images/1500gdp.png}
		\end{figure}
		
		\begin{figure}[h!]
		\centering
		\caption{World GDP shares, 1820 CE}
		\label{fig:1820gdp}
		\includegraphics[width=0.8\textwidth]{C:/Users/Steve/Documents/GitHub/publish/1310utahSoc/images/1820gdp.png}
		\end{figure}
		
		\begin{figure}[h!]
		\centering
		\caption{World GDP shares, 1900 CE}
		\label{fig:1900gdp}
		\includegraphics[width=0.8\textwidth]{C:/Users/Steve/Documents/GitHub/publish/1310utahSoc/images/1900gdp.png}
		\end{figure}
		
		England's GDP share has grown dramatically from the $1$ percent proportional to its population share in $1500$ to $2.5$ times population share in $1820$ to $3$ times population share in $1900$.
		
%\end{comment}
		
		These charts represent highly aggregated data and thus potentially mask important underlying structural and regional differences especially in China. Kenneth Pomeranz for example asserts that the standard of living in regions of China was equivalent to Western Europe in $1800$ (differently than the Maddison data that however is for all of China) and that the standard--of--living adjusted wage levels in the Lower Yangzi region in China were at English levels in $1800$ \cite[107]{pomeranz_great_2001}. Decomposing the standard of living into wages and cost-of-subsistence softens those differences except in the Lower Yangzi but in any case we need to explain the post--$1820$ divergence.
		
		Two main explanatory threads wrestle or perhaps dance with each other: One thread appeals to institutional differences the other to economic and geographic differences exploited by inventor/entrepreneurs. The essential factor to decode is the \textit{prime} mover recognizing that there are interaction effects over time that are surely important.
		
		The study proceeds by questioning the institutional argument that the prime mover in the Industrial Revolution was English institutional exceptionalism and sets up the economic/geographical prime mover hypothesis; this suggests analyzing the growth divergence between China and England as an exercise in comparative micro-- and macroeconomics. But first we should examine the political economies to establish that there exists essential (functional) institutional sufficiency for growth in each country.

\subsubsection{Comparative institutions}
		
		The logic for underweighting English institutional exceptionalism as the primary factor explaining the EIR is that whatever the institutional differences between China and England there were sufficient functional similarities to yield similar economic results up until $1800$ at least in the most comparable Chinese region the Lower Yangzi. It is thus difficult to imagine sufficient institutional differences to cause such a dramatic divergence over the next century. This logic uses the work of R. Bin Wong and Kenneth Pomeranz.
		
		First a comparison of political economies in post--$1500$ late Imperial China and early modern Europe from R. Bin Wong:
		
		\begin{quotation}
		
		``The Chinese state maintained an active interest in the agrarian economy, promoting is expansion over large stretches of territory and its stability through uneven harvest seasons\ldots Despite considerable variation in techniques, there was basic agreement through the eighteenth century about the type of economy officials sought to stabilize and expand. They supported an agrarian economy in which commerce had an important role'' \cite[p.115--116]{wong_china_1997}.
		\end{quotation}
		
		\begin{quotation}
		``Mercantilism, the dominant philosophy of political economy in Europe between the late sixteenth and the early eighteenth century, posed a close relationship between power and wealth. For a state to become powerful, society had to become wealthier. This was achieved by expanding economic production in rich core areas and by extending trade across the country and especially beyond it\ldots competition for wealth on a global scale became a component of European state making. European states promoted the production and commerce of their private entrepreneurs, whose successes contributed to the consolidation and prosperity of competing states'' \cite[p.140]{wong_china_1997}.
		
		\end{quotation}
		
		Wong thus contrasts a Chinese imperial agrarian state interested in social stability with a group of European power elites competing over a zero-sum economic game with military Mercantilism. Yet until the eighteenth--century divergence roughly the same level of subsistence was the norm.

		
		Moving to Kenneth Pomeranz who evaluates Chinese and English and wider Asian and Western European economic levels at more granular scales involving agriculture, transport, and livestock capital, longevity, health and nutrition, birthrates, accumulation, and technology.
		
		\begin{quotation}
		``\ldots as late as the mid-eighteenth century, western Europe was not uniquely productive or economically efficient\ldots many other parts of the Old World were just as prosperous and ``proto-industrial'' or ``proto-capitalist'' as western Europe\ldots What seems likely is that no part of the world was necessarily headed for such a [industrial] breakthrough.''
		
		``\ldots the production of food, fiber, fuel, and building supplies all competed for increasingly scarce land\ldots western Europe\ldots became a fortunate freak only when unexpected and significant discontinuities in the late eighteenth and especially nineteenth centuries enabled it to break through the fundamental constraints of energy use and resource availability that had previously limited \textit{everyone's} horizons\dots the new energy itself came largely from a surge in the extraction and use of English coal\ldots'' \cite[p.~206--207]{pomeranz_great_2001}.
		\end{quotation} 
		
		Pomeranz's detailed comparative evaluation thus somewhat contradicts Maddison's data and highlights both institutional differences and similarities but the differences are irrelevant in the end simply because England uniquely led the organic--to--fossil energy transition that was the revolutionary foundation for and the prime--mover at the center of the Industrial Revolution. Next turn to the economic incentives that England had and China did not to make that transition.	

\section{Toward a theory of industrial revolutions}

%{\huge{from english paper - pick and choose}}

We have already examined the GDP and energy consumption data for the fourth era. To complete the story we can now appeal to economic theory. First,  we summarize the eras using macroeconomic theory illustrated in aggregate demand---aggregate supply charts; second, we examine the transition for industrial and domestic heating from wood--to--coal that unleashed a highly scalable source of heat energy; third, we address the question of what caused the English inventor/entrepreneur to spend the time and money to create the inventions of the first and second phases of the EIR particularly the steam engine that enabled the transition from muscle--power to steam--power using coal as the energy input. To do this we can appeal to standard microeconomic theory.

Figures \ref{fig:asad1} and \ref{fig:asad2} display the four eras in an aggregate demand---aggregate supply (AD---AS) framework. The dotted lines indicate prior locations of AD---AS; solid lines indicate the ending locations. Lines colored red indicate the constraint in each era. These are obviously abstract depictions of the history told above. This is done for two reasons; first to solidify and emphasize the history so that the debate can proceed; second to provide a framework for later projects incorporating the institutional and cultural events into the history. If we can agree on the AD---AS by era then we can hypothesize about those events that might have caused the location or shape to change and then test those ideas in an econometric framework.

\linespread{1.0}
\begin{figure}[h!]
		\caption{Aggregate Supply---Aggregate Demand \\ Four energy/GDP regimes}
		\label{fig:asad1}		
		\centerline{
		\mbox{\includegraphics[width=0.5\textwidth]{C:/Users/Steve/Documents/GitHub/publish/diss2/images/era1}}
		\mbox{\includegraphics[width=0.5\textwidth]{C:/Users/Steve/Documents/GitHub/publish/diss2/images/era2}}
		}
\end{figure}
		
\begin{figure}[h!]		
		\caption{Aggregate Supply---Aggregate Demand \\ continued}
		\label{fig:asad2}
		\centerline{	
		\mbox{\includegraphics[width=0.5\textwidth]{C:/Users/Steve/Documents/GitHub/publish/diss2/images/era3}}
		\mbox{\includegraphics[width=0.5\textwidth]{C:/Users/Steve/Documents/GitHub/publish/diss2/images/era4}}				
		}
\end{figure}
\linespread{1.9}


A notable observation is that energy/GDP era four is the first when aggregate supply was not the constraint; according to the Granger causality tests (see Table \ref{tbl:grangerEnergyGdp}) supply and demand were jointly constraining in that era. Statistically only GDP Granger--causing energy consumption is significant at normal levels but the removal of barriers for consuming energy was likely the uniquely defining event of the era.

Secondly for the theoretical discussion of the EIR it is important to consider at the microeconomic level what can explain the event. Microeconomics is relevant and important to help answer this question as at the end of the economic day people have to have individual incentives to innovate and commercialize no matter what the macroeconomic pressures and/or institutional influences are. This paper mainly discusses the supply--side of the story having already suggested a story of important demand--side factors in Section \ref{sec:EIRstory}. So the question becomes what were the incentives or motivations of the English inventors and entrepreneurs during energy/GDP eras two and three that is from 1500 through 1750.

For this analysis we rely on several sources: John Nef's monumental work documenting the rise of the English coal industry; the contemporaneous comments of a key participant in the EIR; the excellent work of Robert Allen; and an appeal to microeconomic theory.

The microeconomic story of the EIR turns out to be two stories so in effect two energy revolutions. The first revolution or better for comparative work a first--phase industrial revolution tells the story of the essential transition from wood--to--coal for domestic and industrial heating applications. Essential because as important in its own right as it is to continue to scale heat production in the face of rising population and therefore rising aggregate demand the first transition lays the foundation of building a coal extraction, transportation, and distribution infrastructure that is essential for supporting the ever more energy--hungry second--phase industrial revolution. The second--phase's signature development replaces muscle--power with steam--power that largely coal--fueled.

The phase--one revolution lasted through most of the first three AD---AS eras (see Table \ref{tbl:grangerEnergyGdp}) until about 1700. To see this transition's time boundaries refer to Figure \ref{fig:woodCoal} and note the take--off in coal consumption levels after 1700.

Can we appeal to basic microeconomics to help understand this revolution? This is possible with John Nef's help. Examine the data taken from Nef \cite{nef_rise_1932} and shown in Figure \ref{fig:woodprice}. Note that starting about 1540 English wood prices rose by almost a factor of eight by 1700. This results both from rising aggregate demand and deforestation. Importantly even compared to general price inflation wood prices increased by twice the change in the general price level. During the same period, coal prices were declining at least until 1600 and in northern England remained much lower still. See Figure \ref{fig:allen_energy}.

\linespread{1.0}
\begin{figure}[h!]
		\caption{English wood and general price indices \textit{Source:} \cite[pp.~158,221]{nef_rise_1932}}
		\label{fig:woodprice}		
		\center
		\includegraphics[width=0.7\textwidth]{C:/Users/Steve/Documents/GitHub/publish/diss2/images/woodprice}
\end{figure}
\linespread{1.9}

\clearpage

	With the price spread between coal and wood used for such an essential economic input as energy-for-heating moving dramatically in coal's favor the basic economic mechanism of input--price substitution should work. It does explain the transition. To formalize this we can write:

		\begin{equation}
		\label{eq:mrp1}
		\frac{\text{Marginal Product}_{\text{wood Joule}}}{\text{Price}_{\text{wood Joule}}} \ll \frac{\text{Marginal Product}_{\text{coal Joule}}}{\text{Price}_{\text{coal Joule}}},
		\end{equation}

or if one prefers a non-neoclassical writing:

		\begin{equation}
		\label{eq:mrp1}
		\frac{\text{Average Product}_{\text{wood Joule}}}{\text{Price}_{\text{wood Joule}}} \ll \frac{\text{Average Product}_{\text{coal Joule}}}{\text{Price}_{\text{coal Joule}}}.
		\end{equation}

Either writing leads to the same theoretical conclusion: assuming no qualitative difference in the two inputs in terms of work being done (a Joule is a Joule) with the data showing the right--hand--side coal ratio being significantly greater than the wood ratio we would expect entrepreneurs to substitute away from wood to coal. And this is exactly what happened (see Figure \ref{fig:woodCoal}).

This was not an easy transition. Coal was dirtier---perhaps even nastier---than coal and this required new technologies both industrially (for example in iron making) and domestically. But it was a powerful enough economic incentive that the inventors did what they do best---invent.

Some sense of the difficulties that the inventors eventually overcame is related by Robert Allen. Allen argues the following logic chain: Coal was plentiful and cheap in both northwest and northeast England. As London grew rapidly due to English success in international trade London experienced high wages that spread throughout England and faced increasing heating prices due to local deforestation. Thus beginning in the sixteenth century the ``coal--burning house'' (new room and chimney designs were required as well as new stove designs) that was invented in London led English coal demand and production to increase \cite[p.~82]{allen_british_2009}. This invention took more than a century to replace wood--burning stoves but the economic incentives were eventually sufficient. See Figure \ref{fig:woodCoal}.

Moving to the phase--two industrial revolution of replacing muscle--power with steam--power can basic microeconomics help explain this revolution as well? Again the claim is yes. Here we ask Desaguliers, Robert Allen, and theory for assistance.

Jean (or John) Theophilus Desaguliers had a large influence on the EIR. He was an eighteenth century English ``natural philosopher'' (physicist), a member of the Royal Society, colleague of Sir Isaac Newton, and author of \underline{A Course of Experimental Philosophy}. This was an influential 1734 two--volume engineering text that contained a chapter on ``Fire-Engines'' (steam engines). In this chapter Jean Theophilus describes the economic and scalability motives of replacing men and horses with coal-fired steam engines to pump water out of Newcastle mines.  Profit was on his mind \cite[Vol.II, pp.~467--468]{desaguliers_course_1734}.  The age of the industrial capitalism fueled by fossil energy was dawning.

Figure \ref{fig:desagulier} shows a page of his manuscript.

\linespread{1.0}
\begin{figure}[h!]
		\caption{Desaguliers manuscript}
		\label{fig:desagulier}		
		\center
		\includegraphics[width=0.95\textwidth]{C:/Users/Steve/Documents/GitHub/publish/diss2/images/desagulier1}\\
		\includegraphics[width=1.05\textwidth]{C:/Users/Steve/Documents/GitHub/publish/diss2/images/desagulier2}
\end{figure}
\linespread{1.9}

Beyond the quaintness of the 1734 English prose this man demonstrated the soul of a profit--maximizing capitalist. In that context let us examine some data that drove Desaguliers.

Figure \ref{fig:wage-energy} is from Robert Allen and shows the ratios of real wages to energy costs (the cheapest source by location) by benchmark city around 1700.

\linespread{1.0}
\begin{figure}[h!]
	\center
	\caption{Real wage--to--energy price ratios\\\textit{Source:} Robert Allen \cite{allen_british_2009}}
	\label{fig:wage-energy}
	\includegraphics[width=0.7\textwidth]{C:/Users/Steve/Documents/GitHub/publish/diss2/images/wage-energy.png}
\end{figure}
\linespread{1.9}

\clearpage

Clearly Newcastle in 1700 had high wages and very low energy costs exhibiting by far the largest ratio in the sample. Those are the economic fundamentals that faced Desaguliers and motivated his profit comment. London had the second largest ratio and thus the strong economic incentives existed there as well. Beijing had the lowest ratio and that is a topic investigated later.

Intuitively if this wage--to--energy cost ratio is high enough as it was in England entrepreneurs and inventors will have a large incentive to develop the steam technologies to enable the revolution. Refer to Table \ref{tbl:EIRinventions} for a list of the inventions that were converted to steam--power, were originally developed for steam--power, or used steam--power to convert steam--power to a different transmission medium---electricity.

While the economics of these ratios may be intuitive why not appeal to microeconomic theory to help us understand what motivated Desaguliers, Newcomen, Watt and other founding fathers of the EIR. Equation \ref{eq:mrp2} is a variation on production theory that will be familiar to those who remember their Econ 101. A major topic of mainstream production theory is how entrepreneurs maximize profits given the derived demand curves of the various input choices. 

%This equation is a variation on that theme:\footnote{We can proceed either with a neo-classical factor substitution argument, or a more general classical view of normal prices of production. Either approach will react to the enormous productivity-enhancing energy supply shock that was the Industrial Revolution. A more challenging story to tell is one which identifies the sources of aggregate demand that supported expansion of English production. Here, I simply stipulate that aggregate demand existed.}

		\begin{equation}
		\label{eq:mrp2}
		\frac{\text{Average Product}_{\text{labor Joule}}}{\text{Price}_{\text{labor Joule}}} \ll \frac{\text{Average Product}_{\text{steam Joule}}}{\text{Price}_{\text{steam Joule}}}
		\end{equation}

Instead of using different substitutable inputs such as labor and capital we apply the theory to the different sources of energy since that is essentially the only non--substitutable input as in you must have Joules from whatever source to do any economic transformation. If we take the numerators in Equation \ref{eq:mrp2} to be equal abstracting again from the difficulties in invention that were eventually solved then because of the much lower price of English coal--Joules than wages for labor--Joules the relentless (in the face of rising wages) pressure will be for the inventors to invent and the entrepreneurs to commercialize steam--power thus creating the machine age and completing the EIR.

	These equations need additional terms to cover the amortization of whatever research and development and capital equipment is necessary to apply either kind of Joule but clearly from just what is written we see that when wage--to--coal--energy cost ratios are sufficiently high entrepreneur/inventors will be motivated to substitute coal--Joules for human--Joules. And that is what happened at the micro level to drive the EIR first in Newcastle atop the mines, then in the English textile mills, then in other English industries, then in transportation, and later spreading to the world.


What of China? China is our natural experiment; as it turns out China experienced a phase--one industrial revolution---from wood to coal---in the tenth and eleventh century Sung dynasty. We will complete that story in the next paper of this dissertation. For now we can look to later dynasties---the Ming and the Qing--- to see why assuming the Chinese had completed phase--one of a revolution they did not complete phase--two and thus confounding our conference attendees.
			
		As we have seen Robert Allen proposes a relatively simple factor substitution argument that relies on differences in relative labor and energy prices between China and England most dramatically between Newcastle and the rest of the world. Refer to Figure \ref{fig:wage-energy}. Essential to his argument is that England almost uniquely was a high--wage and low--energy--cost economy \cite[p.~34]{allen_british_2009}. 

	
		We can use his supporting data to understand from microeconomic theory what did not happen in China. Refer to figure \ref{fig:allen_wages} and note how low Chinese wages were compared to England in the pivotal 1700 time frame.
		
\linespread{1.0}
		\begin{figure}[h!]
		\centering
		\caption{World wages, 1375--1825 CE. \textit{Source:} Allen (2009) \label{fig:allen_wages}}
		\includegraphics[width=0.6\textwidth]{C:/Users/Steve/Documents/GitHub/publish/1310utahSoc/images/gworldwages.png} 
		\end{figure}			
\linespread{1.9}		

		He also examines world energy prices; we have already noted England had the lowest energy prices in the world. This led to a high English wages--to--energy prices ratio that fuelled the energy transition so notably compared to China \cite[p.~140]{allen_british_2009}. The basis for this argument can be seen in the Figure \ref{fig:allen_energy}. Note that these prices reflect the cheapest energy source usually either wood or coal.

\linespread{1.0}
		\begin{figure}[h!]
		\centering
		\caption{Comparative world energy prices, 1450--1800 CE. \textit{Source:} \cite{allen_british_2009} \label{fig:allen_energy}}
		\includegraphics[width=0.6\textwidth]{C:/Users/Steve/Documents/GitHub/publish/1310utahSoc/images/allen_energy.png} 
		\end{figure}			
\linespread{1.9}		

	Referring back to the Allen ratios in Figure \ref{fig:wage-energy} note that the relative price ratio of wages--to--energy prices was highest in Newcastle and lowest in Beijing. Thus there was a strong economic incentive among inventors and entrepreneurs to substitute coal--power for muscle--power in Newcastle and almost none in Beijing. With little economic incentive for Chinese inventors to invent (though we have seen they were capable of doing so) the technologies needed for an industrial revolution and certainly no wage--energy cost ratio pressure to commercialize the relevant inventions the Chinese did not complete a phase--two industrial revolution. Muscle--power was simply too cheap.
	
\iffalse	
\section{Bibliography}
\bibliographystyle{plainnat}
\bibliography{paper1}
\section*{\end{document}}
\fi

\section{Conclusion}
		The main questions in this paper are about how industrial revolutions come about. By considering the successful English attempt and the unsuccessful Chinese attempt we find that England learns to consume unconstrained energy inputs while China does not. However this story is more generally about economic growth. It is about the English economy spontaneously---no economy had achieved this before---learning to deliver modern economic growth. Simon Kuznets \cite{kuznets_modern_1966} defines this as persistent growth in living standards and population a new economic regime overturning centuries even millennia of Malthusian growth constraints.
		
		Why is learning to consume unconstrained energy inputs so fundamental to the growth story? Many economists agree that growth in living standards requires growth in labor productivity measured most simply as output per labor hour input. Growth economists tell many stories about this often observing comparative institutional and cultural differences in economies with significantly different living standards and conclude logically that those must be the relevant differences. And many tell stories of capital accumulation as the key growth enabler delivered by whatever their important institutional mechanisms might be. These institutional changes and capital growth are indeed observables in the history.
		
		But if you are persuaded that it is energy inputs that fundamentally determine---in fact constrain---economic output and productivity growth then we must fully understand the dynamics that deliver the important outcomes so we can tell policy makers that wish to pursue modern economic growth how to do so. To make growth prescriptions about proximate causes such as institutional changes and capital accumulation may miss the crucial ultimate cause requirements. Examining historical examples occurring before we ``knew'' how to create modern economic growth can help clarify our prescriptions. That is the hoped outcome of this paper.
		
		Briefly recalling the growth data presented above refer to Table \ref{tbl:growthByCentury} and note that English annual per--capita growth rates by century abstracting from the problematic Snooks--influenced early GDP data only approach modern levels of 1.1 percent after 1820---after the English economy collectively learned to remove energy constraints on economic output. This learning is shown in the growth rates of energy consumption in the same table.
		
		Now recall from Figure \ref{fig:capita} how English living standards, with northwest Europe following closely, accelerated away from the rest of the world including China after 1700 and especially after 1820. With this in mind review Table \ref{tbl:histEnergyCons} showing per--capita energy consumption for some relevant economies across time.

\linespread{1.0}		
\begin{table}[h!]
	\caption{Per--Capita Primary Energy Consumption, annual Tonnes of Oil Equivalent. \\ \textit{Source:} Angus Maddison, $^a$de Zeeuw, $^b$US DOE EIA}
	\label{tbl:histEnergyCons}
	\center
	\begin{tabular}{lrrrr}
	\hline
	Year&England&China&Netherlands&India\\
	\hline \hline
	1650$^a$&&&0.63&  \\
	1820&0.61&&&\\
	1840$^a$ &&&0.33& \\
	1870&2.21&\\
	1970$^a$ &&&8.07&0.33 \\
	1973&&0.48&&\\
	1998$^b$&6.56&1.18\\
	2008$^b$&5.99&2.56&9.86&  \\
	\hline
	\end{tabular}
\end{table}
\linespread{1.9}

	For the current argument note that Chinese per--capita energy consumption in 1973 is \textit{significantly less than English per--capita energy consumption in 1820}. This data and the other country data in this table further support the essential claim that regardless of proximate causes energy consumption appears to be the ultimate cause for modern economic growth.
	
	Underweighting cultural, institutional, and social reasons for the great divergence in energy consumption and living standards between China and England raises the question then how to explain it? By appealing to basic economics. The aggregate demand---aggregate supply analysis in Section \ref{sec:EIRstory} sketch out the macroeconomic background in four eras. Importantly this section covers the important demand--side story covered in additional detail in the next paper; but that is not the current focus. They focus here is how to rid the economy of supply--side constraints---primarily energy inputs.
	
	Hypothesizing two phases for the English Industrial Revolution allows a clear microeconomic explanation of the key input factor source substitutions founded on the most basic mechanism---relative price substitution. Phase one substitutes coal for wood in domestic and industrial heating applications essentially removing that energy input constraint. For power applications such as producing commodities using muscle power, phase two substitutes steam--power for muscle--power and thus removes the non--scalable muscle--power constraint on output thus increases labor productivity and living standards.
	
	Note that a crucial political--economy question---distribution---is not covered here. In fact that story is likely where institutional explanations will dominate.
	
	Finally note that once the growth--genie is out of her bottle certain institutions---sometimes autocratic states---are able to take the energy lesson described here and apply it directly to building economies delivering modern economic growth. Japan, the ``Asian Tigers,'' and modern China come to mind. Studying their energy consumption history is a future project.

\iffalse	
\section{Bibliography}

\bibliographystyle{plainnat}
\bibliography{paper1}
\section*{\end{document}}


\section{\end(comma editing)}
%\fi



% from english paper -- pick and choose
	
%		Allen further argues the following logic chain: Coal was plentiful and cheap in both Northwest and Northeast England. As London grew rapidly due to English success in international trade, London experienced high wages that spread throughout England, and faced increasing heating prices due to local deforestation. Thus, beginning in the sixteenth century, the ``coal-burning house'' that was invented in London led English coal demand and production to increase (Allen \citeyear[p.~82]{allen_british_2009}).  See figure \ref{fig:allen_coal}.
		
		English coal mines had an important geological problem with water infiltration; human or animal pumping became increasingly expensive and lacked scale as coal demand increased and the mines went deeper. The first real, practical use of the inefficiently crude Newcomen steam engines was to pump the water from the mines using otherwise surplus coal. As the inventions became more efficient, they became both the literal and figurative engines of the Industrial Revolution (Allen \citeyear[pp.~86 -- 93]{allen_british_2009}).
		
		Kenneth Pomeranz tells a different story, an essentially macro story. Beyond his revisionist and contested view of eighteenth century China and England being at essentially similar development levels, he contends that Western Europe was running out of land, was thus at the precipice of impending land scarcity (Allen \citeyear[p.~264]{allen_british_2009}). Pomeranz further contends the English ``escape'' was due to coal and colonies. To illustrate the growth dilemma refer to the chart in figure \ref{fig:eng_wood}. 
		
\begin{center}
Figure \ref{fig:eng_wood} about here
\end{center}
			

	
		Figure \ref{fig:eng_wood} is the result of a counterfactual exercise asking if it was feasible for England to meet its actual energy consumption demand during the Industrial Revolution by substituting other energy sources for coal -- in this particular experiment wood.  The graph shows that England would indeed have required \textit{all} of its landmass to be forest in order to supply a sustainable fuel source to meet its energy requirements by the last quarter of the nineteenth century (Roger Fouquet \citeyear{fouquet_heat_2008}). England ``escaped'' this bottleneck by learning how highly scalable fossil-energy based macroeconomies are. The graph for the Chinese counterfactual experiment, using recent energy consumption data, shows a similar pattern, and a remarkably similar outcome -- China is now approaching the point at which the entire country would be forested if it had to provide its energy demand with wood.
		
		To illustrate how powerful the virtuous feed-back cycle was from English entrepreneurs individually substituting coal for humans, figure \ref{fig:mtoe_log} is a log of English energy consumption, showing the structural breaks and related change in slopes. The slope changes in this chart indicate a super-exponential growth in energy consumption, a signature of the English energy revolution.

\begin{center}
Figure \ref{fig:mtoe_log} about here
\end{center}
			
				
		Pomeranz further provides a macroeconomic story of the lack of a sustained mineral energy transition in China. What is stunning from his telling is that eleventh-century Sung China \textit{did} start a coal-based energy transition. It was based on the large coal and iron deposits in North and Northwest China close to the then political, demographic, and economic center. Chinese iron production in $1080$ likely exceeded non-Russian European production in $1700$ (Pomeranz \citeyear[p.~62]{pomeranz_great_2001}).
		
		The region was then subjected to a series of ``staggering catastrophes'' (Pomeranz \citeyear[p.~62]{pomeranz_great_2001}) including Mongol invasions and occupations, civil wars, enormous floods, and plague. The demographic and economic centers shifted south, incurring large transportation costs for raw materials. Coal-based industrialization never recovered until well into the twentieth century despite eighteenth century attempts by the government to develop the mines to alleviate fuel shortages in the Lower Yangzi Delta.
		
		Further, while this is an intriguing historical event, there is no evidence that the Chinese had any path-dependent incentives to develop steam engines as the technical problem in the Chinese coalfields was (and is) ventilation to prevent spontaneous combustion, not the water pumping problem of English fields. And thus they did not develop industrial steam engines.



\end{document}

\begin{abstract}
	
	England, during the period leading up to and spanning the first Industrial Revolution, collectively learned how to consume a virtually unconstrained quantity of fossil (carbon) energy. Led by the period's effective aggregate demand growth, this resulted directly in productivity growth which then led to modern economic growth in living standards for the first time in recorded history. 
		
	Studying the event empirically, we can use recent long-period series estimates of levels of English energy consumption, Gross Domestic Product, and population to test the hypothesis that this was primarily an \textit{energy} revolution with important but mostly proximate institutional and cultural support.
	
	Then a natural experiment is run using Ming and Qing China, for which we have some limited data, but important institutional comparisons that would not preclude China from completing an industrial revolution.
	
	The outcome should provide insights into economic development for growth economists, highlighting the importance of energy transitions for growth of economic systems. Additionally, the analytic framework developed can be applied across time and geography, adding insights to ongoing development puzzles.% and to the realistic chances of curbing ecologically damaging mineral (fossil) energy consumption for ecological economists and others interested in that critical topic.
	\end{abstract}
\section{Introduction}	%last

Unravelling the history of the English Industrial Revolution remains in the center ring of economic history. Beyond its historical significance, it holds major lessons for development economists in modern eras.

In this paper, I propose a methodological appeal to data-informed economic principles to explain the miracle. And I conclude that it was primarily an energy revolution; the English learned how to consume virtually unconstrained amounts of fossil energy. This directly led to modern economic growth for the first time in history.

Many, but not all, historians look to primarily institutional or cultural explanations for the event often expressed as a form of English exceptionalism; I propose a taxonomy in table \ref{tbl:taxonomy}. But this is not a paper about institutions; it is about economics. I try to make a strong case that while (a very few) necessary institutions were proximate, they were not sufficient, and do so by telling a compelling economic story, with economics often driving (endogenous) institutional change. The important exogenous institutional/cultural changes likely relate to expanded aggregate effective demand.

\begin{center}
Table \ref{tbl:taxonomy} about here
\end{center}

One must include Max Weber (\citeyear{weber_protestant_2002}) among the canonical exceptionalists, although indirectly bearing on England. Rather than lengthening this paper with details of this taxonomy, those will be in a forthcoming project. So I will proceed with the economics, acknowledging the few potentially causal cultural/institutional events that are required on the demand and supply sides.

As an important example of emphasizing how economic pressures led institutional changes, John Nef (\citeyear{nef_rise_1932}) relates how the economic pressure of English deforestation on wood prices influenced the post-English Reformation transfer of mineral-rich properties from the Church to the Crown, and the Crown's support of enclosures to consolidate mineral rights. Both of these ``institutional'' changes \textit{resulted} from economic pressures, improved the profitability of leasing mineral rights for coal and other mineral extraction, and thus had the macroeconomic result of an increase in the aggregate supply curve. A complete description of institutional changes must await further research.

The contributions I hope to make are to build a framework for analyzing the event which: coherently explains the event; can be extended to test the hypothesized importance of any institutional or cultural events; can accommodate new data series; proposes a structure of different energy/GDP regimes; re-dates the start of the event, moving it considerably earlier than many historians propose (John Nef excepted); uses statistical methods to understand the dynamics of the event; and applies macroeconomic and microeconomic theoretical principles to describe and explain the incentives embedded in this great and \textit{sui generis} event.

%\section{Variations on the story}
\section{Research questions} %1
I seek to identify empirically, economically, and eventually institutionally, what facts constituted the English Industrial Revolution. What was it, why did it occur, why did it happen when it did, why did it happen in England and only England? This paper addresses a subset of this agenda, describing what happened empirically, and suggesting the economic pressures and events that caused this result.

\section{Hypotheses}
The English Industrial Revolution (henceforth EIR) was the first example of modern economic growth (\cite{kuznets_modern_1966}). There were both macroeconomic and microeconomic forces that were causal. The primary driver of the EIR was an energy consumption revolution. There is limited statistical space for a very few exogenous causal institutional or cultural event clusters.	%2

I claim that the English Industrial Revolution was actually two related energy revolutions: the first substituted fossil mineral energy (coal) for wood for heat generation for both industrial and domestic uses; the second and later one substituted fossil mineral energy for labour energy inputs. Both were economically driven; the second one led directly to modern economic growth, and was enabled by the first.

\section{Research approach}
As this is a largely data-driven project, I first describe the data sources and comment on their limitations.

\subsection{Data}
Table \ref{tbl:dataSources} enumerates the primary data sources in this paper.
Figure \ref{fig:overall levels} displays the three data series keyed to the sources.

\begin{center}
Table \ref{tbl:dataSources} about here
\end{center}

The energy consumption data from Roger Fouquet covers England and Wales for the entire study period (1300 -- 1873). A word about why I end analysis in 1873: that is the end date Robert Allen (\citeyear{allen_british_2009}) places on the EIR. I can make a case from the data that it was a few years later, perhaps 1876, but there is little difference.

The gross domestic product data is composed from data series from Graeme Snooks and Lawrence Officer. The normalizing index is 2005 Great Britain Pounds. For this study period, those were the closest to England/Wales gross domestic product data that I have found.

The population data is composed from data series from Graham Snooks and Mitchell. For this study period, these were the closest to England/Wales population data I have found. Figure \ref{fig:overall levels} summarizes the data series by author/time-span.

\begin{center}
Figure \ref{fig:overall levels} about here
\end{center}

All such historical series are clearly composed, modelled, estimated, and thus fraught; a common problem with macroeconomic data to the present day. That said, I reserve special admiration in general for the work of the English economics historians. And these series are generally bounded by their starting point, their ending point, and various benchmarks along the way. The historians use a variety of methods to validate their work. In general, they cannot be too far wrong with the worst case being shifts by several decades in the shape of the curve. And the later data is generally better.

I do not claim these series are definitive for all time, simply the best I know of at this point, and possibly good enough. Their shapes clearly affect the analysis to follow. As better series appear, I will incorporate them into this analytic framework.

\begin{comment}	
compare gdp, energy, pop series. table for sources\\
pop -- maddison vs. my current splice\\
gdp -- maddiosn vs. ?\\
e -- foquet vs. warde
\end{comment}

\subsection{Methodology}		%3
This paper uses largely descriptive statistics of the three data series to describe the EIR. Much of the discussion of results depends on the graphs. I do provide analytic statistics including correlations, sample tests, structural break analyses, bi-variate Granger causality tests (\cite{granger_investigating_1969}), and a scatterplot of energy consumption and gross domestic product. 

I also discuss the results in the context of microeconomic and macroeconomic theory, in a way consistent with the observed data.

I do not estimate a formal empirical model, such as a regression, as that seems redundant after examining the scatterplot. The correlation between energy consumption and gross domestic products is strikingly, and visibly, strong.

%In an appendix, I provide substantial time series analyses as a foundation for formal modelling when this work is extended to examining how

In a future version, I will provide substantial time series analyses as a foundation for formal modelling when this work is extended to examining how important certain historical events are in explaining the outcome. I believe that will be the best use of formal modelling; the approach in this paper is sufficient to support my stated hypothesis.

Anticipating the, potentially many, issues my claims will raise, I enumerate my known ones in a list summarized in table \ref{tbl:issues}. hhhhhhhhhhhhhhhhhhhhhhhhhhhhhhhhhhhhhhhhhhhhh goal, and hope, is that comments will either add issues to or remove them from the list. Further, I encourage comments on approaches to resolving these important historical issues.

\section{Results}		%6

\subsection{Modern economic growth}
Simon Kuznets defined modern economic growth as high rates of growth of per--capita product and population (\cite{kuznets_modern_1966}). Figures \ref{fig:ggdp} and \ref{fig:gdpLog} indicate that England experienced high rates of growth of per-capita product in (possibly) two eras, from 1500 to 1600 that was not sustained, and after 1750 that was mostly sustained. Clearly after about 1820 England had a high and sustained rate of growth in per-capita product here measured as gross domestic product. The annual rate after 1800 was 2.4 percent per-year total growth and 1.1 percent per-capita growth as seen in table \ref{tbl:growthByCentury}. Figure \ref{fig:popLog} shows the log of population growth which, supporting the Kuznets definition, mirrors GDP growth with a lag.

\begin{center}
Figures \ref{fig:ggdp} and \ref{fig:gdpLog} about here\\
Figure \ref{fig:popLog} about here\\
Table \ref{tbl:growthByCentury} about here
\end{center}



\subsection{An energy revolution}
This paper's central assertion is that the EIR was, primarily, an energy revolution. More generally, this was a consumer goods consumption revolution enabled by an energy supply revolution. To support that hypothesis, first I present the data:

Figure \ref{fig:energyLog} displays the log of energy consumption over the study period; the vertical lines are formally determined structural breaks.\footnote{The structural breaks use an F-test methodology on the time series as implemented in the $R$ package struccchange, \cite{zeileis_strucchange:_2002}} The log presentation enhances rate-of-change and potential structural differences in the series. I note four significantly different periods or regimes. The first is from 1300 to 1500, a period dominated by the Black Death epidemic; energy consumption clearly drops, then recovers. The second is from 1500 to roughly 1600 as determined by the structural break. The third is the period from 1600 to roughly 1750; note that the rate-of-change of energy growth in this period is approximately the same as in the prior period; this rate of change similarity is confirmed by the presentation in table \ref{tbl:growthByCentury}. The final period is from 1750 through 1873; clearly the energy consumption rate-of-change accelerates as confirmed by the structural breaks in figure \ref{fig:energyLog} and table \ref{tbl:growthByCentury}.

Based on the structural changes, and based on the hypothesis that the EIR was an energy revolution, I propose that the revolution happened as two main eras: one starting in the mid-to-late sixteenth century,\footnote{This validates John U. Nef's hypothesis of an early start to the British Industrial Revolution \cite{nef_rise_1932}}, and one starting after 1750. The first, under this hypothesis, would have set the stage for the second. The second could not have been possible without the first.

\begin{center}
Figure \ref{fig:energyLog} about here
\end{center}

If we were to overlay the energy levels or logs charts with the GDP levels or logs charts the similarities would be striking; I think a more productive view is figure \ref{fig:energyVsGdp}. This figure shows levels of energy consumption through the study period, and has a standardized series of GDP for the same period. By standardized I mean matched in levels at the first period; the series' evolutions thus show differences in growth rates through continuous time. Again we see four distinct regimes. The most notable features are the period of 1500 to 1600 when growth in GDP clearly leads energy growth, and after 1750 (especially after 1800), when energy growth leads GDP growth.

\begin{center}
Figure \ref{fig:energyVsGdp} about here
\end{center}

The dynamics of GDP growth and energy consumption growth can be seen more clearly by taking the differences of the last graph.

\begin{center}
Figure \ref{fig:energyVsGdpDiff} about here
\end{center}

The Black Death and its aftermath affected the relatively flat net economic performance from 1300 to 1500, but set the stage for a growth boom in the period 1500 to 1600. In the period 1600 to 1750 growth in both relatively flattened, and then boomed again during the period 1750 to 1873.

\subsection{Correlations}

Next, I present some simple analytic statistics to support the hypothesis that the EIR was at its root an energy revolution responding to a positive demand shock.

Starting simply, a Pearson's correlation coefficient and a paired t-test of energy consumption and GDP yields the results in table \ref{tbl:fitTest}: 

\begin{center}
Table \ref{tbl:fitTest} about here
\end{center}

%These simple results suggest that the two series are statistically very similar; a more formal co-integration test could be expected to be positive, and is presented as Appendix B in section \ref{app:Appendix B}. However, for the purposes of this paper, a scatterplot of the series 
These simple results suggest that the two series are statistically very similar; a more formal co-integration test could be expected to be positive, and will be  presented in a future version. However, for the purposes of this paper, a scatterplot of the series 
is shown in figure \ref{fig:scatterplot}. The solid green line is a linear fit; the solid red line is a \textit{lowess} (non-parametric, non-linear) fit.

\begin{center}
Figure \ref{fig:scatterplot} about here
\end{center}

Clearly, there is a very high correlation between the two series. For current purposes, more formal modelling is not needed. Overall statistically, these two series are very close to being the same, that is they share a common data generating process. In a strong sense this is a validation of the thermodynamic view of economic production and growth at least in the long run.

From an economics point of view, this graph suggests a Leontief, fixed-factors production function, which could also be consistent with a Sraffian production interpretation.

However, this overall view does hide important dynamics that the data contain. I examine these more subtle results next, and thereby set the stage for telling a history of the EIR.


\subsection{Causality tests}
I continue by using basic statistical causality tests, specifically the Granger bi-variate test to examine changing dynamics (\cite{granger_investigating_1969}). Table \ref{tbl:grangerEnergyGdp} reports this result for the four main eras already identified.

\begin{center}
Table \ref{tbl:grangerEnergyGdp} about here
\end{center}

During the first energy/GDP era Granger causality between energy and GDP runs both ways at significant levels; while not ignoring these results, I do not want to over-interpret what was happening given the huge shocks of the Black Death. However, it is significant for later eras that the Black Death caused wages to rise, and the European Marriage Pattern (EMP)(\cite{hajnal_european_1965}) increased family incomes entering the early modern period.

During the second energy/GDP era of 1500 to 1600 causality from GDP growth to energy consumption is weakly significant; energy Granger-causing GDP growth is not at all significant. However there is narrative evidence that this was an important proto-industrial period in which home manufacture for markets became important; this is the ``Industrious Revolution'' of Jan de Vries (\citeyear{de_vries_industrial_1994}). Further, there is evidence that the English state supported an early version of Import Substitution Industrialization to replace imports, and to export (\cite{thirsk_economic_1978}). These events support the idea that demand must have been growing, both in domestic consumption markets and for military goods from the government, and eventually for exports.

These events occurred in a backdrop of global population growth during a century of benign agricultural climate; croplands expanded, food was plentiful, real wages likely grew, nuptiality and fertility increased, and England participated in this bounty.

In the third energy/GDP era of 1600 to 1750, neither direction of causality is significant. This will turn out to have important implications as I build the history for the EIR.

In the fourth energy/GDP era of 1750 to 1873, we again see both directions of causality significant, with GDP Granger-causing energy consumption being the stronger.

Notably, over the entire study period GDP Granger-causes energy consumption more significantly than energy Granger-causes GDP, but causality is significant in both directions.

\subsection{Structural breaks}

Figure \ref{fig:structural} juxtaposes frames with logs of energy consumption, gross domestic product, and population, each with formal structural break lines noted. The point here is to note the correspondence of the structural breaks, again suggesting the same underlying data generating process, but with causality-implying lags.

\section{Discussion of results}

I can now present a story of the EIR as supported by the data presented above. 

\subsection{Narrative discussion}

Energy/GDP era one, due to the Black Death, saw both negative demand and supply shocks, but set the stage for the following EIR eras through long-term effects on wages, incomes, and effective aggregate demand. More broadly, the five centuries prior to era one comprise the Medieval Warming Epoch (or Period) supporting higher agricultural output and population levels, both supporting effective aggregate demand through expanded incomes. See figure \ref{fig:temps}.

\begin{center}
Figure \ref{fig:temps} about here
\end{center}


In energy/GDP era two, wages rose due to the negative labor supply shock of era one. Demand had positive shocks, as a result both of wages and  of incomes rising due to later marriages and women working -- the EMP outcomes -- and favorable agricultural conditions.  Expanded household production (\cite{de_vries_industrial_1994}) and explicit import substitution policies starting with Henry VIII, and continuing through Edward VI and Elizabeth I, supported increased aggregate demand. (\cite{thirsk_economic_1978}) See table \ref{tbl:monarchs} for reigns. Supply expanded as can be seen by the stronger growth of energy consumption. Refer to table \ref{tbl:growthByCentury} or figure \ref{fig:energyLog}. This era provided the positive demand shocks and supply constraints that caused the EIR. It started here.

John Nef amplifies this view. He tells the story of era two as the ``age of timber.'' The time frames are a bit different, he says ``\ldots no less appropriate to call the sixteenth and seventeenth centuries an age of timber'' (\citeyear{nef_rise_1932}, p.191). Nef tells a very rich story of rising use of timber for industrial and home heating use, and for construction, and the beginnings of a timber crisis. My dates for era two are 1500--1600, which Nef's dates overlap by going into my era three.

In energy/GDP era three, rates of growth for both GDP and energy consumptions stagnated. This still puzzles scholars including Braudel and Hobsbawm, but there are several potential stories that I will sketch out here. Returning to figure \ref{fig:temps}, notice that a decline in mean temperatures occurred in the early modern era. This era is called the Little Ice Age, and is believed to have been a global phenomenon. This would have opposite effects from the Medieval Warming Epoch, that is reduce agricultural output and population levels, and a negative aggregate demand shock due to reduced income levels. In a sense, this is also a negative energy supply shock, featuring reduced growing space and time due to less effective insolation.  

Scholarly discussion of both the Medieval Warming Epoch and the Little Ice Age seems concentrated among paleoclimatologists; yet they often refer to the effects on the economy, sometimes citing contemporaneous accounts. Jean Grove provides a survey in "The Little Ice Age " (\cite{grove_little_2003}). Hubert Lamb is often cited as an early researcher.\footnote{See, for example, \cite{lamb_aspects_1980}} Lamb describes failed grain harvests in Scotland, and the disappearance of the cod schools in the Atlantic. These examples are typical, though not the focus, in the climatology literature. They do provide a plausible economic explanation for the stagnation in GDP, and the lagged stagnation in population growth.

A related story that fits the data, and the history, is that this era was one of a negative energy supply shock due to deforestation, and growth in the whole economic system thus slowed. This era was the transition between primarily wood-supplied energy to primarily coal-supplied energy for industrial and home heat needs. As London grew because of internal growth, exports, and world trade domination, wood became scarcer and more expensive, driving demand for coal for heating from the north east. You can see this pattern during the 1600 to 1750 era three in the following figure \ref{fig:woodCoal}.

\begin{center}
Figure \ref{fig:woodCoal} about here
\end{center}

Notably, this is also the era Nef calls the ``first energy crisis'' (\cite{nef_early_1977}). During the period 1550 to 1700, according to Nef, increased heating and building demand for wood, and reduced woodlands due to agricultural demands, caused wood prices to rise dramatically.

We can hypothesize that this series of events provided the economic pressure to cause the first phase of the energy revolution -- the transition from wood to coal for heating needs.

A further potential explanation appeals to political events, mainly the large number of wars during the period. By and large the contemporary anecdotes were that war was economically stimulative (\cite{thirsk_economic_1978}). 

In the editing process for this paper, I reviewed further work of Jan de Vries, who reportedly denigrated any climatic explanation; in "The economy of Europe in an age of crisis, 1600-1750" de Vries indeed says the climate evidence is not consistent with population evidence; my work shows population lags GDP, which was plausibly affected by climate change, suggesting a more consistent data set. Separately, I note that my energy/GDP era three has the same year boundaries as de Vries (\citeyear{de_vries_economy_1976}). De Vries also has an extensive empirical look at Dutch temperatures and various measures of agricultural output. In the end he comes to few conclusions except that time-series data are essential, a conclusion I share (\cite{de_vries_measuring_1980}).

This demand for heating coal arising from the first energy crisis and the fortuitous geology of the English coal mines created the path necessary to support energy/GDP era four, in which the second phase of EIR accelerated into modern economic growth via the virtuous, mutually reinforcing, growth cycle between GDP and energy consumption. 

The geology story is that the coal mines were water-infused, and as they were dug deeper, more water had to be pumped out. This provided an economically feasible site for the seminal but very inefficient Newcomen steam engines to pump the water. The coal was essentially free to power the engines. Human or horse power were too expensive. And as the steam engines gained efficiency, they began to be applied to the products of industrial capitalism. That is the story of energy/GDP era four, the age of steam. I turn next to telling that story in more detail; again it is an economic story, supported by the data.


\subsection{Theory discussion}

We have already examined the GDP and energy consumption data for the fourth era. To finish the story, I will retreat to economic theory. First, I summarize the eras in aggregate supply/aggregate demand charts; second, I address the question of what caused the English inventor/entrepreneur to spend the time and money to make the inventions of the first and second phases of the EIR, particularly the steam engine. To do this, I appeal to standard microeconomic theory.

Figure \ref{fig:asad} displays the four eras in an aggregate demand/aggregate supply (AD/AS) framework. The dotted lines indicate prior locations of AD/AS; solid lines indicate the ending locations. Lines colored red indicate the constraint in each era. These are obviously abstract depictions of the history I have told above. I do this for two reasons; first to solidify and emphasize the history so that the debate can proceed; second to provide a framework for later projects incorporating the institutional and cultural events into the history. If we can agree on the AD/AS by era, then we can hypothesize about those events that might have caused the location or shape to change and then test those ideas in an econometric framework.


\begin{center}
Figure \ref{fig:asad} about here
\end{center}

A notable observation is that energy/GDP era four is the first in which supply was not the constraint; according to the Granger causality tests, supply and demand were jointly constraining in that era. Statistically, only GDP Granger-causing energy consumption is significant at normal levels, but the lack of relative barriers in consuming energy was surely the uniquely defining event of the era.

Second for the theoretical discussion of the EIR, it is important to consider at the microeconomic level what can explain the event. At this level I will discuss only the supply side having already suggested a story of the important demand-side factors. So the question becomes what were the incentives or motivations of the English inventors and entrepreneurs during energy/GDP eras two and three, so from 1500 through 1750.

For this analysis I rely on three sources; first the contemporaneous comments of a key participant in the EIR; second the excellent work of Robert Allen; and third an appeal to microeconomic theory.

Jean (or John) Theophilus Desaguliers had a large influence on the EIR. He was an eighteenth century English ``natural philosopher (physicist), member of the Royal Society, colleague of Sir Isaac Newton, and author of ``A Course of Experimental Philosophy.'' This was an influential 1734 two-volume engineering text that contained a chapter on ``Fire-Engines'' (steam engines). In this chapter, Jean Theophilus describes the economic and scalability motives of replacing men and horses with coal-fired steam engines to pump water out of Newcastle mines.  Profit was on his mind.  The age of the industrial capitalism, fueled by fossil energy, was dawning (\cite{desaguliers_course_1734}, vols. II, 467-468).

Figure \ref{fig:desagulier} shows a page of his manuscript.

\begin{center}
Figure \ref{fig:desagulier} about here
\end{center}

Beyond the quaintness of the 1734 English prose, this man demonstrated the soul of a profit-maximizing capitalist. In that context, let us examine some data that drove Desaguliers.

Figure \ref{fig:wage-energy} is from Robert Allen and shows the ratios of real wages to energy costs (the cheapest source) by benchmark city around 1700.

\begin{center}
Figure \ref{fig:wage-energy} about here\footnote{\cite{allen_british_2009}}
\end{center}

Clearly, Newcastle in 1700 had high wages and very low energy costs, by far the largest ratio in the sample. Those were the economic fundamentals that faced Desaguliers and motivated his profit comment. London had the second largest ratio, and thus the economic incentives existed there as well. Beijing had the lowest ratio, a topic I investigate in another research project.

While the economics of these ratios may be intuitive, why not appeal to microeconomic theory to help us understand what motivated Desaguliers, Newcomen, Watt and all the other founding fathers of the EIR. Equation \ref{eq:mrp} is a variation on production theory that will be familiar to those who remember their Econ 101. A major topic of mainstream production theory is how entrepreneurs maximize profits given the derived demand curves of the various input choices. 

This equation is a variation on that theme:\footnote{We can proceed either with a neo-classical factor substitution argument, or a more general classical view of normal prices of production. Either approach will react to the enormous productivity-enhancing energy supply shock that was the Industrial Revolution. A more challenging story to tell is one which identifies the sources of aggregate demand that supported expansion of English production. Here, I simply stipulate that aggregate demand existed.}


\begin{center}
Equation \ref{eq:mrp} about here
\end{center}

Instead of using different substitutable inputs such as labor and capital, I apply the theory to the different sources of energy, energy being essentially the only non-substitutable input as in you must have joules from whatever source to do any economic transformation. 

This equation is written as the profit-maximizing equilibrium that will substitute between different energy sources, say wood and coal for heating as wood becomes scarce; and, say, human-input joules replaced by coal-input joules as wages rise. Clearly, the equation needs additional terms to cover the amortization of whatever equipment is necessary to apply either kind of joule, but also clearly from just what is written we see that when wage-to-coal-energy cost ratios are sufficiently high, entrepreneur/inventors will be motivated to substitute coal joules for human joules. And that is what happened at the micro level to drive the EIR, first in Newcastle atop the mines, then in the English textile mills, then other English industries, later spreading to the world.


\section{Unresolved issues}
This essay is focused on exploring a data-driven economic explanation using energy inputs as primary in causing the Industrial Revolution. As space here is constrained, and given the claims the paper makes, there will be a (possibly very) large number of unresolved issues. I will initiate the list here as table \ref{tbl:issues}.

\begin{center}
Table \ref{tbl:issues} about here
\end{center}

In general E. A. Wrigley and John Nef have the most textured descriptions across the literature of what uniquely occurred in England which was a transition from, in Wrigley's terms, an advanced organic economy to an inorganic economy. So much of the reconciliation given my claims will be to their work.

\section{Conclusion} %second last

The English Industrial Revolution, whatever else it was, was an \textit{energy consumption} revolution. This stands out as its primary feature, a feature that caused, for the first time in history, modern economic growth through large productivity gains. Mankind learned to consume energy in the economic process at a rate that was essentially unbounded, such that there was no longer an energy supply constraint on output.

This happened in England because England had a set of critical conditions that were rare: high wages, high family incomes, sufficient knowledge to construct ``Fire-Engines'', and very low relative energy costs with essentially unlimited supply. Of these critical factors, only England uniquely had very low relative energy costs; this last factor then must be deemed the necessary condition for the EIR.

The EIR happened in distinct eras, each of which can be defined as a specific energy/GDP (aggregate supply -- aggregate demand) regime which frames further research on what exogenous institutions and other factors may have been important. We can use data and simple macroeconomic principles to usefully investigate the EIR. England collectively ``learned'' how to create a positive virtuous macroeconomic growth feedback cycle driven by fossil energy consumption.

Also, there were two distinct energy transitions which are intertwined in a path-dependent story: the first substituting coal for wood in domestic and industrial heating uses; and the second substituting coal energy for labour energy in industrial mechanical uses enabled initially by the steam engine.

Further, the individual behavior of the EIR inventors and entrepreneurs can be explained using simple microeconomic principles. 

Given all this, extant hypotheses of English cultural and/or institutional exceptionalism seem redundant to the outcome. England was a very lucky country, geographically advantaged, at the right place and time for this miracle to occur.

\begin{comment}
\section{journal strategy}
Possible journals:\\
Explorations in Economic History, Economic History Review (European), Journal of Economic History, Cliometrica.\\
Non-history:\\
development journals that accept history?\\
energy journals that accept history?\\
Science? Nature?\\
Reynolds suggests going for AER!\\
Journal of Applied Econometrics (Dave Giles)\\
Journal of International Trade and Economic Development (Dave Giles)\\
\end{comment}

\section{References}
		\bibliographystyle{agsm}	
	\bibliography{diss2}

\listoftables

\listoffigures

\listofmyequations

\newpage

\section{Tables}

\begin{table}[p!]
\caption{Taxonomy of EIR explanations}
\label{tbl:taxonomy}
\begin{tabular}{rl}
Label&Examples\\
\hline
English exceptionalists&Landes (1969), McCloskey (2010), Mokyr (1992,2010)\\
Partial culturalists&Cipolla (1966), Pomeranz (2001), Allen (2009)\\
Primarily energetic&Cottrell (1955), Wrigley (1988,2010), Malanima (2010), Nef (1932)\\
Thermodynamicists&Georgescu-Roegen (1975), Ayres (2003), Garrett (2009)\\
\end{tabular}
\end{table}


\begin{table}[p!]
\caption{Data Sources}
\label{tbl:dataSources}
\begin{tabular}{lrll}
Data series&Year range&Geography&Source\\
\hline
Energy consumption&1300 -- 1873&England/Wales&Roger Fouquet (2008)\\
\hline
Gross domestic product&1300 -- 1700&England&Graeme Snooks (1994)\\
&1741 -- 1873&England/Wales&Lawrence Officer (2009)\\
\hline
Population&1300 -- 1540&England&Graeme Snooks (1994)\\
&1541 -- 1800&England&B. R. Mitchell (1988)\\
&1801 -- 1873&England/Wales&B. R. Mitchell (1988)\\
\end{tabular}
\end{table}

\begin{comment}
\begin{table}[p!]
\caption{t-test of energy and gdp}
\label{tbl:t-testEnergyGdp}
\begin{tabular}{rl}
\end{tabular}
\end{table}
\end{comment}

\begin{table}[p!]
\caption{Wood and total price indices. \textit{Source:} Nef (1932, p.158,221)}
\label{tbl:woodPrice}
\center
\begin{tabular}{lrr}
Period&General price index&Wood price index\\
\hline
1451-1500&100&100\\
1531-1540&105&94\\
1551-1560&132&163\\
1583-1592&198&277\\
1603-1612&251&366\\
1613-1622&257&457\\
1623-1632&282&677\\
1633-1642&291&780\\
1643-1652&331&490\\
1653-1662&308&662\\
1663-1672&324&577\\
1673-1682&348&679\\
1683-1692&319&683\\
1693-1702&339&683\\
\end{tabular}
\end{table}


\begin{table}[p!]
\caption{growth rates by century}
\label{tbl:growthByCentury}
\begin{tabular}{lrrrrrrrr}
Year	&	1300	&	1400	&	1500	&	1600	&	1700	&	1801	&	1873&Total	\\
\hline
GDP Million\\ 2005 GBP	&	3114.7541	&	815.1288	&	994.4571	&	6031.953	&	8361.5911	&	18110	&	102811&	\\
Century-over-century\\rate of growth&&-0.738&0.220&5.066&0.386&1.166&4.677&32.008\\
Compounded annual \\rate of growth&&-0.013&0.002&0.018&0.003&0.008&0.024&0.006\\
\hline
Energy consumption&1.7	&	1	&	1.3	&	2.2	&	3.6	&	11.6	&	66.1&	\\
Century-over-century\\rate of growth&&-0.412&0.300&0.692&0.636&2.222&4.698&37.882\\
Compounded annual \\rate of growth&&-0.005&0.0026&0.005&0.005&0.012&0.024&0.006\\
\hline
Per-capita GDP\\2005 GBP&542&  329&  421& 1,484& 1,663& 1,999& 4,392\\
Century-over-century\\rate of growth&&-0.393& 0.282&2.521&0.121&0.202&1.198& 7.108\\
Compounded annual \\rate of growth&&-0.005&0.002&0.013&0.001&0.002& 0.011&0.004\\
\end{tabular}
\end{table}

\begin{table}[p!]
\caption{Energy and GDP fit tests}
\label{tbl:fitTest}
\begin{center}
\begin{tabular}{lrr}
\hline\hline
Test&Statistic&p-value\tabularnewline
\multicolumn{1}{c}{}\tabularnewline
\hline
Pearson's correlation&$0.998$&\tabularnewline
\hline
Paired t-test&$5.592$&4.991e-07\tabularnewline
\hline
Chi-square&2864&0.0004998\tabularnewline
\end{tabular}
\end{center}
\end{table}

\begin{table}[p!]
\caption{granger tests of energy/gdp}
\label{tbl:grangerEnergyGdp}
\begin{tabular}{lrrl}
Era&Energy $\sim$ GDP Pr($>$F)& GDP $\sim$ Energy Pr($>$F)&AD/AS regime\\
\hline
1300 -- 1500&0.0106&0.0003&EMP, Black Death, \\&&&wages/family income increasing\\
1500 -- 1600&0.1939&0.6126&Positive demand shock\\
1600 -- 1750&0.3529&0.5185&Energy supply constraint\\
1750 -- 1873&0.0024&0.1100&Positive supply shock,\\&&&``virtuous'' macro feedback cycle\\
\hline
1300 -- 1873& 0.0002& 0.0361&Total study period\\
\end{tabular}
\end{table}

\begin{table}[p!]
\caption{Unresolved issues}
\label{tbl:issues}
\center
\begin{tabular}{ll}
Issue&Comment\\
\hline
Reconcile to 1970 English energy study&Humphrey and Stanislaw\\
Reconcile to Broadberry et al. on agricultural revolution, growth&Broadberry et al. vs. PBH\\
Reconcile to English agricultural revolution&Wrigley is the starting place\\
Effects of ``Columbian Exchange'' &Crosby\\
(e.g. potatoes) on English agriculture&\\
Reconcile to English urbanization&Again, Wrigley\\
Paradox of short nineteenth century heights&mentioned by deLong\\
Reconcile to narrow industry energy scope&McCloskey\\
Address general purpose technology story&Bresnehan\\
Reconcile to Nef data&John U. Nef\\
Reconcile to ``little ice age''&de Vries\\
Reconcile to Marc Braudel&Marc Braudel\\
\hline
\end{tabular}
\end{table}

\begin{table}[p!]
\caption{Early modern English monarchs}
\label{tbl:monarchs}
\center
\begin{tabular}{lll}
Monarch&Reign&House\\
\hline
Henry VIII&1509-1547&Tudor\\
Edward VI&1547-1553&Tudor\\
Mary I&1553-1558&Tudor\\
Elizabeth I&1558-1603&Tudor\\
James I&1603-1625&Stuart\\
Charles I&1625-1649&Stuart\\
Oliver Cromwell&1653-1658&Commonwealth\\
Richard Cromwell&1658-1659&Commonwealth\\
Charles II&1660-1685&Stuart\\
James II&1685-1688&Stuart\\
Mary II&1689-1694&Stuart\\
William III&1689-1702&Stuart\\
Anne&1702-1707&Stuart\\
\hline
\end{tabular}
\end{table}


\newpage

\section{Figures}

\begin{figure}[p!]
\center
\caption{Author/time-span series of energy consumption, GDP, and population}
\label{fig:overall levels}
\includegraphics[width=0.9\textwidth]{overallLevels}
\end{figure}

		\begin{figure}[p!]
		\caption{English real gross domestic product, \\
		levels and per--capita }
		\label{fig:ggdp}		
		\centerline{
		\mbox{\includegraphics[width=0.55\textwidth]{ggdp}}
		\mbox{\includegraphics[width=0.55\textwidth]{ggdppop}}
		}
		\end{figure}

		\begin{figure}[p!]
		\caption{English real gross domestic product, \\
		log levels and log per--capita}
		\label{fig:gdpLog}		
		\centerline{
		\mbox{\includegraphics[width=0.55\textwidth]{gdpLog}}
		\mbox{\includegraphics[width=0.55\textwidth]{gdpPopLog}}
		}
		\end{figure}

\begin{figure}[p!]
\center
\caption{Log of population, with structural breaks}
\label{fig:popLog}
\includegraphics[width=0.9\textwidth]{popLog}
\end{figure}

\begin{figure}[p!]
\center
\caption{Log of energy consumption, with structural breaks}
\label{fig:energyLog}
\includegraphics[width=0.9\textwidth]{energyLog1.png}
\end{figure}

\begin{figure}[p!]
\center
\caption{Energy consumption vs. standarized GDP}
\label{fig:energyVsGdp}
\includegraphics[width=0.9\textwidth]{energyVsGdp}
\end{figure}

\begin{figure}[p!]
\center
\caption{Energy consumption vs. standardized GDP, differences}
\label{fig:energyVsGdpDiff}
\includegraphics[width=0.9\textwidth]{energyVsGdpDiff}
\end{figure}

\begin{figure}[p!]
\center
\caption{Scatterplot of energy consumption vs. GDP}
\label{fig:scatterplot}
\includegraphics[width=0.9\textwidth]{scatterplot.png}
\end{figure}

\begin{figure}[p!]
		\caption{Structural break comparison}
		\label{fig:structural}		
		\centerline{
		\mbox{\includegraphics[width=0.33\textwidth]{energyLog1}}
		\mbox{\includegraphics[width=0.33\textwidth]{gbpgdplog}}
		\mbox{\includegraphics[width=0.33\textwidth]{popLog}}
		}
\end{figure}

\begin{figure}[H]
\center
\caption{Late Holocene temperatures. \textit{source:} NASA and IPCC composite}
\label{fig:temps}
\includegraphics[width=0.9\textwidth]{2000_Year_Temperature_Comparison.png}
\end{figure}

\begin{figure}[h!]
\center
\caption{Coal and wood energy sources\\\textit{Source:} Pearson \& Fouquet}
\label{fig:woodCoal}
\includegraphics[width=0.9\textwidth]{woodCoal.png}
\end{figure}

\begin{figure}[h!]
		\caption{Aggregate Supply---Aggregate Demand \\ Four energy/GDP regimes}
		\label{fig:asad}		
		\centerline{
		\mbox{\includegraphics[width=0.25\textwidth]{era1}}
		\mbox{\includegraphics[width=0.25\textwidth]{era2}}
		\mbox{\includegraphics[width=0.25\textwidth]{era3}}
		\mbox{\includegraphics[width=0.25\textwidth]{era4}}				
		}
\end{figure}

\begin{figure}[h!]
		\caption{Desaguliers manuscript}
		\label{fig:desagulier}		
		\center
%		\mbox{\includegraphics[width=0.95\textwidth]{desagulier1}}\\
%		\mbox{\includegraphics[width=0.95\textwidth]{desagulier2}}
		\includegraphics[width=0.95\textwidth]{desagulier1}\\
		\includegraphics[width=1.05\textwidth]{desagulier2}
%		}
\end{figure}

\begin{figure}[h!]
\center
\caption{Real wage to energy ratios\\\textit{Source:} Robert Allen (2009)}
\label{fig:wage-energy}
\includegraphics[width=0.9\textwidth]{wage-energy.png}
\end{figure}

\begin{figure}[h!]
\center
\caption{Standardized English energy intensity of GDP}
\label{fig:energyIntensity}
\includegraphics[width=0.9\textwidth]{energyIntensity}
\end{figure}

\begin{figure}[h!]
\center
\caption{Log of GDP, with structural breaks}
\label{fig:gbpgdplog.png}
\includegraphics[width=0.9\textwidth]{gbpgdplog.png}
\end{figure}

\newpage

\section{Equations}

		\begin{equation}
		\label{eq:mrp}
%		\frac{\text{Marginal Revenue Product}_{\text{ organic energy joule}}}{\text{Price}_{\text{ organic energy joule}}} = \frac{\text{Marginal Revenue Product}_{\text{ fossil energy joule}}}{\text{Price}_{\text{ fossil energy joule}}}
		\frac{\text{Marginal Product}_{\text{ organic energy joule}}}{\text{Price}_{\text{ organic energy joule}}} = \frac{\text{Marginal Product}_{\text{ fossil energy joule}}}{\text{Price}_{\text{ fossil energy joule}}}
		\end{equation}
		\myequations{Microeconomic theory - marginal product} 
		
%granger tests over at least two regimes
%add back in for full paper
%\section{Appendix A. Detailed Granger test output} 
%\label{app:Appendix A}

%\section{Appendix B. Time series analyses}
%\label{app:Appendix B}

\begin{comment}
\section{Appendix C. Future research}
survey on institutions/culture\\
empirical tests of institutional/cultural events\\
population curves\\
\label{app:Appendix C}
\end{comment}


\begin{comment}

\begin{verbatim}
> grangertest(per1eng ~ per1gdp,n)
Granger causality test

Model 1: per1eng ~ Lags(per1eng, 1:1) + Lags(per1gdp, 1:1)
Model 2: per1eng ~ Lags(per1eng, 1:1)
  Res.Df Df      F  Pr(>F)  
1     16                    
2     17 -1 8.3703 0.01059 *
---
Signif. codes:  0 �***' 0.001 �**' 0.01 �*' 0.05 �.' 0.1 � ' 1 
 grangertest(per1gdp ~ per1eng,n)
Granger causality test

Model 1: per1gdp ~ Lags(per1gdp, 1:1) + Lags(per1eng, 1:1)
Model 2: per1gdp ~ Lags(per1gdp, 1:1)
  Res.Df Df      F    Pr(>F)    
1     16                        
2     17 -1 21.772 0.0002582 ***
---
Signif. codes:  0 '***' 0.001 '**' 0.01 '*' 0.05 '.' 0.1 ' ' 1 
> grangertest(per2eng ~ per2gdp,n)
Granger causality test

Model 1: per2eng ~ Lags(per2eng, 1:1) + Lags(per2gdp, 1:1)
Model 2: per2eng ~ Lags(per2eng, 1:1)
  Res.Df Df      F Pr(>F)
1      7                 
2      8 -1 2.0643 0.1939
> grangertest(per2gdp ~ per2eng,n)
Granger causality test

Model 1: per2gdp ~ Lags(per2gdp, 1:1) + Lags(per2eng, 1:1)
Model 2: per2gdp ~ Lags(per2gdp, 1:1)
  Res.Df Df      F Pr(>F)
1      7                 
2      8 -1 0.2808 0.6126
> grangertest(per3eng ~ per3gdp,n)
Granger causality test

Model 1: per3eng ~ Lags(per3eng, 1:1) + Lags(per3gdp, 1:1)
Model 2: per3eng ~ Lags(per3eng, 1:1)
  Res.Df Df      F Pr(>F)
1      6                 
2      7 -1 1.0136 0.3529
> grangertest(per3gdp ~ per3eng,n)
Granger causality test

Model 1: per3gdp ~ Lags(per3gdp, 1:1) + Lags(per3eng, 1:1)
Model 2: per3gdp ~ Lags(per3gdp, 1:1)
  Res.Df Df      F Pr(>F)
1      6                 
2      7 -1 0.4703 0.5185
> grangertest(per4eng ~ per4gdp,n)
Granger causality test

Model 1: per4eng ~ Lags(per4eng, 1:1) + Lags(per4gdp, 1:1)
Model 2: per4eng ~ Lags(per4eng, 1:1)
  Res.Df Df      F   Pr(>F)   
1     22                      
2     23 -1 11.735 0.002418 **
---
Signif. codes:  0 '***' 0.001 '**' 0.01 '*' 0.05 '.' 0.1 ' ' 1 
> grangertest(per4gdp ~ per4eng,n)
Granger causality test

Model 1: per4gdp ~ Lags(per4gdp, 1:1) + Lags(per4eng, 1:1)
Model 2: per4gdp ~ Lags(per4gdp, 1:1)
  Res.Df Df      F Pr(>F)
1     22                 
2     23 -1 2.7737   0.11

\end{verbatim}
\end{comment}

\clearpage
\hbox{}\newpage


\end{document}
\section{end}

\section{Defence proposal below here}
\newpage
			
	\section{Table of Contents}
	\tableofcontents
	\listoffigures
	\listoftables
	
	\newpage
	

	\section{The Research Question}
	
	The origins and causes of the English Industrial Revolution remain among economic history's most contested puzzles.  This slowly evolving revolution was likely the most important event in economic history since the \gls{neorev} approximately 10,000 years ago, and eclipses even that unquestionably cataclysmic transition in economic importance as measured by growth in per capita income.  Table \ref{tbl:gdpcapita} shows the growth in world per capita real income between CE 1 and 1900. \footnote{Data from Maddison 2007 \cite{maddison_world_2007}} Yet, even this primary outcome of the Industrial Revolution remains contested.
	
	In the first 17 centuries of the Current Era per capita output increased by about 32 percent, surely still hewing closely to subsistence.  By stark contrast, in the two hundred years between 1700 and 1900 world average per capita output increased by  over 100 percent.  If one considers the very narrow population base to which the increased income actually inured (about 15 percent of the 1900 world population, measured by nation-state, produced about 40 percent of world GDP), this is a remarkable outcome. The richest country in 1900, the United Kingdom, enjoyed per capita income of about 4,500 \gls{gkusd}, over three times the world average.
	
		\begin{table}[h!]
		\centering

%		\begin{adjustwidth}{-0.75in}{}

		\caption[World GDP per capita]{This table shows the growth of world GDP per capita in two periods covering almost two millennia. Maddsion's data, author's calculations. Dollars in 1900 Geary-Khamis International (US) Dollars}\label{tbl:gdpcapita}
		
		\begin{tabular}{lrr}
		\toprule
		Benchmark&GDP per capita,&Percent Increase\\
		Year&1990 G-K\$& from prior benchmark\\
		\midrule \midrule
		CE 1&467&-\\
		1700&615&31.7\%\\
		1900&1,261&105.0\%\\
		\bottomrule

		\end{tabular}
%		\end{adjustwidth}
		\end{table}
	
	In some respects, the English consumers and inventors between the reigns of \gls{lizIreign} and \gls{vicreign} forged a Faustian bargain with the future to achieve this miracle.  They almost surely did so unknowingly by unleashing a historically unprecedented positive feedback cycle of demand and production that has now lasted for 400 years; this event thus has the characteristic of an emergent macroeconomic effect.  %History also inserts great economic inequalities, instabilities and conflicts, environmental damages, and other existential threats into this record of unprecedented growth in incomes, surplus, and, thus, wealth for wide swaths of humanity.
	
	What happened?  How did it happen?  Why did it happen first in England when it did and nowhere else in accumulated history?  Why has it not regressed, as every known preceding surge in per capita income has done?  
	
	How should our understanding of this Revolution inform our understanding of economic development in the nation-states and global economy that followed England -- that is, how did the Revolution evolve and spread?  What can we learn about future economic development and growth?
	
	And very specifically, what role did effectively unconstrained rates of \gls{energy} play in the Revolution? What social and institutional changes help account for this ``miracle invention" by which the English for the first time in history, and uniquely at the beginning, ``escaped the constraints of an advanced organic society?" \footnote{Wrigley 2010, p. 239 \cite{wrigley_energy_2010}} In turn, what social and institutional changes did this revolution cause in England and the other economies which followed in her path?
	
	I consider whether this discovery--\textit{the ability to convert virtually unlimited energy to output through the macro-invention of industrial-scale machine-capitalism} \footnote{Eckel \textit{Coal, Iron, War}, 1921  \cite{eckel_coal_1921}}---is the real, although in significant senses accidental, invention of the Revolution.  If so, we must consider this history seriously as we gaze into the physical and economic future of our species and planet.

		\subsection{What is the Study About?}
		
			\subsubsection{Nature and significance}
			This study explores the role of energy consumption in \gls{growth} and development.  The research question arises as one views long-period time-series of both per capita output and per capita energy consumption.  Both can be best described visually as super-exponential growth curves; thus, it is natural to ask how they are related.  The characteristic shape is seen in Figure \ref{fig:road}.\footnote{Economist Magazine, December 23, 1999  \cite{_road_1999}}
			
			\begin{figure}
			\centering
			\caption[Road to riches hockey stick]{From a 1999 Economist magazine article, based on the Angus Maddison data, illustrating the ``hockey stick'' effect of per capita GDP growth in Europe. Usefully annotated with historical events}\label{fig:road}
			\includegraphics[width=0.8\textwidth]{../images/roadtoriches.pdf}
			\end{figure}
		
			A word on the title: Sir Arthur A. Lewis in his iconic 1954 ``Economic Development with unlimited supplies of labour'' \footnote{Lewis, 1954.\cite{lewis_economic_1954}} develops as a central theme that, in underdeveloped economies, labour will be used even if its marginal product is zero due to its relative abundance.
			
			I can construct similar examples in modern energy use; suppose I left the computer on which I am composing this unattended but on for several hours before writing this paragraph. It is not clear what the marginal product of that unattended energy consumption is: The computer did no writing or editing while I was absent. If I were writing by quill pen and candlelight, I would likely not have left the candle burning unattended due to fire risks and the high per lumen cost of medieval candles. The abundance of modern energy changes the expected marginal cost/marginal benefit equation.
			
			Another way to think about modern energy consumption is that the ratio of mineral energy available for whatever task, be it commodity manufacturing, service provision of various kinds including transportation, or writing a dissertation, is virtually unconstrained relative to the amount of the very largest amount of human energy I could possibly marshal for any of those tasks.
			
			So the title contextualizes the way in which I view modern energy consumption, and emphasizes how uniquely important the great English invention was.
			
			\subsubsection{Contributions of the study}
			\begin{enumerate}
			\item I clarify the historical explanation of this Revolution.  The explanation has fragmented over several economic generations into several ``camps" from which I hope to abstract what is both necessary and sufficient to explain the magical event. The story is complex, enriched with feed-back mechanisms, and clearly not mono-causal in the choices of supporting historical events.
			
			One can view the range of prior explanations of the English Industrial Revolution as a continuum from purely cultural to purely thermodynamic, with religious, cultural, social, institutional, technological, and geographical explanations arrayed along the line. By far, the most common explanations are weighted toward some form of English cultural exceptionalism. 
			
			\label{par:deirdre} Perhaps the most eloquent of this class is the recent series of books by Deirde McCloskey currently being published under the banner of the ``Bourgeois Era.''  Professor McCloskey details (in a promised six volumes) the great cultural changes that surrounded the English Industrial Revolution.  The series describes processes which are the antithesis of Marxian historical materialism. While there are many compelling stories in this history, this paper counters her thesis with a conviction that the actual story is, in fact and on the evidence, primarily a materialist story which caused and was supported by the evolving cultural and institutional environments. \footnote{The series currently consists of two published volumes, ``The Bourgeois Virtues'' (2007), and ``Bourgeois Dignity'' (2010) \cite{mccloskey_bourgeois_2010}}
			
			
			\item I propose a new definition of the English Industrial Revolution: a historically unique development that enabled the English macroeconomy to consume a virtually unconstrained amount of energy, and thus to escape any practical supply side limitation on output.  For perhaps the first time in history demand, not supply, became the limit of output; output growth exceeded population growth for an extended period of time.  Unprecedented growth of both macroeconomic productivity and \gls{surplus} accumulation became the new normal.
			
			\item Further, I explore my intuitions with the most analytically useful analysis I am aware of for this kind of problem.  Specifically, I  use the methodology of the so-called \gls{copenhagen} to investigate and attempt to explain the relationships and dynamics among energy consumption rates, output growth, and population growth. The method allows admitting exogenous events, so that if, for example, a statistical structural break appears in the data starting near the time of Gutenberg's invention, one can both recognize that in the model in a control sense and hypothesize that it was an important event. While the historically constructed series I use can surely be problematic, in a sense testing them with rigorous statistics could possibly highlight weaknesses in the construction methods.
			
			The method is capable of decomposing feedback loops into both short-run and long-run components.  To my knowledge, this decomposition method has never been attempted over long historical-period data sets.
			
			\item I apply the empirical methodology to two relatively new long-period histories of English energy consumption. One is from Roger Fouquet, \footnote{Fouquet's new study on English energy \cite{fouquet_heat_2008}} and one from Paul Warde. \footnote{Energy consumption from Warde's 2007 work. \cite{warde_energy_2007}} My strategy is to empirically test both inputs and compare the results both economically and statistically.
			
			\item As a future research agenda, 
			
			\begin{enumerate}
			
			\item I will extend this methodology to examine other economies in phases of significant growth; clearly the United States and China should be included, but any economy with sufficient historical data is a potential target. \label{par:china}
			
			\item I will use the empirical methods to decompose global growth over the period with available meaningful data.  
			
			\item Finally, I may consider an energy-based theory of income, growth, and wealth, and contrast that with extant micro-economic marginal productivity based theories including "new growth" New-Keynesian theories. If energy consumption is proven foundational to economic growth, then a measure of \gls{surplus} as energy consumption above \gls{bio} \footnote{Brad DeLong 1998. DeLong discusses GDP and population estimates, and uses this evocative term. \cite{delong_estimates_1998}} may prove interesting and useful.
			\end{enumerate}
			\end{enumerate}

		\subsection{Statement of the Research Question and Main Hypothesis}
		The research question asks how English energy consumption and economic output are structurally related, considering population levels, over a long period preceding and spanning the English Industrial Revolution.
		
		My main hypothesis is that learning to consume unconstrained energy was responsible, at least in the statistically causal sense, for the massive increases in per capita output first demonstrated in the English Industrial Revolution. Refer to Figure \ref{fig:road} and Table \ref{tbl:gdpcapitauk} for historical English per capita GDP growth.\footnote{Data from Maddison 2007 \cite{maddison_world_2007}}
		
		\begin{table} \centering
		\caption[UK GDP per capita]{This table shows the growth of English GDP per capita in two periods covering almost two millennia. Maddsion's data, author's calculations. Dollars in 1900 Geary-Khamis International (US) Dollars}\label{tbl:gdpcapitauk}.
				
		\begin{tabular}{lrr}
		\toprule
		Benchmark&UK GDP per capita,&Percent Increase\\
		Year&1990 G-K\$& from prior benchmark\\
		\midrule \midrule
		CE 1&400&-\\
		1700&1,250&212.5\%\\
		1900&4,492&259.4\%\\
		\bottomrule

		\end{tabular}
%		\end{adjustwidth}
		\end{table}
		
		Thus, an understanding of that event is crucial to understanding \gls{growth}.


		\subsection{Subsidiary Hypotheses}
		\begin{enumerate}
		\item While the primary and clear discovery of the English Industrial Revolution was the ability to consume unlimited energy at the level of the macro economy, there are a series of necessary social and institutional precursors. 
		
		These are in dispute, but I focus on identifying those that contributed to the incentives and skills of consuming unlimited energy. My initial candidates are the rise of consumer demand including (though at a later time) the export orientation of the English economy, and the accelerated accumulation and spread of knowledge after Gutenberg's \footnote{Johannes Gutenberg, 1398 - 1468, widely credited with inventing a printing press using mechanical moveable type} printing invention.
		
		Following Cottrell \footnote{W. F. Cottrell 1955 \cite{cottrell_energy_1955}} (and Karl Marx), there should also be testable hypotheses on the energy revolution causing social and institutional changes. This is an area for future research.
		
		\item While England clearly led, the revolution spread fairly quickly. I hypothesize that the structural dynamics of the early followers were similar to those of England.  Via the framework I am developing, this should be a testable hypothesis in future research.
		

		
		\item I further hypothesize that the rate of energy consumption is a much better predictor of \gls{surplus} than any other factor including ``capital'' and ``technical progress,'' as they would enter a common specification of the Solow macro production function.
		\end{enumerate}


  		\subsection{Limitations and Delimitations}
		
		In this proposal, the focus is on the English experience; I do extend the model to a global economy by testing specifications which admit English exports as support for growth in overall demand. In the research for this proposal, country comparisons arise and I reference them. As I continue this work, I will extend the methods to other key economies as described on page \pageref{par:china}.

		\subsubsection{Limitations}
		The primary limitation in the study is the accuracy of the data series since they are historically constructed by various economic historians.  I will evaluate several sources for the data and attempt to make a logical choice of which series I will use.  As questions arise on the accuracy, my considered response is ``what choice do we have?"  We should let the data tell us whether the model(s) using the series are credible statistically. Thus, doing the econometric analysis in this way may be peripherally useful to the historians who have been compiling this data, as it will suggest the validity of the data.
		
		\subsubsection{Delimitations}
		The primary delimitation that I have imposed is to use only three time series as core variables in the proposed study; Gross Domestic Product (GDP), energy consumption, and population are treated as endogenous variables. I will allow historical events suggested by structural breaks in these endogenous series to enter the model.  Depending on model fits, I may consider adding other theoretically justifiable series, for example, energy prices. This parsimonious ``simple to complex'' approach is the preferred methodology of the \gls{copenhagen}.
		
		I prefer to use Bayesian inference rather than classical methods for it's superior statistical characteristics. However, the applied methodology I need to deploy is not yet up to the challenge, so I will limit this study to classical inference.

		
	\section{Literature Review}
	There are two main bodies of literature that bear on my topic.  The first deals with the historical causes of the English Industrial Revolution, and the subsequent histories of the other countries which bear on their path to modern economic growth.
	
	The second body of literature deals with the econometric analysis of energy and modern economic growth.
	
	This study will both deal with the intersection of these bodies of literature, and extend them in the following ways:
	\begin{enumerate}
	\item Use econometric methods to fully describe the structural dynamics, both long- and short-run, relating economic growth and energy consumption.
	\item Identify the most important historical-institutional changes that led to modern economic growth as first experienced in the English Industrial Revolution.
	\item Since the prime cause of the English Industrial Revolution remains under debate, I will attempt to add to the debate based on the evidence.
	\end{enumerate}
	
	\subsection{Contributions of the core literature}

	\paragraph{Primarily cultural explanations of the English Industrial Revolution -- the English cultural exceptionalists.} In line with my metaphor of a continuum of primary reasons to explain why the Industrial Revolution was English and happened in the $18^{th}$ century, I first describe some of the more prominent cultural or institutional primacy authors. Note that almost all historians who approach this problem end up with multi-causal explanations, so this categorization is based on my judgement. 
	
	Deirdre McCloskey, who I referenced earlier, provides an acutely observed and detailed history of the cultural, social, and institutional changes that contributed to the Industrial Revolution. As is common among these scholars, coal gets a mention. In fact it gets a full chapter (22) in McCloskey's ``Bourgeois Dignity'', \footnote{McCloskey 2010 \cite{mccloskey_bourgeois_2010}} but is only one of very many factors (and chapters -- a total of 46) for McCloskey.
	
	David Landes, with his ``The Unbound Prometheus'', \footnote{Landes 1969 \cite{landes_unbound_1969}} became the mainstream doyen of recent documenters of the Industrial Revolution, and discusses in some detail the proximate technical and industrial institutional changes. While primarily a supply-side explanation, he does address the role of demand in leading technological change. He recognizes the role of energy substitution.  But in the end, he does not satisfactorily explain why England and why then without appealing later work to the theme of English cultural exceptionalism.
	
	Jack Goldstone perhaps epitomizes modern economic historians. Goldstone is a premier member of the ``California School'' of anti-Eurocentric revisionist historians, and a prolifically good one. He thus has the dual burden of explaining ``why England?'' and ``why not China?'' In the end, while at root a revisionist, he lands not that far from Landes. From ``The Rise of the West -- Or not,'' he states `` ...a very accidental combination of events in the late seventeenth century placed England on a peculiar path, leading to industrialization and constitutional democracy. These accidents included the compromise between the Anglican Church and Dissenters, and between Crown and Parliament, in the settlements of 1689; the adoption of Newtonian science as part of the cosmology of the Anglican Church and its spread to craftsmen and entrepreneurs throughout Britain; and the opportunity to apply the idea of the vacuum and mechanics to solve a particular technical problem: pumping water out of deep mine shafts in or near coal mines. Without these particular accidents of history, there is no reason to believe that Europe would have ever been more advanced than the leading Asian civilizations of the eighteenth and nineteenth centuries.'' \footnote{Goldstone 2000 \cite{goldstone_rise_2000}}
	
	So Goldstone asserts an accidental path-dependent institutional explanation while giving a brief hat tip to the development of steam engines. He is a self-described anti-Eurocentric California School member, but exposes the contradictions that this group faces. For example, he brackets the arguments of the intellectual space by ending up in a multi-causal culturally biased explanation that describes a great deal, but in the end explains little about primary causes that are prescriptively useful for development economists; and in my opinion he does not fully explain ``what happened,'' ``why it happened,'' and ``when it happened.'' He understands that the pumping machines for coal mines were important, but misses that it was the coal use itself that was the primary driver of the Industrial Revolution.
	
	Joel Mokyr is perhaps the premier purveyor of the view of the ``scientific-practical culture of England-the engineers, craftsmen, and entrepreneurs who specialized in applying the Newtonian science into machines useful for production.'' So he is the chronicler of the tinkerer class, (allegedly a quote of Peer Vries - for whom I am searching for a citation) the core of the English who were culturally uniquely capable of developing the technology of the Industrial Revolution. \footnote{Joel Mokyr 1992 \cite{mokyr_lever_1992}}
	
	It is not possible to conclude this far too abbreviated summary of Eurocentric historians without briefly mentioning Max Weber -- in important ways the most Eurocentric of all observers. While too reductionist in fact, what I take from Weber is that he essentially said that European protestant work ethics were the primary factors in the rise of the west. \footnote{ Max Weber, \textit{The Protestant Ethic} 2002(1904-1904,revised 1920) (\cite{weber_protestant_2002}}


	\paragraph{Somewhat cultural explanations of the English Industrial Revolution.}  
	This group of Economic historians is smaller than the first group because in my judgement they have developed a more coherent view of the English Industrial Revolution. It is worth noting that the first group (excepting Weber) are all American historians. This group is only partly American and is not Eurocentric. Kenneth Pomeranz is American. Robert Allen is also, but works professionally in England. Carlo Cippola was Italian who later in his career taught at Berkeley. It is Cipolla with whom I start.
	
	Carlo Cipolla had a unique perch from which to develop his worldview -- from Pavia west of Venice, thus astride the great medieval land trade routes and geographically centered on the land of the Italian city states, he could see and think both to the west and east. His wry and witty histories are the antecedents of Mokyr and later historians in the sense of describing the technical advances of the European adventure, but within the context of a world historian.

	One of his most acute set of observations, in his 1966 ``Guns, Sails and Empire,'' \footnote{Cipolla 1966 \cite{cipolla_guns_1966}} describes the new technology of early modern era European sailing ships as an energy revolution that ``transcended the limitations of human energy and obtained a decisive advantage over non-Europeans'' who were still using oarsmen and armed boarding parties. The far more energy intensive European war ships carried heavy cannon, yet could both out-run, out-manoeuvre, and out-fight their rivals. The Europeans, first the Spanish and Portuguese, then the Dutch, and finally the English used this highly energy-intensive technology to conquer and dominate the world's ocean trade routes and trade with a potent blend of military mercantilism. 
	
	Thus, while not ignoring cultural differences, Cipolla for me defines the true essential elements of the successful rise of the west by describing the first major energy revolution since the Neolithic, and prefiguring the $18^{th}$ century mineral energy revolution that was yet to come.

						
	California School stalwart Kenneth Pomeranz describes the crucial determinants in the rise of the west in terms of coal, colonies and cotton. As an anti-Eurocentric and sinologist, he describes fully the cultural and social similarities and differences between England and the most relevant Chinese comparable region, the Yangze Valley, and concludes that since the differences were not sufficient causes, the ``Great Divergence'' in the outcome must be instead explained through coal, colonies and cotton (as a substitute for English land-intensive wool). \footnote{Pomeranz 2001 \cite{pomeranz_great_2001}}
	
	Robert Allen uses a largely quantitative microeconomic approach to explaining the English Industrial Revolution. He even attempts an econometrically estimated theoretical model which includes some institutional variables. And in the end he comes close to my hypothesized truth, concluding that ``The British were simply luckier in their geology...there was only one route to the twentieth century -- and it traversed northern Britain''. \footnote{Allen 2009 \cite{allen_british_2009}}
	
	\paragraph{Primarily energetic explanations of the English Industrial Revolution.} The reference count reduces as we traverse the continuum. My primary references here are only two: Edward Anthony (E. A.) Wrigley, the great English economic demographer, and the little known Fred Cottrell. I also cite a couple of Dutch economic historians who for me complete the story of the English geological exceptionalism that Allen and Pomernz describe. Tony Wrigley strays from his demography ``day job'' into the fray as explainer of the English Industrial Revolution starting with his 1988 ``Continuity, chance and change: the character of the industrial revolution in England'' \footnote{E. A. Wrigley 1988, \cite{wrigley_continuity_1988}} in which he clearly lays out the prime cause as the transition from an ``organic'' economy, albeit an advanced one harvesting energy from \gls{insolation}, to a mineral economy in which the limitations of organic energy sources were transcended.
	
	In his 2010 ``Energy and the English Industrial Revolution'' \footnote{Wrigley 2010 \cite{wrigley_energy_2010}} he advances and summarizes his arguments and focuses on the advances in productivity from the mineral energy transition. Wrigley cites and builds on the much less known and highly contrarian work of the iconoclastic American sociologist Fred Cottrell, who thought deeply about the role of energy in human history.
	
	Cottrell wrote ``Energy and Society'' in 1955. \footnote{Cottrell 1955 \cite{cottrell_energy_1955}} A sociologist, he clearly had a significant background in economics. This foundational book describes human activity, including economic activity, in terms of its net energy requirements. The expansion of civilization and its standard of living is directly related to increasing access to energy supplies with an expanding net surplus of available energy output over what is required to harvest the energy, a term now called Energy Return on Investment (EROI).\footnote{A term dating to 1984 and attributed to Charles A. S. Hall, cf. \cite{cleveland_energy_1984}} The greater the energy surplus, the more the society can use energy for uses other than just extracting energy. So the wealth of nations becomes a function of their net energy surpluses.
	
	Cottrell provides interesting examples of net energy calculations, including the great increase in energy surplus from the transition to early modern European sailing ships, a topic to which Carlo Cippola returned a decade later. While discussing economic activity he does not use the term ``capital'' but rather the concept of high energy intensive ``converters'' (i.e. industrial machines). This he naturally contrasts with low energy intensity converters (man and other animals).
	
	Even while his approach is very close to that of the thermodynamicists, and usefully for this study, Cottrell uniquely and strongly asserts that changes in access to energy surpluses shapes civilization, its institutions, its social relations, and its culture. His causality runs from energy access to culture at the sociological level rather than the other way around as argued at least implicitly by many of the references here. The concept has lately been anointed ``sociological thermodynamics.''
	
	Finally for this group of energy-biased scholars, I cite two Dutch economic historians. Jan Luiten van Zanden is probably closest to a new institutionalist, and his recent compendium ``The Long Road to the Industrial Revolution: The European economy in a global perspective, 1000-1800'' \cite{van_zanden_long_2009} largely covers the institutional changes in north west Europe that preceded the English Industrial Revolution, changes such as the European Marriage Pattern to which can be attributed the Middle Ages' rise in incomes and consumption that set the stage for the demand-led revolution. 
	
	Additionally however, van Zanden discusses the Dutch growth experience and it is that experience that is fundamental to understanding how geographically privileged the English were in the following sense: the Dutch had essentially everything from a social, cultural, institutional and entrepreneurial standpoint that the English had. They attempted to move from a pre-industrial regime to an industrial one by using their only readily available energy source -- peat. They were successful until the peat ran out -- and the abundantly coal-fueled English prevailed. This experience is in essence a natural experiment that negates the arguments that the revolution was culturally driven.
	
	Prefiguring van Zanden in 1978, J. W. de Zeeuw's article ``Peat and the Dutch Golden Age'' \footnote{de Zeeus 1978 \cite{de_zeeuw_peat_1978}} directly attributes the Republic's economic rise to energy consumption, and provides economy-wide energy consumption calculations, including comparisons with other economies, giving empirical support to van Zanden's assertion.

	\paragraph{Thermodynamic explanations of growth.} There is a vanishingly small group of scholars who describe economic activity almost exclusively in thermodynamic terms. It is worth reflecting that from this standpoint all work, including all economic work, is a function of an energy transformation. Energy input causes output. A rare case of clear physical causality in economics -- thermodynamics does after all rule.
	
	The most well known thermodynamicist is the renowned mathematician and economist Nicolas Georgescu-Roegen. Georgescu-Roegen tied the second law of thermodynamics to production theory, pointed out the implications for the sustainability of economic growth, and thus laid the foundation for the modern fields of ecological and evolutionary economics. Ironically, his work has had little impact in the field of development economics. \footnote{Georgescu-Roegen 1975 \cite{georgescu-roegen_energy_1975}}
	
	More recently, Ayres et al. \footnote{Robert Ayres et al. 2003 \cite{ayres_exergy_2003}} have taken a microeconomic approach, using various production functions to empirically estimate coefficients on energy as well as labor and capital. They conclude that energy input is sufficient to explain away the infamous ``Solow residual'' that underpins modern growth theory, and in fact finding that one can drop both the labour and capital inputs, with energy input explaining most of the output. %\textit{Pace} Karl Marx. removed 10/13/11 per Garrett
	
	Much more recently, the compelling thermodynamic description of economic activity attracted a physical scientist, Timothy Garrett, who has modelled economic activity as a thermodynamic system and estimated the coefficient, and thus advanced Georgescu-Roegen's work on (the lack of) carbon-fueled output sustainability by simplifying the sometimes arcane calculations of carbon dioxide output. \footnote{ Timothy Garrett 2009 \cite{garrett_are_2009}} Garrett removes labour from the ouput estimation model, showing that value equals the rate of energy consumption with an estimate of $9.7 \pm 0.3$ Mw per 1990 USD.
		
	\paragraph{A brief summing up.}	Before reviewing the empirical methodology literature, a summary of the lines of thought discussed to date may be useful. What is clear is that the prime causes of economic growth and development are a much debated and unsettled topic. I have modelled that by the notion of a continuum of views with the ``culturalists'' on one end and the ``thermodynamicists'' on the other. The culturalists are by far in the majority, while the thermodynamicists are unquestionably correct at the physical level. 
	
	What the science needs to try to unpack are the important dimensions of the emergent social systems in which the thermodynamics operate, and according to Cottrell, which are largely a function of societies seeking a high energy surplus. By important, I mean those necessary social changes without which high energy surplus societies will not emerge. After all, the answer to why do some societies develop and others do not remains one of the great remaining mysteries of economics, and at least one key to unlock the mystery seems to be in a measure of energy surplus.
	
	Thus the empirical methodology becomes important. In general, there are two options: deductive theoretical models or inductive empirical models. In reality, one cannot inductively model without some notion of a theory. Would Newton have inferred gravity without the apple falling from the tree? Surely one's starting position is a matter of philosophy -- theory first or data first, not theory only or data only. One must at some point specify and test a model, or at least select the variables about which one wishes to speculate; theory is eventually implicit in those acts. My proposed methodology, data first, is guided initially by Christopher Sims, and fleshed out by S\o ern Johansen and Katarina Juselius. I briefly review that literature as well as an empirical survey by James Payne.
	
	\paragraph{Empirical methodology.} Christopher Sims' 1980 attack on the deductive methodologies which yielded the large system-of-equations models common after the Cowles commission work has reverberated through econometrics every since. Sims essentially said that the methods of restricting the systems-of-equations so that they were uniquely soluble were ``incredible'' by which he strongly implied ``impossible.'' \footnote{Sims \textit{Macroeconomics and Reality}, 1980 \cite{sims_macroeconomics_1980}} 
	
	Sims' core point is that systems-of-equations models require the modeller to make \textit{a priori} model (coefficient) restrictions which are theoretically unsupportable, in fact so many restrictions ($m^2$ in a reduced-form system of $m$ equations) that it becomes impossible to model with consistent theory. He further offered the ``solution'' of unrestricted \gls{var} which make no \textit{a priori} coefficient restrictions and, initially, no theoretical limits.
	
	Building on his work, and the cointegration work of Engle and Granger, \footnote{Granger 1986 \cite{granger_developments_1986}} Johansen and Juselius developed the \gls{copenhagen} some methods of which have been econometrically popular since the 1990's. \footnote{Johansen \textit{Likelihood-based Inference on Cointegration if the Vector Autoregressive Model} \cite{johansen_likelihood-based_1995}} The applied methodology is summarized in Juselius' 2007 book ``The Cointegrated {VAR} Model: Methodology and Applications'' \footnote{Juselius 2007 \cite{juselius_cointegrated_2007}} which is the analytic strategy I follow as described in section \ref{sec:genmethod}. To my knowledge, this method has not been applied in the type of historical context I propose to study.
	
	
	In his 2010 ``Survey of the International Evidence on the Causal Relationship between Energy Consumption and Growth'' \footnote{Payne 2010 , p. 86 \cite{payne_survey_2010}} James Payne concludes that the 101 recent studies he surveys yield no consensus of the relationship even at the country level. He attributes this failure to several methodological issues, one of which is not paying sufficient attention to the coefficients the various methods yield. 
	
	Payne's survey validates my choice of \gls{var} as a modelling strategy for my research question, and his criticisms of existing work validates my choice of using the \gls{copenhagen} methodology. According to Payne, the current researchers, in order to fully understand the causal relationship, must ``examine the coefficients with respect to both the sign (positive or negative) and magnitude of the relationship between energy consumption and economic growth.'' 
	
	None of the surveyed studies apparently use the full Copenhagen methodology which I will apply, and which should satisfy the methodological issues Payne identifies. None of these studies appear to attempt to model exogenous time intervention variables which, as I demonstrate in table \ref{tbl:station}, can make a significant difference in model specification. Further, the Copenhagen methodology maximizes the information extracted from the empirical model by decomposing short-, medium-, and long-run dynamics, little of which was reported in the surveyed works.
	
	While I hypothesize energy consumption as fundamental to economic growth, the short run dynamics of an economy learning how to consume energy at an unconstrained rate should provide insights useful for development economists.
	

	\subsection{Categorical Enumeration of Literature and Sources}

	Table \ref{tbl:cite} categorizes all the references considered during preparation for this topic defense; there will be additions and deletions in the final dissertation.
	\setlength{\LTleft}{-0.50 in}

	\begin{longtable}{llll} 

	\caption[Categorized references]{References categorized by function, school of thought, geographical coverage, and author \label{tbl:cite}}\\ 
		\hline\hline
	Historical, Empirical,&School of Thought&Country&Citation\\
	Theoretical, or Data&&&\\
	\hline\hline\hline
	\endhead
%%%%%%%%%%%%%%%%%%% Historical	
	Historical&Culture, Institutions&World System&Janet Abu-Lughod\cite{abu-lughod_before_1991}\\
	
	&Culture, Institutions&China&Rhea C. Blue\cite{blue_argumentation_1948}\\

	&Culture, Institutions&England&Coleman and Cuthbert \cite{coleman_economy_1977}\\

	&Culture, Institutions&U.S.&Davis et al. \cite{davis_american_1972}\\

	&Culture, Institutions&England, Holland&Jan de Vries \cite{de_vries_industrial_1994}\\

	&Culture, Institutions&England&Jack Goldstone\\
	&&&\cite{goldstone_why_2008,goldstone_whose_2000,goldstone_rise_2000, goldstone_capitalist_1983, goldstone_cultural_1987, goldstone_trend_1993, goldstone_initial_1998, goldstone_efflorescences_2002}\\

	&Culture, Institutions&England&Floud and McCloskey \cite{floud_economic_1994}\\
	
	&Culture, Institutions&World System&Andre Gunder Frank \cite{frank_reorient:_1998}\\
	
	&Culture, Institutions&China&Robert Hartwell \cite{hartwell_revolution_1962, hartwell_markets_1966, hartwell_cycle_1967}\\
	
	&Culture, Institutions&Multi-Country&William McNeill \cite{mcneill_rise_1963,mcneill_rise_1990}\\

	&Culture, Institutions&Sung China&Shiba and Elvin \cite{shiba_commerce_1970}\\

	&Culture, Institutions&England&Graeme Snooks \cite{snooks_was_1994}\\

	&Culture, Institutions&U.S.&Peter Temin \cite{temin_causal_1975}\\
	
	&Culture, Institutions&England, Holland&Jan Luiten van Zanden 
\cite{van_zanden_long_2009}\\

	&Culture, Institutions&England&Max Weber\cite{weber_protestant_2002}\\
	
	&Culture, Institutions&U.S.&Harold Williamson \cite{williamson_growth_1951, williamson_long_1962}\\

	&Culture, Institutions&China&R. B. Wong \cite{wong_china_1997}\\

	&Culture, Institutions&China&Xu and Wu \cite{xu_chinese_2000}\\

	\midrule
	
	&Technology&England&T.S. Ashton\cite{ashton_industrial_1966}\\
	
	&Technology&Global&Fred Cottrell \cite{cottrell_energy_1955}\\
	
	&Technology&England&Engels and Kelley \cite{engels_condition_1892}\\

	&Technology&China&Hsien-Chun Wang \cite{hsien-chun_wang_discovering_2009}\\
	
	&Technology&England&David Landes\cite{landes_unbound_1969}\\
	
	&Technology&England&Joel Mokyr \cite{mokyr_enlightened_2010,mokyr_lever_1992}\\
	
	\midrule
%%%%%%%%%%%%% Geographical;	
	&Geographical&England&Robert Allen\cite{allen_british_2009,allen_great_2001}\\
	
	&&England&William Stanley Jevons\cite{jevons_coal_1965}\\

	&&China, England&Kenneth Pomeranz\cite{pomeranz_great_2001}\\

	&Energy&England&E.A. Wrigley \cite{wrigley_continuity_1988, wrigley_transition_2006, wrigley_energy_2010}\\

	\midrule
	
%%%%%%%%%%%%%%%%%%%% Diverse	
	&War/energy revolution&England&Carlo Cippola \cite{cipolla_guns_1966,cipolla_clocks_1967,cipolla_before_1983}\\
	
	&No single opinion&England&N.F.R. Crafts \cite{crafts_industrial_1977}\\

	&No single opinion&England&D.C. Coleman\cite{coleman_economy_1977}\\
	\midrule
	\midrule

%%%%%%%%%%%%%%%%%%%% Empirical	
	Empirical&Malthusian growth&Multi-Country&Ashraf and Galor\cite{ashraf_dynamics_2011}\\
	
	&Energy intensity decline&U.S.&Baksi and Green\cite{baksi_calculating_2007}\\

	&Medieval Warm Epoch&W. Europe&Bradley et al.\cite{bradley_climate_2003}\\

	&Agricultural revolution&England&Gregory Clark\cite{clark_price_2004}\\

	&Coal importance&Multi-Country&Edwin Eckel\cite{eckel_coal_1921}\\

	&Coal, iron&England&G. Hammersley \cite{hammersley_charcoal_1973}\\

	&Charcoal use&Multi-Country&Peter Harris \cite{harris_charcoal_????}\\

	&Modern growth&Multi-Country&Simon Kuznets \cite{kuznets_modern_1966}\\

	&Growth&U.S.&Douglass North \cite{north_economic_1966}\\

	&Energy/growth&Multi-Country&James Payne \cite{payne_survey_2010}\\

	&Energy/growth&Multi-Country&Vaclav Smil \cite{smil_energy_2008}\\

	&Energy/growth&OPEC&Jay Squalli \cite{squalli_electricity_2007}\\

	&Growth&England&Peter Temin \cite{temin_two_1997}\\

	&High energy intensity&Holland&J. W. de Zeeuw \cite{de_zeeuw_peat_1978}\\


	\midrule
	\midrule
%%%%%%%%%%%%%%%%%%%% Theoretical	
	Theoretical&Energy&U.S.&Robert Ayres\cite{ayres_exergy_2003}\\
	
	&Ecological/Thermodynamic&Global&Herman Daly et al. \cite{daly_valuing_1993}\\
	
	&Statistical Equlibria&Global&Duncan Foley \cite{foley_statistical_1996}\\
	
	&Demographic transition&Multi-Country&Oded Galor \cite{galor_demographic_2011}\\

	&Thermodynamic Economics&Global&Timothy Garrett \cite{garrett_are_2009}\\

	&Thermodynamic Economics&Global&Nicholas Georgescu-Roegen \cite{georgescu-roegen_energy_1975}\\
	
	&Technology and Malthus&Multi-Country&Michael Kremer \cite{kremer_population_1993}\\

	&Labour Supply&Global South&Sir W. Arthur Lewis\cite{lewis_economic_1954}\\

	&Residual Growth Theory&U.S.&Robert Solow \cite{solow_technical_1957}\\
	
	\midrule
	\midrule
	
%%%%%%%%%%%%%%%%%%%% Data	
	Data&GDP&Global&Brad DeLong \cite{de_long_estimates_1998}\\
	
	&Energy Consumption&England&Roger Fouquet \cite{fouquet_heat_2008}\\
	
	&Output and demographics&U.S.&Haines and ICPSR \cite{haines_historical_2010}\\
	
	&Population&Global&Michael Kremer \cite{kremer_population_1993}\\
	
	&GDP, population&Global&Angus Maddison \cite{maddison_world_2007}\\

	&Economic, demographic&England&B. R. Mitchell \cite{mitchell_british_1988}\\

	&Energy consumption&U.S.&Energy Information Agency \cite{u.s._energy_information_administration_energy_????, u.s._energy_information_administration_u.s._????, u.s._energy_information_administration_u.s._????-1}\\

	&GDP, population&England&Lawrence Officer \cite{officer_what_2009}\\

	&Energy consumption&England&Paul Warde \cite{warde_energy_2007}\\
	
	\midrule
	\midrule
%%%%%%%%%%%%%%%%%%%% Methodology

	Methodology&Structural breaks&&S\o ern Johansen et al. \cite{johansen_cointegration_2000,johansen_likelihood-based_1995,johansen_testing_2010}\\
	
	&Cointegration&&Katrina Juselius \cite{juselius_cointegrated_2007}\\

	&Survey&&James Payne \cite{payne_survey_2010}\\

	&VAR model specificaton&&Christopher Sims \cite{sims_macroeconomics_1980}\\

	&Unit root&&Zivot and Andrews \cite{zivot_further_1992}\\

	\bottomrule

	\end{longtable}

%	\end{adjustwidth}	
	
\begin{comment}%adds
johansen I2
bannister
garrett
sims
theil
koopmans
sangar
phillips
haavelmo
\end{comment}



	\section{Research Procedure}

		\subsection{Theoretical Framework}
		I propose to study the first dramatic example of sustained development and growth in recent history, that is, at least in the last millennium.  This study of the English Industrial Revolution should yield both answers about that event as well as a framework I can use to study other important development and growth stories.
		
		For England, I examine the role energy consumption played in the history of economic development and growth.  I use both historical-institutional and econometric analyses.  I hope to discover time-varying patterns of growth, their relation to energy consumption, and the critical institutional framework in which the patterns develop. Preliminary analytic results are included later in this proposal in section \ref{sec:analytic}.
		
		\textit{Ex ante}, my sole favoured formal economic deductive theory is that energy consumption, population, and GDP are strongly interrelated and these relations are primary for sustained \gls{growth}. My intuition is that learning to consume essentially unconstrained supplies of energy is \textit{the} critical factor determining whether a nation-state experiences \gls{growth}. \footnote{Kuznets 1966 \cite{kuznets_modern_1966}}. %I describe the reasons for these initial beliefs in %an unpublished work, which is the foundation for the study here of the English %experience. \footnote{Bannister 2010} 
		
\begin{comment}
		In general theoretically, this work may lie clearly within the realm of Foley and Sidrauski's \footnote{\textit{Statistical Equilibrium Models in Economics} \cite{foley_statistical_1996}} ``flow'' equilibrium models of consumption and labour markets, and thus formally rejects any requirement for a more traditional rational expectations equilibrium process or aggregative representative agents.  
		
		Foley and Sidrauski redefine classical static economic equilibriums as probability distributions of flows of transactions over the space of all possible transactions.  In their terms, ``flow'', or statistical, equilibriums map to my chosen empirical framework of cointegration relations pushed and pulled by various equilibrating forces. My methodology identifies those cointegrating long-run \textit{statistical} relationships which I test for consistency with my overall theory of the strong relation between energy consumption rates and GDP growth.
		
		Additionally I search for historical-institutional changes as model-exogenous explanatory variables.  I will allow a demand-oriented model to guide my search for these important facts, essentially a \gls{pk}, largely \gls{verdoorn}, view that demand leads output, and productivity follows.  The questions I ask at every historical period where structural changes occur are: From where does demand emerge, and what is its structure? How is the demand satisfied, especially, what are the incentives for and knowledge available to whomever the inventor class may be?
\end{comment}

		The method I use to test this hypothesis is one that makes no formal assumptions about production functions, rational expectations, aggregative representative agents, or any of the other traditional apparatus typically used to build (usually static) macro models that explore such questions.
		
		Instead, the methodology relies on testing for long-term statistical equilibria among the core variables, then defines which short and medium term shocks and actions push or pull toward statistical equilibrium.
		
		Beyond the substantial research support for using \gls{cvar}, I also appeal to the theoretical work of Foley and Sidrauski's \footnote{\textit{Statistical Equilibrium Models in Economics} \cite{foley_statistical_1996}} ``flow'' equilibrium models of consumption and labour markets. In this work, Foley and Sidrauski redefine classical static economic equilibriums as probability distributions of flows of transactions over the space of all possible transactions, with equilibrium becoming statistically defined.
		 
		\subsection{General Research Methodology}
		The historical-institutional methodology is both descriptive and quantitative. Where historically long-period times series are available, I use time-series econometric techniques including \gls{var}, \gls{cvar}, and structural econometric methods to describe structural dynamics and  regime changes in the time paths of the key variables. The primary data series I anticipate using are energy consumption data, economic output data, and population data.

		\subsection{A word on the statistical methodology: the ``Copenhagen'' school of time series econometrics}
		
		In the 1980's Robert Engle and Sir Clive Granger did the seminal work on \gls{coint} for which they shared a Nobel prize. \footnote{\textit{Developments in the Study of Cointegrated Economic Variables}\cite{granger_developments_1986}}
		
		In the 1990's the University of Copenhagen mathematical statistician S\o ern Johansen collaborated with a University of Copenhagen econometrician, Katarina Juselius, in further developing the mathematics and applied tools for cointegration studies generally categorized as Cointegrated Vector Autoregressive (CVAR) models. The models are specified as equilibrium correction models (ECM). Their work is widely recognized by the shorthand ``Johansen cointegration method'' and the resulting models are often called error correction models. 
		
		A discussion with a committee member makes it clear that I need to further justify this methodology. The nub of the conversation was that if indeed I can show these super-exponential series, why go to all the econometric trouble I am about to embark on. Fair question. The curves are overwhelmingly descriptive and suggestive.
		
		Upon reflection, I am interested in looking inside the dynamics that are supported by the long period time series. What this examination should tell is whether the prime dynamic drivers, that is the leading-in-time variables, changed places in the either the short run or the long run in the models I specify. While I purposefully avoid the term causality at this point, depending on the strength of the results, the data may support either the presence or absence of statistical causality.
		
		What I want to understand, beyond my hypothesis of the centrality of mineral energy consumption as the defining invention of the Industrial Revolution, is what implications this has for modern development and for sustained per capita economic development in an age of potentially emission constrained economies. It may be possible to comment on the importance of institutional timing in the process with sufficient time series dynamics. 
		
		The time series methodology I propose has the ability to do this. It further has the capability of incorporating important time related events that enter the time series as discontinuities. Both of these capabilities are core to my research.
		
		The statistical attractions of the methodology are several, perhaps manifest:
		\begin{itemize}
			\item  It does not require prior coefficient identification, permitting data-driven model exploration. Thus the iconic 1980 ``incredibility'' assertion by Christopher Sims is fully recognized and embraced. \footnote{Sims asserted that the then common large scale systems of equation models were subject to incredible identification restrictions, and thus could not be credibly valid. He recommended using \gls{var} methods instead. \textit{Macroeconomics and Reality}\cite{sims_macroeconomics_1980}}
			\item It makes no prior assumptions on endogeneity or exogeneity, reducing the risks of model uncertainty (misspecification). Thus, it is a \textit{symmetric} modelling approach.
			\item It decomposes systems into long-run cointegrated, or statistical, equilibriums and encourages their economic interpretation, and shorter run forces which either pull a system from disequilibrium toward equilibrium or push the system along an equilibrium path.  The equilibriums so far described are purely statistical.  Part of the methodology involves labelling the identified forces in economically meaningful terms.
			\item By identifying short-run pushing and pulling forces within the system, I develop a rich semantic for describing structural dynamics and how they react to stochastic shocks.
			\item Thus, I seek a very empirically rich description of dynamic interactive systems which hopefully provoke useful, perhaps unique, economic insights. The methodology is described in section \ref{sec:method}.
		\end{itemize}
		On the other hand, this is an analytically demanding methodology with no pre-packaged ``push-button'' solutions, though there is good statistical software support.  My intent, partially through this dissertation project, is to become an expert in this methodology.

		\subsection{Specific Research Methodology} 
		\label{sec:genmethod}
		Following Juselius \footnote{\textit{The Cointegrated {VAR} Model: Methodology and Applications }\cite{juselius_cointegrated_2007}}, I formalize my theoretical model in $VAR$ (levels) form as follows:
		\begin{equation}
		\begin{split}
		\bold{x}_t &= \boldsymbol{\Pi}_1 \bold{x}_{t-1} + \cdots + \boldsymbol{\Pi}_k \bold{x}_{t-k} +\\
		 &\boldsymbol{\Phi}_{s1} D_{sDemandChange_t} + \boldsymbol{\Phi}_{s2} D_{sTechnicalChange_t} +
		 \boldsymbol{\Phi}_{s3} D_{sExports_t} +\\
		& \boldsymbol{\Phi}_{trend} t + \boldsymbol{\mu}_0 + \boldsymbol{\epsilon}_t,
		\end{split}
		\end{equation}
		where $\bold{x}' = \{ln(energyConsumption), ln(GDP), ln(population)\}$ each in units that I will determine based on the series I choose, $k$ is the number of lags for the independent variables, $\boldsymbol{\Phi_{s1}}$ is a mean-shift dummy variable for the social-institutional change in income levels and distribution reflecting a rise in consumer demand, $\boldsymbol{\Phi_{s2}}$ is a mean-shift dummy variable for an increase in the ability to accumulate and disseminate technical knowledge for inventor/entrepreneurs,$\boldsymbol{\Phi_{s3}}$ is a mean-shift dummy variable for the social-institutional change in export levels reflecting a rise in external demand, $\boldsymbol{\Phi_{trend}}$ is a deterministic time trend, and $\boldsymbol{\mu}_0$ is a vector of constants.
		
		I hypothesize the mean-shift in consumer demand to have arisen out of the changing institutional and social patterns from the high middle ages through the early modern period, including the Black Death in 1348. I hypothesize the mean-shift in accumulated and disseminated technical knowledge to have started with the 1448 invention of the Gutenberg printing press. These are variables to control for significant structural shifts in the data. The nulls in each case are that the coefficients are not statistically different from zero.
		
		By formulating the main information set, i.e. energy consumption, economic output (GDP), and population, as a vector autoregressive system, I will allow the data to determine which variables importantly affect others, and whether they do so in the short-run, the long-run, or both.  The methodology admits the possibility of feedback cycles in the data, and allows their description.
		
		For historical-institutional research required to understand the social changes that necessarily preceded the Revolution, I will rely on the research of English, Dutch, Chinese, and American historians, mostly economic historians who have, in some cases, examined the causes back to the High Middle Ages.  These histories shed light on what was potentially different in the English case.
	
		\subsubsection{Empirical Methodology} 
		\label{sec:method}
		
		 I provide detailed step-by-step methodology in Appendix \ref{sec:Appendix A}. This section describes the general approach.
		
		\paragraph{Summary of methodological steps.}
		From the theoretical \gls{var} in section (\ref{sec:genmethod}), the methodology writes several forms for specific purposes:
		
		\begin{itemize}
		\item First, the model is written as a structural vector equilibrium correction model to present the economic theory.
		\item Second, the model is rewritten as an equivalent moving average model to emphasize and test possible linear trends.
		\item Third, the model is extended to include intervention variables, or time dummies, to account for major historical/institutional events.
		\item Fourth, the model is rewritten as a reduced-form vector equilibrium correction model so it is soluble, and is estimated and tested.
		\end{itemize}
		
		\paragraph{Cointegrated vector equilibrium correction model reduced-form specification.}
		As an example, the general VAR model is written in this reduced form that can be estimated via maximum likelihood:
		$$
		\Delta \bold{x}_t = \boldsymbol{\mu}_0 + \boldsymbol{\mu}_1 t + \boldsymbol{\Pi} \bold{x}_{t-1}+ \sum_{i=1}^{p-1}\boldsymbol{\Gamma}_i \Delta \bold{x}_{t-i} +\boldsymbol{\epsilon}_t
		$$
		where: $\Delta \bold{x}_t$ is an m x 1 vector of first differences of endogenous dependent variables (energy consumption, GDP, and population) with $m=3$ in this case. Economic time series modelling commonly uses discrete data and represents the data as difference equations. This is functionally similar to representing continuous data as first derivatives. The continuous data analog of $\Delta \bold{x}_t$ is $\frac{dx}{dt}$. $\boldsymbol{\mu}_0$ is an m x 1 vector of intercepts; $\boldsymbol{\mu}_1 t$ is an m x 1 vector of coefficients describing the deterministic component of the series related to time; $\boldsymbol{\Pi}$ is an m x m matrix of coefficients describing the long term equilibrium correction relationships among the core variables. This is later decomposed into vectors of weightings and cointegrating relationships; $\boldsymbol{\Gamma}_i$ is an m x m matrix of coefficients representing the short-term adjustments to economic shocks for each lag that is relevant in the system up to a lag length of $p$; $\Delta \bold{x}_{t-i}$ represents the first differences of the lagged core variables; $\boldsymbol{\epsilon}_t$ is an m x 1 vector of error terms.
		
		\paragraph{Appendix B contains descriptions of the symbols used in Appendix A}
		
		\subsection{Preliminary Analytic Results} \label{sec:analytic}
		\subsubsection{Origin of time series data}
		\paragraph{Data series used for preliminary analytic results.}
		All preliminary analysis has been on English or English and Welsh data. The initial series have the sources as indicated in Table \ref{tbl:data}.
		
		\paragraph{Additional English data series to be evaluated.}
		I intend to use these additional data series to determine which series yield the best econometric model: Angus Maddison's data on GDP and population \footnote{Maddison 2007 \cite{maddison_world_2007}}; and Paul Warde's data on English energy consumption \footnote{Warde 2007 \cite{warde_energy_2007}}.
		
		\begin{table}[h!]
		\centering
		\caption{Data series for preliminary analytic results}\label{tbl:data}
		\begin{tabular}{lll}
		\toprule \hline
		Series&Range    &Source\\
		\midrule
		GDP (2005 GBP)&1300-1700&Snooks \cite{snooks_was_1994}\\
					  &1701-1874&Officer \cite{officer_what_2009}\\
					  &&\\
		Population	  &1300-1540&Snooks  \cite{snooks_was_1994}\\
					  &1541-1874&Mitchell \cite{mitchell_british_1988}\\
					  &&\\
		Energy Consumption (MTOE)&1300-1874&Fouquet \cite{fouquet_heat_2008}\\
		\bottomrule
		
		\end{tabular}
		\caption*{\textit{Notes:} These time series are prepared by the various Economic historians using several methods; the methods often result in benchmark dates rather than tables with complete annual data. In those cases, I used Stineman interpolation as implemented in the $R$ package \texttt{stinepack} \footnote{Bjornsson 2009 \cite{bjornsson_stinepack:_2009}}}
		\end{table}
		
		\subsubsection{List of preliminary tests}
			\begin{itemize}
			\item Initial structural tests
			\item Visual inspection of levels and diff()
			\item Univariate analysis with structural inputs
			\item Model with unit roots using KPSS
			\item Bivariate and multivariate cointegration tests
			\end{itemize}
		
		\subsubsection{Important attributes}
		\begin{itemize}
		\item In the methodology, theoretical models are nested in an empirical model. This implies that there is more than one possible theoretical model contained in the data, with one model possibly being more dominant at certain time frames and less dominant at others.  Thus, structurally dynamic empirical models can exhibit time dependent parameter changes in their theoretical interpretations.
		
		In fact, this particular econometric model likely has a feature that is rare in cointegration studies: mean reversion or, equivalently, stationarity does not appear to be a long-run feature (or even a very long-run feature) of the series in the observation set. If proven so, then in fact a unit-root process, if substantiated, will be a deep structural (economic) feature of the model, rather than a statistical convenience as is typically the case.
		
		An equivalent interpretation is that the important shocks to the empirical system will be permanent in nature rather than transitory. I evaluate these important (historical, economic, and statistical) attributes.
		
		\item structural breaks
		\item unit roots
		\item cointegration
		\end{itemize}
		
		
		\subsubsection{Initial structural analyses}
		
		As summarized in Table \ref{tbl:struct}, the series \gls{mtoe} has significant structural breaks at years 1590, 1736, and 1844. GDP has significant breaks at years 1556, 1731, and 1869. Population has significant breaks at years 1206, 1367, 1580, and 1774. In all cases I have not yet attempted to assign actual historical events to these breaks which statistically represent inter-regime parameter changes. I use the most significant of these breaks as time intervention variables in the following ARIMA and cointegration modeling. The structural breaks are analyzed using the $R$ package \texttt{strucchange} \footnote{Zeileis et al. 2002 \cite{zeileis_strucchange:_2002}} following methodology developed by Achim Zeileis et al. \footnote{Zeileis 2003 \cite{zeileis_testing_2003}}.
		
		\begin{table}[h!]
		\centering
%		\begin{adjustwidth}{1.00in}{}

		\caption{Structural analysis results}\label{tbl:struct}
		
		\begin{tabular}{lrllll}
		\toprule \hline
		Variable&Number of Breaks&Years\\
		\toprule
		Million Tonnes of Oil Equivalent (MTOE)&3&&1590&1736&1844\\
		Gross Domestic Product (GDP)&3&&1556&1731&1869\\
		Population&4&1206&1367&1580&1774\\
		\bottomrule
		\end{tabular}
%		\end{adjustwidth}

		\end{table}
		
%		\newpage		
		\subsubsection{Visualizing levels and differences}		
		
		
		Figures \ref{fig:globfig1}, \ref{fig:globfig2}, and \ref{fig:globfig3} provide plots of the levels, the base 10 log levels, and the first differences of the log levels of each of the three data series. The vertical lines on the log level plots represent the structural breaks identified in table \ref{tbl:struct}.
		
		One noteworthy result is that there is a fairly remarkable correspondence in  separately analyzed breaks in the $16^{th}$ and $17^{th}$ centuries. This supports the notion that there were significant historical events that should be identified.
		
		Another result from the first difference plots is that all three series do not visually appear to be stationary. This suggests that these unmodified series are probably $I(2)$, though the structural breaks have not yet been incorporated into the first differences. The ARIMA modeling which will incorporate the structural dating should clarify the statistical nature of the series.
		
		Also, in future analysis, I may drop the observations before 1500 to see if the econometric results improve. This will be part of the testing of alternative data series from other historians.
		
		All plots, in fact all statistical analyses, are done with the $R$ statistical language system \footnote{$R$ Team 2011 \cite{team_r:_2011}} and additional add-in functionality as cited.

%		\begin{figure}[h!]
		\begin{figure}
%		\centering

		\caption[Levels, \textit{log} of levels and diffs of the time series]{Levels, \textit{log} of levels and first differences of logs of English energy consumption, with statistically significant breakpoints indicated by vertical lines on the log chart.}						\label{fig:globfig1}
		
		\subfloat[\textit{level} energy consumption][English energy consumption in levels]{
		\includegraphics[width=0.33\textwidth]{../data/Analytics/notes/gmtoe.png}
		\label{fig:subfig1}}
%%		\qquad %%%%leave no blank lines to get them side by side
		\subfloat[\textit{log} energy consumption][\textit{log} of English energy consumption]{
		\includegraphics[width=0.33\textwidth]{../data/Analytics/notes/gbpmtoelog.png}
		\label{fig:subfig2}}
		\subfloat[diff \textit{log} energy consumption][Difference of \textit{log} of English energy consumption]{
		\includegraphics[width=0.33\textwidth]		{../data/Analytics/notes/gdifflogmtoe.png}
		\label{fig:subfig3}}
		
		\end{figure}
		

		\begin{figure}
%		\centering

		\caption[Levels, \textit{log} of levels and diffs of the time series]{Levels, \textit{log} of levels and first differences of logs of English GDP, with statistically significant breakpoints indicated by vertical lines on the log chart.}			\label{fig:globfig2}
				
		\subfloat[\textit{level} GDP][English GDP in levels]{
		\includegraphics[width=0.33\textwidth]{../data/Analytics/notes/ggdp.png}
		\label{fig:subfig4}}
		\subfloat[\textit{log} GDP][\textit{log} of English GDP]{
		\includegraphics[width=0.33\textwidth]{../data/Analytics/notes/gbpgdplog.png}
		\label{fig:subfig5}}
		\subfloat[diff \textit{log} energy consumption][Difference of \textit{log} of 				English energy consumption]{
		\includegraphics[width=0.33\textwidth] 	{../data/Analytics/notes/gdiffloggdp.png}
		\label{fig:subfig6}}
		
		\end{figure}

		\begin{figure}
		\centering

		\caption[Levels, \textit{log} of levels and diffs of the time series]{Levels, \textit{log} of levels and first differences of logs of the English population, with statistically significant breakpoints indicated by vertical lines on the log chart.}						\label{fig:globfig3}
		
		\subfloat[\textit{ln} population][\textit{ln} of English population in levels]{
		\includegraphics[width=0.33\textwidth]{../data/Analytics/notes/gpop.png}
		\label{fig:subfig7}}
		\subfloat[\textit{log} GDP][\textit{log} of English GDP]{
		\includegraphics[width=0.33\textwidth]{../data/Analytics/notes/gbppoplog.png}
		\label{fig:subfig8}}
		\subfloat[\textit{ln} diff population][\textit{ln} of first differences of English population]{
		\includegraphics[width=0.33\textwidth]{../data/Analytics/notes/gdifflogpop.png}
		\label{fig:subfig9}}
		
		\end{figure}		
		
%		\includepdfmerge[nup=2x3,noautoscale=true,scale=0.4,frame=true,offset=0 				-50,delta=0 10]						{../data/Analytics/notes/bpmtoe.png,../data/Analytics/notes/diffmtoe.png, ../data/Analytics/notes/bpgdp.png,../data/Analytics/notes/diffgdp.png, ../data/Analytics/notes/bppop.png,../data/Analytics/notes/diffpop.png}

%	\caption[
%	\textit{Ln} of levels and diffs of the time series]{\textit{Ln} of levels and first differences of the English time series, with statistically significant breakpoints indicated by vertical lines on the levels charts}


\newpage		
		\subsubsection{Stationarity}
		
		To further characterize the time-series, I tested each as a univariate \gls{arima} model using the $R$ \texttt{forecast} package \cite{hyndman_forecast:_2010}. The \texttt{auto.arima} method in \texttt{package::forecast} selects the appropriate model and allows incorporating exogenous (non-stochastic) inputs such as the time-series intervention variables described above.
		
		I note that after allowing for event interventions, both MTOE and GDP are $I(1)$ (non-stationary, or integrated of order one, mean a first difference or first derivative will be stationary. The population series is modeled as $I(0)$ (stationary), though the event interventions are not significant in this analysis. Each series analyzed without event dummies indicates $I(2)$, confirming the visual inspection results from figure \ref{fig:globfig1}.

		

		\begin{table}[h!]

		\begin{adjustwidth}{-0.75in}{}

		\caption[Summary \texttt{auto.arima} results]{This table contains the 		summary of the \texttt{auto.arima} analyses showing the results for no time intervention variables, and for results with time intervention variables}\label{tbl:station}

		\begin{tabular}{llrrrrrrrrr}
		\toprule \hline
		Breaks&Model&&Model&Variables&&&&&&\\
		\toprule \hline
		&MTOE&ar1&ar2&ar3&ar4&ma1&S1&S2&S3&drift\\
		\hline
		None&ARIMA(4,2,1)&0.8385& 0.5742&  -0.2776&  -0.3054&  -0.9334&&&\\
	    &s.e.        &0.0457&  0.0531&   0.0541&   0.0431&   0.0306&&&\\
		\hline
1590&ARIMA(4,1,1)&0.1880&1.3171&0.0014&-0.6435&0.8857&0.0065&0.0001&0.0004&-0.0049\\
1736&with drift&0.0354&0.0374&0.0341&0.0325&0.0270&0.0016&0.0016&0.0016&0.0019\\
1844&&&&&&&&&&\\
		\midrule \hline
		&GDP&ar1&ar2&ma1&ma2&ma3&ma4&S1&S2&S3\\
		\toprule
		None&ARIMA(1,2,4)&-0.6596&&0.3130&-0.6897&-0.1270&-0.1022&&&\\
	    &s.e.        &0.0608 &&0.0684&0.0492 &0.0613 &0.0531&&&\\
		\midrule
1556&ARIMA(2,1,3)&0.2427&0.7438&0.4158&-0.7589&-0.1777&&-0.0067&0.0030&-0.0156\\
1731&s.e.        &0.0467&0.0466&0.0634&0.0358&0.0600&&0.0030&0.0025&0.0062\\
1869&&&&&&&&&&\\
		\midrule \hline
		&Population&ar1&ar2&ma1&ma2&ma3&intercept&S1&S2&S3\\
		\toprule
		None&ARIMA(0,2,3)&&&-0.2994&-0.2525&-0.1521&&&&\\
	    &s.e.        &&&0.0410&0.0426&0.0411&&&&\\
		\midrule
1373&ARIMA(2,0,3)&1.9912&-0.9912&-0.2956&-0.2498&-0.1502&15.3036&-0.0016&-0.0017&-0.0017\\
1579&s.e.        &0.0072& 0.0072& 0.0416& 0.0429& 0.0415&    NaN& 0.0033&0.0033&0.0033\\
1774&&&&&&&&&&\\
		\bottomrule

		\end{tabular}
		\end{adjustwidth}
		\end{table}
		
		\newpage
		\subsubsection{Cointegration}
		
		Finally in this preliminary analytic exercise, I investigate the cointegration properties of the information set. To do so, I use the $R$ package \texttt{urca} as implemented from work done by Bernard Pfaff \cite{pfaff_analysis_2008}.
		
		The results are summarized in Table \ref{tbl:coint}. I model each series as a bivariate pair to assess the strength of any indicated cointegration relations and to support the multivariate modeling for the main analysis. I also model with different lag structures to assess robustness. The most robust bivariate result is between the MTOE and GDP pairs, with the second between the GDP and population pairs. These results support the multivariate modeling which indicates two cointegrating relations.
		
		The noteworthy result is that the cointegration relation between energy consumption and GDP is very strong, supporting my research hypothesis. These are preliminary, but encouraging, results. Further statistical testing awaits approval of this proposal.
		
		I also need to switch to the most robust cointegration software available, which is CATS in RATS as my analysis proceeds.
		

		\begin{table}[h!]
		\begin{center}
\caption[Johansen cointegration tests]{Johansen cointegration tests for English energy demand system; sample period 1300 - 1874 CE}\label{tbl:coint}
	
\begin{tabular} {llllrll}
\toprule \hline
&&No.&&&&\\
&Deterministic&of lagged&&Test& \multicolumn{2} {c} {Critical values} \\
\cmidrule (lr) {6-7}
Variables&terms&differences&$H_0:r=r_0$&statistic&$10\%$&$5\%$\\
\midrule
$e,g$&$tr,shift$&1&$r_0=0$&45.18&22.76&25.32\\
&&&$r_0=1$&0.81&10.49&12.25\\
&&&&&&\\
&&2&$r_0=0$&33.21&22.76&25.32\\
&&&$r_0=1$&0.79&10.49&12.25\\
&&&&&&\\
$e,p$&$tr,shift$&1&$r_0=0$&41.49&22.76&25.32\\
&&&$r_0=1$&12.45&10.49&12.25\\
&&&&&&\\
&&2&$r_0=0$&33.51&22.76&25.32\\
&&&$r_0=1$&10.20&10.49&12.25\\
&&&&&&\\
$g,p$&$tr,shift$&1&$r_0=0$&43.29&22.76&25.32\\
&&&$r_0=1$&6.02&10.49&12.25\\
&&&&&&\\
&&2&$r_0=0$&35.73&22.76&25.32\\
&&&$r_0=1$&5.97&10.49&12.25\\
&&&&&&\\
$e,g,p$&$tr,shift$&1&$r_0=0$&78.54&39.06&42.44\\
&&&$r_0=1$&41.45&22.76&25.32\\
&&&$r_0=2$&5.16&10.49&12.25\\
&&&&&&\\
&&2&$r_0=0$&63.06&39.06&42.44\\
&&&$r_0=1$&28.94&22.76&25.32\\
&&&$r_0=2$&5.42&10.49&12.25\\
\bottomrule
\end{tabular}

\caption*{\textit{Notes:} \textit{e}-ln of MTOE (million tonnes of oil equivalent), \textit{g}-ln of GDP, \textit{p}-ln of population. For the model's deterministic components: \textit{c}-constant, \textit{tr}-linear trend, \textit{shift}-shift dummy $dsubS3$; critical values from Johansen.}
\end{center}
\end{table}

		\newpage


	\section{Completion Timetable}
		\begin{tabular}{ll}
		Event&Date\\
		\midrule
		Draft Proposal&August 2011\\
		Topic Defence&Spring 2012\\
		Draft Dissertation&Fall 2012\\
		Final Defence&Fall 2012\\
		\bottomrule
		\end{tabular}
	
	
	\section{Annotated Bibliography}
		\bibliography{scbdissertation}
%		\bibliography{exporteditems}
	
	\section{Glossary}
	\printglossaries
	
\begin{comment}
$12^{th}$ century renaissance \url{http://en.wikipedia.org/wiki/Renaissance_of_the_12th_century}
\end{comment}

\appendix
\section{Detailed Statistical Methodology} 
\label{sec:Appendix A}

	Note that these step-by-step details are mainly notes to myself regarding the Copenhagen methodology at a detailed level. I include them here for interested readers, but the proposal should be understandable without this detail for most.

			\begin{enumerate}

				\item Description of the logic of the methodology (the probability approach in VAR econometrics):
				
				 ``The vector autoregressive (VAR) process based on Gaussian (normally distributed) errors has frequently been a popular choice as a description of macroeconomic time-series data.
				
			Theory-based economic models have traditionally been developed as non-stochastic mathematical entities and often applied to empirical data by adding a stochastic error process to the mathematical model.
			
			From an econometric point of view, the two approaches are fundamentally different: one starting  from an explicit stochastic formulation of \textit{all} data and then \textit{reducing} the general statistical (dynamic) model by imposing testable restrictions on the parameters, the other starting from a mathematical (static) formulation of a theoretical model and then \textit{expanding} the model by adding stochastic components.''
			\footnote{Juselius 2006 chapters 1--3} %%Juselius 2006 ch1-3

				\item Investigate the unrestricted VAR
				\begin{enumerate}
					\item Estimate the unrestricted VAR for the information set
					\item Select and form the \gls{ecm} representation
					\item Perform misspecification tests
					\begin{enumerate}
						\item Perform specification checking
						\item Test residual correlations and use information criteria to identify lag lengths
						\item Test residual autocorrelation 
						\item Test residual heteroskedasticity
						\item Perform normality tests
					\end{enumerate}
				\end{enumerate}
				\item Investigate deterministic components in the model. \footnote{Juselius 2006 chapter 6} %% Juselius ch6
				\begin{enumerate}
					\item Identify possible deterministic cases. The reference model is the simple $VAR(1)$ containing a constant, $\boldsymbol{\mu}_0$, and a trend, $\boldsymbol{\mu}_1t$ in $AR$ form:
					\begin{equation}
					\Delta \bold{x}_t = \boldsymbol{\alpha \beta}'\bold{x}_{t-1} + \boldsymbol{\mu}_0 + \boldsymbol{\mu}_1t + \boldsymbol{\epsilon}
					\end{equation}
					\begin{enumerate}
						\item Case 1. $\boldsymbol{\mu}_1$, $\boldsymbol{\mu}_0 = \bold{0}$. This case corresponds to a model with no deterministic components in the data, i.e. $E(\bold{\Delta} \bold{x}_t)=\bold{0}$ and $E(\boldsymbol{\beta}^{'} \bold{x}_t) = \bold{0}$
						\item Case 2. $\boldsymbol{\mu}_1=\bold{0}$, $\boldsymbol{\gamma}_0=\bold{0}$ but $\boldsymbol{\beta}_0 \neq \bold{0}$, where $\boldsymbol{\gamma}_0$ is defined as a term in the decomposition of $\boldsymbol{\mu}_0=\boldsymbol{\alpha \beta}_0 + \boldsymbol{\gamma}_0$.
						\item Case 3. $\boldsymbol{\mu}_1= \bold{0}$, i.e. $(\boldsymbol{\beta}_1, \boldsymbol{\gamma}_1)= \bold{0}$, and the constant term $\boldsymbol{\mu}_0$ is \textit{unrestricted}, i.e. no linear trends in the $VAR$ model, but linear trends in the variables, where $\boldsymbol{\gamma}_1$ is defined as a term in the decomposition of $\boldsymbol{\mu}_1=\boldsymbol{\alpha \beta}_1 + \boldsymbol{\gamma}_1$.
						\item Case 4. $\boldsymbol{\gamma}_1 = \bold{0}$, but $(\boldsymbol{\gamma}_0,\boldsymbol{\beta}_0,\boldsymbol{\beta}_1) \neq \bold{0}$, i.e. the trend is \textit{restricted} only to appear in the conintegrating relations, but the constant is unrestricted in the model.
						\item Case 5. No restrictions on $\boldsymbol{\mu}_0, \boldsymbol{\mu}_1$, i.e. trend and constant are \textit{unrestricted} in the model.
					\end{enumerate}
					\item In $MA$ common trends form to clarify the source of linear trends:
					\begin{equation}
					\bold{x}_t = \boldsymbol{\beta}_\bot(\boldsymbol{\alpha}_\bot ' \boldsymbol{\Gamma} \boldsymbol{\beta}_\bot)^{-1} \boldsymbol{\alpha}_\bot '\left\lbrace \boldsymbol{\gamma}_0 t + \frac{1}{2} \boldsymbol{\gamma}_1 t + \frac{1}{2} \boldsymbol{\gamma}_1 t^2 \right\rbrace + \bold{C}^*(L) \boldsymbol{\mu}_1 t
					 + \bold{C}^*(1) \boldsymbol{\mu}_0 + \bold(C) \displaystyle\sum\limits_{i=1}^t \boldsymbol{\epsilon}_t + \bold{C}^*(L) \boldsymbol{\epsilon}_t + \bold{\tilde{X}}_0
					\end{equation}
					Thus, the linear trends in the variables can originate from three different sources in the $VAR$ model:
					\begin{enumerate}
						\item[1.] the $\boldsymbol{\alpha}$ component $(\bold{C}^*(L) \boldsymbol{\mu}_1 t)$ of the unrestricted linear trend $\boldsymbol{\mu}_1 t$;
						\item[2.] the $\boldsymbol{\beta}_\bot$ component $(\boldsymbol{\gamma}_1 t)$ of the unrestricted linear trend $\boldsymbol{\mu}_1 t$;
						\item[3.] the $\boldsymbol{\beta}_\bot$ component $(\boldsymbol{\gamma}_0 t)$ of the unrestricted constant term $\boldsymbol{\mu}_0$.
					\end{enumerate}
					\item Consider intervention \textit{(dummy)} variables. Similarly as for the trend and constant, consider a simple regression model for $x_t$ containing three different types of dummy variables:
					\begin{equation}
					x_t = \phi_s D_{s,t} + \phi_p D_{p,t} + \phi_{tr} D_{tr,t} + u_t + x_0
					\end{equation}
					where $D_{s,t}$ is a mean-shift dummy $(\dots,0,0,0,1,1,1,\dots)$, $D_{p,t}$ is a permanent intervention dummy $(\dots,0,0,1,0,0,\dots)$, and $D_{tr,t}$ is a transitory shock dummy $(\dots,0,0,1,-1,0,0.\dots)$ and the residual is a first order autoregressive process with form:
					\begin{equation}
					u_t = \frac{\epsilon_t}{1-\rho L}
					\end{equation}
					Which for difference non-stationary series results in a change of form as follows:
					\begin{equation}
					\Delta x_t = \phi_s \Delta D_{p,t} + \phi_p \Delta_{tr,t} + \phi_{tr} \Delta_{dtr,t} +  \epsilon_t,
					\end{equation}
					i.e. a shift in the levels of a variable becomes a `blip' in the difference variable, a permanent `blip' in the levels becomes a transitory blip in the differences, and a transitory blip in the levels becomes a double transitory blip, $D_{dtr,t}$, in the differences.
					
					My plan is to analyse the time series for structural breaks, formulate dummy variable(s) appropriately, then test whether the dummies are significant, and for which equations in the $VAR$.
				\end{enumerate}
				\item Estimate the $I(1)$ model. \footnote{Juselius 2006 chapter 7} %%Juselius ch7
				\begin{enumerate}
					\item Preliminary estimates of the English data indicate that the series, \textit{with time dummies included}, are integrated of order one $I(1)$.  This, of course, is essential in determining the formally correct methodology as $I(2)$ forms require a modified technique.  However, practical considerations, for example lack of $I(2)$ support in current software, dictate that I will use $I(1)$ methodology, BUT will check for signalling of $I(2)$ problems.
					\item Concentrating the general $VAR$ model.
					
					Proceeding under the $I(1)$ assumption, and under the assumption that the empirical $VAR$ can describe the data, we can state the $I(1)$ condition as:
					\begin{equation}
					\boldsymbol{\Pi} = \boldsymbol{\alpha \beta'},
					\end{equation}
					where $\boldsymbol{\alpha}$ and $\boldsymbol{\beta}$ are $p \times r$ matrices.  If $r = p$, then $\bold{x}_t$ is stationary and classical inference applies.  If $r = 0$, then there are $p$ autonomous trends in $\bold{x}_t$ so that each $x_{i,t}$ is non-stationary with its own individual trend.  In this case the vector process is driven by $p$ different stochastic trends and it is not possible to obtain stationary cointegration relations between the levels of the variables.  Also, the variables have no stochastic trends in common and, hence, do not move together over time.
					
					Should this be the case, I will reformulate the $VAR$ model in levels as a $VAR$ model in differences \textit{without any consequential loss of long-run information} and classical inference applies.
					
					If $p > r > 0$, then $\bold{x}_t \sim I(1)$ and there exist $r$ directions into which the process can be made stationary by linear combinations.  These are the cointegrating relations, which often can be interpreted as deviations from economic steady-state relations, and are thus economically meaningful.
					\item Derive the maximum likelihood (ML) estimator of the concentrated model.
					
					Assuming a finding of cointegrating relations, I proceed to use ML to estimate the long-run equilibrium relations from a concentrated model of the form (disregarding for the moment deterministic elements):
					\begin{equation}
						\bold{R}_{0t} = \boldsymbol{\alpha \beta'}\bold{R}_{1t} + \boldsymbol{\epsilon}_t, \quad \boldsymbol{\epsilon}_t \sim N_p(\bold{0},\boldsymbol{\Omega}).
					\end{equation} Note that at this point the estimating task has been separated into long-run and short-run processes, an essential part of the methodology, and allowed under the Frisch-Waugh theorem.
					\item Normalize the results.
					
					This step allows an economic interpretation of a cointegrating relation, and thus one should choose a meaningful normalizing variable. Note, however, that the cointegrating relations are not sensitive to the choice of normalizing variable. Note that this is the so-called ``Johansen" procedure.
					\item Interpret the unrestricted results.
					
					While the unrestricted $\boldsymbol{\hat{\alpha}, \,\hat{\beta}, \text{ and } \hat{\Pi}}$ are not expected to yield directly interpretable results, they may, so I will proceed by inspecting them.
				\end{enumerate}
				\item Determine the cointegration rank. \footnote{Juselius 2006 chapter 8}  %%Juselius ch 8
				
				This is an essential step as it influences subsequent inference and economic interpretation. The actual test statistics are likelihood ratios (LR), and are compared to carefully specified asymptotic distributions to yield critical values.
				
				The cointegrating rank divides the data into $r$ relations towards which the process is adjusting, and $p - r$ relations which are pushing the process. The former are interpreted as (initially) statistical equilibrium errors (deviations from a statistical steady-state), and the latter as common driving trends in the system.
				\item Cointegration hypotheses testing--recursive tests of parameter constancy. \footnote{Juselius 2006 chapter 9}
				
				In this step I will perform four classes of tests for parameter constancy, using both statistical and graphical methods.
				\begin{enumerate}
					\item Recursive tests of the full model, essentially a recursive test of the likelihood function, which will indicate whether the model is approximately acceptable.  This is conceptually similar to recursive Chow tests for single-equation models.
					\item Recursive tests on the eigenvalues, $\lambda_i$ and transformations of them, which will provide more detailed constancy information about individual cointegrating relations.
					\item Recursive tests of the constancy of the cointegration space, $\boldsymbol{\beta' x_t}$.
					\item Recursive tests of predictive failure for both the full system and individual series.
					\item Judging the effects of any found parameter instability.
				\end{enumerate}
				\item Cointegration hypotheses testing--testing restrictions on $\boldsymbol{\beta}$. \footnote{Juselius 2006 chapter 10}
				
				This process will help in spotting potentially relevant long-run relations, and requires several steps:
				\begin{enumerate}
					\item Formulate hypotheses as restrictions on $\boldsymbol{\beta}$.
					\item Testing the same restriction on all cointegration relations.
					\item Testing the stationarity of a known $\boldsymbol{\beta}$ vector.
					\item Testing the stationarity of a cointegration relation when some coefficients are know and others must be estimated.
				\end{enumerate}
				\item Cointegration hypotheses testing--testing restrictions on $\boldsymbol{\alpha}$. \footnote{Juselius 2006 chapter 11}
				
				Tests on $\boldsymbol{\alpha}$ are closely associated with interesting hypotheses about the common driving forces in the system.  
				
				The test of zero row in $\boldsymbol{\alpha}$ is the equivalent of testing whether a variable can be considered weakly exogenous for the long-run parameters $\boldsymbol{\beta}$.  These define a common driving trend as cumulative sums of empirical shocks to variables.
				
				The test of a unit vector in $\boldsymbol{\alpha}$ defines a variable which is exclusively adjusting, and whose shocks have no permanent effect on any of the variables in the system.
				
				Thus, the two types of test can identify the pushing and pulling forces of the system.
				\begin{enumerate}
					\item Testing for long-run weak exogeneity.
					\item Testing for weak exogeneity in partial models.
					\item Testing a known vector in $\boldsymbol{\alpha}$.
					\item Interpret the results on $\boldsymbol{\beta}$ and $\boldsymbol{\alpha}$ in terms of economic scenarios.
				\end{enumerate}
				\item Identification--identify the long-run structure. \footnote{Juselius 2006 chapter 12} %%Juselius 2006  ch 12
				\begin{enumerate}
					\item Formulate identifying hypotheses and degrees of freedom
					\item Consider just-identifying restrictions
					\item Consider over-identifying restrictions
					\item Consider lack of identification
					\item Perform recursive diagnostic tests of $\alpha$ and $\beta$
				\end{enumerate}
				\item Identification--identify the short-run structure. \footnote{Juselius 2006 chapter 13}	%% Juselius 2006 ch 13
				\begin{enumerate}
					\item Formulate identifying restrictions
					\item Interpret shocks
					\item Formulate the short-run economic questions
					\item Consider restrictions on the short-run reduced-form model
					\item Construct \textit{VAR} in triangular form
					\item Formulate empirically identifiable current effects
					\item Construct the preferred structure
				\end{enumerate}
				\item Identification--identify common trends. \footnote{Juselius 2006 chapter 14} %%Juselius 2006 ch 14
				\begin{enumerate}
					\item Formulate the common trends representation
					\item Formulate the unrestricted MA representation
					\item Formulate the MA representation subject to restrictions on $\alpha$ and $\beta$
					\item Impose exclusion restrictions on $\beta_\bot$
					\item Assess the economic model scenario
				\end{enumerate}
				\item Identification--identify a structural MA model. \footnote{Juselius 2006 chapter 15} %% Juselius ch 15
				\begin{enumerate}
					\item Reparameterize the VAR model
					\item Separate between transitory and permanent shocks
					\item Formulate and interpret structural shocks
					\item Test the credibility of the labels on the economic shocks
				\end{enumerate}
			\end{enumerate}
			
\section{Table of variable, coefficient, and function definitions} 
\label{sec:Appendix B}
\centering
%\begin{center}
\begin{longtable}{ll}
\toprule
Variable&Description\\
\toprule
$\bold{x}_i , i=0 \cdots k-1$&m x 1 vector of endogenous variables\\
$m$&number of endogenous variables\\
$k$&number of lags for independent endogenous variables\\
$\Delta \bold{x}_i$&first difference of endogenous variables\\
$\boldsymbol{\mu}_0$&m x 1 vector of intercept coefficients\\
$\boldsymbol{\mu}_1$&m x 1 vector of time-dependent coefficients\\
$t$&sequentially enumerated time period\\
$AR$&auto-regressive specification\\
$E$&probabilistic expectations operator\\
$\boldsymbol{\Pi}(=\boldsymbol{\alpha}\boldsymbol{\beta}')$&m x m matrix of dynamic coefficients relating $\Delta \bold{x}_t$ to past values of $\bold{x}_t$\\
$\boldsymbol{\alpha}$&m x r matrix of adjustment coefficients describing how equilibrium deviations feed back onto $\bold{x}_t$\\
$\boldsymbol{\beta}'$&r x m matrix of deviations from equilibrium\\
$r$&number of cointegrating relationships\\
$\boldsymbol{\gamma}$&a term which allows us to decompose $\boldsymbol{\alpha\beta}'$ into further usable terms\\
$MA$&moving average common trends specification\\
$\boldsymbol{\beta}_\bot$&orthogonal complement of $\boldsymbol{\beta}'$\\
$\boldsymbol{\alpha}'_\bot$&orthogonal complement of $\boldsymbol{\alpha}$\\
$\boldsymbol{\Gamma}$&m x m matrix of coefficients relating $\Delta \bold{x}_t$ to past values of $\Delta \bold{x}_{t-i}$\\
$\bold{C}$&m x m matrix of coefficients in moving average representation of VAR to evaluate deterministic trends in cointegrating relationships\\
$L$&lag operator representing the number of past periods of endogenous variable data to be included\\
$\bold{R}$&concentrated notation of VAR model for notational simplicity\\


\end{longtable}
%\end{center}

\end{document} \section{End}

\section{notes}
\subsection{2013 EHA notes}
Nef, Habakuk\\
Agriculture transition\\

\subsection{Thirsk 05/13}
ISI, government encouraged, sponsored\\
demand?\\

\subsection{committee notes}
Tim -- remove personal references e.g. Marx  %done 10/13/11
Tim -- quantitative measure of unlimited energy % so lewis talked about labour with at zero marginal product. my idea is that, relative to any labor supply, the amount of mineral energy was essentially unlimited for the initial developing economies. another way of thinking of it is the human energy chart. an interesting question is whether pure energy has diminish marginal product, I think not, thus that you have to look at specific sources and their physical requirements to determine the amount of diminishing marginal product, and in any case it is less than human labour.  10/13/11. %done 11/14/11.

Richard -- other metric approaches. %Why, since looking at graphs of gdp and energy consumption per capita presents such a clear relationship, do I consider a more detailed statistical look? I am interested in looking inside the dynamics that are supported by the long period time series. What this examination should tell is whether the the prime dynamic drivers, so the leading-in-time variables, changed places in the short run and the long run in the economies I study. While I purposefully avoid the term causality at this point, depending on the strength of the results, I may find causality in the data. What I want to understand, beyond my hypothesis of centrality of mineral energy consumption as the defining invention of the Industrial Revolution, is what implications this has for modern development and for sustained per capita economic development in an age of potentially emission constrained economies. The time series methodology I propose has the ability to do this. And it has the capability of incorporating important time related events that enter the time series as discontinuties. Both of these capabilities are core to my research. 10/14/11. %done 11/14/11.

11/17
Rudi - 5 major items. mainly tighten. He thinks if I do a 2-3 page abstract of intro section, Lance might be interested.

You've done already a lot of work. On a lot of this I am no expert (neither history nor metrics). So bear with me.
Let me jot down just a few quick notes:


\begin{itemize}
\item 1) Presentation: Put hypothesis first; don't make the reader wait or search. In this draft, and in a later paper, and in your presentation, the juxtaposition of the various explanations should come much earlier. I.e.: On the first page (!!), there should be a paragraph saying that "the industrial revolution in England is commonly explained by (1) cultural exceptionalism, wihch essentially means ... , (2) ... , (3) energy, .... (4) thermodynamics. My hypothesis is that (x) matters more than previously believed, and my dissertation tests the evidence for that. The anecdotal evidence includes the lack of Dutch industrialization ... etc. " 

\item 2) Hypothesis: Sharpen it. How distinct is your hypothesis really from culture/institution-driven explanations? You do argue that all, even cultural exceptionalists, place emphasis on coal. It might be reasonable to say that the explanation must be found in a combination of these factors -- and your contribution is to "shift the weights." Is that about right? It could be fleshed out more, sharpened at the edges.

\item 3) Metrics: Institutional variables. I didn't go through the details. Do you use some proxies for institutional development, or cultural factors? Should that be there, if you want to compare the influence of those variables to energy? 

\item 4) Contributions: Limit yourself. Section 2.1.2 is too broad, and contains too much, I would say. It is not clear what you mean by 5., and, if clarified, it is by itself possibly a whole dissertation. I would drop it for now. 3. and 4. should go together. Arguably, 4. might be post-dissertation research. You want a classic 3 goals (and papers) in here: (1) Literature review, which you've done in part;  (2) Explanation and re-evaluation of the industrial revolution in England in the language of economic history; (3) Explanation and re-evaluation of the industrial revolution in England in the language of econometrics. 

\item 5) The link to growth theory: Rather than 5., you might want to consider "old" growth theories. You make reference to Lewis. So combine Lewis (or Marx and Kalecki, really) with Malthus, and see where that takes you. Lewis warns of the turning point, when surplus labor is exhausted, and real wages must rise. If energy is the real labor, then we might be there. What does that mean for future growth? That's an interesting question, and you can fruitfully feed future research --- but I would try to make these three papers quite focused on what you have in mind now. 

\end{itemize}


thinks I could do england vs holland and nail that. would need dutch energy series.

Tim - wants me to change graphs, support his method, has many notes, likes my lit review.

\begin{itemize}
\item 
\end{itemize}

\begin{itemize}
\item Why time series
\item Various ts approaches
\item univariate arima
\item static
\item ardl
\item var
\item cointegrated var

\end{itemize}

may use the Diebold 98 article for organizing, or Dougherty

\subsection{sequence for bibliography}

pdflatex
bibtex on .bib F11
pdflatex
pdflatex
bibtex on .tex  F11
pdflatex
pdflatex

sequence for glossary

    Build LaTeX document, this will generate the files used by makeglossaries
    Run the indexing/sorting, the recommended way is to use makeglossaries (a script that runs xindy or makeindex depending on options in the document with correct encoding and language settings):

    makeglossaries <myDocument>

    Build LaTeX document again to get document with glossary entries


\begin{comment}
% save just for reference...covered by table and annotated bib
		\subsection{Data Sources}
		I anticipate using a wide variety of sources. Should I just do a bibliography here?
			\subsubsection{English historical-institutional sources}
			\begin{itemize}
			\item Wrigley
			\item van Zanden
			\item for reference, list in Wrigley discussion
			\item Landes
			\item Crafts
			\item Temin
			\item Pomeranz
			\item Allen
			\item Mokyr
			\item Goldstone
			\item McCloskey
			\item de Vries
			\item Jevons
			\item Clark
			\item TS Ashton
			\end{itemize}
			\subsubsection{English long-period time series sources}
			\begin{itemize}
			\item Mitchell
			\item Snooks
			\item Officer
			\item Fouquet
			\item Warde
			\item Maddison
			\end{itemize}
			\subsubsection{U.S. historical-institutional sources}
			\begin{itemize}
			\item Vaclav Smil
			\item Daly and Townsend
			\item Lewis
			\item North
			\end{itemize}
			\subsubsection{U.S. long-period time series sources}
			\begin{itemize}
			\item Historical Statistics of the United States 1780-1945
			\item Historical Statistics of the United States Millennial Edition Online
			\item Historical, Demographic, Economic, and Social Data: The United States, 1790-2002
			\item Gordon Whitney
			\item Kindleberger
			\end{itemize}
			\subsubsection{Chinese historical-institutional sources}
			\begin{itemize}
			\item Xu, Dixin, and Zhengming Wu, eds. 2000. Chinese Capitalism, 1522-1840. New York: St. Martin's Press.
			\item Hsien-Chun Wang. 2009. ``Discovering Steam Power in China, 1840s-1860s." Technology and Culture 51(1): 31-54.
			\item ``Pomeranz, K.: The Great Divergence: China, Europe, and the Making of the Modern World Economy."
			\item Andre Gunder Frank
			\item Janet Abu-Lughod
			\item Jack Goldstone
			\end{itemize}
			\subsubsection{Chinese long-period time series sources}(May not be available)
			\begin{itemize}
			\item Sinton, Jonathan E. 2001. ``Accuracy and reliability of China's energy statistics." China Economic Review 12(4): 373-383.
			\item Chinese Statistical Yearbook
			\end{itemize}
			\subsubsection{World long-period time series sources}
			\begin{itemize}
			\item Payne - a literature review of recent energy-gdp empirical studies
			\item US DOE EIA
			\item OECD
			\item UN
			\item Katarina Juselius. 2010. On the Role of Theory and Evidence in Macroeconomics. University of Copenhagen. Department of Economics.
			\item Epic of Gilgamesh. The tablets VII and  XI story of the flood. BC 2700. As a hook on which to illustrate the length of time between the Neolithic and Industrial revolutions. One of very earliest histories. So we know about half the 10K years, and there is essentially nothing economically that happened.
			\item Michael Kremer, 1993. Population Growth and Technological Change: One Million B.C. to 1990.
			\item Johansen et al. 2000 Cointegration analysis in the presence of structural breaks
			\item Foley 1996 Statistical Equilibrium Models in Economics
			\end{itemize}
\end{comment}			

\begin{comment}
%leave this out as the glossary is linked		
		\subsection{Definition of Terms}
		\begin{itemize}
		\item \gls{organic}
		\item \gls{mineral}
		\item \gls{neorev}
		\item \gls{insolation}
		\item \gls{areal}
		\item \gls{punctiform}
		\item \gls{indrev}
		\item \gls{early}
		\item \gls{high}
		\item \gls{earmodern}
		\item \gls{modern}
		\item \gls{energy}
		\item \gls{growth}
		\end{itemize}
\end{comment}		

in the full writeup, I need to highlight the ironic effloressence description of jack goldstone.

data sources -- maddison chinese stats 1998 oecd

data sources -- mitchell on american stats

\subsection{Methodology notes}

Cleaning time series data\\
http://www.r-bloggers.com/cleaning-time-series-and-other-data-streams/
\\
package::pracma, method::outlierMAD. Instances of 'moving window data cleaning'\\

\fi
