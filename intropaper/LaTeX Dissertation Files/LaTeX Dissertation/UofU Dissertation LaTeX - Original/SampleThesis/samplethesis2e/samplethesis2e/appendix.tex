\chapter{Classical identities}\label{appendix}
\fixchapterheading
\section*{Rellich's identity}\label{rellich.section}
\setcounter{thrm}{0}
%
%

Standard developments of Pohozaev's identity used an identity by
Rellich~\cite{rellich:der40}, reproduced here.

\begin{lem}[Rellich]
Given $L$ in divergence form and $a,d$ defined above, $u\in C^2
(\Omega )$, we have
\begin{equation}\label{rellich}
\int_{\Omega}(-Lu)\nabla u\cdot (x-\overline{x})\, dx
= (1-\frac{n}{2}) \int_{\Omega} a(\nabla u,\nabla u) \, dx
-
\frac{1}{2} \int_{\Omega}
d(\nabla u, \nabla u) \, dx
\end{equation}
$$
+
\frac{1}{2} \int_{\partial\Omega} a(\nabla u,\nabla u)(x-\overline{x})
\cdot \nu  \, dS
-
\int_{\partial\Omega}
a(\nabla u,\nu )\nabla u\cdot (x-\overline{x}) \, dS.
$$
\end{lem}
{\bf Proof:}\\
There is no loss in generality to take $\overline{x} = 0$. First
rewrite $L$:
$$Lu = \frac{1}{2}\left[ \sum_{i}\sum_{j}
\frac{\partial}{\partial x_i}
\left( a_{ij} \frac{\partial u}{\partial x_j} \right) +
\sum_{i}\sum_{j}
\frac{\partial}{\partial x_i}
\left( a_{ij} \frac{\partial u}{\partial x_j} \right)
\right]$$
Switching the order of summation on the second term and relabeling
subscripts, $j \rightarrow i$ and $i \rightarrow j$, then using the fact
that $a_{ij}(x)$ is a symmetric matrix,
gives the symmetric form needed to derive Rellich's identity.
\begin{equation}
Lu = \frac{1}{2} \sum_{i,j}\left[
\frac{\partial}{\partial x_i}
\left( a_{ij} \frac{\partial u}{\partial x_j} \right) +
\frac{\partial}{\partial x_j}
\left( a_{ij} \frac{\partial u}{\partial x_i} \right)
\right].
\end{equation}

Multiplying $-Lu$ by $\frac{\partial u}{\partial x_k} x_k$ and integrating
over $\Omega$, yields
$$\int_{\Omega}(-Lu)\frac{\partial u}{\partial x_k} x_k \, dx=
-\frac{1}{2} \int_{\Omega}
\sum_{i,j}\left[
\frac{\partial}{\partial x_i}
\left( a_{ij} \frac{\partial u}{\partial x_j} \right) +
\frac{\partial}{\partial x_j}
\left( a_{ij} \frac{\partial u}{\partial x_i} \right)
\right]
\frac{\partial u}{\partial x_k} x_k \, dx$$
Integrating by parts (for integral theorems see~\cite[p. 20]
{zeidler:nfa88IIa})
gives
$$= \frac{1}{2} \int_{\Omega}
\sum_{i,j} a_{ij} \left[
\frac{\partial u}{\partial x_j}
\frac{\partial^2 u}{\partial x_k\partial x_i} +
\frac{\partial u}{\partial x_i}
\frac{\partial^2 u}{\partial x_k\partial x_j}
\right] x_k \, dx
$$
$$
+
\frac{1}{2} \int_{\Omega}
\sum_{i,j} a_{ij} \left[
\frac{\partial u}{\partial x_j} \delta_{ik} +
\frac{\partial u}{\partial x_i} \delta_{jk}
\right] \frac{\partial u}{\partial x_k} \, dx
$$
$$- \frac{1}{2} \int_{\partial\Omega}
\sum_{i,j} a_{ij} \left[
\frac{\partial u}{\partial x_j} \nu_{i} +
\frac{\partial u}{\partial x_i} \nu_{j}
\right] \frac{\partial u}{\partial x_k} x_k \, dx
$$
= $I_1 + I_2 + I_3$, where the unit normal vector is $\nu$.
One may rewrite $I_1$ as
$$I_1 = \frac{1}{2} \int_{\Omega}
\sum_{i,j} a_{ij} \frac{\partial}{\partial x_k}\left(
\frac{\partial u}{\partial x_i}
\frac{\partial u}{\partial x_j}
\right) x_k \, dx
$$
Integrating the first term by parts again yields
$$I_1 = -\frac{1}{2} \int_{\Omega}
\sum_{i,j} a_{ij} \left(
\frac{\partial u}{\partial x_i}
\frac{\partial u}{\partial x_j}
\right) \, dx
+
\frac{1}{2} \int_{\partial\Omega}
\sum_{i,j} a_{ij} \left(
\frac{\partial u}{\partial x_i}
\frac{\partial u}{\partial x_j}
\right) x_k \nu_k \, dS
$$
$$
-
\frac{1}{2} \int_{\Omega}
\sum_{i,j} \left(
\frac{\partial u}{\partial x_i}
\frac{\partial u}{\partial x_j}
\right) x_k \frac{\partial a_{ij}}{\partial x_k}\, dx.
$$
Summing over $k$ gives
$$\int_{\Omega}(-Lu)(\nabla u\cdot x)\, dx =
-\frac{n}{2} \int_{\Omega}
\sum_{i,j} a_{ij} \left(
\frac{\partial u}{\partial x_i}
\frac{\partial u}{\partial x_j}
\right) \, dx
$$
$$
+
\frac{1}{2} \int_{\partial\Omega}
\sum_{i,j} a_{ij} \left(
\frac{\partial u}{\partial x_i}
\frac{\partial u}{\partial x_j}
\right) (x\cdot \nu ) \, dS
-
\frac{1}{2} \int_{\Omega}
\sum_{i,j} \left(
\frac{\partial u}{\partial x_i}
\frac{\partial u}{\partial x_j}
\right) (x\cdot  \nabla a_{ij}) \, dx
$$
$$
+
\frac{1}{2} \int_{\Omega}
\sum_{i,j,k} a_{ij} \left[
\frac{\partial u}{\partial x_j}
\frac{\partial u}{\partial x_k} \delta_{ik} +
\frac{\partial u}{\partial x_i}
\frac{\partial u}{\partial x_k} \delta_{jk}
\right] \, dx
$$
$$- \frac{1}{2} \int_{\partial\Omega}
\sum_{i,j} a_{ij} \left[
\frac{\partial u}{\partial x_j} \nu_{i} +
\frac{\partial u}{\partial x_i} \nu_{j}
\right] (\nabla u\cdot x) \, dS.
$$
Combining the first and fourth term on the right-hand side
simplifies the expression
$$\int_{\Omega}(-Lu)(\nabla u\cdot x)\, dx
=
(1-\frac{n}{2}) \int_{\Omega}
\sum_{i,j} a_{ij} \left(
\frac{\partial u}{\partial x_i}
\frac{\partial u}{\partial x_j}
\right) \, dx
$$
$$
+
\frac{1}{2} \int_{\partial\Omega}
\sum_{i,j} a_{ij} \left(
\frac{\partial u}{\partial x_i}
\frac{\partial u}{\partial x_j}
\right) (x\cdot \nu ) \, dS
-
\frac{1}{2} \int_{\Omega}
\sum_{i,j} \left(
\frac{\partial u}{\partial x_i}
\frac{\partial u}{\partial x_j}
\right) (x\cdot  \nabla a_{ij}) \, dx
$$
$$
-
\frac{1}{2} \int_{\partial\Omega}
\sum_{i,j} a_{ij} \left[
\frac{\partial u}{\partial x_j} \nu_{i} +
\frac{\partial u}{\partial x_i} \nu_{j}
\right] (\nabla u\cdot x) \, dS.
$$
Using the notation defined above, the result follows.


%
%
%
\section*{Fortran code}\label{code}
%
%

%% The following was constructed by a very handy program called
%% tgrind.  tgrind is a filter to convert C or fortran files into
%% formatted tex.  Starting with a fortran subroutine rhs.f:
%%
%% tgrind -lf -f >rhs.tex rhs.f
%%
%% This creates the tex file rhs.tex.  This may be directly included
%% in a latex document via the special command \tgrind (latex command
%% here):
%%
%% \begin{singlespace}
%% \begin{small}
%% \tgrind{rhs.tex}
%% \end{small}
%% \end{singlespace}
%%
%% Otherwise, the file rhs.tex needs to be edited (rhs_mod.tex)
%% to be included into a latex document (it is a stand-alone
%% tex file).  The line with the \File command (top) needs to be
%% removed or commented out: \File{rhs.f},{14:32},{Jul  5 1993} and
%% the \end command at the bottom also needs to be commented out.  The
%% file rhs.tex can then be included into the document:
%%
%% \begin{singlespace}
%% \begin{small}
%% \input tgrindmac
%\File{rhs.f},{14:32},{Jul  5 1993}
\L{\LB{}}
\L{\LB{      \K{subroutine} rhs(neq,v,rhsf)}}
\L{\LB{      \K{save}}}
\L{\LB{c}}
\L{\LB{c This \K{subroutine} computes the \K{function} values. Inputs are neq and }}
\L{\LB{c v, and on output the values of f are stored in the array of rhsf}}
\L{\LB{c}}
\L{\LB{      \K{include} \S{}\'parabolic.inc\'\SE{}}}
\L{\LB{}}
\L{\LB{      \K{integer} neq}}
\L{\LB{      \K{integer} i}}
\L{\LB{      \K{integer} j}}
\L{\LB{      \K{integer} k}}
\L{\LB{      \K{integer} ind}}
\L{\LB{      \K{integer} inde}}
\L{\LB{      \K{integer} indw}}
\L{\LB{      \K{integer} indn}}
\L{\LB{      \K{integer} inds}}
\L{\LB{      \K{integer} ind0}}
\L{\LB{      \K{integer} ind1}}
\L{\LB{      \K{integer} ind2}}
\L{\LB{}}
\L{\LB{      \K{double} \K{precision} v(neq)}}
\L{\LB{      \K{double} \K{precision} rhsf(neq)}}
\L{\LB{      \K{double} \K{precision} u(nv)}}
\L{\LB{      \K{double} \K{precision} diff}}
\L{\LB{      \K{double} \K{precision} diffn}}
\L{\LB{      \K{double} \K{precision} diffxn}}
\L{\LB{      \K{double} \K{precision} diffyn}}
\L{\LB{      \K{double} \K{precision} nl}}
\L{\LB{}}
\L{\LB{c      \K{write}(*,*)\S{}\'funct begin\'\SE{}}}
\L{\LB{}}
\L{\LB{c}}
\L{\LB{c     Compute F for the local dynamics, written as  F(u)= \-du\/dt + f(u)}}
\L{\LB{c     }}
\L{\LB{c}}
\L{\LB{c the system parameters}}
\L{\LB{c}}
\L{\LB{c      p1              ! \K{parameter} F}}
\L{\LB{c      p2              ! \K{parameter} k}}
\L{\LB{}}
\L{\LB{      \K{do} j = 1, ny }}
\L{\LB{         \K{do} i = 1, nx}}
\L{\LB{c}}
\L{\LB{c set up index}}
\L{\LB{c}}
\L{\LB{            ind = (i\-1)*nv + (j\-1)*meq}}
\L{\LB{c}}
\L{\LB{c Extract the jth component at current time}}
\L{\LB{c}}
\L{\LB{            nl = v(1+ind)*v(2+ind)*v(2+ind)}}
\L{\LB{}}
\L{\LB{            rhsf(1+ind) =  (\- nl + p1*(1.0d0 \- v(1+ind)))*local}}
\L{\LB{            rhsf(2+ind) =  (  nl \- (p1+p2)*v(2+ind))*local}}
\L{\LB{}}
\L{\LB{         \K{end} \K{do}}}
\L{\LB{      \K{end} \K{do}}}
\L{\LB{}}
\L{\LB{c \-\-\-\-\-\-\-\-\-\-\-\-\-\-\-\-\-\-\-\-\-\-\-\-\-\-\-\-\-\-\-\-\-\-\-\-\-\-\-\-}}
\L{\LB{c}}
\L{\LB{c     add diffusion for all species (zero diffusion }}
\L{\LB{c     coefficient takes care of those that \K{do} not diffuse). }}
\L{\LB{c }}
\L{\LB{c
\-\-\-\-\-\-\-\-\-\-\-\-\-\-\-\-\-\-\-\-\-\-\-\-\-\-\-\-\-\-\-\-\-\-\-\-\-\-\-
}}
\L{\LB{}}
\L{\LB{      \K{do} j = 1, ny}}
\L{\LB{         \K{do} i = 1, nx}}
\L{\LB{}}
\L{\LB{c indexing}}
\L{\LB{c}}
\L{\LB{            ind0 = (i\-1)*nv + (j\-1)*meq   ! point}}
\L{\LB{            indw = (i\-2)*nv + (j\-1)*meq   ! west point}}
\L{\LB{            inde = (i)*nv + (j\-1)*meq     ! east point}}
\L{\LB{            indn = (i\-1)*nv + (j)*meq     ! north point}}
\L{\LB{            inds = (i\-1)*nv + (j\-2)*meq   ! south point}}
\L{\LB{}}
\L{\LB{            \K{if}(i.eq.1) indw = (nx\-1)*nv + (j\-1)*meq}}
\L{\LB{            \K{if}(i.eq.nx) inde = (j\-1)*meq }}
\L{\LB{            \K{if}(j.eq.1) inds = (i\-1)*nv + (ny\-1)*meq}}
\L{\LB{            \K{if}(j.eq.ny) indn = (i\-1)*nv}}
\L{\LB{}}
\L{\LB{            \K{do} k = 1, 2}}
\L{\LB{}}
\L{\LB{c}}
\L{\LB{c First compute the contribution within a row at the current time}}
\L{\LB{c and at the preceding time. }}
\L{\LB{c}}
\L{\LB{               ind = k + ind0}}
\L{\LB{               ind1 = k + indw}}
\L{\LB{               ind2 = k + inde}}
\L{\LB{}}
\L{\LB{               diffxn = v(ind1) \- 2.0d0*v(ind) + v(ind2)}}
\L{\LB{}}
\L{\LB{c}}
\L{\LB{c Compute the contribution from the columns}}
\L{\LB{c}}
\L{\LB{               ind1 = k + indn}}
\L{\LB{               ind2 = k + inds}}
\L{\LB{}}
\L{\LB{               diffyn = v(ind1) \- 2.0d0*v(ind) + v(ind2)}}
\L{\LB{}}
\L{\LB{c}}
\L{\LB{c Multiply by  other factors and sum}}
\L{\LB{c}}
\L{\LB{               diff = d(k)*hxx*(diffxn + diffyn)*diffus}}
\L{\LB{}}
\L{\LB{               rhsf(ind) = rhsf(ind) + diff}}
\L{\LB{               }}
\L{\LB{}}
\L{\LB{            \K{end} \K{do}}}
\L{\LB{         \K{end} \K{do}}}
\L{\LB{      \K{end} \K{do}}}
\L{\LB{}}
\L{\LB{       }}
\L{\LB{      \K{return}}}
\L{\LB{      \K{end}}}
\L{\LB{}}
\L{\LB{}}
\L{\LB{}}
\vfill\eject
%\end

%% \end{small}
%% \end{singlespace}
%%
%% In either case, the \File command probably will need removing
%% because it places the page numbers differently than the normal
%% thesis style. Using both gives:

 \begin{singlespace}
 \begin{small}
% \let\end\relax \def\File#1,#2,#3{}
 \tgrind{rhs_mod.tex}
 \end{small}
 \end{singlespace}
