\chapter{Quadratic nonlinearities}\label{quad}
%
% Don't use \fixchapterheading here. Chapter is followed by a
% paragraph, not a heading.
%
In this chapter we derive results for the quadratic equation.

\section{Derivation of the quadratic formula}
A quadratic equation is one of the form
\begin{equation}\label{quadratic}
ax^2 + bx + c = 0
\end{equation}
where $a,b,c$ are known constants and $x$ is the unknown.
The results are summarized in Table \ref{pde.tab1} and Table
\ref{pde.tab2} below.

%
%
\section{Application of the quadratic formula}
%
%

If the differential operator generates a nonnegative form, then an
inequality is based on the following considerations. See
Figure \ref{gelfand.fig1} for $n=1,2$,
Figure \ref{gelfand.fig2} for $3\leq n \leq 9$
and
Figure \ref{gelfand.fig3} for $n\geq 10$.

% Example of a table:
% Table caption can be selected as paragraph style or centered style
% (for an inverted pyramid title). Use \oldstylecaptiontrue (paragraph)
% or \oldstylecaptionfalse (centered) to select the style.
%
\begin{table}[b]
\centering
\caption{\label{timing1} PDE solve times, $15^3+1$
equations.\label{pde.tab1}}
\plusline
\begin{tabular}{||l|l|l|l|l|l||}\hline
Precond. & Time & Nonlinear & Krylov
& Function & Precond. \\
 & & Iterations & Iterations & calls & solves \\ \hline
None & 1260.9u & 3 & 26 & 30 & 0  \\
 &(21:09) & & & &  \\ \hline
FFT  & 983.4u & 2  & 5  & 8  & 7 \\
&(16:31) & & & & \\ \hline
MILU & 629.7u & 3  & 11 & 15 & 14 \\
& (10:36) & & & & \\ \hline
\end{tabular}
\end{table}
\clearpage

\begin{table}[t]
\caption{Convergence properties of RQI.\label{pde.tab2}}
\centering
\plusline\small
\begin{tabular}{l|lll} \hline
Object & Normal Matrices & Diagonalizable Matrices
& Defective Matrices \\ \hline
$\rho$ & Stationary at ev's. &
Stationary at ev's. &
Stationary at ev's. \\
$\| r_k\|$ & $\to 0$ as $k\to\infty$. &
Can oscillate. &
Can oscillate. \\
$\rho_k$ & Converges. &
Unknown. &
Unknown. \\
Convergence to & is cubic. & is quadratic. & is linear. \\
eigensets & & & \\
\hline
\end{tabular}\normalsize
\end{table}

% Figures in LaTeX will go on the bottom of the same page, or the top of
% the next page, but never before the first reference. All figures must
% be referenced. The syntax is below. See uuguide for control.
%
%       \begin{figure}[x]    % x = b, t, h, p
%       ...
%       \caption{My title.}  % Captions are below the figure!
%       \end{figure}
%
%
% These graphs were created by gnuplot. For simple graphs this is a
% great utility.  For example, if you want a sin curve in your thesis
% try the following:
%
% (terminal window): gnuplot
% (in gnuplot):
%                 set terminal latex
%                 set output "foo.tex"
%                 plot sin(x)
%                 quit
%
\begin{figure}[b]       % Place it on the bottom of page
\centering              % Put \label{} into \caption.
\inputpicture{fig1.tex}
\caption{Gelfand equation on the ball, $n=1,2$.
\label{gelfand.fig1}}    % Use \ref{gelfand.fig1} for references
\end{figure}


\begin{figure}[p]       % Likely it will go on the top of the page
\centering
\inputpicture{fig2.tex}
\caption{Gelfand equation on the ball, $3\leq n \leq 9$.
\label{gelfand.fig2}}    % If not, then change [t] to [p]
\end{figure}

\begin{figure}[p] % Likely it will go on the top of the next page
                  % If not, then change [h] to [p]
\centering
\inputpicture{fig3.tex}
\caption{Gelfand equation on the ball, $n\geq 10$.
\label{gelfand.fig3}}
\end{figure}
\clearpage % dump figures where they below
