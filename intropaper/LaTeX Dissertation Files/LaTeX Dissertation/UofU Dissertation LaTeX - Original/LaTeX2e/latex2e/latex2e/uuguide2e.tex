% Updated Nov-1993 by GBG. Version 1.8 of the guide.
%================================================================
% LaTeX file
% revised May 1988 by Grant B. Gustafson for April 1988 uuthesis style
% revised Feb 1990 by Grant B. Gustafson for Feb 1990 uuthesis style
% revised Feb 1992 by Grant B. Gustafson. References. Addresses.
% Revised Apr 1992 by Grant B. Gustafson. \qed, \proof, \ulabel.
% Revised Nov 1992 by Grant B. Gustafson. \chapter, \titlepage, etc.
% 581-6879, 113 JWB, Math Dept.
% Revised Nov 1993 by Grant B. Gustafson. Major revision.
% Revised Feb 1994 by Grant B. Gustafson.
% Revised Dec 1994 by Grant B. Gustafson.
% Revised Jan 1995 by Grant B. Gustafson.
% Revised Apr 1996 by Grant B. Gustafson.
% Revised Apr 1998 by Grant B. Gustafson.
% See account at end of document.
%

\documentstyle{article}
\setlength{\oddsidemargin}{0in}
\setlength{\evensidemargin}{0in}
\setlength{\textwidth}{6.4in}
\setlength{\parskip}{5pt}
\setlength{\parindent}{0pt}
\date{1 April 1998}

\title{Using \LaTeX{} for University of Utah Theses \\
       {\large \tt uuthesis2e.cls Version 1.8d, April 1998}}
\author{Stan Shebs
\footnote{Updated by Grant Gustafson, 1988-1998.}
}

\begin{document}

\maketitle

\section{Introduction}
This document\footnote{Updated by Grant Gustafson, 1988-1998.}
tells how to format a University of Utah thesis or
dissertation using the computer typesetting system \LaTeX. Formatting
standards are
under the control of a document class called \verb|uuthesis|.

This guide is made to be used in conjunction with
{\bf A Handbook for Theses and Dissertations}, which is
published by the graduate school of the University of Utah. It is
recommended that you obtain a recent copy of the above handbook during a
personal visit to the thesis office. Make this first visit {\em before}
starting on a thesis or dissertation. The number of contacts with the
thesis office averages five: one initial visit, one after chapter 1 is
finished, some question and answer sessions followed by a reading before
the thesis defense and a final reading after all revisions.

The thesis office, like professional journals in science and engineering,
has their own set of publication standards. If you want to publish your
work, then you have to meet the standards. In the case of the thesis
office, the standards and requirements are printed in the {\em
Handbook}.

For mathematics theses,
the editors request that you and your thesis advisor decide on a style
guide. This does not mean that you have settled upon {\em A Manual for
Authors of Mathematical Papers} published by the American Mathematical
Society. The latter is {\em not a style guide}. You are expected to
choose {\em The ACS Style Guide} or {\em The Chicago Manual of Style}.
Hybrids adopted from standard journals may also be acceptable.

Presently, only the {\tt Chicago style} exists. Nevertheless, this option
must be manually selected.

For computer science theses, the same question will be asked: will it be
the Chicago Manual or the ACS Guide? More manuals of style are listed in
the references of the {\em Handbook}.

The basic problem for everyone who writes a thesis or dissertation is
the arrangement of topics and results into parts and chapters, sections
and subsections, appendices, tables and figures. The {\em rule book} for
this process is the graduate school's {\em Handbook}.
The {\tt uuthesis} style was created to ease the job of meeting the
requirements. It is part of the solution to the problem, but not the
entire solution, and it is certainly not the {\em Handbook}.

Some knowledge of \TeX\ and \LaTeX{} is assumed. If you haven't used
either before, then format a simpler and shorter document before
tackling a thesis. One document that will be useful is the test file in
item 4 below.

After a review session with \LaTeX{} features, begin a serious study of
the sample thesis prepared by Jeff McGough, Grant Gustafson and Nelson
Beebe, 1993. See item 5 below.

Beginners with \LaTeX{} should obtain the following documents and
sources:

\begin{enumerate}
\item Leslie Lamport, {\em \LaTeX{} User's Guide and Reference Manual},
Addison-Wesley Publishing Company, Reading, Massachusetts (1986). ISBN
0-201-15790X (\LaTeX{} version 2.09 released 19 April 1986).

\item Grant B. Gustafson, {\em \LaTeX{} Article Style}, Department of
Mathematics, University of Utah, SLC, UT 84112 (801--581--6851), 1988,
1992, 50 pages. Copy from {\tt /u/ma/gustafson/tex/art.tex}, or request
via email:
{\tt gustafson@math.utah.edu}

\item The University of Utah thesis style files:
\begin{quote}
UUGUIDE.TEX    --- source for this guide. \\
UUTHESIS.STY   --- Thesis style file version 1.8
\end{quote}
These are installed and maintained on computer systems in Computer
Science and The Department of Mathematics. Previous versions use sources
{\tt uut11.sty} and {\tt uut12.sty}. These last sources are not used now
since they are included in {\tt uuthesis.sty}.

\item Small sample thesis test file available on mathematics department
unix machines as the single source

\verb"  /usr/local/lib/tex/uuthesis/latex2e/test2e.tex"

The source uses {\tt uuthesis2e.cls} and {\tt Chicago.clo}. Copy these sources
also, if it is not
available at your computer site. This source is small but contains
figures and tables, bibliography, table of contents, list of figures,
list of tables, all the front matter, acknowledgements, abstract, and
uses all the sectioning commands. It is useful as an initial test file
when setting up your thesis and also as a debugging source when you have
problems. Use it to understand figure placement and basic format of a
thesis. This source can be obtained by email. Send your request to Beebe
or Gustafson at the address listed at the end of this guide.

\item Sample thesis file available on mathematics department unix
machines as sources in root file

\verb"  /usr/local/lib/tex/uuthesis/samplethesis/thesis.tex"

There are many parts to this sample thesis. Copy the entire directory
and examine it closely to understand how a thesis is managed. If you
cannot access this directory, then request email copies or copies on
diskette. Send your request to Beebe or Gustafson at the address listed
at the end of this guide.

\item Sample Computer Science and Mathematics theses, by Stan Shebs
(CS), David Eyre (Math), Paul Joyce (Math) and Jeff McGough (Math).
These documents are available through the University Library and serve
as specific examples of how to organize a thesis. The sources to these
may also be available; contact the people listed at the end of this
guide.

\item Documentation for {\tt xfig}, {\tt fig2dev}, {\tt gnuplot}, {\tt
maple}. These unix programs are invaluable for making figures and
drawings in \LaTeX{} using {\tt latex} or {\tt pictex} macros. Also
learn about {\tt graph} and {\tt x79} for manipulation of plots, and
{\tt pltde} and {\tt phase} for solving and plotting differential
equations.

\end{enumerate}

Some additional references are available, related to thesis production,
although not in so direct a manner:

\begin{enumerate}
\item Nelson H.\ F.\ Beebe, {\em A Bibliography of Publications about
\TeX{}}, 1993. {\footnotesize Available by anonymous FTP from site {\tt
science.utah.edu} in directory {\tt pub/tex/bib}, files {\tt
texbook1.bib} and {texbook1.ltx}} .

\item David J.\ Buerger, {\em {\LaTeX{}} for Engineers and Scientists},
McGraw-Hill, New York (1990), ISBN 0-07-008845-4.\footnote{ Some basic
ideas about spacing control in this book make poor advice for a thesis.
It is a useful reference that is perhaps easier to read than Lamport's
{\em Manual}, since it is not as energetic and does not qualify at all
as a reference manual.}

\item Jane Hahn, {\em {\LaTeX{}} for Everyone}, PTI, 1991.\footnote{
Recommended for beginners in \LaTeX{}.}

\item Michael D. Spivak, {\em The Joy of {\TeX} --- A Gourmet Guide to
Typesetting with the {AmsTeX} macro package}, AMS, 2nd revised edition,
1990, ISBN 0-8218-2997-1.

\end{enumerate}

\section{The Document Style}

The \verb|uuthesis| document style is largely adapted from the
\verb|report| style with {\em doublespace} option. \verb|Uuthesis|
shares many of the characteristics of the \verb|report| style. For
instance, double-sided formatting is available (for convenience), but
double columns are not. The 12-point font happens by default.
An 11-point is available by specifying the \verb|11pt| option, which can
cut many pages off a long thesis. No other type sizes are available.
Remarks hereafter are about the 12pt option. The main text of the thesis
will be double-spaced automatically, which results in {\tt baselineskip}
of about 21pt (72.27pt per inch). If you need to insert single-spaced
text, then use the {\tt singlespace} environment, which results in a
baselineskip of 13.5pt. Certain pre-defined options in the {\tt
uuthesis} format already use the {\tt singlespace} environment and
therefore only wizards should change the basic baselineskip dimensions.

In cases where the Thesis Manual is not explicit, the style follows
the {\em Chicago Manual}. This is because only the class {\tt Chicago.clo}
is currently supported.

\subsection{Document Style Options}

There are several options that may be supplied:
\begin{description}

\item[11pt] Set the text in 11pt type.
This size uses 5pt fonts for {\tt tiny}, which is less than the required
2mm font height for University Microfilms. Consequently, do not use {\tt
tiny} with the 11pt option (this can happen in math mode).

\item[12pt] Set the text in 12pt type (this is the default).

\item[twoside] Format the pages for two-sided printing.

\item[draft] Relax the requirements on box overflow. Do {\em not\/} use
this for what the thesis editor will see!

\item[report] Format things appropriately for a technical report. It
sets the default to single spacing, omits the two signature pages, and
generates the title page from the contents of the definition
\verb|reporttitlepage|. This control sequence is defined in the format
to be a warning message. It is expected that users will redefine the
control sequence for their own purpose. Otherwise, it produces a
harmless but useless title page.

\item[honors] Sets the title page generation from the contents of the
control sequence \verb|HONORSTITLE|, which is pre-defined in the format,
using the following definitions:
\begin{quote}
{\tt honorsdepartment} --- e.g., English \\
{\tt honorsadvisor} --- e.g., Charles D.\ Smith \\
{\tt honorssupervisor} --- e.g., Alta V.\ Wilcox \\
{\tt honorsdirector} --- e.g., Richard Cummings
\end{quote}
The thesis approval pages are not generated.

\item[csreport] Sets the title page generation from the contents of the
control sequence \verb|CSREPORTTITLE|, which is pre-defined in the
uuthesis format to satisfy the Computer Science Department, for
production of technical reports. The approval pages are not generated.
CS technical reports are single-spaced, which can be accomplished by
using in addition the \verb|report| option above.

\item[Chicago] Sets the default style to the Chicago manual of style.
No other manual options currently exist.
\end{description}


\section{What the Style Defines}

This section lists all the control sequences defined or modified by the
style. Later sections will describe these in more detail.

\noindent The following standard \LaTeX{} environments are modified by
this style:

\begin{quote}
\verb|description| environment \\
\verb|equation| environment \\
\verb|figure| environment \\
\verb|quotation| environment \\
\verb|quote| environment \\
\verb|verse| environment \\
\verb|table| environment \\
\verb|thebibliography| environment
\end{quote}

\noindent Although these environments are generally similar to those in
other \LaTeX{} style, it is prudent to experiment with the thesis style
before assuming anything about the output.
\bigskip

\noindent In addition, there are some declarations for new environments,
some of which are in common use in mathematics papers. The details may
be found in a later section.

\begin{quote}
\verb|Proof|  environment\\
\verb|singlespace|  environment\\
\verb|doublespace|  environment\\
\verb|normalspace|  environment\\
\verb|index| environment \\
\verb|epigraph|  environment \\
\verb|topics|  environment
\end{quote}

The following optional environments are enabled by the control string
\verb"\theoremsetup" in the preamble. They are undefined if the control
string does not appear in the preamble.

\begin{quote}
\verb|theorem|  environment\\
\verb|proposition|  environment\\
\verb|corollary|  environment
\end{quote}

\noindent
Many new control sequences are defined, and a number of standard ones
modified. None of those listed here should be modified unless you know
what you are doing and are willing to experiment. Arguments to the
control sequences below are indicated as \verb|#1|, \verb|#2|, and so
forth. These arguments are usually ordinary text, but in a few special
cases there are extreme limitations on content.

\noindent Modified standard \LaTeX{} commands:

\begin{quote}
\verb|\appendix| \\
\verb|\author{#1}| \\
\verb|\part{#1}| \\
\verb|\chapter{#1}| ~ {\bf Warning}: Read about this one!\\
\verb|\section{#1}| \\
\verb|\subsection{#1}| \\
\verb|\subsubsection{#1}| \\
\verb|\subsubsubsection{#1}| \\
\verb|\paragraph{#1}| \\
\verb|\index| \\
\verb|\subitem| \\
\verb|\subsubitem| \\
\verb|\titlepage| \\
\verb|\thebibliography{#1}| \\
\verb|\listoffigures| \\
\verb|\listoftables| \\
\verb|\tableofcontents|
\end{quote}

\noindent New commands (the arguments are never optional):

\begin{quote}
\verb|\abstracttitlepage| \\
\verb|\chairtitle{#1}| \\
\verb|\captionONfalse| \\
\verb|\captionONtrue| \\
\verb|\captionlineskip| \\
\verb|\committeeapproval| \\
\verb|\committeechair{#1}| \\
\verb|\copyrightpage| \\
\verb|\copyrightyear{#1}| \\
\verb|\dedicationpage| \\
\verb|\dedication{#1}| \\
\verb|\degree{#1}| \\
\verb|\departmentchair{#1}| \\
\verb|\department{#1}| \\
\verb|\firstreader{#1}| \\
\verb|\fivelevels| \\
\verb|\fixchapterheading| \\
\verb|\fourlevels| \\
\verb|\fourthreader{#1}| \\
\verb|\frontmatter{#1}{#2}{#3}| \\
\verb|\frontmatterformat| \\
\verb|\graduatedean{#1}| \\
\verb|\honorsadvisor{#1}| \\
\verb|\honorsdepartment{#1}| \\
\verb|\honorsdirector{#1}| \\
\verb|\honorssupervisor{#1}| \\
\verb|\listoffiguresfalse| \\
\verb|\listoffigures| \\
\verb|\listoftablesfalse| \\
\verb|\listoftables| \\
\verb|\mainheader{#1}| \\
\verb|\mainheadingwidth| \\
\verb|\maintext| \\
\verb|\minusfourthline| \\
\verb|\minushalfline| \\
\verb|\minusline| \\
\verb|\noisyfalse| \\
\verb|\noisytrue| \\
\verb|\normalspace| \\
\verb|\optionalfront{#1}{#2}| \\
\verb|\plusfourthline| \\
\verb|\plushalfline| \\
\verb|\plusline| \\
\verb|\prefacesection{#1}| \\
\verb|\preface{#1}{#2}| \\
\verb|\pf|\\
\verb|\proof{#1}|\\
\verb|\proofline{#1}|\\
\verb|\qed| \\
\verb|\rawbibliographytrue| \\
\verb|\rawbibliographyfalse| \\
\verb|\readingapproval| \\
\verb|\reportitlepage| \\
\verb|\secondreader{#1}| \\
\verb|\singlespace| \\
\verb|\submitdate{#1}| \\
\verb|\tableofcontents| \\
\verb|\thesistype{#1}| \\
\verb|\thirdreader{#1}| \\
\verb|\threelevels| \\
\verb|\titlepage| \\
\verb|\twopagefigure{#1}{#2}| \\
\verb|\ulabel{#1}{#2}|\\
\verb|\vita|
\end{quote}

\section{Front Matter}

\noindent The Front Matter consists of the extra pages at the front of
the thesis, such as the title page and reading approval pages. In
particular, the chapters of the thesis are specifically excluded from
the front matter. In the initial stages of writing, the front matter is
unimportant, and you can just start with the \verb|\maintext| command
described in the next section.

Almost all of the front matter pages can be generated automatically,
including the signature pages. A number of declarations must be supplied
in the preamble, so that the appropriate places get filled in with the
right information (all declarations have one argument \#1, which is
ordinary text). Declarations must appear in the preamble, that is, {\em
after} \verb|\documentstyle{uuthesis}| and {\em before} the control
\verb|\begin{document}|.

\begin{description}

\item \verb|\thesistype{type}|
is the type of thesis. {\em Type\/} is either \verb|thesis| for Master's
theses, or \verb|dissertation| for doctorates. The word goes through
directly, so be sure of the spelling!

\item \verb|\title{text}|
declares the full title of your thesis. If it is too long to fit on one
line, then designate line breaks (using \verb|\\|) at appropriate
locations.

\item \verb|\author{names}| declares your full legal name (as the Thesis
Manual says). Multiple authors will have to use explicit line breaks.

\item \verb|\degree{name}| is the name of the degree. Sometimes the
standard phrases are insufficient, such as when getting a specific
degree from an omnibus department.

\item \verb|\department{name}| is the full name of the department, for
example, {\em Department of Mathematics}.

\item \verb|\departmentchair{name}| is the name of the department
chairperson.

\item \verb|\submitdate{month year}| is the date of submission, which
must be the end of a quarter, as specified in the manual.

\item \verb|\copyrightyear{year}| is the year that should appear in the
copyright notice.

\item \verb|\committeechair{name}| is the name of your advisor or
committee chairperson. Multiple names are unexpected and may not be
processed correctly.

\item \verb|\chairtitle{name}| is the title of your advisor or
committee chairperson, usually {\em professor}.

\item \verb|\firstreader{name}|
\item \verb|\secondreader{name}|
\item \verb|\thirdreader{name}|
\item \verb|\fourthreader{name}|
declares the names of committee members. The committee approval page
will end up with the appropriate number of lines for signatures.
The command \verb|fifthreader| emits an error message and defaults back
to four readers.

\item \verb|\graduatedean{name}| is the name of the graduate dean (B.
Gale Dick, as of this writing).

\item \verb|\dedication{words}| declares a dedication.
No special formatting, but you can use \verb|\\| to get multiple lines
at least.  A multi-paragraph dedication is unlikely to work.
The dedication page will be omitted if this declaration is not supplied.

\item \verb|\honorsadvisor{name}| is the name of the Honors
advisor in the candidate's home department. Used only for Honors theses.

\item \verb|\honorsdepartment{name}| is the name of the department,
e.g., English. Used only for Honors theses.

\item \verb|\honorsdirector{name}| is the name of the director of the
Honors Program, e.g., Richard Cummings. Only for Honors theses.

\item \verb|\honorssupervisor{name}| is the name of the person who
supervised the Honors thesis. Used only for Honors theses.

\end{description}

You have two ways to format the front matter; the easy way and the hard way.
The easy way is automatic, somewhat restrictive, but sufficient for 99\%
of all theses.  The hard way gives more control, but takes more work to
get right.

\subsection{Automatic Front Matter}

The ``one command to do it all'' is

\hspace{4em} \verb|\frontmatter{abstract}{acknowledge}{acktitle}|

\noindent
where \verb|abstract| is the name of the \LaTeX{} source file
containing the abstract's text,
\verb|acknowledge| is the name of the \LaTeX{} source file for the
preface, and \verb|acktitle| is the heading that will be used for the
preface (``Preface'' and ``Acknowledgments'' are the two most common
choices).  The \verb|\frontmatter|
command will generate all pages up to the beginning of the text proper.
The abstract may not contain citations. Further, the University of Utah
requires the total length of the abstract to be less than 350 words.

The lists of figures and tables will appear by default. To disable them,
use the commands below:
\begin{quote}
\verb|\listoffiguresfalse|\\
\verb|\listoftablesfalse|
\end{quote}
These commands eliminate the lists and any possible reference to the
lists that might have appeared in the table of contents.

Lists are followed by the acknowledgments or preface, which are at the
end of the front matter.  The title is one of the arguments to the
\verb|\frontmatter| command. The title will always be capitalized.

\subsection{Manual Front Matter}

This method requires that you specify each front matter page
individually. It is the most versatile. For example, it allows your
thesis to include things like a glossary, if it needs one. The
declarations are to be typed in the order below, just {\em after}
\verb|\begin{document}|:

\begin{description}

\item \verb|\frontmatterformat|
This is required in order to set up spacing and page numbering for the
front matter. Note that the page numbers are Roman in the front matter
and arabic in the main thesis. If you get a strangely numbered table of
contents, then suspect this control sequence was somehow omitted.

\item \verb|\titlepage|
Uses text set up by preamble commands {\em title, author, thesistype,
department, degree}. Creates a separate
title page with empty page style. Automatically ejects the page.

\item \verb|\copyrightpage|
Uses preamble commands {\em author, copy\-right\-year}.
Creates a special copy\-right page required by the Thesis Office.
Automatically ejects the page.

\item \verb|\committeeapproval|
Uses text set up by preamble commands. The commands are {\em
thesistype}, {\em author}, {\em committeechair}, {\em firstreader}, {\em
secondreader}, {\em thirdreader}, {\em fourthreader}.
Produces a {\bf Supervisory Committee Approval} form
including signature lines. This form certifies that each member of the
committee has read the thesis and that the thesis has been approved by
majority vote. Signatures on this form are absolute: refusal
to sign at the thesis defense is not unknown, because the form
can be used as a weapon to procure manuscript revisions.

\item \verb|\readingapproval|
Uses text set up by preamble commands {\em thesistype, author,
committeechair, departmentchair, graduatedean}.
Produces a {\bf Final Reading Approval} form
including signature lines. This form is addressed to the Graduate
Council and certifies that the thesis has been read and approved.
Signatures are generally postponed until all revisions have been made
and the committee approval form is completed.

\item \verb|\prefacesection{Abstract}|
Creates a separate abstract. The abstract in this case is set up with
the command \verb|\prefacesection{Abstract}|, then your abstract text
follows normally.  Figures, references, and such are not allowed.
Don't use this one {\em and} the next one: use only one or the other.

\item \verb|\preface{Abstract}{filename}| Creates the abstract, getting
the contents from the file \verb|filename|.
This is useful if the abstract file is to be used by something else
at the same time.

\item \verb|\dedicationpage|
Uses the text set up by the preamble command \verb|\dedication|
to create a dedication page.

\item \verb|\tableofcontents|
Uses text set up by sectioning commands. Part, Chapter and Section
commands cause automatic entries in the \verb|toc| file. This
file is made by a pass through \LaTeX{}. It will take two passes to get
it right (with no intervening aborts). On each pass, the last table of
contents is inserted into the output, and a revised table of contents is
written onto the \verb|toc| file.

The table of contents might need extra blank lines and additional text.
The command which does this is called
\begin{quote}
\verb|\addcontentsline{#1}{#2}{#3}|
\end{quote}
For this purpose, \#1 always equals \verb|toc|. The allowed values of
\#2 are: {\em part, chapter, section, subsection, groupheader}. The {\em
appendix} uses {\em chapter}, so {\bf appendix} is not an allowed keyword.

Use \#2 equal to {\em groupheader} to insert text without page
references and dot leaders. The value of \#3 is any text you want to put
on the line. For
example,

\verb|\addcontentsline{toc}{groupheader}{\protect\newline}|

adds a blank line into the TOC (vertical space).
Another useful idea:

\verb|  \addcontentsline{toc}{groupheader}{\vspace{-1pc}}|

which subtracts 12pts of vertical white space from the \verb|toc| file.
A macro {\tt oneline} written for this purpose is:

\verb|  \def\oneline{\addcontentsline{toc}{groupheader}{\vspace{-1pc}}}|

See page 159 of the \LaTeX{} manual.

\item \verb|\listoftables|
Uses text set up by table environment captions. The list is
single-spaced with double-spacing between entries. Table titles are
supposed to agree with the first sentence used in the caption,
regardless of the complexity of the table caption. The formatting of TOC
entry and title may be different. The \LaTeX{} manual suggests 100
characters maximum for the list of tables entry and 300 characters
maximum for the table title. See the section below on figure and table
placement for details about captions.

An entry in the list of tables is created by the \verb"\caption"
control, provided the control \verb"\captionONtrue" is set (the
default). The setting of \verb"\captionONfalse" caused the
\verb"\caption" control to abort incrementing any counters and no entry
is made in the list of tables.


{\bf Figure and table captions}.
Please do not confuse figure captions and table captions. They are
different. A figure caption might be a narrative, but a table caption is
usually restricted to one or two lines, in general. Anything to be said
about the table and its data is either in the text or else is a
paragraph {\em below} the table. The SIAM journals tend to have table
captions in paragraphs.  So, it is {\em allowed} to follow a SIAM
paragraph style for tables. How you set the style of tables depends a
lot upon the journal where you hope to publish. See journals in your
specialty and the standard style guides for examples.

\item \verb|\listoffigures|
Uses text set up by figure environments.
List is single-spaced with double-spacing
between entries. The best titles are one line long or less. Follow the
recommendations above for table titles and list of tables entries.
In particular, read above about the {\tt caption} control and the two
boolean controls {\tt captionONtrue} and {\tt captionONfalse}.

\item \verb|\preface{Acknowledgments}{filename}|
Generates a preface page, using the text written in the file called
\verb|filename|. The contents of this \LaTeX{} source file is up to the
author. A {\em prefacesection} command can do the same job, if the
text is included directly.

\item  \verb|\optionalfront{name}|
Additional sections may be added using this command.  The \verb|name| is
the name of the section, such as ``Glossary'', ``List of Symbols'', or
``List of Abbreviations''. This command just sets up the page heading
and adds an entry into the table of contents.  The format of the section
is up to you, and it is advisable first to follow the general format of
the other front matter sections, and then to check with the thesis
editor before trying to get Format Approval.

\end{description}

\section{Vertical space controls}

A common problem in a thesis is to control the vertical white space in a
way that can be easily undone. It is a mistake to use the common
controls of \LaTeX{}, because they cannot be turned off without
affecting the internal workings of the typesetter.

Below are control sequences that can be inserted into the document to
add and subtract vertical white space. If the document is later used to
produce a journal article, then the controls can be disabled by a few
lines of definitions in the preamble. For example, a relevant control
\verb"\foo" can be easily turned off with \verb"\def\foo{}". Not so with
\verb"\vspace{...}" commands, because they are used internally also, and
cannot safely be turned off.


\begin{description}
\item \verb|\minusfourthline|
Subtracts vertical space equal to one fourth of a blank line.
The
definition:
\begin{quote}
\verb|\def\minushalfline{\vspace{-0.25\normalbaselineskip}}|
\end{quote}

\item \verb|\minushalfline|
Subtracts vertical space equal to one half of a blank line. The
definition:
\begin{quote}
\verb|\def\minushalfline{\vspace{-0.5\normalbaselineskip}}|
\end{quote}

\item \verb|\minusline|
Subtracts vertical space equal to one blank line. The definition:
\begin{quote}
\verb|\def\minusline{\vspace{-\normalbaselineskip}}|
\end{quote}

\item \verb|\plusfourthline|
Adds vertical space equal to one fourth of a blank line. The definition:
\begin{quote}
\verb|\def\plushalfline{\vspace{0.25\normalbaselineskip}}|
\end{quote}

\item \verb|\plushalfline|
Adds vertical space equal to one half of a blank line. The definition:
\begin{quote}
\verb|\def\plushalfline{\vspace{0.5\normalbaselineskip}}|
\end{quote}

\item \verb|\plusline|
Adds vertical space equal to one blank line. The definition:
\begin{quote}
\verb|\def\plusline{\vspace{\normalbaselineskip}}|
\end{quote}

\item \verb|\captionONfalse|
This boolean causes the {\tt caption} macro to turn off incrementing of
the table and figure counters and it causes each use of {\tt caption} to
abort producing an entry in the list of figures or the list of tables.
Generally used before a continuation table.

\item \verb|\captionONtrue|
The default.
This boolean causes the {\tt caption} macro to turn on incrementing of
the table and figure counters and it causes each use of {\tt caption} to
produce an entry in the list of figures or the list of tables, as
appropriate.

\item \verb|\captionlineskip|
The spacing of lines in table and figure captions is controlled
by
the dimension definition
\begin{quote}
\verb|\def\captionlineskip{13.5pt}|
\end{quote}
\noindent
which is the standard for single spacing.
To change the spacing, specify the dimension, for example
\begin{quote}
\verb|\def\captionlineskip{14.5pt}|
\end{quote}
\noindent
could be used to control spacing in a caption that contains many
uppercase keywords.
The control can be set and reset many times, once for
each caption, if needed. Be aware that consistency is required in
captions: any major change is expected to be uniform throughout the
thesis.

\item \verb|\mainheadingwidth|
The horizontal width in a heading for {\tt part}, {\tt chapter}, {\tt
section} commands is controlled by the dimension definition
\begin{quote}
\verb|\def\mainheadingwidth{4.25in}|
\end{quote}
\noindent
To change the spacing increase or decrease the value 4.25, for example:
\begin{quote}
\verb|\def\mainheadingwidth{4.5in}|
\end{quote}
\noindent
The control can be set and reset many times, once for each main heading,
if needed. This may happen if the title contains a large math symbol or
the breaking of a title into inverted pyramids fails to work properly.

This dimension is also used for headings in the front matter. It is
suggested that you change the dimension only after \verb"\maintext" in
the \LaTeX{} source. After this point, it can be reset at will.


\item \verb|\EMX|
Controls displayed math formulas that have depth, forcing them to have
the same vertical space before and after the display, as is used on
other displays. The acronym ``EMX'' comes from EXtra Math space. It can also
be used on inline formulas to keep them from encroaching upon nearby
lines. Usage:
\begin{quote}
        \verb|$$\EMX math-formula $$| \\
        \verb|$$\EMX math-formula \eqno{(1.1)}$$| \\
        \verb|$$\EMX math-formula \leqno{(1.2)}$$| \\
        \verb|\begin{equation}\EMX math-formula \end{equation}| \\
        \verb|\begin{eqnarray}\EMX math-formula \\| \\
                        \verb|\EMX math-formula \\ ...| \\
        \verb|\end{eqnarray}|
\end{quote}
Jose Burillo used \verb|\EMX| throughout his thesis. To turn off this
effect, after the thesis is all full of \verb|\EMX|'s, make the
following definition at any point before the disablement is desired.
\begin{quote}
\verb|\def\EMX{}|
\end{quote}
The intent of the macro is to invoke it once for each time the action
is to be taken. However, if executed outside a math environment, then
the action taken is global, for all following math environments.
The definition of \verb|\EMX|:
\begin{verbatim}
\def\EMX{%
\ifx\@optionONE\@ptsize
  \abovedisplayskip 24pt plus 3pt minus 7pt
\else
  \abovedisplayskip 22pt plus 2pt minus 5pt
\fi
\belowdisplayskip \abovedisplayskip
\abovedisplayshortskip  \abovedisplayskip
\belowdisplayshortskip  \belowdisplayskip
}
\end{verbatim}
Micrometer adjustment of the display spacing after a short
line can be done by replacing \verb|\abovedisplayskip| in the above
definition by a percentage, e.g., \verb|0.9\abovedisplayskip|.

{\bf Fixing errors in sectional titles}.
Long titles for all sectional commands except {\tt part}, {\tt chapter}
and {\tt paragraph} produce an error message if the title exceeds the
width limit:

\narrower{\tt
Title Error: (62.04767pt too wide)\\
Page 1, Title="Second section: consequences of the first section"
}

This can be
turned off by \verb"\noisyfalse". Titles are
checked for dimension \verb"\mainheadingwidth" less half an inch for the
section number. The maximum title size is 4.5in including the numbering.
A little more space (4.8 or 4.9) can be used for a particular title in
order to make it look good.

Part and Chapter titles are set in a box of width defined by the control
\verb"\mainheadingwidth". Error messages are automatically emitted for
exceeding the box width. The setting of \verb"\noisyfalse" does not
affect these error messages.

Long title problems can be fixed individually. To produce 5pt
more vertical space on the second line of a long title, the line break
can be coded \verb"\protect\\[5pt]". All line breaks should be coded
\verb"\protect\\" and not simply as \verb"\\". It is good practice to
define a control  \verb"\def\BR{\protect\\}" and then use \verb"\BR" in
all titles to insert a break.

{\bf Fixing errors in the table of contents}.
Table of contents entries sometimes have to be supplied with breaks and
separate entries for the TOC and the title in the text. Normally, the
TOC is set with entries up to 4.5in wide, with breaks supplied by
\LaTeX{}. For example:

\narrower{
\verb"\chapter[The Longest Day of the War]" \\
\verb"        {The Longest Day \protect\\ of the War}"
}

The errors in title lengths for table of content entries are emitted by
\LaTeX{} during the input of the {\tt toc} file. These are {\em serious
messages} which direct you to fix the {\tt toc} entry that originates
from a sectioning command. For example, the \LaTeX{} error message

\narrower{
\verb"[5] (thesis.toc"\\
\verb"Overfull \hbox (6.62102pt too wide) in paragraph at lines 16--16"\\
\verb" \twlrm sec-tion, first sub-sub-sub-sec-tion, first para-graph:"\\
\verb")"
}

means that some sectioning command title had a bad break in the TOC.
Furthermore, the bad break was on the word {\tt paragraph}.
Insert a manual break to get it right. Find the title in the source and
do something like this (\verb"\BR" is defined by
\verb"\def\BR{\protect\\}"):

\narrower{
\verb"\paragraph[Third section, third subsection, first subsubsection,"\\
\verb"first subsubsubsection, first \BR paragraph: must be a sentence.]"\\
\verb"{Third section, third subsection, first subsubsection, first"\\
\verb"subsubsubsection, first paragraph: must be a sentence.}"
}

\item \verb|\doublespacedheadings|
The thesis format requires every other line to be blank throughout the
thesis. Chapter titles are included, they must have every other line
blank. Titles of sections, subsections and subsubsections must have a
single style.
The {\em first style}
is every other line blank (double-spaced titles). It is an option that
can be specifically selected with the above control sequence.

\item \verb|\singlespacedheadings|
The {\em second style} is text on every line and no
interleaved blank lines (single-spaced titles). This is used only for
the {\tt subsection}, {\tt subsubsection}, {\tt subsubsubsection}
commands. The others remain the same as in the {\tt
doublespacedheadings} format. This option is the default as a result of
consultation with the thesis office (1993).

\item \verb|\fixchapterheading|
The space after a main heading is too large if a chapter command
is followed by a section command with no text between. Use
\verb"\fixchapterheading" to remove the extra space. Place this command
after the {\tt chapter} command and before the {\tt section} command.
This command applies also to an appendix section that directly follows
an appendix heading (chapter followed by section in an appendix).
The default definitions for single and double spaced headings:

\verb|    \def\fixchapterheading{\vspace{-24pt}}|\\
\verb|    \def\fixchapterheading{\vspace{-19pt}}|

\end{description}

\section{Main Text}

The chapters are to follow the above special page-generating commands,
but to set the counters correctly, the command
\begin{quote}
    \verb|\maintext|
\end{quote}
must precede the \verb|\include{...}| or \verb|\input{...}| commands for
the chapters.

The spacing after a chapter command is considered to be a critical
parameter by the thesis editor. A thesis should be coded in \LaTeX{} as
follows:

\begin{verbatim}
\chapter{Preliminaries}
\fixchapterheading              % Used when no text appears
\section{Introduction}
\end{verbatim}

There is no way to predict the contents of the first line of text
following a chapter command. It could be a sectional command or it could
be a paragraph of text.

If you want a draft  without the front matter pages, then comment out
the front matter command or commands, and leave just the command
\verb|\maintext| instead.  The draft
will begin unceremoniously with the first page of the first chapter.  It will
still include all the back matter. If your thesis has a {\tt part}
command, then it should follow {\tt maintext} so that the page numbering
begins correctly (PART I is on page 1).

Each chapter or part should be in its own file, and included into the thesis
using the \LaTeX{} command \verb|\include|.  The command \verb|\input|
may also be used, but it is more suitable for direct textual inclusion,
while \verb|\include| has some additional capabilities that are useful
while working on drafts.
In particular, the \verb|\includeonly| command is very useful to restrict
processing to selected \verb|\include| files---important when the whole
thesis may take 30 minutes to format!
A disadvantage of the \verb|\include{...}| command is that each use
clears the page. If the chapters are broken due to length problems, then
extra white space may appear (so break on section or chapter
headings only).

Part and chapter headings are defined by the thesis format and never
change, but the exact format of section headings depends
on how many levels of subheading you use.
Since it would be pretty hard to decide by looking
directly at the text, you will need a command \verb|\threelevels|,
\verb|\fourlevels| or \verb"\fivelevels" in the preamble (before the
\verb|begin{document}|).

\section{Font sizes and styles for sectional commands}

Section headings can be {\tt part}, {\tt chapter}, {\tt section}, {\tt
subsection}, {\tt subsubsection}, {\tt subsubsubsection} or {\tt
paragraph}.

The {\tt part} and {\tt chapter} headings are the largest fonts used.
The format of these headings is: maximum 4.5 inches wide, centered,
every other line blank. The default size is 4.25in, changed with the
control \verb"\mainheadingwidth".

The test for blank line height is made by placing two duplicate pages
together in registration back lit by a strong light source. The two pages
together must make a full page of text without collision of characters
from adjacent lines. This test implies that blank line height depends
upon the font height of adjacent lines, generally measured by
uppercase {\tt M}. This format is called {\em double-spaced}. Normal
\LaTeX{} spacing is called {\em single-spaced}.

The sectional commands {\tt part} and {\tt chapter} are always produce a
double-spaced heading.

The commands {\tt section}, {\tt subsection}, {\tt subsubsection} and
{\tt subsubsubsection} may be all single-spaced or all double-spaced
(but not mixed). See the command \verb"\doublespacedheadings" to select
the non-default style.

The {\tt section} command produce a number which is part of the first
centered title line. Additional lines are also centered. Specific line
breaks must be supplied by the author to achieve inverted pyramid
format.

In {\tt threelevels},
the {\tt subsection} commands produces a number
in its own box and the title is set in the selected font in a paragraph
of width 4.25 inches. It is up to the author to set specific line breaks
so that the title is left-justified and forms an inverted pyramid on the
right (maximum width is 4.5in).

In {\tt fourlevels}, the {\tt subsection} is centered like {\tt
section}. The {\tt subsubsection} appears as outlined above for the {\tt
subsection} control in {\tt threelevels}.

In {\tt fivelevels}, the {\tt section} and {\tt subsection} are
centered.
The {\tt subsubsection} and{\tt subsubsubsection} are essentially the
same, with {\tt subsubsection} being the same as {\tt threelevels}.

In {\tt threelevels}, {\tt fourlevels} and {\tt fivelevels}, the command
{\tt paragraph} produces a boldface sentence at the start of a
paragraph. The text should always end in a period and form a sentence. A
number is generated at the beginning of the line. The paragraph always
appears in double-spaced text. Generally the first sentence should be
less than 4.5in long (it will appear in the TOC).

A sectional command must be followed by two lines of text. That means a
sectional command cannot end a page and clearly it should not split on a
page boundary. There is no uuthesis intervention on placement, it is up
to the author. The placement can be forced by writing \verb"\vbox{...}"
around the sectional command and the paragraph that follows.

The following scheme is used for assignment of font sizes and styles for
double-spaced titles in sectional commands.

\begin{center}
\def\sm{\footnotesize}
\begin{tabular}{@{}p{1.2in}p{1.55in}p{3.2in}@{}}
Command & Font size and style  &  Details \\
\hline
\\
\tt part         &\sm \verb"\HFpart\bf"  & uppercase title, inverted
                                          pyramid, doublespaced.\par
                                          Use whole page, centered
                                          horizontally and vertically.\\
\tt chapter      &\sm  \verb"\HFchapter\bf"  & Centered uppercase title in
                                          4.25in box, double-spaced,
                                          inverted pyramid. Box can be
                                          up to 4.5in wide. Chapters are
                                          expected to start a page. \\
\tt section     &\sm \verb"\HFsection\bf"   & Centered, dual-case title,
                                          doublespaced.\par Inverted
                                          pyramid up to
                                          4.5in wide. \\
\tt subsection &\sm \verb"\HFsubsection\bf" &Dual-case title,
                                             doublespaced, inverted
                                             pyramid up to 4.5in wide.
                                             For {\tt fourlevels} and
                                             {\tt fivelevels} the text
                                             is centered. For {\tt
                                             threelevels} it is
                                             left-justified.
                                             \\
\tt subsubsection&\sm\verb"\HFsubsubsection\bf" & Left justified, dual-case
                                             title. The subsubsection
                                             number uses 7 characters
                                             and the title is a
                                             doublespaced paragraph
                                             left-justified 10
                                             characters from the edge.
                                             The number is separated
                                             from the title by 2 spaces.
                                             One blank line follows the
                                             title. \\
\tt subsubsubsection&\sm\verb"\HFsubsubsubsection\bf" & Left justified,
                                             dual-case title. The
                                             subsubsubsection
                                             number uses 9 characters
                                             and the title is a
                                             doublespaced paragraph
                                             left-justified 13
                                             characters from the edge.
                                             The number is separated
                                             from the title by 2 spaces.
                                             One blank line follows the
                                             title. \\
\tt paragraph   &\sm\verb"\HFparagraph\bf" & Standard paragraph indent,
                                             dual-case title. The
                                             paragraph number uses 9
                                             characters. After 2 spaces
                                             the title is run into the
                                             doublespaced paragraph.
                                             Without the paragraph
                                             number this amounts to
                                             bold-facing the title at the
                                             start of a paragraph.
                                             \\
\\
\hline
\end{tabular}
\end{center}

The 12pt font sizes are listed below. All titles are set in bold-face
style. The sizes and styles can be changed with special permission from
the thesis office.

\begin{tabular}{ll}
\verb"\HFmainhead"           & Uppercase \verb"\large" \\
\verb"\HFpartHT"             & Uppercase \verb"\large" \\
\verb"\HFchapterHT"          & Uppercase \verb"\large" \\
\verb"\HFsectionHT"          & Mixed case \verb"\large" \\
\verb"\HFsubsectionHT"       & Mixed case \verb"\normalsize" \\
\verb"\HFsubsubsectionHT"    & Mixed case \verb"\normalsize" \\
\verb"\HFsubsubsubsectionHT" & Mixed case \verb"\normalsize" \\
\verb"\HFparagraphHT"        & Mixed case \verb"\normalsize" \\
\end{tabular}

The baseline skips for double-spaced headings are invoked by the control
\verb"\doublespacedheadings":

\begin{quote}
\verb"\def\HFmainheadHT{24pt}" \\
\verb"\def\HFpartHT{24pt}" \\
\verb"\def\HFchapterHT{24pt}" \\
\verb"\def\HFsectionHT{20pt}" \\
\verb"\def\HFsubsectionHT{20pt}" \\
\verb"\def\HFsubsubsectionHT{20pt}" \\
\verb"\def\HFsubsubsubsectionHT{20pt}" \\
\verb"\def\HFparagraphHT{14.5pt}"
\end{quote}

For 12pt single-spaced headings, the current {\em default}, the
following changes are made by the control sequence
\verb"\singlespacedheadings":

\begin{quote}
\verb"\def\HFsectionHT{15.5pt}" \\
\verb"\def\HFsubsectionHT{13.5pt}" \\
\verb"\def\HFsubsubsectionHT{13.5pt}" \\
\verb"\def\HFsubsubsubsectionHT{13.5pt}"
\end{quote}

A typical sectional font definition is as follows:

\verb"\def\HFchapter{\@setsize\HFchapter{\HFchapterHT}\xivpt\@xivpt}"

This command defines {\tt HFchapter} as a heading font size using 14pt
pre-loaded \LaTeX{} fonts. For example, \verb"\HTchapter\bf" is used by
the \verb"\chapter" command to switch fonts for a chapter heading.

If you want something other than the two defaults supplied in uuthesis,
then it is up to you to learn about the sectioning commands and the
above font definitions. Install changes in your own private file called
{\tt thesis.sty} by copying the appropriate code from uuthesis.sty and
entering the modifications. The file {\tt thesis.sty} is read {\em
after} the file {\tt uuthesis.sty} and hence new definitions replace the
standard definitions. Simple changes like \verb"\bf" to \verb"\rm"
should be easy. For font size changes, be prepared to show sample output
and be able to justify why your thesis should be different.


\section{Back Matter}

The back matter of a thesis consists of the pages that follow the
chapters. These pages consist of one or more appendices,
a bibliography, perhaps an index and vita.

{\bf Appendices}.
The control \verb|\appendix| applies in the usual way, as documented in
the \LaTeX{} manual. To get the formatting right for the number of
appendices, set the variable \verb|\numberofappendices=n|, where
\verb|n| can range from 0 up to any value. The style of the table of
contents and the chapter headings will be different for 1 appendix than
for 2 appendices.

Appendices can have sections and subsections.
These will not appear in the table of contents.
\begin{quote}
\begin{verbatim}
\numberofappendices=2           % "Appendices" appears in TOC
\appendix                       % Switch into appendix mode
\chapter{Fortran code}          % appendix "A.1 Fortran code" in TOC
\fixchapterheading              % Fix spacing
\section{Common variables}      % Not written to TOC.
\section{Subroutine init}       % Not in TOC
\chapter{Numerics}              % appendix "A.2 Numerics" in TOC
\end{verbatim}
\end{quote}
The lower order sectional commands like {\tt section} will not generate
entries in the TOC. Larger subheadings in the appendix that need no TOC
entry can use \verb"\chapter*{...}".

Presently there can be only one set of appendices by the rules of the
graduate school. To make the controls complete, there is
\verb"\noappendix" to shut off the appendix mechanism and start chapters
again where the numbering left off. The control records how to shut off
the appendix mechanism, in case that documentation is required. This
could become necessary for a thesis that has several parts, because the
appendices in that case have a slightly different structure.

{\bf Bibliographies}. There are at least two ways to do bibliographies
--- manually, or using Bib\TeX{}. The latter is a far superior
alternative, especially if you intend to use the same references in
several places (such as the papers you're going to extract from the
thesis). Significant bibliographies may already exist for the references
you want.  For instance, the PASS group in Computer Science has a file
with about 1000 bib entries, mostly in programming languages. Another
advantage of Bib\TeX\ is that it will format the entries according to a
predefined style declared by the command \verb|\bibliographystyle|.

To date, there is no universal mathematics database. Bib\TeX{} use by
mathematicians has been enthusiastic. Most theses in mathematics now do
Bib\TeX{} bibliographies. Most researchers have access to large
databases of bibliographic data for their specialty, largely compiled by
individual effort.

Many bibliography styles have been defined.  Computer Science people
should use the \verb|acm| style, which follows the ACM journal style.
There is no universal style file for mathematics theses. A number of
mathematics theses have used the {\tt plain} and {\tt SIAM} styles.
Several Computer Science theses have successfully used the ACM style
file. The ACM style uses lowercase article titles. Proper names are
protected by using braces in the title.

\verb|\rawbibliographytrue| \\
This control defines the bibliography environment suitable for placing
references after a section title, as is done in the Chemistry
Department. The section used is the last of a chapter and each chapter
collects its own set of references, appearing in a numbered section,
which also appears in the Table of Contents. Place this control in the
preamble of the \LaTeX{} document to enable the feature. When enabled,
the {\tt bibliography} environment only creates a list of references, no
headers or table of contents entries are generated (the section heading
does this already).

\verb|\rawbibliographyfalse| \\
This control defines the bibliography environment suitable for placing
references at the end of the thesis, as is done in mathematics. The
Table of Contents has the word ``REFERENCES''. The references appear on
a new page with 2-inch main heading. This is the default. The control
can be used many times within the thesis to control the output of a page
of references.


{\bf Index}. The thesis format allows for an index.  Its format should
be like that in the {\em Handbook} from the thesis office, which is
basically the same as the standard \LaTeX{} behavior.

An index is a separate \LaTeX{} source file which produces a typical
index with page references, not different from an index found in a
textbook.
The \LaTeX{} source for the index might reside in a file
\verb|thesis.ind|. The index is inserted into the thesis root file
by the
following code (the end of the root file):

\verb|    %% ====================================================================%%
%% thesis.tex - sample root file for UofU thesis or dissertation       %%
%% ====================================================================%%
%% This contains portions of a thesis which met the graduate school    %%
%% standards with only a few modifications.                            %%
%% ====================================================================%%
%%% NOTE: Writing a thesis implicitly involves meeting standards of the%%
%%% publisher, in this case, the Graduate School of the University of  %%
%%% Utah. The thesis office distributes "A Handbook for Theses and     %%
%%% Dissertations" which is the publication standard of the graduate   %%
%%% school (the AMS has a similar document for AMS publications).      %%
%%% Uuthesis can help you meet the standard, but uuthesis does not set %%
%%% any standards -- standards are set by the graduate school.         %%
%% ====================================================================%%
%% The command ``latex thesis'' will build the document.  To produce a %%
%% final version you may also type ``make thesis''.                    %%
%% ====================================================================%%
%%
\documentclass[11pt,Chicago]{uuthesis2e}
\usepackage{amssymb}
\usepackage{thesis}
\usepackage{diagram}
\usepackage{tgrind}
\let \tenrm = \rm 		% This is used in fig*.tex
%%
%% For drafts of one or more chapters, uncomment the relevant line
%% by removing the percent %:
%%
%\includeonly{}                    % Only front matter and back matter
%\includeonly{chap1}               %  plus chapter 1
%\includeonly{chap1,chap2}         %  plus chapter 1 and 2
%\includeonly{chap1,chap2,chap3}   %  plus all chapters
\includeonly{chap1,chap2,chap3,%  % BEWARE: First % kills white space
             appendix}            %  plus all chapters, appendix
%\includeonly{chap1,chap2,%        % BEWARE: First % kills white space
%             appendix}            %  plus chapters 1-2, appendix
%\includeonly{chap3}               % Front + chapter 3 + back
%
%\tracingstats=2                % show TeX memory usage
\title{REPRESENTATION OF SOLUTIONS TO LINEAR\protect\\
       ALGEBRAIC EQUATIONS}
\author{Fred Krylov}
\thesistype{dissertation}
\graduatedean{Ann W. Hart}
\department{Department of Mathematics}
\degree{Doctor of Philosophy}
\departmentchair{Paul Fife}
\committeechair{Fletcher Gross}
\firstreader{Hans Othmer}
\secondreader{Jim Carlson}
\thirdreader{Grant Gustafson}
\fourthreader{Nick Korevaar}
\chairtitle{Professor}
\submitdate{March 1993}
\copyrightyear{1993}
% Chapter is one level, section and subsection are the next two levels.
\fourlevels
\dedication{For my cat, Mouse, a few lines only.}
 \inputpicturetrue  % By Jeff McGough. See uuguide and private thesis.sty
%\inputpicturefalse % To NOT produce pictures, uncomment this line
\begin{document}
%% Comment out items by inserting a percent % character
\frontmatterformat
\titlepage
\copyrightpage
\committeeapproval
\readingapproval
\preface{abstract}{Abstract}
\dedicationpage
\tableofcontents
\listoffigures
\listoftables
%
% Optional front page, made from source "notation.tex".
% If you don't need it, then don't use it.
%
\optionalfront{Notation and Symbols}{%\vspace*{.2cm}
\begin{normalsize}
    \renewcommand{\arraystretch}{1.655} 
% 1.655 = 24pt/14.5pt = baselineskip/normalsize
Most of the following may be found in \cite{gilbarg:epd83}.\\ 
\begin{tabular}{ll} 
${\Bbb R}^n$ & $n$-dimensional Euclidean space. \\ 
$\Omega$ & A bounded open subset of ${\Bbb R}^n$. \\
$\partial\Omega$ & The boundary of $\Omega$. \\ 
$B_r$ & $=\{ x\, : \, |x|<r\}$, the ball of radius $r$.\\ 
$D^{\beta}f$ & $=\partial^{|\beta |}f/\partial x_{\beta_1}\partial
x_{\beta_2} \cdots \partial x_{\beta_n}$, $|\beta | \equiv \sum_i
\beta_i$. \\
$\nabla f$ & $=(\partial f/\partial x_1,\partial f/\partial x_2,\dots ,
\partial f/\partial x_n)$, the gradient of $f$. \\
div$\{ \bf g \}$ & $=\partial g_1/\partial x_1 + \partial g_2/\partial x_2
+ \dots + \partial g_n/\partial x_n$, the divergence of $\bf g$. \\
$\Delta f$ & $=\mbox{div}\{\nabla f\}$, the Laplacian of $f$. \\
$C^k(\Omega )$ & Functions defined
on $\Omega$ which have $k$ continuous derivatives. \\ 
$C^k_0(\Omega )$
& $C^k(\Omega )$ functions which vanish at the boundary. \\
$C^{0,\alpha}(\Omega )$ & H\"{o}lder continuous functions with
H\"{o}lder constant $\alpha$. \\ 
$C^{k,\alpha}(\Omega )$ & $C^k(\Omega
)$ functions with $C^{0,\alpha}(\Omega )$ derivatives (up to order
$k$). \\ 
$C^{k,\alpha}_0(\Omega )$ & $C^{k,\alpha}(\Omega )$ functions
which vanish at the boundary. \\ 
$\| f\|_{L^p(\Omega )}$ & $= \left(
\int_{\Omega}|f|^p dx \right)^{1/p}$, the $L^p$ norm.  \\
$L^p(\Omega )$ & The space of $p$ integrable functions (the $L^p$ norm
is bounded). \\ $\| f \|_{W^{k,p}(\Omega )}$ & $= \left( \sum_{|\beta |
\leq k} \int_{\Omega}|D^{\beta} f|^p dx \right)^{1/p}$, the Sobolev norm. \\
$W^{k,p}(\Omega )$ & The space of functions with bounded Sobolev norm.
\\ $W^{k,p}_0(\Omega )$ & $W^{k,p}(\Omega )$ functions that vanish a.e.
at the boundary.  
\end{tabular} 
\end{normalsize}
}
\preface{acknowledge}{Acknowledgements}
\maintext       % Start normal page numbering. Parts and chapters follow.
%%%
%%% This is the beginning of the actual thesis.  If you don't know latex
%%% then start with the LaTeX manual by Lamport and another easy
%%% reference, like the paperback by Jane Hahn, LaTeX for Everyone, PTI,
%%% 1991. See also $TEX/latex/sample.tex and $TEX/doc/story.tex, where
%%% $TEX==/usr/local/lib/tex
%%%
% Start this dissertation....
%
\chapter{Introduction}\label{introduction}   % level 1
%
%% Stolen from the sample.tex file.  There have been a few
%% modifications to fit in the thesis here.
%
% This is a sample LaTeX input file.  (Version of 28 May 1985)
%
% A '%' character causes TeX to ignore all remaining text on the line,
% and is used for comments like this one.
%
% \author{Leslie Lamport} % For this section Lamport is the author.
% \title{A Sample Document}
% \date{December 12, 1984}
%
%
\fixchapterheading % Use this if section follows chapter immediately
\section{The Sample.tex file}  % Produces section heading.  % level 2
%
    % Lower -level sections are begun with similar
    % \subsection and \subsubsection commands.


\subsection{Ordinary Text}   % level 3

The ends of words and sentences are marked by spaces. It doesn't matter
how many spaces you type; one is as good as 100.  The end of a line
counts as a space.\footnote{
This is a sample input file.  Comparing it with the output it
generates can show you how to produce a simple document of
your own.
}

One or more blank lines denote the end of a paragraph.

Since any number of consecutive spaces are treated like a single one,
the formatting of the input file makes no difference to \LaTeX,
but it makes a difference to you. When you use
\LaTeX,       % The \LaTeX command generates the LaTeX logo.
making your input file as easy to read as possible will be a great help
as you write your document and when you change it.  This sample file
shows how you can add comments to your own input file.

Because printing is different from typewriting, there are a number of
things that you have to do differently when preparing an input file than
if you were just typing the document directly.
Quotation marks like
       ``this''
have to be handled specially, as do quotes within quotes:
       ``\,`this'                  % \, separates the double and single quote.
        is what I just
        wrote, not  `that'\,.''

Dashes come in three sizes: an
       intra-word
dash, a medium dash for number ranges like
       1--2,
and a punctuation
       dash---like
this.

A sentence-ending space should be larger than the space between words
within a sentence.  You sometimes have to type special commands in
conjunction with punctuation characters to get this right, as in the
following sentence.
       Gnats, gnus, etc.\    % `\ ' makes an inter-word space.
       all begin with G\@.   % \@ marks end-of-sentence punctuation.
You should check the spaces after periods when reading your output to
make sure you haven't forgotten any special cases.
Generating an ellipsis
       \ldots\    % `\ ' needed because TeX ignores spaces after
                  % command names like \ldots made from \ + letters.
                  %
                  % Note how a `%' character causes TeX to ignore the
                  % end of the input line, so these blank lines do not
                  % start a new paragraph.
with the right spacing around the periods
requires a special  command.

\LaTeX\ interprets some common characters as commands, so you must type
special commands to generate them.  These characters include the
following:
       \$ \& \% \# \{ and \}.

In printing, text is emphasized by using an
       {\em italic\/}  % The \/ command produces the tiny extra space that
                       % should be added between a slanted and a following
                       % unslanted letter.
type style.

\begin{em}
   A long segment of text can also be emphasized in this way.  Text within
   such a segment given additional emphasis
          with\/ {\em Roman}
   type.  Italic type loses its ability to emphasize and become simply
   distracting when used excessively.
\end{em}

It is sometimes necessary to prevent \LaTeX\ from breaking a line where
it might otherwise do so.  This may be at a space, as between the
``Mr.'' and ``Jones'' in
       ``Mr.~Jones,''        % ~ produces an unbreakable interword space.
or within a word---especially when the word is a symbol like
       \mbox{\em itemnum\/}
that makes little sense when hyphenated across
       lines.

Footnotes\footnote{This is an example of a footnote.}
pose no problem.

\LaTeX\ is good at typesetting mathematical formulas like
       $ x-3y = 7 $
or
       $$ a_{1} > x^{2n} / y^{2n} > x'. $$
Remember that a letter like
       $x$        % $ ... $  and  \( ... \)  are equivalent
is a formula when it denotes a mathematical symbol, and should
be treated as one.

\subsection{Displayed Text}

Text is displayed by indenting it from the left margin.

\subsubsection{Quotations}

Quotations are commonly displayed.  There are short quotations
\begin{quote}
   This is a short a quotation.  It consists of a
   single paragraph of text.  There is no paragraph
   indentation.
\end{quote}
and longer ones.
\begin{quotation}
   This is a longer quotation.  It consists of two paragraphs
   of text.  The beginning of each paragraph is indicated
   by an extra indentation.

   This is the second paragraph of the quotation.  It is just
   as dull as the first paragraph.
\end{quotation}

\subsubsection{Lists}

Another frequently-displayed structure is a list.

\paragraph{Itemize.}
The following is an example of an {\em itemized} list.

\minusline % Part of uuthesis.sty to remove extra vertical space.

\begin{quote}
\begin{itemize}
   \item  This is the first item of an itemized list.  Each item
          in the list is marked with a ``tick''.  The document
          style determines what kind of tick mark is used.

   \item  This is the second item of the list.  It contains another
          list nested inside it.  The inner list is an {\em enumerated}
          list.
          \begin{enumerate}
              \item This is the first item of an enumerated list that
                    is nested within the itemized list.

              \item This is the second item of the inner list.  \LaTeX\
                    allows you to nest lists deeper than you really should.
          \end{enumerate}
          This is the rest of the second item of the outer list.  It
          is no more interesting than any other part of the item.
   \item  This is the third item of the list.
\end{itemize}
\end{quote}

\paragraph{Verse.}
You can even display poetry.

\minusline % Part of uuthesis.sty to kill one line

\begin{quote}
\begin{quote}
\begin{verse}
   There is an environment for verse \\    % The \\ command separates lines
   Whose features some poets will curse.   % within a stanza.

                           % One or more blank lines separate stanzas.

   For instead of making\\
   Them do {\em all\/} line breaking, \\
   It allows them to put too many words on a line when they'd
   rather be forced to be terse.
\end{verse}
\end{quote}
\end{quote}

\subsubsection{Mathematics}
Mathematical formulas may also be displayed.  A displayed formula is
one-line long; multiline formulas require special formatting
instructions.
   \[  x' + y^{2} = z_{i}^{2}\]
Don't start a paragraph with a displayed equation, nor make
one a paragraph by itself.

\section{More examples: Jeff McGough's Thesis}

Equations like
$\gamma = 0$ that don't need numbering may
be
set inline by the coding \verb"$\gamma = 0$" or displayed by
\par
\begin{singlespace}
\begin{verbatim}
$$
\gamma = 0.
$$
\end{verbatim}
\end{singlespace}
\par
Numbered equations are set as shown in the next paragraph. They use the
theorem environments defined in \verb"thesis.sty":
\par
\begin{singlespace}
\begin{verbatim}
\newtheorem{thrm}{Theorem}
\newtheorem{lem}[thrm]{Lemma}
\newtheorem{cor}[thrm]{Corollary}
\newtheorem{rem}[thrm]{Remark}
\newtheorem{defn}[thrm]{Definition}
\newtheorem{exmpl}[thrm]{Example}
\end{verbatim}
\end{singlespace}
\par

The Gelfand problem is the following elliptic boundary value problem:
%
% The equation-array feature in LaTeX is a bad idea.  For centered
% numbers you should set your own equations and arrays as follows:
%
\def\dd{\displaystyle}
\begin{equation}\label{gelfand}
\begin{array}{rl}
\dd \Delta u + \lambda e^u = 0, &
\dd u\in \Omega,\\[8pt] % add 8pt extra vertical space. 1 line=23pt
\dd u=0, & \dd u\in\partial\Omega.
\end{array}
\end{equation}
The previous equation had a label.  It may be referenced as
equation~(\ref{gelfand}).

%
%
\section{History of the Gelfand problem}
%
%

According to Bebernes and Eberly \cite[p.46]{bebernes:mpc89},
Gelfand was ``the first to make an in-depth
study'' of (\ref{gelfand}). Following this statement they briefly
outline the history of the Gelfand problem.
\par
% Quotes need to forced single space:
\begin{singlespace}
\begin{quote}
For dimension $n=1$, Liouville~\cite{liouville:edp53} first studied and
found an explicit solution in 1853. For $n=2$, Bratu~\cite{bratu:ein14}
found an explicit solution in 1914.  Frank-Kamenetski~\cite{frank:dhe55}
rediscovered these results in his development of thermal explosion
theory.  Joseph and Lundgren~\cite{joseph:qdp73} gave an elementary
proof via phase plane analysis of the multiple existence of solutions
for dimensions $n\geq 3$.
\end{quote}
\end{singlespace}
\par

% Several things to note here.  Latex sometimes breaks equations, this
% can be restricted by the samepage command.  The spacing in the array
% mode is also important for some structures.

From Zeidler~\cite{zeidler:nfa88IIa}:
{\samepage
\begin{equation}\label{station}
\begin{array}{rcll}
\dd\mbox{div } j& = &\dd f, &\dd x\in\Omega ,\\[8pt]
\dd u& = & \dd g_1, & \dd x\in\partial\Omega_1 , \\[8pt]
\dd j\nu & = & \dd g_2, & \dd x\in\partial\Omega_2 ,
\end{array}
\end{equation} }
where
\begin{equation}\label{current}
j =  h(|\nabla u|^2)\nabla u
\end{equation}
and $\Omega$ is a bounded domain in ${\Bbb R}^n$ with
smooth boundary $\partial\Omega = \overline{\partial\Omega_1}\cup
\overline{\partial\Omega_2}$, $\partial\Omega_1 \cap
\partial\Omega_2 = \emptyset$ and $\nu$ is the normal vector to
$\partial\Omega$.

% There is a lot of shorthand set up for structures, for example a
% lemma:
\begin{lem}
Assuming that $\partial\Omega_2 = \emptyset$ and that $h(t) = 1$, we
have $$
\begin{array}{lr}
\dd\Delta u = f, & \dd x\in\Omega ,\\[8pt]
\dd u =  g_1, & \dd x\in\partial\Omega .
\end{array}
$$
\end{lem}

% another ...
\begin{cor}
If $g_2 = 0$ then
$$
\begin{array}{lr}
\dd \Delta u = f, & \dd x\in\Omega ,\\[8pt]
\dd u =  0, & \dd x\in\partial\Omega .
\end{array}
$$
\end{cor}

% Look in thesis.sty for more structures.

\section{Fundamental results}
The investigation of the Gelfand problem begins with examining the
..... (this paragraph continues for many lines).

%
% This is an example of a big ugly technical theorem.  It has two
% levels of lists, referencing, citations and names.
%
\begin{thrm}[Joseph-Lundgren~\cite{joseph:qdp73}]
Boundary value problem (\ref{gelfand}) has positive radial
solutions $u$ on the unit ball which depend on $n$ and $\lambda$
in the following manner.
\begin{enumerate}
\item For $n=1,2$, there exists $\lambda^* >0$ such that
\begin{enumerate}
\item for $0< \lambda < \lambda^*$ there are two positive
solutions,
\item for $\lambda =\lambda^*$ there is a unique solution, and
\item for $\lambda > \lambda^*$ there are no solutions.
\end{enumerate}
\item For $3\leq n \leq 9$, let $\overline{\lambda}=2(n-2)$; then
there exist positive constants $\lambda_*$, $\lambda^*$ with
$0< \lambda_* < \overline{\lambda} < \lambda^*$, such that
\begin{enumerate}
\item for $\lambda = \lambda^*$ there is a unique solution,
\item for $\lambda > \lambda^*$ there are no solutions,
\item for $\lambda = \overline{\lambda}$ there is a countably infinite number
of solutions,
\item for $\lambda \in (\lambda_*,\lambda^*)$, $\lambda \neq
\overline{\lambda}$, there is a finite number of solutions,
\item for $\lambda < \lambda_*$ there is a unique solution.
\end{enumerate}
\item For $n\geq 10$, let $\lambda^* = 2(n-2)$ then
\begin{enumerate}
\item for $\lambda \geq \lambda^*$ there are no solutions,
\item for $\lambda \in (0,\lambda^*)$ there is a unique solution.
\end{enumerate}
\end{enumerate}
\end{thrm}

\chapter{Quadratic nonlinearities}\label{quad}
%
% Don't use \fixchapterheading here. Chapter is followed by a
% paragraph, not a heading.
%
In this chapter we derive results for the quadratic equation.

\section{Derivation of the quadratic formula}
A quadratic equation is one of the form
\begin{equation}\label{quadratic}
ax^2 + bx + c = 0
\end{equation}
where $a,b,c$ are known constants and $x$ is the unknown.
The results are summarized in Table \ref{pde.tab1} and Table
\ref{pde.tab2} below.

%
%
\section{Application of the quadratic formula}
%
%

If the differential operator generates a nonnegative form, then an
inequality is based on the following considerations. See
Figure \ref{gelfand.fig1} for $n=1,2$,
Figure \ref{gelfand.fig2} for $3\leq n \leq 9$
and
Figure \ref{gelfand.fig3} for $n\geq 10$.

% Example of a table:
% Table caption can be selected as paragraph style or centered style
% (for an inverted pyramid title). Use \oldstylecaptiontrue (paragraph)
% or \oldstylecaptionfalse (centered) to select the style.
%
\begin{table}[b]
\centering
\caption{\label{timing1} PDE solve times, $15^3+1$
equations.\label{pde.tab1}}
\plusline
\begin{tabular}{||l|l|l|l|l|l||}\hline
Precond. & Time & Nonlinear & Krylov
& Function & Precond. \\
 & & Iterations & Iterations & calls & solves \\ \hline
None & 1260.9u & 3 & 26 & 30 & 0  \\
 &(21:09) & & & &  \\ \hline
FFT  & 983.4u & 2  & 5  & 8  & 7 \\
&(16:31) & & & & \\ \hline
MILU & 629.7u & 3  & 11 & 15 & 14 \\
& (10:36) & & & & \\ \hline
\end{tabular}
\end{table}
\clearpage

\begin{table}[t]
\caption{Convergence properties of RQI.\label{pde.tab2}}
\centering
\plusline\small
\begin{tabular}{l|lll} \hline
Object & Normal Matrices & Diagonalizable Matrices
& Defective Matrices \\ \hline
$\rho$ & Stationary at ev's. &
Stationary at ev's. &
Stationary at ev's. \\
$\| r_k\|$ & $\to 0$ as $k\to\infty$. &
Can oscillate. &
Can oscillate. \\
$\rho_k$ & Converges. &
Unknown. &
Unknown. \\
Convergence to & is cubic. & is quadratic. & is linear. \\
eigensets & & & \\
\hline
\end{tabular}\normalsize
\end{table}

% Figures in LaTeX will go on the bottom of the same page, or the top of
% the next page, but never before the first reference. All figures must
% be referenced. The syntax is below. See uuguide for control.
%
%       \begin{figure}[x]    % x = b, t, h, p
%       ...
%       \caption{My title.}  % Captions are below the figure!
%       \end{figure}
%
%
% These graphs were created by gnuplot. For simple graphs this is a
% great utility.  For example, if you want a sin curve in your thesis
% try the following:
%
% (terminal window): gnuplot
% (in gnuplot):
%                 set terminal latex
%                 set output "foo.tex"
%                 plot sin(x)
%                 quit
%
\begin{figure}[b]       % Place it on the bottom of page
\centering              % Put \label{} into \caption.
\inputpicture{fig1.tex}
\caption{Gelfand equation on the ball, $n=1,2$.
\label{gelfand.fig1}}    % Use \ref{gelfand.fig1} for references
\end{figure}


\begin{figure}[p]       % Likely it will go on the top of the page
\centering
\inputpicture{fig2.tex}
\caption{Gelfand equation on the ball, $3\leq n \leq 9$.
\label{gelfand.fig2}}    % If not, then change [t] to [p]
\end{figure}

\begin{figure}[p] % Likely it will go on the top of the next page
                  % If not, then change [h] to [p]
\centering
\inputpicture{fig3.tex}
\caption{Gelfand equation on the ball, $n\geq 10$.
\label{gelfand.fig3}}
\end{figure}
\clearpage % dump figures where they below

\chapter{Systems}\label{systems}
\fixchapterheading
\section{Diagrams made with diagram.sty}
% \captionstyleparagraph

%% The full documentation is in the file: diagram.sty

An example diagram appears below in Figure \ref{diagram.fig1}. This is
typical of what can made with the diagram package.

\begin{figure}[b]               % Place at bottom of this page
$$
\begin{diagram}
\node{U} \arrow{e,t}{i_1} \arrow{s}
\node{X} \arrow{s,r}{\pi} \\
\node{Y-\partial Q} \arrow{e,t}{j_1} \node{Y}
\end{diagram}
$$
\caption{Diagram example\label{diagram.fig1}}
\end{figure}

\section{Sample diagrams from diagram.tex}

Example diagrams reproduced here were taken from various sources.
Compare the three diagrams of increasing sizes in
Figure \ref{file.fig1}, Figure \ref{file.fig2}, Figure \ref{file.fig3}
with the three diagrams in Figure \ref{file.fig4}, Figure
\ref{file.fig5},
Figure \ref{file.fig6}.


\begin{figure}[b]
$$
\setlength{\dgARROWLENGTH}{3.0em}
\begin{diagram}[\strut A]
\node{A} \arrow{e} \arrow{s} \arrow{se} \node{B} \arrow{s} \\
\node{C} \arrow{e}                      \node{D}
\end{diagram}
$$
\caption{Base diagram, Arrowlength = 3.0em
\label{file.fig1}}
\end{figure}

\begin{figure}[p]
$$
\setlength{\dgARROWLENGTH}{6.0em}
\begin{diagram}[\strut A]
\node{A} \arrow{e} \arrow{s} \arrow{se} \node{B} \arrow{s} \\
\node{C} \arrow{e}                      \node{D}
\end{diagram}
$$
\caption{Same as Figure \protect\ref{file.fig1}, but Arrowlength = 6.0em
\label{file.fig2}}
\end{figure}

\begin{figure}[p]
$$
\setlength{\dgARROWLENGTH}{12.0em}
\begin{diagram}[\strut A]
\node{A} \arrow{e} \arrow{s} \arrow{se} \node{B} \arrow{s} \\
\node{C} \arrow{e}                      \node{D}
\end{diagram}
$$
\caption{Same as Figure \protect\ref{file.fig1}, but Arrowlength =
12.0em \label{file.fig3}}
\end{figure}

\begin{figure}[p]
$$
\setlength{\dgARROWLENGTH}{3.0em}
\begin{diagram}[\strut A]
\node{A} \arrow{e} \arrow{s} \arrow{se} \node{B} \arrow{s} \\
\node{C} \arrow{e}                      \node{D}
\end{diagram}
$$
\caption{Base figure, same as Figure \protect\ref{file.fig1}.
\label{file.fig4}}
\end{figure}
\clearpage % make page of floats

\begin{figure}[t]
$$
\setlength{\dgARROWLENGTH}{3.0em}
\begin{diagram}[\strut\hspace{6.0em}]
\node{A} \arrow{e} \arrow{s} \arrow{se} \node{B} \arrow{s} \\
\node{C} \arrow{e}                      \node{D}
\end{diagram}
$$
\caption{Same as Figure \protect\ref{file.fig4}, but Bignode = strut
hspace 6.0em. \label{file.fig5}}
\end{figure}

\begin{figure}[t]
$$
\setlength{\dgARROWLENGTH}{3.0em}
\begin{diagram}[\strut\hspace{12.0em}]
\node{A} \arrow{e} \arrow{s} \arrow{se} \node{B} \arrow{s} \\
\node{C} \arrow{e}                      \node{D}
\end{diagram}
$$
\caption{Same as Figure \protect\ref{file.fig1}, but Bignode = strut
hspace 12.0em \label{file.fig6}}
\end{figure}

Below we show diagrams from the manual with a few modifications. The
first in Figure \ref{file.fig7} is essentially as it appears in the
manual, whereas the second, Figure \ref{file.fig8} has been
rescaled to a larger size.
\begin{figure}[p]
$$
\begin{diagram}[B^*]
\node{A} \arrow{e,t}{a} \arrow{s,l}{c} \arrow{ese,b,1}{u}
   \node{B^*} \arrow{e,t}{b^*}
      \node{C} \arrow{s,r}{d} \arrow{wsw,b,1}{v} \\
\node{D} \arrow[2]{e,b}{e}
   \node[2]{E}
\end{diagram}
$$
\caption{First diagram from manual
\label{file.fig7}}
\end{figure}

\begin{figure}[p]
$$
\setlength{\dgARROWLENGTH}{.75em}
\begin{diagram}[B^*]
\node{A} \arrow[2]{e,t}{a} \arrow[2]{s,l}{c} \arrow[2]{ese,b,1}{u}
   \node[2]{B^*} \arrow[2]{e,t}{b^*}
      \node[2]{C} \arrow[2]{s,r}{d} \arrow{wsw,b,-}{v}
\\
        \node[3]{} \arrow{wsw}
\\
\node{D} \arrow[4]{e,b}{e}
   \node[4]{E}
\end{diagram}
$$
\caption{First diagram from manual, rescaled.\label{file.fig8}}
\end{figure}

Below are several diagrams created by Bill Richter. The first, Figure
\ref{file.fig9} is modified slightly to produce Figure
\ref{file.fig10}. Both use fractur fonts. The last one, Figure
\ref{file.fig11}, is a complicated example illustrating the limits of
what can be done with diagrams.

The diagram below in Figure \ref{file.fig12}, the last of our series of
illustrations, is by Anders Thorup (\verb"thorup@math.ku.dk"),
originally done with a package developed by himself and Steven Kleiman
(\verb"kleiman@math.mit.edu"):


%%%%
%%%% If you're missing fractur fonts, then comment out these next 5
%%%% lines and type instead;
%%%% \let\frak=\bf
%%%%
\font\tenfrak=eufm10 scaled \magstep1
\font\sevenfrak=eufm7 scaled \magstep1
\font\fivefrak=eufm5 scaled \magstep1
\newfam\frakfam \def\frak{\fam\frakfam\tenfrak} \textfont\frakfam=\tenfrak
\scriptfont\frakfam=\sevenfrak  \scriptscriptfont\frakfam=\fivefrak
%%%%
%%%%
\def\a{ \alpha }
\def\d{ \delta }
\def\s{ \sigma }
\def\l{ \lambda }
\def\p{ \partial }
\def\st{{\tilde\s}}
\def\O{ \Omega }
\def\S{\Sigma}
\def\Z{{   \Bbb Z }}
\def\@{ \otimes }
\def\^{ \wedge }
\def\({ \left( }
\def\){ \right) }
\def\K#1{{ K\(\Z/2,#1\) }}
\def\KZ#1{{K\(\Z/4,#1\) }}
\def\id{ \mathop{id}\nolimits }
\def\h{ {\frak h} }
\def\e{ {\frak e} }
\def\G{ G }
\def\pinch{{ \mathop{{\rm pinch}} }}
\def\tuber{{ \bar\tau }}
\begin{figure}[p]
$$
\setlength{\dgARROWLENGTH}{1.5em}
\begin{diagram}[ \KZ{8n-1}  ]
\node[4]{ \K{8n+1} } \\
\node[2]{ \KZ{8n-1}  } \arrow{e} \arrow{ene,t}{Sq^2}
   \node{E} \arrow{ne,b}{\Theta} \arrow{s,l}{\pi} \\
\node{ \S\O X \^ \O X  } \arrow{e,t}{H_\mu} \arrow{ne,t}{\s(\a\@\a)}
   \node{ \Sigma \O X } \arrow{e,t}{\sigma} \arrow{ne,t}{\st}
       \node{ X }  \arrow{e,t}{\a^2}
           \node{ \KZ{8n}. }
\end{diagram}
$$
\caption{Bill Richter, first diagram\label{file.fig9}}
\end{figure}

\begin{figure}[p]
$$
\setlength{\dgARROWLENGTH}{-2.75em}
\begin{diagram}[ \O^2 \( \S A \^ \S A \) ]
\node[3]{\O\S A} \arrow[2]{e,t}{\l_2}
  \node[2]{\O^2 \( \S A \^ \S A \)}
\\
\node[4]{\#}
\\
% Note: the next two lines are like
% \node{\O B}  \arrow[2]{e,t,1}{\d}     \arrow[2]{ne,t}{\O\(\p\)}
% but put a gap in first arrow to make room for crossing arrow
\node{\O B}  \arrow{e,t,-}{\d}  \arrow[2]{ne,t}{\O\(\p\)}
  \node{} \arrow{e}
    \node{F} \arrow[2]{e,t}{\h}  \arrow[2]{s,r}{\pi} \arrow[2]{n,r}{J}
      \node[2]{\O^2 \( B \^ \S A \)} \arrow[2]{n,r}{\O^2\(\p\^\id\)}
\\
\\
\node{A} \arrow[2]{ne,t}{\e} \arrow[2]{e,t}{f}  \arrow[2]{nne,t,1}{E}
  \node[2]{X}  \arrow[2]{e,t}{h}
    \node[2]{B.}
\end{diagram}
$$
\caption{Bill Richter, second diagram\label{file.fig10}}
\end{figure}
\clearpage % Expunge all figures


\begin{figure}[p]
$$
\setlength{\dgARROWLENGTH}{-3.9em}
\begin{diagram}[ J\(S^4\^S^4\) ]
\node[9]{\O S^5}
\\
\\
\\
\node[8]{\beth}
\\
\node{\O\( M^5_{2\i}\)} \arrow[4]{e,t}{\O\(\pinch\)}
        \node[4]{\O S^5} \arrow[2]{e,t,-}{\d}
                                \arrow[4]{ne,t}{\O\(2\i\)}
                \node[2]{} \arrow[2]{e}
                        \node[2]{\G}    \arrow[4]{e,t}{\h_2}
                                        \arrow[2]{s,r,-}{\pi} \arrow[4]{n,r}{J}
                                \node[4]{J\(S^4\^S^4\)}
\\
\\
\node[3]{J_2\( M^4_{2\i}\)} \arrow[3]{e,t,3,-}{\d_2}
                                \arrow[2]{ne,t}{\i}
        \node[3]{}      \arrow{e}
                \node{\G_2} \arrow[4]{e,t,3}{\h_2}
                                \arrow[2]{ne,t}{\i}
                        \node[2]{}      \arrow[2]{s}
                                \node[2]{S^8} \arrow[2]{ne,b}{E}
\\
\node[4]{\aleph}
\\
\node{M^{12}_{2\i}} \arrow[4]{e,t}{\tuber} \arrow[2]{ne,t}{\tau}
        \node[4]{S^4} \arrow[4]{e,t}{\i}
                                \arrow[2]{ne,t}{\e}
                                        \arrow[4]{nne,t,3}{E}
                \node[4]{M^5_{2\i}} \arrow[4]{e,t}{\pinch}
                        \node[4]{S^5}
\end{diagram}
$$
\caption{Bill Richter, third diagram\label{file.fig11}}
\end{figure}

\begin{figure}[p]
$$
\setlength{\dgARROWLENGTH}{-6em}
\begin{diagram}[H^k(B_G\times N;Q)=H^k_G(N;Q)]
\node{H^k(B_G\times N;Q)=H^k_G(N;Q)}
      \arrow[2]{e,t}{f^*_j} \arrow[2]{s,l}{p^*} \arrow{se,t}{\tilde f^*}
   \node[2]{H^k_G(F_j;Q)}
      \arrow[2]{s,r}{q^*_j} \\
\node[2]{H^k_G(M;Q)}
      \arrow{ne,t}{i^*_j} \arrow[2]{s,l,1}{i^*} \\
\node{H^k(N;Q)}
      \arrow{e,t,-}{\tilde f^*_j=f^*_j} \arrow{se,b}{\tilde f^*=f^*}
   \node{}
      \arrow{e}
   \node{H^k(F_j;Q)} \\
\node[2]{H^k(M;Q)}
      \arrow{ne,b}{i^*_j}
\end{diagram}
$$
\caption{Anders Thorup diagram\label{file.fig12}}
\end{figure}
\clearpage

\numberofappendices=1   % Set 0 for none, else number of appendices.
\appendix       % Chapters, sections are now appendix style
\chapter{Classical identities}\label{appendix}
\fixchapterheading
\section*{Rellich's identity}\label{rellich.section}
\setcounter{thrm}{0}
%
%

Standard developments of Pohozaev's identity used an identity by
Rellich~\cite{rellich:der40}, reproduced here.

\begin{lem}[Rellich]
Given $L$ in divergence form and $a,d$ defined above, $u\in C^2
(\Omega )$, we have
\begin{equation}\label{rellich}
\int_{\Omega}(-Lu)\nabla u\cdot (x-\overline{x})\, dx
= (1-\frac{n}{2}) \int_{\Omega} a(\nabla u,\nabla u) \, dx
-
\frac{1}{2} \int_{\Omega}
d(\nabla u, \nabla u) \, dx
\end{equation}
$$
+
\frac{1}{2} \int_{\partial\Omega} a(\nabla u,\nabla u)(x-\overline{x})
\cdot \nu  \, dS
-
\int_{\partial\Omega}
a(\nabla u,\nu )\nabla u\cdot (x-\overline{x}) \, dS.
$$
\end{lem}
{\bf Proof:}\\
There is no loss in generality to take $\overline{x} = 0$. First
rewrite $L$:
$$Lu = \frac{1}{2}\left[ \sum_{i}\sum_{j}
\frac{\partial}{\partial x_i}
\left( a_{ij} \frac{\partial u}{\partial x_j} \right) +
\sum_{i}\sum_{j}
\frac{\partial}{\partial x_i}
\left( a_{ij} \frac{\partial u}{\partial x_j} \right)
\right]$$
Switching the order of summation on the second term and relabeling
subscripts, $j \rightarrow i$ and $i \rightarrow j$, then using the fact
that $a_{ij}(x)$ is a symmetric matrix,
gives the symmetric form needed to derive Rellich's identity.
\begin{equation}
Lu = \frac{1}{2} \sum_{i,j}\left[
\frac{\partial}{\partial x_i}
\left( a_{ij} \frac{\partial u}{\partial x_j} \right) +
\frac{\partial}{\partial x_j}
\left( a_{ij} \frac{\partial u}{\partial x_i} \right)
\right].
\end{equation}

Multiplying $-Lu$ by $\frac{\partial u}{\partial x_k} x_k$ and integrating
over $\Omega$, yields
$$\int_{\Omega}(-Lu)\frac{\partial u}{\partial x_k} x_k \, dx=
-\frac{1}{2} \int_{\Omega}
\sum_{i,j}\left[
\frac{\partial}{\partial x_i}
\left( a_{ij} \frac{\partial u}{\partial x_j} \right) +
\frac{\partial}{\partial x_j}
\left( a_{ij} \frac{\partial u}{\partial x_i} \right)
\right]
\frac{\partial u}{\partial x_k} x_k \, dx$$
Integrating by parts (for integral theorems see~\cite[p. 20]
{zeidler:nfa88IIa})
gives
$$= \frac{1}{2} \int_{\Omega}
\sum_{i,j} a_{ij} \left[
\frac{\partial u}{\partial x_j}
\frac{\partial^2 u}{\partial x_k\partial x_i} +
\frac{\partial u}{\partial x_i}
\frac{\partial^2 u}{\partial x_k\partial x_j}
\right] x_k \, dx
$$
$$
+
\frac{1}{2} \int_{\Omega}
\sum_{i,j} a_{ij} \left[
\frac{\partial u}{\partial x_j} \delta_{ik} +
\frac{\partial u}{\partial x_i} \delta_{jk}
\right] \frac{\partial u}{\partial x_k} \, dx
$$
$$- \frac{1}{2} \int_{\partial\Omega}
\sum_{i,j} a_{ij} \left[
\frac{\partial u}{\partial x_j} \nu_{i} +
\frac{\partial u}{\partial x_i} \nu_{j}
\right] \frac{\partial u}{\partial x_k} x_k \, dx
$$
= $I_1 + I_2 + I_3$, where the unit normal vector is $\nu$.
One may rewrite $I_1$ as
$$I_1 = \frac{1}{2} \int_{\Omega}
\sum_{i,j} a_{ij} \frac{\partial}{\partial x_k}\left(
\frac{\partial u}{\partial x_i}
\frac{\partial u}{\partial x_j}
\right) x_k \, dx
$$
Integrating the first term by parts again yields
$$I_1 = -\frac{1}{2} \int_{\Omega}
\sum_{i,j} a_{ij} \left(
\frac{\partial u}{\partial x_i}
\frac{\partial u}{\partial x_j}
\right) \, dx
+
\frac{1}{2} \int_{\partial\Omega}
\sum_{i,j} a_{ij} \left(
\frac{\partial u}{\partial x_i}
\frac{\partial u}{\partial x_j}
\right) x_k \nu_k \, dS
$$
$$
-
\frac{1}{2} \int_{\Omega}
\sum_{i,j} \left(
\frac{\partial u}{\partial x_i}
\frac{\partial u}{\partial x_j}
\right) x_k \frac{\partial a_{ij}}{\partial x_k}\, dx.
$$
Summing over $k$ gives
$$\int_{\Omega}(-Lu)(\nabla u\cdot x)\, dx =
-\frac{n}{2} \int_{\Omega}
\sum_{i,j} a_{ij} \left(
\frac{\partial u}{\partial x_i}
\frac{\partial u}{\partial x_j}
\right) \, dx
$$
$$
+
\frac{1}{2} \int_{\partial\Omega}
\sum_{i,j} a_{ij} \left(
\frac{\partial u}{\partial x_i}
\frac{\partial u}{\partial x_j}
\right) (x\cdot \nu ) \, dS
-
\frac{1}{2} \int_{\Omega}
\sum_{i,j} \left(
\frac{\partial u}{\partial x_i}
\frac{\partial u}{\partial x_j}
\right) (x\cdot  \nabla a_{ij}) \, dx
$$
$$
+
\frac{1}{2} \int_{\Omega}
\sum_{i,j,k} a_{ij} \left[
\frac{\partial u}{\partial x_j}
\frac{\partial u}{\partial x_k} \delta_{ik} +
\frac{\partial u}{\partial x_i}
\frac{\partial u}{\partial x_k} \delta_{jk}
\right] \, dx
$$
$$- \frac{1}{2} \int_{\partial\Omega}
\sum_{i,j} a_{ij} \left[
\frac{\partial u}{\partial x_j} \nu_{i} +
\frac{\partial u}{\partial x_i} \nu_{j}
\right] (\nabla u\cdot x) \, dS.
$$
Combining the first and fourth term on the right-hand side
simplifies the expression
$$\int_{\Omega}(-Lu)(\nabla u\cdot x)\, dx
=
(1-\frac{n}{2}) \int_{\Omega}
\sum_{i,j} a_{ij} \left(
\frac{\partial u}{\partial x_i}
\frac{\partial u}{\partial x_j}
\right) \, dx
$$
$$
+
\frac{1}{2} \int_{\partial\Omega}
\sum_{i,j} a_{ij} \left(
\frac{\partial u}{\partial x_i}
\frac{\partial u}{\partial x_j}
\right) (x\cdot \nu ) \, dS
-
\frac{1}{2} \int_{\Omega}
\sum_{i,j} \left(
\frac{\partial u}{\partial x_i}
\frac{\partial u}{\partial x_j}
\right) (x\cdot  \nabla a_{ij}) \, dx
$$
$$
-
\frac{1}{2} \int_{\partial\Omega}
\sum_{i,j} a_{ij} \left[
\frac{\partial u}{\partial x_j} \nu_{i} +
\frac{\partial u}{\partial x_i} \nu_{j}
\right] (\nabla u\cdot x) \, dS.
$$
Using the notation defined above, the result follows.


%
%
%
\section*{Fortran code}\label{code}
%
%

%% The following was constructed by a very handy program called
%% tgrind.  tgrind is a filter to convert C or fortran files into
%% formatted tex.  Starting with a fortran subroutine rhs.f:
%%
%% tgrind -lf -f >rhs.tex rhs.f
%%
%% This creates the tex file rhs.tex.  This may be directly included
%% in a latex document via the special command \tgrind (latex command
%% here):
%%
%% \begin{singlespace}
%% \begin{small}
%% \tgrind{rhs.tex}
%% \end{small}
%% \end{singlespace}
%%
%% Otherwise, the file rhs.tex needs to be edited (rhs_mod.tex)
%% to be included into a latex document (it is a stand-alone
%% tex file).  The line with the \File command (top) needs to be
%% removed or commented out: \File{rhs.f},{14:32},{Jul  5 1993} and
%% the \end command at the bottom also needs to be commented out.  The
%% file rhs.tex can then be included into the document:
%%
%% \begin{singlespace}
%% \begin{small}
%% \input tgrindmac
%\File{rhs.f},{14:32},{Jul  5 1993}
\L{\LB{}}
\L{\LB{      \K{subroutine} rhs(neq,v,rhsf)}}
\L{\LB{      \K{save}}}
\L{\LB{c}}
\L{\LB{c This \K{subroutine} computes the \K{function} values. Inputs are neq and }}
\L{\LB{c v, and on output the values of f are stored in the array of rhsf}}
\L{\LB{c}}
\L{\LB{      \K{include} \S{}\'parabolic.inc\'\SE{}}}
\L{\LB{}}
\L{\LB{      \K{integer} neq}}
\L{\LB{      \K{integer} i}}
\L{\LB{      \K{integer} j}}
\L{\LB{      \K{integer} k}}
\L{\LB{      \K{integer} ind}}
\L{\LB{      \K{integer} inde}}
\L{\LB{      \K{integer} indw}}
\L{\LB{      \K{integer} indn}}
\L{\LB{      \K{integer} inds}}
\L{\LB{      \K{integer} ind0}}
\L{\LB{      \K{integer} ind1}}
\L{\LB{      \K{integer} ind2}}
\L{\LB{}}
\L{\LB{      \K{double} \K{precision} v(neq)}}
\L{\LB{      \K{double} \K{precision} rhsf(neq)}}
\L{\LB{      \K{double} \K{precision} u(nv)}}
\L{\LB{      \K{double} \K{precision} diff}}
\L{\LB{      \K{double} \K{precision} diffn}}
\L{\LB{      \K{double} \K{precision} diffxn}}
\L{\LB{      \K{double} \K{precision} diffyn}}
\L{\LB{      \K{double} \K{precision} nl}}
\L{\LB{}}
\L{\LB{c      \K{write}(*,*)\S{}\'funct begin\'\SE{}}}
\L{\LB{}}
\L{\LB{c}}
\L{\LB{c     Compute F for the local dynamics, written as  F(u)= \-du\/dt + f(u)}}
\L{\LB{c     }}
\L{\LB{c}}
\L{\LB{c the system parameters}}
\L{\LB{c}}
\L{\LB{c      p1              ! \K{parameter} F}}
\L{\LB{c      p2              ! \K{parameter} k}}
\L{\LB{}}
\L{\LB{      \K{do} j = 1, ny }}
\L{\LB{         \K{do} i = 1, nx}}
\L{\LB{c}}
\L{\LB{c set up index}}
\L{\LB{c}}
\L{\LB{            ind = (i\-1)*nv + (j\-1)*meq}}
\L{\LB{c}}
\L{\LB{c Extract the jth component at current time}}
\L{\LB{c}}
\L{\LB{            nl = v(1+ind)*v(2+ind)*v(2+ind)}}
\L{\LB{}}
\L{\LB{            rhsf(1+ind) =  (\- nl + p1*(1.0d0 \- v(1+ind)))*local}}
\L{\LB{            rhsf(2+ind) =  (  nl \- (p1+p2)*v(2+ind))*local}}
\L{\LB{}}
\L{\LB{         \K{end} \K{do}}}
\L{\LB{      \K{end} \K{do}}}
\L{\LB{}}
\L{\LB{c \-\-\-\-\-\-\-\-\-\-\-\-\-\-\-\-\-\-\-\-\-\-\-\-\-\-\-\-\-\-\-\-\-\-\-\-\-\-\-\-}}
\L{\LB{c}}
\L{\LB{c     add diffusion for all species (zero diffusion }}
\L{\LB{c     coefficient takes care of those that \K{do} not diffuse). }}
\L{\LB{c }}
\L{\LB{c
\-\-\-\-\-\-\-\-\-\-\-\-\-\-\-\-\-\-\-\-\-\-\-\-\-\-\-\-\-\-\-\-\-\-\-\-\-\-\-
}}
\L{\LB{}}
\L{\LB{      \K{do} j = 1, ny}}
\L{\LB{         \K{do} i = 1, nx}}
\L{\LB{}}
\L{\LB{c indexing}}
\L{\LB{c}}
\L{\LB{            ind0 = (i\-1)*nv + (j\-1)*meq   ! point}}
\L{\LB{            indw = (i\-2)*nv + (j\-1)*meq   ! west point}}
\L{\LB{            inde = (i)*nv + (j\-1)*meq     ! east point}}
\L{\LB{            indn = (i\-1)*nv + (j)*meq     ! north point}}
\L{\LB{            inds = (i\-1)*nv + (j\-2)*meq   ! south point}}
\L{\LB{}}
\L{\LB{            \K{if}(i.eq.1) indw = (nx\-1)*nv + (j\-1)*meq}}
\L{\LB{            \K{if}(i.eq.nx) inde = (j\-1)*meq }}
\L{\LB{            \K{if}(j.eq.1) inds = (i\-1)*nv + (ny\-1)*meq}}
\L{\LB{            \K{if}(j.eq.ny) indn = (i\-1)*nv}}
\L{\LB{}}
\L{\LB{            \K{do} k = 1, 2}}
\L{\LB{}}
\L{\LB{c}}
\L{\LB{c First compute the contribution within a row at the current time}}
\L{\LB{c and at the preceding time. }}
\L{\LB{c}}
\L{\LB{               ind = k + ind0}}
\L{\LB{               ind1 = k + indw}}
\L{\LB{               ind2 = k + inde}}
\L{\LB{}}
\L{\LB{               diffxn = v(ind1) \- 2.0d0*v(ind) + v(ind2)}}
\L{\LB{}}
\L{\LB{c}}
\L{\LB{c Compute the contribution from the columns}}
\L{\LB{c}}
\L{\LB{               ind1 = k + indn}}
\L{\LB{               ind2 = k + inds}}
\L{\LB{}}
\L{\LB{               diffyn = v(ind1) \- 2.0d0*v(ind) + v(ind2)}}
\L{\LB{}}
\L{\LB{c}}
\L{\LB{c Multiply by  other factors and sum}}
\L{\LB{c}}
\L{\LB{               diff = d(k)*hxx*(diffxn + diffyn)*diffus}}
\L{\LB{}}
\L{\LB{               rhsf(ind) = rhsf(ind) + diff}}
\L{\LB{               }}
\L{\LB{}}
\L{\LB{            \K{end} \K{do}}}
\L{\LB{         \K{end} \K{do}}}
\L{\LB{      \K{end} \K{do}}}
\L{\LB{}}
\L{\LB{       }}
\L{\LB{      \K{return}}}
\L{\LB{      \K{end}}}
\L{\LB{}}
\L{\LB{}}
\L{\LB{}}
\vfill\eject
%\end

%% \end{small}
%% \end{singlespace}
%%
%% In either case, the \File command probably will need removing
%% because it places the page numbers differently than the normal
%% thesis style. Using both gives:

 \begin{singlespace}
 \begin{small}
% \let\end\relax \def\File#1,#2,#3{}
 \tgrind{rhs_mod.tex}
 \end{small}
 \end{singlespace}

%
% The choice of bibliography style is a major decision, jointly made
% by you, your thesis advisor and the thesis editor. Common choices are
% "siam", "acm", "amsplain", "plain", "chicago".
%
\bibliographystyle{siam}
\bibliography{thesis}

\end{document}
|

One mystery is how to select the items to index. A second mystery is how
to automate the matching of the page numbers to insure that the index is
always correct. The mysteries are solved below.

Support within \LaTeX{} for an index is restricted to {\em earmarking
words and phrases} for output to a low-level file with extension
\verb|idx|. For example, {\tt thesis.idx} is made from root file {\tt
thesis.tex} by placing \verb"\makeindex" in the preamble and running
\LaTeX{}. The file {\tt thesis.idx} contains all the earmarked words and
phrases plus their page number in the text. This is a major service,
because it solves the problem of matching page numbers to index items.
It also insures that page number references to the index are correct.
Unfortunately, the contents of {\tt thesis.idx} are unsuitable for
use as an index {\em directly}, because the list is unsorted and
unformatted.

\begin{quote}
Words and phrases are earmarked in the document using the control
\verb"\index{#1}". Nelson Beebe recommends using a control \verb"\X{#1}"
defined by

\verb"\def\X#1{{#1}\index{{#1}}}"

because it is easier to tag the words and phrases. The
\verb"\index{...}" command produces itself no text in the document, only
in the {\tt idx} file. Actual entries in the {\tt idx} file depend upon
definitions of \verb"\item", \verb"\subitem" and \verb"\subsubitem"
given in {\tt uuthesis.sty}.

The low-level {\tt thesis.idx} source file is as primitive as an {\tt
aux} file. It is made by inserting the control \verb"\makeindex" into
the preamble (and only there) of the \TeX{} document.
\end{quote}

The second mystery, that of sorting and formatting the index, will now
be discussed. The creation of the sorted and formatted {\tt thesis.ind}
from the \LaTeX{} low-level file {\tt thesis.idx} is done in a terminal
window outside of \LaTeX{} (and also outside of {\tt emacs}). We run a
special unix command {\tt makeindex} or {\tt makeidx} to create the
source {\tt thesis.ind} from {\tt thesis.idx}. The syntax for the
command is given in a manual page: {\tt man makeindex}. The 1993 version
runs like this:

\verb"    makeindex thesis.idx"

It produces \verb"thesis.ind" which is the sorted and formatted version
of the index.

The formatting of the index itself for uuthesis style is two-column
newspaper format with header {\bf Index} and normal numbering. If you
are going to insert an index, it must be before the {\em vita} and after
the {\em appendices}. Because of numbering problems it may be necessary
to turn off the appendix switch with \verb"\noappendix".

The special separate executable system program called {\tt makeindex} or
{\tt makeidx} exists on SUN and DEC unix systems. It is also available
for microcomputers and other operating systems. Contact Nelson Beebe
for details about other operating systems, or look in the
directories {\tt pub/tex/pub/makeindex}, {\tt pub/ibmpc} and {\tt
pub/macintosh} via anonymous ftp to site {\tt science.utah.edu}

{\bf Vita}. The vita is optional, but the absolute last thing if it
appears. It is set up by the \verb|\vita| command, which starts a new
page and titles it. The text for your vita is to immediately follow the
\verb|\vita| command. It must follow \LaTeX{} rules, but it is otherwise
totally up to you. The Thesis {\em Handbook} makes some suggestions.

\section{Figure and Table Placement}

There is deep problem with figures and tables. They trouble most persons
who use \LaTeX{}. The placement seems sometimes out of control, with the
figure or table ending up on the wrong page. The purpose of this section
is to discuss how to adjust the placement and tweak various parameters
to make everything happen according to plan.

The thesis format dictates that figures {\em should be placed so they
look balanced on the page}. Defined limits: If text appears, then it
makes a block of six or more lines mid-page. Pages of floats are not
constrained. Figures and tables are separated by triple spacing (2ex).
The counting of figures and tables refers to large boxes that take the
full text width. A figure that itself consists of three small
horizontally placed graphs is considered one figure.

The basic setup for figures and tables is to define them as floating
boxes of text that take up the full text width. As such, a box can be
placed at the top \verb"[t]", at the bottom \verb"[b]" or in the middle
of the page \verb"[h]". Figures and tables can appear by themselves on a
page of floats. This is a page on which no text appears. The option
\verb"[p]" selects this placement.

The graduate school style disallows a figure mid-page, that is, text
cannot appear both before and after a figure on the same page.
A chapter can end with a figure. If too little of the page is used it
may look better placed with \verb"[h]" option as the last item on the
page.

Further, a figure must appear {\em after} it is referenced, that is, the
figure is physically below the reference, and no later than the next
page, or the one after that, in case there is a page of floats
generated. In summary, the thesis office allows options \verb"[b]",
\verb"[t]", \verb"[p]". However, \verb"[b]" and \verb"[t]" have
restrictions: \verb"[t]" cannot generate a figure before the figure is
referenced, and \verb"[b]" cannot generate a figure on the following
pages after the reference. The \verb"[h]" option is used rarely to force
a figure placement at the end of a chapter. It should be mentioned that
at chapter end this is the {\em only way} to get the figure in the
right place on the page!

The figure and table counters are reset in each chapter. Default
placement is \verb"[p]". Figure and table numbers look like {\tt 2.5},
where {\tt 2} is the chapter number and {\tt 5} is the figure number in
that chapter. The figure captions use {\bf Figure 2.5} followed by the
{\bf Title} which is in the normal Roman 12pt font. The numbers and
titles are inserted by the \verb"\caption" control using a special
syntax to generate cross reference labels. If the {\tt caption} control
is not used, then no counters are incremented and all labeling and
numbering is up to you.

\footnotesize\begin{verbatim}
\newcounter{figure}[chapter]                    figure counter
\def\fps@figure{p}                              Default placement [p]
\def\thefigure{\thechapter.\@arabic\c@figure}   Number is "2.5"
\def\fnum@figure{\bf Figure \thefigure}         Label is "Figure 2.5"

\newcounter{table}[chapter]                     table counter
\def\fps@table{p}                               Default placement [p]
\def\thetable{\thechapter.\@arabic\c@table}     Number is "2.5"
\def\fnum@table{\bf Table \thetable}            Label is "Table 2.5"
\end{verbatim}\normalsize

For example, if you don't like the bold label, then insert this into
your private {\tt thesis.sty}:
\footnotesize\begin{verbatim}
\def\fnum@figure{Figure \thefigure}         Label is "Figure 2.5"
\def\fnum@table{Table \thetable}            Label is "Table 2.5"
\end{verbatim}\normalsize

The secret to getting figures and tables to dump in the right place is a
function of several delicate parameters. They are described more fully
in the \LaTeX{} manual. There is no discussion of the effect of change
of {\em two} parameters at once. To understand the full effect requires
a deep knowledge of the float algorithm in {\tt latex.tex}. Suffice it
to say that the parameters below, set as indicated, cause the figures to
be dumped in a predictable way. Still, it may not produce what you want,
and hence you have license to change them locally mid-document to handle
a particular positioning problem. {\em Exercise great caution} in
changing these parameters. After they produce the desired change, {\em
reset them to the defaults below}.

\footnotesize\begin{verbatim}
\def\topfraction{1}             % Option [t] can use the whole page
\def\bottomfraction{1}          % Option [b] can use the whole page
\def\textfraction{0}            % Page can be all figures
\def\floatpagefraction{.5}      % Float page dump if half full
\end{verbatim}\normalsize


The maximum number of figures on a page is controlled by some variables
that you will likely never have to reset. Option \verb"[t]" and option
\verb"[b]" can be used together to generate 20 figures on one page.
The thesis office has no rule about the number of figures per page.
Manual intervention is required. The variables and default values (don't
change):

\footnotesize\begin{verbatim}
\setcounter{topnumber}{10}      % Option [t] can place 10 figures max
\setcounter{bottomnumber}{10}   % Option [b] can place 10 figures max
\setcounter{totalnumber}{20}    % Up to 20 figures per page
\end{verbatim}\normalsize

The distance between figures and text and tables is set by a number of
glue variables. It is possible to change them locally within the thesis
to solve fitting problems with figures and tables. Always reset the
values after use!

\footnotesize\begin{verbatim}
\floatsep 2em plus 2pt minus 4pt        % vertical space between floats
\textfloatsep 2em plus 2pt minus 4pt    % vertical space between text and floats
\intextsep 2em plus 4pt minus 4pt       % vertical space around mid-page float
\end{verbatim}\normalsize


{\bf Figure captions}.
Captions for tables and figures are centered in the current text width.
A figure caption is {\em below} the figure. A table caption is {\em
above} the table. The captions are to be consistent and follow a
recognized style guide. If a style guide is not followed, then the
thesis office asks that a caption not extend beyond the table or figure
edges and that the title be set in a centered inverted pyramid style.
Rules for tables are different than rules for figures. See your style
guide or find journal samples that support your style.

The captions in figures may be of a different style than found in style
manuals such as the {\em ACS Manual} or the {\em Chicago Manual}. In
this style, the caption is a paragraph set in block format across the
text width. The title can be set in boldface font, followed by a
paragraph of table notes.

The {\em Journal of Computational Physics}, Vol 108, No 1, September
1993, p 173, shows a non-pyramid form table with a paragraph of text
instead of a short title. The {\em SIAM Journal of Scientific Computing}
shows similar formats for tables in the July 1993 issue, Vol 14, pages
930-931. The {\em SIAM Journal of Numerical Analysis} Volume 30, No 3,
June, 1993, page 625, shows very long table captions in block format.

A common style borrowed from many style manuals is in use by journals
such as {\em Applied Numerical Mathematics}, {\em Journal of
Combinatorial Theory}, {\em Journal of Number theory} and {\em Journal
of Multivariate Analysis}. In this style, the title is above the table
in a centered inverted pyramid format. The title is brief. If notes
about the table are needed, then they appear below the table in a text
paragraph or else in the main text near the reference to the figure.

The control

\verb"    \captionstyleparagraph"

sets the caption in a text width paragraph. The size can be changed with
a {\tt parbox} construct or a {\tt minipage} environment wrapped around
the caption.

The control

\verb"    \captionstylecentered"

sets the caption in a centering environment using the current text
width. Again, the width can be changed by {\tt parbox} and {\tt
minipage} constructions.

The text below a table, or table notes, as they are sometimes called, is
set in a text width paragraph. This takes no special command inside a
{\tt table} environment, except possibly a \verb"\par" command to insert
space after the last line of the table. In particular, this text is {\em
not} defined by a {\tt caption} command!

The title used in a caption must also appear in the {\bf List of Tables}
or the {\bf List of Figures} with the same words. The breaking of long
titles will be different in the {\em List}. For example, in the Robert
Dillon thesis, table entries and figure entries in the {\em Lists} were
the first sentence in the paragraph that made up the caption. Caption
placement is {\em below} the figure and {\em above} the table.

The controls {\tt captionONtrue} and {\tt captionONfalse} turn on and
off, respectively, incrementing of counters and list of tables of list
of figures entries, for usage of the {\tt caption} macro. The best use
of this feature is to create a continuation table on the next page which
has the same table number and title {\em Continued}. All the features of
\LaTeX{} floats are available to place the table on the next page.

A figure caption may be a paragraph of text describing the table. The
caption is expected to be in paragraph format or else in centered
inverted pyramid format. Titles wider than the text width are
unacceptable. The entry in the list of figures is expected to be in a
different format, namely a plain text paragraph, and one line is best.

A table caption is usually not a paragraph, but in certain styles that
also is possible. In paragraph style, the first sentence is kept short
so it can be the same in the {\em List}. Explanation of the table is to
appear {\em below} the table in a text width paragraph, or in the main
text, or if you use the style of the {\em SIAM Journal of Numerical
Analysis}, as a paragraph or less continuing the title. The keynote here
is consistency: don't mix styles, use one style. Exceptions to this rule
for tables are an agreement between the thesis office, the student and
the thesis advisor. A style guide is the best way to procure agreement.

An example follows showing how to control the effects for a table of
unspecified width of about 3in having a two-line title that is to be set
in centered inverted pyramid format (ACS style). The attempt here is to
constrain all formatting to the table width. It is more common to use
the entire text width for the table with horizontal line across the page
following the table. If that style is preferred, then replace
\verb"\TMPsize" by \verb"\textwidth" in the example (so that footnotes
appear below the table).

In this example, {\em Table 2.15.\ Principal Coordinates for the
Fundamental
Dynamical System} will appear in the list of tables, while an inverted
pyramid title will appear above the table, as follows:


\def\topfraction{1}
\def\bottomfraction{1}
\def\textfraction{0}
\def\TMP{\begin{tabular}{|c|c|c|c|c|c|c|c|} \hline
 $x$ & 0.00 & 0.25 & 0.50 & 0.75 & 1.00 & 1.50 & 2.00 \\ \hline
 $y$ & 1.10 & 1.08 & 1.00 & 0.86 & 0.66 & 0.35 & 0.27 \\ \hline
\end{tabular}}
\newdimen\TMPsize
\settowidth{\TMPsize}{\TMP}

\begin{table}[h]
\begin{center}
\begin{minipage}{\TMPsize}
\def\Makecaption#1#2{\begin{center}{\bf Figure 2.15}. #2\end{center}}
\makeatletter\let\@makecaption\Makecaption\makeatother
\caption{ Principal Coordinates for the \protect\\ Fundamental Dynamical
System }
\TMP\medskip
\par\rule{\hsize}{1pt}\par
\leftline{\footnotesize The value $x$ is the position coordinate.}
\leftline{\footnotesize The value $y$ is the velocity coordinate.}
\end{minipage}
\end{center}
\end{table}
%\clearpage

\begin{quote}\footnotesize
\verb"\captionstylecentered" \\
\verb"\def\TMP{\begin{tabular}{|c|c|c|c|c|c|c|c|} \hline" \\
\verb" $x$ & 0.00 & 0.25 & 0.50 & 0.75 & 1.00 & 1.50 & 2.00 \\ \hline" \\
\verb" $y$ & 1.10 & 1.08 & 1.00 & 0.86 & 0.66 & 0.35 & 0.27 \\ \hline" \\
\verb"\end{tabular}}" \\
\verb"\newdimen\TMPsize\settowidth{\TMPsize}{\TMP}"
\par
\verb"\begin{table}[b]" \\
\verb"\begin{center}" \\
\verb"\begin{minipage}{\TMPsize}" \\
\verb"\caption{Principal Coordinates for the \protect\\ Fundamental" \\
\verb"Dynamical System}" \\
\verb"\TMP\medskip" \\
\verb"\par\rule{\hsize}{1pt}\par" \\
\verb"\leftline{\footnotesize The value $x$ is the position coordinate.}" \\
\verb"\leftline{\footnotesize The value $y$ is the velocity coordinate.}" \\
\verb"\end{minipage}" \\
\verb"\end{center}" \\
\verb"\end{table}"
\end{quote}

The {\bf minipage} construct is used to restrain the footnote width and
the caption width to the table width. The full text width is used by
default for the caption. Explicit line breaks were needed to produce an
inverse pyramid.

The best style uses the caption macro with a short title so that no
engineering needs to take place in the list of tables.


\begin{center}\large\bf Check list for figures and tables\end{center}
\begin{itemize}
\item[1.] A figure must follow its reference, at the bottom of the
          current page or immediately on the top of the next page.
          Start with \verb"[b]" and change to \verb"[t]" if it doesn't
          fit.

\item[2.] An option of \verb"[t]", which causes proper figure
          placement after the reference on the preceding page, can be
          followed by a figure with [\verb"[b]" option. Both figures can
          be used on the same page. Warning: Three usually won't fit and
          still have six lines of text mid-page.

\item[3.] Three figures in a row should generate one figure at the
          bottom and two figures on the next page. This means one
          \verb"[b]" option followed by two \verb"[t]" options. The
          deferred figures may have to be physically moved in the source
          to cause the dumping onto the next page. Often the text looks
          best if the two deferred figures are on a page of floats by
          themselves. Use \verb"[p]" on the last two figures to make it
          happen. The decision is based on six lines of text. If six
          lines won't fit on the same page as the two figures (at the
          bottom), then make a page of floats for the two figures.

\item[4.] If a page of floats has a widow figure at the end, then it
          is to appear as a \verb"[t]" option on the next page of text.

\item[5.] A \verb"\clearpage" command is sometimes needed to empty pages
          of floats before the next paragraph is read. The required
          \verb"\clearpage" is inserted after the final figure.

\item[6.] At the end of a chapter it is sometimes desired to place a
          figure as the last addition to the last page. The  last
          \verb"\begin{figure}[b]" \ldots \verb"\end{figure}" will
          mysteriously appear on the next page.
          The required \verb"\clearpage" is inserted after the figure to
          fix the problem, and \verb"[b]" and \verb"[t]" usually have to
          be changed to \verb"[h]" if the page is not full.

          If a good portion of the last page is blank, then the final
          page might look better with a ragged lower edge, which can be
          obtained by using \verb"[h]" instead of \verb"[b]". The thesis
          office endorses this exception to the mid-page figure rule.

\item[7.] Expect the thesis editor to move the figures and tables
          around to make the document look balanced. Expect that you
          will move figures also, as more text is added or subtracted.
          The flux ends when the thesis is approved and the final copies
          are delivered. A healthy attitude of expected change is more
          useful than an attitude of permanent location.

\item[8.] Check all figure pages for rule violations: caption,
          numbering, at least six lines of text or zero lines of text
          per figure page. Tables have one style, figures another.
          The style must be consistent: a violation is inverted pyramid
          title on one table and paragraph format title on another.

\end{itemize}


\section{Making Figures}

There are a number of methods for creating diagrams and figures for use
in a \LaTeX{} document. The choice of method depends upon the type of
figure and the method of generation.

{\bf Photographs}. If these are to be inserted into a thesis, then the
photos must be approved well in advance of publication by the thesis
office. There are copyright restrictions that may apply, which could
prevent acceptance of the thesis.

A number of theses have used color photos. The photos are pasted onto
the master copy and the page is copied on special bond paper in a color
xerox machine. Bring your own paper and ask that it be used --- if they
won't, go somewhere else. It is important to understand the mechanics,
because it simplifies the preparation of the paste-up master. Also, it
is easier to plan the photo reproduction, knowing that only one copy
will be lost to pasting. Color xerox copies cost less than an archival
photographic print. The necessary bound copies are about 10. In some
cases, too much detail is lost by xerographic methods and only the
paste-up technique can be used.

Mounting of monochrome photos requires dry mount tissue. Rubber cement
cannot be used in a library copy. Cibachrome color uses a special
mounting adhesive and neither dry mount tissue nor rubber cement can be
applied. In short, seek technical advice on photographs.

Discussed below is the method for paste-up of photographs with final
copies produced by color xerography. It must be emphasized that this
method applies only in limited cases where resolution loss is not
important to the photograph.

The basic method for photos is to create a framed box smaller than the
photo and write into the box important information about the photo, such
as where it came from, the size, the negative number, the date it was
taken, and other data. The boxed area will be pasted over and the
information lost, but not to you: it remains in the \LaTeX{} source.
Suggestion: use rubber cement or special cement for color plus alignment
tools to position the photo. A drafting table with T-square is helpful.

The framed box is placed into the text by a figure environment. In
Nelson Beebe's Postscript macros there is feature for doing what appears
below. After you fight the basic methods in \LaTeX{}, as done below,
Nelson's macros will make sense and you will be able to use them
effectively. An example to illustrate what is possible directly from
\LaTeX{}:
\def\Xsize{4.5} \def\Ysize{3.5} % Photo size by ruler is 4.25 by 3.25
\def\Msize{3in} % Message size 3inches
\def\Xp{0.75}   % Put at X=0.75in = (\Xsize - \Msize)/2
\def\Yp{2.5}    % Put at Y=2.5in = \Ysize - \Xp
\def\PHOTO{
 \parbox{3in}{
   \small\rm\medskip
   Taken 5-19-1981 at LBL. Nikon with strobe.
   f2.8 on Royal Pan. Negative 11 in 'LBL Trip'
   dated May 1981. I own it, no copyright troubles.
   \par\bigskip\par
   \centerline{\large\rm PHOTO 10a}
 }
}

\begin{figure}[hb]
\begin{center}
\fbox{
\setlength{\unitlength}{1in}
\begin{picture}(\Xsize,\Ysize)
\put(\Xp,\Yp){\PHOTO}
\end{picture}
}
\caption{High-Speed photograph of a bullet.}
\end{center}
\end{figure}

\footnotesize\begin{verbatim}
\documentstyle[12pt]{article}
\setlength{\textwidth}{6.4in}
\begin{document}
\def\Xsize{4.5} \def\Ysize{3.5} % Photo size by ruler is 4.25 by 3.25
\def\Msize{3in} % Message size 3inches
\def\Xp{0.75}    % Put at X=0.75in = (\Xsize - \Msize)/2
\def\Yp{2.5}    % Put at Y=2.5in = \Ysize - \Xp
\def\PHOTO{
 \parbox{3in}{ \small\rm\medskip
   Taken 5-19-1981 at LBL. Nikon with strobe. f2.8 on Royal Pan.
   Negative 11 in 'LBL Trip' dated May 1981. I own it, no copyright
   troubles. \par\bigskip\par \centerline{\large\rm PHOTO 10a} }
}
\begin{figure}[b]
\begin{center}\fbox{\setlength{\unitlength}{1in}
\begin{picture}(\Xsize,\Ysize)\put(\Xp,\Yp){\PHOTO}\end{picture}}
\caption{High-Speed photograph of a bullet.}
\end{center}
\end{figure}
\end{document}
\end{verbatim}\normalsize
\clearpage

{\bf Paste-up Figures}. Figures can be pasted up to make the final copy
of a thesis. Still, it is handy to have \LaTeX{} decide the figure
positions, numbering and captions. The ideas used above for photographs
can be applied to paste-up figures.


{\bf Computer-drawn Figures}. The best unix tool is called {\tt xfig}.
It is a drawing program for X-windows that emits source code for
\LaTeX{}, PostScript or Pic\TeX{}. The output can be inserted into \LaTeX{}
documents by an \verb"\input{filename}" command or
\verb"\inputgraphics{filename}".

The program {\tt xfig} is maintained by Brian Smith at Lawrence Berkeley
Laboratory. The 1993 version is {\tt xfig 2.1.6}, available on {\tt
export.lcs.mit.edu} in {\tt /contrib/R5fixes/xfig-patches/}. It exists
for unix machines, in particular, it runs on newer SUN hardware.

The newest 1998 versions of {\tt xfig 3.1}, also available for linux, have
the advantage of being able to load PS files. Complex figures can be
imported for the purpose of adding text, axes, labels.  The most useful
export is PS/LaTeX, which splits the PS and TEXT portions. The main advantage
of this method is the automatic use (and control) of figure fonts, which
change automatically with any change of main font in the LaTeX source file.
In particular, the size and style of figure fonts can be universally selected
by macro definition, independently of the rest of the text.

Simple PS figures, such as 2D graphs, can be imported into {\tt xfig}
and traced using spline curves (click once for each desired node), hence
producing a smaller source file that can be edited directly in {\tt xfig}.
The PS file is used only for the tracing, then discarded. The saved file
is an {\tt xfig} source, which is plain text. This is recommended for
PS 2D graphics that have inconsistencies with other figures, eg, different
axes or labels or text fonts.

Text input in {\tt xfig} for PS/LaTeX must obey the usual rules for
\LaTeX{} sources. For example, axes labels can be entered in math mode,
typing the dollar signs as part of the {\tt xfig} input.


{\bf Commutative Diagrams}. There are two packages that can be used to
produce commutative diagrams and box-style diagrams involving lines and
arrows and text. The first is called {\tt diagram}. See the sample
theses below for syntax. The second is called {\tt xypic}, by Kris Rose,
Copenhagen, Denmark. Both packages use special fonts to create the
diagrams and they can be used in various flavors of \TeX. This method of
producing diagrams is preferred over using {\tt xfig} because of
versatility and ease of use. Unfortunately, there is a steep learning
curve. A large investment of time is required to produce the first
diagram.

{\bf Maple Graphs}. Plots that can be generated by equations or
differential equations or by easily programmed algorithms can be
produced in {\tt maple}. The plot output can be chosen to be {\tt
unixplot} or {\tt Postscript}.

Postscript output can be used by the methods outlined by Nelson Beebe.
See below for a reference to his long article on the subject of
insertion of postscript figures into \LaTeX{} documents.

The {\tt unixplot} output has the advantage of being editable in {\tt
xfig}. The output file, usually called {\tt plotoutfile}, is filtered
twice and passed into {\tt xfig} for editing of labels, axes and extra
things like shading and captioning.

In the 1998 and later versions of {\tt xfig}, ghostscript is invoked to
display a postscript insertion. Therefore, the insert can be seen on the
screen, but unfortunately not edited. Most 2D postscript figures can be
traced in {\tt xfig}, using splines, and then the PS figure can be discarded,
keeping the {\tt xfig} replica, which is of course editable. For 3D figures,
the PS graphic should be stripped of all fonts and axes, then imported
into {\tt xfig}. Axes and text are then added, as \LaTeX{} constructs,
which insures consistent fonts and font sizes throughout the thesis.


To make a maple figure with the unixplot standard, proceed as follows.
First, plot a curve like
$y=\sin(x)$. The maple V.3 command is {\tt plot(sin(x),x=0..Pi)}. Save
it as a {\tt unixplot} file called {\tt plotoutfile}.
Translate from {\tt unixplot} format to {\tt pic} format with the shell
script {\tt unixplot2pic.sh}.
Then apply the filter \verb"xfig2pictex" to
\verb"plotoutfile" (.xfig extension assumed) to produce file
\verb"plotoutfile.pictex". \footnote{In you don't have access to these
tools send mail to {\tt gustafson@math.utah.edu}. Sources are in the
directory {\tt /u/ma/gustafson/xfig??}. The files are {\tt
xfig2pictex.sh} and {\tt unixplot2pic.sh}. }

The output is a coded source in \LaTeX{} macros that is suited for use
with a macro package called {\tt pictex}, made for generating figures in
\LaTeX{} by drawing a pixel at a time. It is slow but accurate and
produces figures directly into the DVI file, hence the figures can be
viewed in any good DVI screen viewer. The use of {\tt pictex} macros is
free. The manual costs money, but with {\tt xfig} no programming ability
in {\tt pictex} is required.

The resulting pictex file is inserted into a \LaTeX{} document as
follows:
\footnotesize\begin{verbatim}
Preamble:   \input{pictex.sty} % also works as a style file option

\begin{figure}[b]
  \Figure {The function $y=\sin(x)$.} {plotoutfile.pictex}
\end{figure}
\end{verbatim}\normalsize

The command {\tt Figure} itself is defined below.
Figure size is controlled by the commands in the unix shell source
\verb"xfig2pictex.sh".
Hand edit of the pictex file is possible to shrink or blow up the
figure. Figure size changes can also be done inside xfig.
\footnotesize\begin{verbatim}
\newcommand{\Figure}[2]
% #1=title
% #2=pictex file name
{% begin command \Figure
    \protect\begin{center}%
    \input{#2}%
    \protect\end{center}%
    \protect\caption{#1}%
}% end of command \Figure
\end{verbatim}\normalsize\rm


The editing features of {\tt xfig}, which adds \LaTeX{} commands over
the top of figures, is invaluable. After getting used to the environment
your productivity should increase and the quality of your figures should
better than ever before.

{\bf Maple Postscript Output}.
Below are some examples for controlling the maple postscript output,
to remove the default labels, default tick marks, the border on the
outside of the plot and to change the orientation to portrait. Finally,
recommendations on fixing the plot output size and thickness of lines,
fonts and title.

\begin{verbatim}
f:=x->sin(x):
xmarks:=[1=`1.0`,2=`2.0`,3=`3.00`]:
ymarks:=[0.5=`00.5`,1.0=`1.00`]:
plot(f,0..2*Pi,
  style=line,
  thickness=3,
  tickmarks=[xmarks,ymarks],
  axesfont=[TIMES,BOLD,16],
  titlefont=[TIMES,BOLD,30],
  title=`sin(x) plot`);

This plots f(x) without x and y labels.
The line thickness is wider, for reproduction.
The tick marks are set by hand. Gets rid of overlapping ticks at (0,0).
The tick marks appear in a none-default font in a bolder and larger style.
The title is specified, but in a larger and different font.
  See ?plot,options for other FONTS, STYLES, SIZES.

Other options:

    numpoints=100       Smoother curves
    resolution=300      Smoother curves
    linestyle=2         Dashed line for second plot
    linestyle=3         Another kind of dashed line for third plot, etc

Controlling the output:

  with(plots):
  setoptions(style=line, thickness=3,
             axesfont=[TIMES,BOLD,16], titlefont=[TIMES,BOLD,30],
             title=`sin(x) plot`);
  plot(f,0..2*Pi);

    This allows you to set all the defaults so they don't appear in the
    plot commands. Very handy dith DISPLAY from plots package.
    Unfortunately, tickmarks=[xmarks,ymarks]
    cannot be set this way. It has to be in each plot command!

  plotsetup(ps,plotoutput=`plot.ps`,plotoptions=`portrait,noborder`);

    Select postscript output instead of x11 screen output.
    This sets the plot output to the file "plot.ps".
    Output will be portrait format instead of the default landscape.
    The border is removed from the plot.

  plotsetup(x11);

    Sets the plot back to the screen. Undoes all the above.

  plotsetup(ps,plotoutput=`plot.ps`,
            plotoptions=`portrait,noborder,height=288,width=144`);

    Select postscript output instead of x11 screen output.
    This sets the plot output to the file "plot.ps".
    Output will be portrait format instead of the default landscape.
    The border is removed from the plot.
    The size of the plot will be 288/72 inches by 144/72 inches in
    portrait format and 144/72 inches by 288/72 inches in landscape
    format.

    Font sizes are not scaled. Any special fonts stay full size.
    If you intend to have fonts in the figure, then be prepared to add
    them separately in XFIG. That will cure all the problems of font
    consistency, size and location on the graphic.

\end{verbatim}


{\bf Postscript Figures}. It is possible to directly insert Postscript
figures into \LaTeX{} documents. Many microcomputer programs produce
Postscript output and therefore it may be an advantage to insert this
type of figure directly. Some unix programs which can create the
postscript images: {\tt gnuplot, xfig, idraw, freehand, matlab, xmath,
maple} and {\tt mathematica}.

Many theses in mathematics have successfully used postscript figure
insertions. If there is a standard, then possibly it is postscript. This
standard is not without problems. The biggest trouble is the size of
labels and the pen size for plotting. Letters have to be 2mm high to
pass the microfilm test (less than 2mm will vanish on the microfilm).
The xerox test is applied to the pen size: it must not lighten or vanish
upon being Xeroxed. If your figures are Postscript, then choose a large
point size for the fonts to insure that reduction leaves the letters 2mm
high. For example, 14pt \verb"\large" fonts reduced 50\% become 7pt
fonts, which is almost exactly 2mm high. Recommended is the use of {\tt xfig}
to add labels in \LaTeX{} mode. It is important to keep consistent fonts
and font size, but PS fonts are from the outset inconsistent with all
the \LaTeX{} fonts.

Nelson Beebe has written an extensive article on insertion of Postscript
figures, called {\em \TeX{} and Encapsulated Postscript}. Since this
article is published and also updated periodically with new ideas, it is
best to contact Nelson at {\tt beebe@math.utah.edu}. The 1991 version
0.38 of the article was 44 pages (with index and figures).

Nelson recommends reading first the short article (5 pages, 9 figures)
of December 1992 called {\em Incorporating Encapsulated PostScript in
\TeX{} and \LaTeX{} Documents}, available as {\tt exepsf.ps} in math
directory {\tt /u/cl/doc} or email a request to Nelson, {\tt
beebe@math.utah.edu}.

{\bf NCAR Graphics}. The graphs produced by NCAR can be exported to
Postscript format. The remarks of Nelson Beebe in his article on
Postscript Figures apply to the exported figure files. See above for
information on that article. Presently, there is no export of NCAR
graphics files that can be filtered into {\tt xfig}. Therefore, editing
of the graphics is limited to available tools for postscript figures.

{\bf Editing Postscript Figures}.
There was an announcement of editing tools for postscript sources by
\begin{quote}
 Craig Barratt, \verb"craig@ISL.Stanford.EDU".
\end{quote}
The package
{\tt PsFrag} is available via anonymous ftp from {\tt isl.stanford.edu}
(internet address 36.60.0.10).  Get the compressed tar file
{\tt  pub/boyd/psfrag/psfrag.tar.Z}.
Extract the archive using {\tt pdtar xzf}.
See the files {\small\tt README, USAGE, INSTALL} for detailed
information. The package is a set of shell scripts that employ {\tt
dvips} and {\tt ghostscript} to replace postscript text items by
\LaTeX{} code that overlay the postscript image.

The advent of this same feature in {\tt xfig} makes the psfrag package
nearly useless. Nevertheless, there are features in psfrag that are
not in {\tt xfig}.


{\bf Gnuplot Figures}. The program {\tt gnuplot} can produce publication
quality figures with \LaTeX{}, Postscript or {\tt xfig} output. The
latter is best if you want to make a quick figure and then edit it in
{\tt xfig} to add labels, captions and change things like the line size,
axes and fonts. Everything that {\tt gnuplot} produces can be edited
manually in {\tt xfig}.

There is a document available on {\tt gnuplot} figures called {\em
\LaTeX{} and the GNUPLOT Plotting Program} by David Kotz, {\tt
dfk@cs.duke.edu}, February 10, 1990. The \LaTeX{} file that produces the
documentation is available at a number of sites on internet. Send mail
to {\tt info-gnuplot@dartmouth.edu} (from version 3.5, 1993). This
document is included in the {\tt Tutorial} directory of the gnuplot
distribution.

Graphs created by gnuplot can be exported in \LaTeX{} format for direct
inclusion in a figure environment.
For example, if you want $y=\sin(x)$ in your thesis try the following:
\begin{verbatim}
 (terminal window): gnuplot
 (in gnuplot):
                 set terminal latex
                 set output "sin.tex"
                 plot sin(x)
                 quit
\end{verbatim}
The resulting file {\tt sin.tex} can be inserted into a figure
environment with the command \verb"\input{sin.tex}". The figure size can
be changed by editing {\tt sin.tex} on the line containing {\tt
unitlength}.

Gnuplot figures can be produced from raw data files of
$(x,y)$-values. Consider the following command file
\verb"fig1.2.1.gnuplot" invoked as
\begin{quote}
 \verb"gnuplot fig1.2.1.gnuplot"
\end{quote}
in a terminal window:
\begin{verbatim}
set nokey
set data style lines
set tics out
set xrange [0:24]
set yrange [120000:240000]
set xtics 0,12
set ytics 120000,60000
plot "fig1.2.1.data"                            View result in X
pause -1                                        Wait for key
set term bfig                                   Output large xfig file
set output "fig1.2.1.fig"
replot
set output                                      Back to X11 output
!xfig fig1.2.1.fig                              Run xfig and
fig2dev -L pictex fig1.2.1.fig fig1.2.1.pictex  Make pictex file
!emacs fig1.2.1.pictex fig1.2.1.gnuplot         Hand edit sources
\end{verbatim}

{\bf Graphs from data using maple}. The computer algebra system {\tt
maple} can make a figure from a given data file by writing the data into
a list {\tt L} and then invoking {\tt plot([L])} to produce the graph.
Virtually everything done in gnuplot above can be reproduced by a maple
script. The real problem is that neither maple nor gnuplot provide any
editing capabilities --- a program like {\tt xfig} is still needed along
with translators and filters for import and export of the various
graphics formats.

\section{More Features}

\subsection{Additional Commands}

Some additional convenience commands have been defined.

\verb|\begin{epigraph}| {\em words} \verb|\end{epigraph}|

\noindent Creates an epigraph.
An epigraph is a quotation or motto at the beginning of a book, a
chapter or section. The text will be indented and single-spaced, but the
rest of the formatting (such as a blank line between the text and
attribution) must be done by hand. Overdoing the number of epigraphs
will cause you to be known as an {\em epigraphist}, one versed in {\em
graffiti}.

\medskip

\verb"\begin{topics}{template} ... \end{topics}"

\noindent This special environment can be used to solve display problems
that might otherwise use {\tt tabular} or {\tt array} constructions. It
is similar to the {\tt itemize}, {\tt description} and {\tt enumerate}
environments which have the special feature of allowing page breaks
between items. The definition:

\begin{center}\begin{tabular}{l}
\verb"\newcommand{\topicslabel}[1]{\mbox{#1}\hfil}" \\
\verb"\newenvironment{topics}[1]{" \\
\verb"    \begin{list}{}{%" \\
\verb"      \let\makelabel\topicslabel" \\
\verb"      \settowidth{\labelwidth}{\topicslabel{#1}}" \\
\verb"      \setlength{\leftmargin}{1.1\labelwidth}}" \\
\verb"}{\end{list}}"
\end{tabular}\end{center}

The special feature of the {\tt topics} environment is seen in the
example below. The alignment of items cannot be done in the usual list
environments. But the new feature has a price: the longest item must be
entered as a template at the start of the environment (similar to the
{\tt tabbing} environment). The template has a natural length that is
used as the item width (see the above definition) and therefore custom
widths can be designed for the item column which will persist and be
uniform from page to page. The \verb"\item[...]" is never optional: the
square brackets must occur even if no text is to be entered!


\newcommand{\topicslabel}[1]{\mbox{#1}\hfil}
\newenvironment{topics}[1]{
    \begin{list}{}{%
      \let\makelabel\topicslabel
      \settowidth{\labelwidth}{\topicslabel{#1}}
      \setlength{\leftmargin}{1.1\labelwidth}}
}{\end{list}}

\begin{minipage}{2.5in}\footnotesize
\begin{verbatim}
 \begin{topics}
       {\bf Crocodiles and Reptiles.}
 \item[\bf How X works:]
    \begin{topics}{\sf Windows.}
    \item[\sf Windows.]
         Usually made of glass but
         X-windows are an exception.
    \item[\sf Errors.]
         They appear in the console
         window from time to time.
    \item[\sf Exit.]
         For the twm manager, exit
         by pull-down menu option.
    \end{topics}
  \item[\bf Alligators.]
       In many places in the world
       there can be found
       crocodiles and alligators.
  \item[\bf Crocodiles and Reptiles.]
       Besides crocodiles and alligators
       there are brown and green lizards.
 \end{topics}
\end{verbatim}
\end{minipage}
%
\hfill
\fbox{\parbox{3.5in}{\footnotesize
 \begin{topics}
       {\bf Crocodiles and Reptiles.}
 \item[\bf How X works:]
    \begin{topics}{\sf Windows.}
    \item[\sf Windows.]
         Usually made of glass but
         X-windows are an exception.
    \item[\sf Errors.]
         They appear in the console
         window from time to time.
    \item[\sf Exit.]
         For the twm manager, exit
         by pull-down menu option.
    \end{topics}
  \item[\bf Alligators.]
       In many places in the world
       there can be found
       crocodiles and alligators.
  \item[\bf Crocodiles and Reptiles.]
       Besides the crocodiles and alligators
       there are brown and green lizards.
 \end{topics}
}}


\medskip

\verb|\noisytrue|, \verb|\noisyfalse|

\noindent Enables and disables the printing of progress messages. This
is useful for debugging if front matter is coming out wrong, or if you
like programs to give out reassuring patter about what they're up to.
The default is \verb|\noisytrue|.

\medskip

\verb|\begin{singlespace}| {\em words} \verb|\end{singlespace}|

\noindent This environment allows single-spaced text to be inserted into
the thesis at any point. Use it with caution: the Thesis Manual is very
specific about double-spaced text, which is the default in this style. A
warning: a control \verb"\par" is required before invoking single
spacing and directly after as well. This can also be done by blank
lines. Most complaints about single spacing can be resolved by inserting
blank lines before and and the environment.


\medskip

\verb|\singlespace|\\
\verb|\doublespace|\\
\verb|\normalspace|

\noindent In some circumstances, the singlespacing environment may not
work. It usually requires a \verb"\par" control before it is used and
immediately after before changing to other spacing. Each of these
commands changes the inter-line spacing for everything up to the next
space-setting command. \verb|\singlespace| and \verb|\doublespace| do
the obvious things, while \verb|\normalspace| sets to the default
spacing (double for theses, single for tech reports).  For this reason,
\verb|\normalspace| is preferable to \verb|\doublespace|, when you only
want to back to the usual spacing.

\medskip

\verb|\twopagefigure{#1}{#2}|

\noindent
This special command is made for inserting photographic paste-ups; the
first page gets the caption, and the second is completely blank. Its
arguments are the figure caption and a label. The figure caption should
contain a label also in order to build a list of figures entry. Its
definition may be helpful to understanding how it is to be used:

\begin{verbatim}
\newcommand{\twopagefigure}[2]{
\begin{figure}[p]
\vspace{4in}
\caption{#1}
\label{#2}
\vspace{4in}
\end{figure}
\begin{figure}[p]
\vspace{8in}
\end{figure}
}
\end{verbatim}


\verb|\theoremsetup|

\noindent This setup control defines theorem-type environments {\tt
theorem}, {\tt proposition} and {\tt corollary} in a default manner. Put
the control into the preamble of the document if you wish to enable the
feature. Remove the reference to disable the loading of these default
definitions. Warning: once defined, there is no convenient way to erase
the definitions.

The reason for the control \verb"\theoremsetup", clumsy as it seems, is
to allow for custom definitions of {\tt theorem}, {\tt proposition},
{\tt corollary}. The \verb"\newtheorem" feature disallows a definition
to be turned off or re-defined. A common tweak of the theorem
environment is to insert Roman text in place of italics into the theorem
body. Of course, it is desirable to call the result {\tt theorem} and
not some other name.

\medskip

\verb|\begin{theorem}| {\em words} \verb|\end{theorem}|

\noindent This theorem environment follows the suggestion in the
\LaTeX{} manual. It defines a new counter called {\tt theorem} which is
reset in each section and incremented upon each use of the environment.
See the control \verb|\theoremsetup| above to enable this environment.
New environments can be made from a base theorem environment:

\medskip

\verb|\newenvironment{Theorem}{\begin{theorem}\rm}{\end{theorem}}|

This environment, used in place of {\tt theorem}, has all the features
of the theorem environment but sets the text body in Roman. The {\tt
Theorem} environment is {\em not defined} in the uuthesis style --- it
is a customization to be coded into the preamble.


\medskip

\verb|\begin{proposition}| {\em words} \verb|\end{proposition}|

\noindent Like the {\tt theorem} environment, this pre-defined
environment follows the standard use of propositions and theorems, with
the text of the proposition set in italics. The counter used is the {\tt
theorem} counter: there is no separate counter for propositions. See
\verb|\theoremsetup| above to enable this environment.

\medskip

\verb|\begin{corollary}| {\em words} \verb|\end{corollary}|

\noindent Like the {\tt theorem} environment, text is set in italics.
The same counter is used: {\tt theorem}. See the control
\verb|\theoremsetup| above to enable this environment.

\noindent Other inventions commonly used for a thesis can be included in
the preamble of the document. Here are some normal definitions, which
should be self-explanatory. Please observe that the items below are {\bf
not pre-defined}! To make them useful, enter them into the preamble of
the \LaTeX{} document.

\begin{verbatim}
\newtheorem{theorem}{Theorem}[chapter]          The common definitions
\newtheorem{proposition}{Proposition}[chapter]  enabled by control
\newtheorem{corollary}[proposition]{Corollary}     \theoremsetup

\newtheorem{lemma}[theorem]{Lemma}
\newtheorem{define}[theorem]{Definition}
\newtheorem{hope}[theorem]{Conjecture}
\newtheorem{example}[theorem]{Example}
\newtheorem{obs}[theorem]{Observation}

\newtheorem{lemma}{Lemma}[section]
\newtheorem{co}[lemma]{Corollary}
\newtheorem{df}[lemma]{Definition}
\newtheorem{example}[lemma]{Example}
\newtheorem{question}[lemma]{Open Question}
\newtheorem{remark}[lemma]{Remark}
\newtheorem{th}[lemma]{Theorem}

\renewcommand{\thelemma}{\arabic{section}.\arabic{lemma}}
\renewcommand{\theequation}{\arabic{section}.\arabic{equation}}
\end{verbatim}

\noindent The font used for these environments is by default italic. If
you want a normal roman font, then new environments must be defined ---
see the \LaTeX{} manual for details.

\medskip

\noindent
Rudolf A. Roemer has suggested a clever macro for inserting
equation numbers in the form, e.g., (1.2a), (1.2b).
\begin{verbatim}
% Put this into your "thesis.sty" file
\newcounter{eqletter}\newcounter{eqdummy}%
\newenvironment{mathletters}%
{\setcounter{eqletter}{0}\refstepcounter{equation}%
\setcounter{eqdummy}{\value{equation}}\setcounter{equation}{\value{eqletter}}%
\renewcommand{\theequation}{\arabic{chapter}.\arabic{eqdummy}\alph{equation}}%
}%
{\setcounter{equation}{\value{eqdummy}}%
\renewcommand{\theequation}{\arabic{chapter}.\arabic{equation}}}%
\end{verbatim}

\long\def\YY{
\begin{minipage}{3.8in}
\noindent
An example of how to use Rudolf Roemer's macro appears below. The
\LaTeX{} source appears on the right.\par
\begin{mathletters}
\label{eq-all}
\begin{equation}
2T+4X=\pi.
\label{eq-veryfirst}
\end{equation}
\begin{eqnarray}
T &= &2\pi \int_0^1 f(t)dt,
 \label{eq-first} \\
X &= &2\pi \int_0^1 g(t)dt.
 \label{eq-last}
\end{eqnarray}
\end{mathletters}\par
\noindent
First, there is equation \ref{eq-first}, preceded by equation
\ref{eq-veryfirst}, then there is equation \ref{eq-last},
and they are all part of equation \ref{eq-all}.
\end{minipage}
}

\newcounter{eqletter}\newcounter{eqdummy}%
\newcounter{chapter}\setcounter{chapter}{1}
\newenvironment{mathletters}%
{\setcounter{eqletter}{0}\refstepcounter{equation}%
\setcounter{eqdummy}{\value{equation}}\setcounter{equation}{\value{eqletter}}%
\renewcommand{\theequation}{\arabic{chapter}.\arabic{eqdummy}\alph{equation}}%
}%
{\setcounter{equation}{\value{eqdummy}}%
\renewcommand{\theequation}{\arabic{chapter}.\arabic{equation}}}%

{\YY}
\hfill
{
\noindent
\begin{tabular}{@{}p{2.4in}@{}}
\verb|\begin{mathletters}| \newline
\verb|\label{eq-all}| \newline
\verb|\begin{equation}| \newline
\verb|2T+4X=\pi.| \newline
\verb|\label{eq-veryfirst}| \newline
\verb|\end{equation}| \newline
\verb|\begin{eqnarray}| \newline
\verb|T &= &2\pi \int_0^1 f(t)dt,| \newline
\verb| \label{eq-first} \\| \newline
\verb|X &= &2\pi \int_0^1 g(t)dt.| \newline
\verb| \label{eq-last}| \newline
\verb|\end{eqnarray}| \newline
\verb|\end{mathletters}| \newline
\end{tabular}
}
\par


\medskip

\verb|\begin{Proof} ... \end{Proof}|

\noindent The  Proof environment is by Nelson Beebe.
It uses normal Roman text, with a trailing
black square at the end of the text, supplied automatically
by the \verb|\end{Proof}|. The definition:

\begin{verbatim}
\newcommand{\boxx}{\unskip \nopagebreak \hfill ${\blacksquare}$% }
\newenvironment{Proof}%
    {\noindent{\bf Proof.}\begin{rm}}%
    {\boxx\end{rm}}
\end{verbatim}

\verb|\proofline{#1}|

\noindent Insert a line of text used to identify a proof. It can be used
also to invent a proof environment, which might be more convenient. The
definition allows for easy re-definition:

\begin{quote}
\verb|\def\proofline#1{\protect\par\noindent{\bf #1}:~}|
\end{quote}

\medskip

\verb|\proof{#1}|

\noindent Inserts a proof line of the form ``{\bf Proof \verb|#1|}:''
which is convenient for items of the form {\bf Proof of Theorem 7:}. The
definition allows it to be redefined easily:

\begin{quote}
\verb|\def\proof#1{\proofline{Proof #1}}|
\end{quote}

\medskip

\verb|\pf|

\noindent Inserts {\bf Proof:} into the text with no indent. This is a
common way of marking proofs. The definition:

\begin{quote}
\verb|\def\pf{\protect\proofline{Proof}}|
\end{quote}

\medskip

\verb|\qed|

\noindent This control sequence is used to end a proof. It has the following
definition in the uuthesis style:
\begin{quote}
\verb|\def\qed{\protect\par\noindent{}$\square$\protect\par}|
\end{quote}
Other possible definitions, placed in the preamble, would be:
\begin{quote}
\verb|\def\qed{\nobreak \hfill \rule{1ex}{1em}}|\\
\verb|\def\qed{\nobreak\ \mbox{~~\rule{1ex}{1em}}}|
\end{quote}

\medskip

\verb|\mainheading{#1}|

\noindent The two-inch top margins that you see in the thesis are
produced by this control sequence. It may be useful in solving some
formatting problems. It is best used after an \verb|\eject| or
\verb|\clearpage| command, so that the two inches of space is at the top
of the page. The argument is the text that you want centered in
\verb|\large\bf| print, just below the 2-inch white space. This command
actually uses a special font size \verb|\HFmainhead| with baselineskip
\verb|\HFmainheadHT|, basically large boldface. The text is set in a box
of width \verb|\mainheadingwidth|, normally 4.25in.

\medskip

\verb|\ulabel{#1}{#2}|

\noindent This feature supplies an alternative to the \verb|\label|
control sequence. Its purpose is to prefix entries in the AUX file with
a string given in the second argument \verb|#2|. An example showing how
{\tt ulabel} works:
\begin{verbatim}
Example:
         \begin{theorem}% Theorem 2.4
         \ulabel{hottheorem}{Theorem}
          The group ${\cal G}$ is simple.
         \end{theorem}
\end{verbatim}
Later on, \verb"\ref{hottheorem}" produces string \verb"Theorem 2.4"
because \verb"\thetheorem" equals \verb"2.4". The usual
\verb"\label{hottheorem}" supplied by \LaTeX{} would produce only
\verb"2.4" in the AUX file and \verb"\ref{hottheorem}" in this case
would be \verb"2.4" instead of \verb"Theorem 2.4".


\noindent Greg Conner used this control sequence in 1992 to reference
theorems and corollaries. The advantage is that references can be tied
to the particular theorem, corollary or lemma by words and number,
resulting in a consistent system of references.


\noindent The definition:

\begin{verbatim}
\def\ulabel#1#2{\@bsphack\if@filesw {\let\thepage\relax
   \def\protect{\noexpand\noexpand\noexpand}%
   \xdef\@gtempa{\write\@auxout{\string
   \newlabel{#1}{{#2 \@currentlabel}{\thepage}}}}}\@gtempa
   \if@nobreak \ifvmode\nobreak\fi\fi\fi\@esphack}
\end{verbatim}

\medskip

\verb|\inputpicture|

% Nov 1992
\noindent Jeff McGough used the control \verb|\inputpicture| in November
1992 to input figure files in his thesis.  He found that complicated
figures slowed the evaluation process considerably and it became
convenient to compile the DVI file with {\em phantom figures} in place
of the real ones. The {\em phantom figure} was a box as seen below made
with this \LaTeX{} source:

\begin{minipage}{2.3in}
\footnotesize\rm
\verb|\setlength{\unitlength}{0.0063in}%| \\
\verb|\begin{picture}(573,357)(114,408)| \\
\verb|\thicklines| \\
\verb|\put(114,408){\framebox(573,357){}}| \\
\verb|\end{picture}|
\end{minipage}
\hfill
\begin{minipage}{4in}
\setlength{\unitlength}{0.0063in}%
\begin{picture}(573,357)(114,408)
\thicklines
\put(114,408){\framebox(573,357){}}
\end{picture}
\end{minipage}

\bigskip


The generation of figures is controlled by
\begin{quote}
\verb|\newcommand{\inputpicture}[1]{%| \\
\verb|   \ifinputpicture \input{#1} \else % GNUPLOT: LaTeX picture made by gnuplot.
\setlength{\unitlength}{0.240900pt}
\ifx\plotpoint\undefined\newsavebox{\plotpoint}\fi
\sbox{\plotpoint}{\rule[-0.175pt]{0.350pt}{0.350pt}}%
\begin{picture}(1500,900)(0,0)
\tenrm
\sbox{\plotpoint}{\rule[-0.175pt]{0.350pt}{0.350pt}}%
\put(264,158){\rule[-0.175pt]{282.335pt}{0.350pt}}
\put(264,158){\rule[-0.175pt]{282.335pt}{0.350pt}}
\put(1436,158){\rule[-0.175pt]{0.350pt}{151.526pt}}
\put(264,787){\rule[-0.175pt]{282.335pt}{0.350pt}}
\put(264,158){\rule[-0.175pt]{0.350pt}{151.526pt}}
\sbox{\plotpoint}{\rule[-0.350pt]{0.700pt}{0.700pt}}%
\end{picture}
 \fi }| \\
\verb|\newif\ifinputpicture \inputpicturetrue|
\end{quote}
where file {\tt box.tex} contains the contents given above.

To use the \verb|\inputpicture| control, enter after the beginning of
the document one of these two commands:
\begin{quote}
\verb|\inputpicturetrue| \\
\verb|\inputpicturefalse|
\end{quote}

To input a picture file \verb|pic.tex| use the following syntax:
\begin{quote}
\verb|\inputpicture{pic.tex}|
\end{quote}
The picture will appear in the DVI file if the
control sequence \verb|\inputpicturetrue|
was
set and a box will appear in the DVI file if \verb|\inputpicturefalse|
was set.


\subsection{Other Documents}

The thesis itself is not the only document that you may need to produce.
It is useful to have a file \verb|forms.tex| which has all the
declarations of the regular thesis, but only generates, say, the
committee approval page, so it can be printed on bond paper and signed
before the thesis goes through Format Approval.

It may be desirable to use a title page for the abstracts that are
submitted separately; \verb|\abstracttitlepage| works just like
\verb|\titlepage|, but produces a slightly different output.

Another possibility is for the thesis to be turned into a departmental
technical report.  This is required in Computer Science.  However, the
thesis format does not follow the report format, so as mentioned
earlier, the \verb|report| option to the style makes appropriate changes
(there are really only a few that need to be made). The most significant
change is the use of a different title page, which is generated by the
command \verb|\reporttitlepage|. The default definition of the command
is specific to Computer Science, and it should be redefined if used
elsewhere.

\section{Warnings and Problems}

The uuthesis style will not correct forms of references, spelling
mistakes, or poor writing.  Proofread everything.

A common complaint is inconsistency of use. An example: in the
introduction the terminology {\bf 4--dimensional} used is whereas later
on it becomes {\bf four--dimensional}. While not a difficulty for
American mathematicians, it is definitely a problem for persons whose
native language is not English. A thesis is an international document,
and therefore the manuscript must pass a consistency test of this
nature. Nelson Beebe recommends using a control sequence for each piece
of special terminology, to insure consistency throughout the document.

Double words are a common defect in theses. These can be eliminated by a
unix program {\tt dw} that checks for this situation. See the sample
thesis sources in {\tt /usr/local/lib/tex/uuthesis} on mathematics
department unix machines or send email to Nelson Beebe.

Missing words are another common problem. Careful proofing of text is
the only sure cure for such defects.

Placement of punctuation is determined by the style guide. It is to be
consistent. Equations at the end of a sentence should end with a period,
in general. Commas often appear at the end of a displayed equation.
Quotations should be treated as per the style guide: sometimes they get
quote marks and sometimes not. Italics and periods are used according
to the dictionary: is it `et al.', `{\em et al.}', `et.\ al.' or `{\em
et.\ al.}? Hyphenation is used according to the dictionary or a style
guide but must be consistent throughout the text.

Overuse of certain words is a common thesis fault that can be reduced
by using a dictionary and a word count program. For example, the word
{\bf note} or phrase {\bf note that} tend to be used too much. Other pet
combinations known to irritate readers are {\bf because} and {\bf
since}, essentially the same in mathematics, but not so for persons
trained in other disciplines. A thesaurus can be useful for finding other
combinations of words that replace the troublesome words.

A spelling program like {\tt ispell} on the Unix systems is a safeguard
against simple typographical errors. Other spell programs exist on
microcomputer hardware. Most are useful, especially WordPerfect and
MS-Word spelling checkers.

The most commonly misused controls are

\begin{quote}
\verb"\eject" \\
\verb"\newpage" \\
\verb"\clearpage" \\
\verb"\vspace" \\
\verb"\vskip" \\
\verb"\hskip" \\
\verb"\hspace" \\
\verb"\\" \\
\verb"\\[2pt]" \\
\verb"\mbox{...}" \\
\verb"\smallskip" \\
\verb"\medskip" \\
\verb"\bigskip"
\end{quote}

As a general rule, leave hard space controls out of the document! Use
environments and control sequences wherever possible. The {\tt newpage}
command creates a document that the thesis office will have to reject:
the page heights are to be identical.

Use \verb"\clearpage" only to cause a dump of a page of floats (figures
or tables entered with option \verb"[p]") or to force a figure dump at
an important point. This command is a page-filling command, but it can
backfire and cause figures and tables to end up on the wrong page.

Avoid using font size \verb|\tiny|. It uses 7pt font size, which is
right at the reproduction limit of 2mm for microfilm. Small subscripts
in math mode can be increased in size by the use of
\verb"\displaystyle". Commands like

\verb"        \def\dd{\displaystyle}"

as often seen in the preamble to decrease the typing of math mode
equations, especially in the {\tt array} environment, where compressions
always seem to appear.

{\bf Error messages}.
There is a requirement that all section and chapter headings must be
less than 4.5 inches long.  This requirement has been met for the
\verb|uuthesis| style, technically, but you can enter a longer title,
and hence explicit line breaks are required to make the title into an
inverted pyramid.

The errors in this case show up in the preview but also in error
messages during the \TeX{} compile, especially when the TOC is input.
Please {\em don't edit the TOC}! It is recreated every time \TeX{} runs.
The place to edit is the sectioning commands deep inside the \LaTeX{}
document, where probably no error message is emitted. The fix is to
insert an optional argument \verb"[...]", for example,

\verb"\section{Main results on normal subgroups}"

becomes

\verb"\section[Main results on reduction \protect\\ of normal subgroups]"\\
\verb"        {Main results on reduction of normal subgroups}"

It takes two runs of \LaTeX{} to put the new TOC into the DVI output. On
the second run the error given during read of the TOC should vanish.

The control \verb"\def\mainheadingwidth{4.25in}" is standard and you
may have to define it as 4.5in or 4.8in to make a particular title into
an inverted pyramid. After such exceptions, it is important to set it
back to 4.25in prior to typesetting the next title.

The entry in the table of contents may have to be crafted independently
of the actual title in order to create line breaks that are acceptable.
This is done by the optional argument to a sectioning command:

\footnotesize
\verb"\chapter[Isomorphisms and Polymorphisms Revisited:"\\
\verb"         \protect\\A Survey]"\\
\verb"        {Isomorphisms and Polymorphisms"\\
\verb"         \protect\\ Revisited: A survey}"
\normalsize

The sectioning commands that take the optional argument in square
brackets are {\tt part}, {\tt chapter}, {\tt section}, {\tt subsection},
{\tt subsubsection}, {\tt paragraph}. In addition, most commands that
automatically enter data into the table of contents also take a square
bracket optional argument, for example, {\tt caption} in the table and
figure environments.

The coding of chapter headings followed immediately by a section head is
controlled by control
\begin{quote} \verb|\fixchapterheading|.\end{quote}
\noindent If you did not read these remarks above, then do so now.

Captions on figures and tables have been revised several times. Long
captions are centered, and the \verb|\protect\\| command can be used in
both the TOC and TITLE arguments.  Long captions are completely
feasible, but explicit line breaks may be required for extra long
titles. There are two styles for captions: paragraph and centered.

Default figure/table placement can be wrong. \LaTeX{} tries to put the
figure/table on the same page following the reference with option
\verb"[b]". Use of the \verb|[t]| option after previewing is usually
sufficient to move the figure to the next page, if it didn't fit on the
right page to begin with. It helps to have the figure follow the
reference as close as possible (i.e., within the paragraph, right after
the sentence with the reference). Sometimes \verb"\clearpage" is needed
to force a figure onto the right page, especially at the end of a
chapter.

It is required by the thesis office that a figure appear {\bf after} it
is cited, and never before the citation, even at the top of the present
page. So sometimes option \verb"[t]" doesn't work and it has to be
changed to \verb"[b]" or \verb"[p]" or \verb"[h]".

The \verb|figure*| and \verb|table*| environments have been removed,
since the format does not support double column printing in the first place.

Double and other spacing in tables may be gotten using \verb|\arraystretch|.
Default is for single spacing.  Table format does require at least one
horizontal line between caption and body; use \verb|\hline|.

This style has been used with a number of different laser printers,
including Imagens, Laserwriters, and Laserjets.
The different engines used by
these printers may cause the margins to differ when printed out on
two different printers by as much as 2-3 character widths. This may be
corrected by issuing print commands with offsets (if that is
possible), or by overriding the default margins.

The signature pages required for theses and dissertations are formatted
directly here, although you can get them from the thesis editor. These
forms are correct now, but if the format of the signature pages changes,
this style will have to be modified, or else you'll have to get the
forms from the editor. Correspondence on changes is most appreciated
(see below for contacts).

\section{Bug fixes and improvements}
The initial style files and documents were produced by Stan Shebs in
1987. Several improvements were incorporated with bug fixes until
version 1.2 in 1988, when the style files stabilized. Since that time,
only minor fixes have been added. The most recent, by Nelson Beebe,
1990-1993, and Conner and Gustafson, 1992. A major revision took place
in October, 1993 by Gustafson. The fixes:

\begin{quote}
Rewrote APPROVAL pages to match the Handbook.\\
Rewrote 11pt and 12pt options, eliminated uut11.sty, uut12.sty. \\
Added {\tt topics} environment. \\
Made special font sizes for all sectioning and header commands. \\
Built commands for singlespaced and doublespaced titles. \\
Rebuilt threelevels, fourlevels, fivelevels. \\
Rebuilt parameters for sectioning commands to match thesis manual. \\
Rebuilt {\tt chapter} and {\tt part} commands to correct TOC entries. \\
Fixed maintext command to be useful everywhere (standardized). \\
Fixed Part command so page numbering is correct. \\
Fixed figure problems with positioning. \\
Fixed figure captions and added paragraph/centering options. \\
Isolated theorem control sequences to make them selectable. \\
Added control sequences for chapter heading width. \\
Changed TOC entry indent to 0in. Fixed TOC entry widths. \\
Appendix controls rewritten. Added {\tt noappendix} control. \\
Rewrote documentation, internal and external. \\
Add double spacing for multi-line chapter titles (UofU
thesis office request 1992).\\
Fixed REFERENCES spacing, ABSTRACT spacing. \\
Paragraph style parameters now have glue removed.\\
Proof environment added. Theorem environments added. \\
Titlepage repairs. \\
Appendix repairs. \\
Caption repairs.\\
Added {\tt csreport} and {\tt honors} options.
\end{quote}

If you find something broken in the style file, or find a severe
limitation, then please send email to one of the names below, so that
the changes and fixes get applied to the master files.

Contributors for bug fixes in 1993, 1994: Bill Brimley, Rudolf Roemer,
Rick Gee, Elisabeth Manderscheid and Eric Eide. Rudolf contributed some
clever macros for numbering math equations in the form, e.g., (1.2a),
(1.2b). These were not implemented in uuthesis, because of their special
nature, but have been placed in this document instead, for reference.
Rick Gee contributed to spacing fixes in the LOF and TOC. Elisabeth
identified the pieces needed for an honors thesis. Eric attacked the
problem of widow section titles and \verb"\vbox" controls. The actual
problem was cured by adding a {\tt nopagebreak} control into the
sectioning commands. No one has found an easy solution for the problem
of keeping segments of text together on the page.

The \LaTeX{}2e version of {\tt uuthesis} was initially created by Ken
Parker (Computer Science) in May of 1995. The sources remained privately
used for 1995 to 1997. In January 1998, the sources were activated for
general use by anyone who wanted to experiment with \LaTeX{}2e.


\section{Author addresses}
If you have problems with this format, or suggestions for improvements
and features, contact (in this order):
\begin{center}\footnotesize
\begin{tabular}{lll}
Nelson Beebe & Math, 581--5254 & \verb|beebe@math.utah.edu|, 1998 \\
Grant B.\ Gustafson & Math, 581--6879 & \verb|gustafso@math.utah.edu|,
1998\\
Leigh B.\ Stoller & CS, 585-3733 & \verb|stoller@cs.utah.edu|, 1995 \\
Stan Shebs & Apple (CA) & \verb|shebs@cygnus.com|, April 1995 \\
\end{tabular}
\end{center}

\section{Acknowledgments}

The \verb|uuthesis| style file has passed through many hands.
It seems to have started life both as the standard \LaTeX\ report style
file \verb|report.sty| by Leslie Lamport, and as a Stanford thesis format
\verb|suthesis.sty|, done by an unknown person at Stanford. Mike Jordan
adapted
it to the University of Utah style and wrote the format of signature pages.
Mark Watkins did some cleanup, and added macro definitions to
parameterize committee members, graduate deans, and so forth, as well as
writing a few of the paragraphs above.  Other contributions have been
made by Nelson Beebe, Ed Cetron, Greg Conner, David Eyre, Grant
Gustafson, Ming He, Paul Joyce, Tina Ma, Jeff McGough, Eric Muehle, John
W. Peterson, Robert Dillon and Jeff Yost.  Finally, Linda Zillman,
thesis editor, was very helpful in adapting the format to the
capabilities and restrictions of \LaTeX. Continued support since 1993
has kindly been given by Christine Pickett, thesis editor, and many of
the newest features and fixes are due to her enthusiasm and cooperation.

\section{Sample Root Files}

\noindent The following is a root file for a dissertation. Made in 1993
by Jeff McGough and Grant Gustafson. It uses separate files for
everything, including the abstract, acknowledgments, chapters and
bibliography. Note in particular the preamble items versus the control
sequences that appear after the start of the document.

\begin{verbatim}
\documentstyle[11pt,amssymbols,thesis,diagram,tgrind]{uuthesis2}
\includeonly{chap1,chap2,chap3,appendix}
\title{REPRESENTATION OF SOLUTIONS TO LINEAR\protect\\
       ALGEBRAIC EQUATIONS}
\author{Fred Krylov}
\thesistype{dissertation}
\graduatedean{Ann W. Hart}
\department{Department of Mathematics}
\degree{Doctor of Philosophy}
\departmentchair{Paul Fife}
\committeechair{Fletcher Gross}
\firstreader{Hans Othmer}
\secondreader{Jim Carlson}
\thirdreader{Grant Gustafson}
\fourthreader{Nick Korevaar}
\chairtitle{Professor}
\submitdate{March 1993}
\copyrightyear{1993}
\fourlevels
\dedication{For my cat, Mouse, a few lines only.}
 \inputpicturetrue  % By Jeff McGough.
\begin{document}
\frontmatterformat
\titlepage
\copyrightpage
\committeeapproval
\readingapproval
\preface{abstract}{Abstract}
\dedicationpage
\tableofcontents
\listoffigures
\listoftables
\optionalfront{Notation and Symbols}{%\vspace*{.2cm}
\begin{normalsize}
    \renewcommand{\arraystretch}{1.655} 
% 1.655 = 24pt/14.5pt = baselineskip/normalsize
Most of the following may be found in \cite{gilbarg:epd83}.\\ 
\begin{tabular}{ll} 
${\Bbb R}^n$ & $n$-dimensional Euclidean space. \\ 
$\Omega$ & A bounded open subset of ${\Bbb R}^n$. \\
$\partial\Omega$ & The boundary of $\Omega$. \\ 
$B_r$ & $=\{ x\, : \, |x|<r\}$, the ball of radius $r$.\\ 
$D^{\beta}f$ & $=\partial^{|\beta |}f/\partial x_{\beta_1}\partial
x_{\beta_2} \cdots \partial x_{\beta_n}$, $|\beta | \equiv \sum_i
\beta_i$. \\
$\nabla f$ & $=(\partial f/\partial x_1,\partial f/\partial x_2,\dots ,
\partial f/\partial x_n)$, the gradient of $f$. \\
div$\{ \bf g \}$ & $=\partial g_1/\partial x_1 + \partial g_2/\partial x_2
+ \dots + \partial g_n/\partial x_n$, the divergence of $\bf g$. \\
$\Delta f$ & $=\mbox{div}\{\nabla f\}$, the Laplacian of $f$. \\
$C^k(\Omega )$ & Functions defined
on $\Omega$ which have $k$ continuous derivatives. \\ 
$C^k_0(\Omega )$
& $C^k(\Omega )$ functions which vanish at the boundary. \\
$C^{0,\alpha}(\Omega )$ & H\"{o}lder continuous functions with
H\"{o}lder constant $\alpha$. \\ 
$C^{k,\alpha}(\Omega )$ & $C^k(\Omega
)$ functions with $C^{0,\alpha}(\Omega )$ derivatives (up to order
$k$). \\ 
$C^{k,\alpha}_0(\Omega )$ & $C^{k,\alpha}(\Omega )$ functions
which vanish at the boundary. \\ 
$\| f\|_{L^p(\Omega )}$ & $= \left(
\int_{\Omega}|f|^p dx \right)^{1/p}$, the $L^p$ norm.  \\
$L^p(\Omega )$ & The space of $p$ integrable functions (the $L^p$ norm
is bounded). \\ $\| f \|_{W^{k,p}(\Omega )}$ & $= \left( \sum_{|\beta |
\leq k} \int_{\Omega}|D^{\beta} f|^p dx \right)^{1/p}$, the Sobolev norm. \\
$W^{k,p}(\Omega )$ & The space of functions with bounded Sobolev norm.
\\ $W^{k,p}_0(\Omega )$ & $W^{k,p}(\Omega )$ functions that vanish a.e.
at the boundary.  
\end{tabular} 
\end{normalsize}
}
\preface{acknowledge}{Acknowledgements}
\maintext
%%%
%%% This is the beginning of the actual thesis.  If you don't know latex
%%% then start with the LaTeX manual by Lamport and another easy
%%% reference, like the paperback by Jane Hahn, LaTeX for Everyone, PTI,
%%% 1991. See also $TEX/latex/sample.tex and $TEX/doc/story.tex, where
%%% $TEX==/usr/local/lib/tex
%%%
% Start this dissertation....
%
\chapter{Introduction}\label{introduction}   % level 1
%
%% Stolen from the sample.tex file.  There have been a few
%% modifications to fit in the thesis here.
%
% This is a sample LaTeX input file.  (Version of 28 May 1985)
%
% A '%' character causes TeX to ignore all remaining text on the line,
% and is used for comments like this one.
%
% \author{Leslie Lamport} % For this section Lamport is the author.
% \title{A Sample Document}
% \date{December 12, 1984}
%
%
\fixchapterheading % Use this if section follows chapter immediately
\section{The Sample.tex file}  % Produces section heading.  % level 2
%
    % Lower -level sections are begun with similar
    % \subsection and \subsubsection commands.


\subsection{Ordinary Text}   % level 3

The ends of words and sentences are marked by spaces. It doesn't matter
how many spaces you type; one is as good as 100.  The end of a line
counts as a space.\footnote{
This is a sample input file.  Comparing it with the output it
generates can show you how to produce a simple document of
your own.
}

One or more blank lines denote the end of a paragraph.

Since any number of consecutive spaces are treated like a single one,
the formatting of the input file makes no difference to \LaTeX,
but it makes a difference to you. When you use
\LaTeX,       % The \LaTeX command generates the LaTeX logo.
making your input file as easy to read as possible will be a great help
as you write your document and when you change it.  This sample file
shows how you can add comments to your own input file.

Because printing is different from typewriting, there are a number of
things that you have to do differently when preparing an input file than
if you were just typing the document directly.
Quotation marks like
       ``this''
have to be handled specially, as do quotes within quotes:
       ``\,`this'                  % \, separates the double and single quote.
        is what I just
        wrote, not  `that'\,.''

Dashes come in three sizes: an
       intra-word
dash, a medium dash for number ranges like
       1--2,
and a punctuation
       dash---like
this.

A sentence-ending space should be larger than the space between words
within a sentence.  You sometimes have to type special commands in
conjunction with punctuation characters to get this right, as in the
following sentence.
       Gnats, gnus, etc.\    % `\ ' makes an inter-word space.
       all begin with G\@.   % \@ marks end-of-sentence punctuation.
You should check the spaces after periods when reading your output to
make sure you haven't forgotten any special cases.
Generating an ellipsis
       \ldots\    % `\ ' needed because TeX ignores spaces after
                  % command names like \ldots made from \ + letters.
                  %
                  % Note how a `%' character causes TeX to ignore the
                  % end of the input line, so these blank lines do not
                  % start a new paragraph.
with the right spacing around the periods
requires a special  command.

\LaTeX\ interprets some common characters as commands, so you must type
special commands to generate them.  These characters include the
following:
       \$ \& \% \# \{ and \}.

In printing, text is emphasized by using an
       {\em italic\/}  % The \/ command produces the tiny extra space that
                       % should be added between a slanted and a following
                       % unslanted letter.
type style.

\begin{em}
   A long segment of text can also be emphasized in this way.  Text within
   such a segment given additional emphasis
          with\/ {\em Roman}
   type.  Italic type loses its ability to emphasize and become simply
   distracting when used excessively.
\end{em}

It is sometimes necessary to prevent \LaTeX\ from breaking a line where
it might otherwise do so.  This may be at a space, as between the
``Mr.'' and ``Jones'' in
       ``Mr.~Jones,''        % ~ produces an unbreakable interword space.
or within a word---especially when the word is a symbol like
       \mbox{\em itemnum\/}
that makes little sense when hyphenated across
       lines.

Footnotes\footnote{This is an example of a footnote.}
pose no problem.

\LaTeX\ is good at typesetting mathematical formulas like
       $ x-3y = 7 $
or
       $$ a_{1} > x^{2n} / y^{2n} > x'. $$
Remember that a letter like
       $x$        % $ ... $  and  \( ... \)  are equivalent
is a formula when it denotes a mathematical symbol, and should
be treated as one.

\subsection{Displayed Text}

Text is displayed by indenting it from the left margin.

\subsubsection{Quotations}

Quotations are commonly displayed.  There are short quotations
\begin{quote}
   This is a short a quotation.  It consists of a
   single paragraph of text.  There is no paragraph
   indentation.
\end{quote}
and longer ones.
\begin{quotation}
   This is a longer quotation.  It consists of two paragraphs
   of text.  The beginning of each paragraph is indicated
   by an extra indentation.

   This is the second paragraph of the quotation.  It is just
   as dull as the first paragraph.
\end{quotation}

\subsubsection{Lists}

Another frequently-displayed structure is a list.

\paragraph{Itemize.}
The following is an example of an {\em itemized} list.

\minusline % Part of uuthesis.sty to remove extra vertical space.

\begin{quote}
\begin{itemize}
   \item  This is the first item of an itemized list.  Each item
          in the list is marked with a ``tick''.  The document
          style determines what kind of tick mark is used.

   \item  This is the second item of the list.  It contains another
          list nested inside it.  The inner list is an {\em enumerated}
          list.
          \begin{enumerate}
              \item This is the first item of an enumerated list that
                    is nested within the itemized list.

              \item This is the second item of the inner list.  \LaTeX\
                    allows you to nest lists deeper than you really should.
          \end{enumerate}
          This is the rest of the second item of the outer list.  It
          is no more interesting than any other part of the item.
   \item  This is the third item of the list.
\end{itemize}
\end{quote}

\paragraph{Verse.}
You can even display poetry.

\minusline % Part of uuthesis.sty to kill one line

\begin{quote}
\begin{quote}
\begin{verse}
   There is an environment for verse \\    % The \\ command separates lines
   Whose features some poets will curse.   % within a stanza.

                           % One or more blank lines separate stanzas.

   For instead of making\\
   Them do {\em all\/} line breaking, \\
   It allows them to put too many words on a line when they'd
   rather be forced to be terse.
\end{verse}
\end{quote}
\end{quote}

\subsubsection{Mathematics}
Mathematical formulas may also be displayed.  A displayed formula is
one-line long; multiline formulas require special formatting
instructions.
   \[  x' + y^{2} = z_{i}^{2}\]
Don't start a paragraph with a displayed equation, nor make
one a paragraph by itself.

\section{More examples: Jeff McGough's Thesis}

Equations like
$\gamma = 0$ that don't need numbering may
be
set inline by the coding \verb"$\gamma = 0$" or displayed by
\par
\begin{singlespace}
\begin{verbatim}
$$
\gamma = 0.
$$
\end{verbatim}
\end{singlespace}
\par
Numbered equations are set as shown in the next paragraph. They use the
theorem environments defined in \verb"thesis.sty":
\par
\begin{singlespace}
\begin{verbatim}
\newtheorem{thrm}{Theorem}
\newtheorem{lem}[thrm]{Lemma}
\newtheorem{cor}[thrm]{Corollary}
\newtheorem{rem}[thrm]{Remark}
\newtheorem{defn}[thrm]{Definition}
\newtheorem{exmpl}[thrm]{Example}
\end{verbatim}
\end{singlespace}
\par

The Gelfand problem is the following elliptic boundary value problem:
%
% The equation-array feature in LaTeX is a bad idea.  For centered
% numbers you should set your own equations and arrays as follows:
%
\def\dd{\displaystyle}
\begin{equation}\label{gelfand}
\begin{array}{rl}
\dd \Delta u + \lambda e^u = 0, &
\dd u\in \Omega,\\[8pt] % add 8pt extra vertical space. 1 line=23pt
\dd u=0, & \dd u\in\partial\Omega.
\end{array}
\end{equation}
The previous equation had a label.  It may be referenced as
equation~(\ref{gelfand}).

%
%
\section{History of the Gelfand problem}
%
%

According to Bebernes and Eberly \cite[p.46]{bebernes:mpc89},
Gelfand was ``the first to make an in-depth
study'' of (\ref{gelfand}). Following this statement they briefly
outline the history of the Gelfand problem.
\par
% Quotes need to forced single space:
\begin{singlespace}
\begin{quote}
For dimension $n=1$, Liouville~\cite{liouville:edp53} first studied and
found an explicit solution in 1853. For $n=2$, Bratu~\cite{bratu:ein14}
found an explicit solution in 1914.  Frank-Kamenetski~\cite{frank:dhe55}
rediscovered these results in his development of thermal explosion
theory.  Joseph and Lundgren~\cite{joseph:qdp73} gave an elementary
proof via phase plane analysis of the multiple existence of solutions
for dimensions $n\geq 3$.
\end{quote}
\end{singlespace}
\par

% Several things to note here.  Latex sometimes breaks equations, this
% can be restricted by the samepage command.  The spacing in the array
% mode is also important for some structures.

From Zeidler~\cite{zeidler:nfa88IIa}:
{\samepage
\begin{equation}\label{station}
\begin{array}{rcll}
\dd\mbox{div } j& = &\dd f, &\dd x\in\Omega ,\\[8pt]
\dd u& = & \dd g_1, & \dd x\in\partial\Omega_1 , \\[8pt]
\dd j\nu & = & \dd g_2, & \dd x\in\partial\Omega_2 ,
\end{array}
\end{equation} }
where
\begin{equation}\label{current}
j =  h(|\nabla u|^2)\nabla u
\end{equation}
and $\Omega$ is a bounded domain in ${\Bbb R}^n$ with
smooth boundary $\partial\Omega = \overline{\partial\Omega_1}\cup
\overline{\partial\Omega_2}$, $\partial\Omega_1 \cap
\partial\Omega_2 = \emptyset$ and $\nu$ is the normal vector to
$\partial\Omega$.

% There is a lot of shorthand set up for structures, for example a
% lemma:
\begin{lem}
Assuming that $\partial\Omega_2 = \emptyset$ and that $h(t) = 1$, we
have $$
\begin{array}{lr}
\dd\Delta u = f, & \dd x\in\Omega ,\\[8pt]
\dd u =  g_1, & \dd x\in\partial\Omega .
\end{array}
$$
\end{lem}

% another ...
\begin{cor}
If $g_2 = 0$ then
$$
\begin{array}{lr}
\dd \Delta u = f, & \dd x\in\Omega ,\\[8pt]
\dd u =  0, & \dd x\in\partial\Omega .
\end{array}
$$
\end{cor}

% Look in thesis.sty for more structures.

\section{Fundamental results}
The investigation of the Gelfand problem begins with examining the
..... (this paragraph continues for many lines).

%
% This is an example of a big ugly technical theorem.  It has two
% levels of lists, referencing, citations and names.
%
\begin{thrm}[Joseph-Lundgren~\cite{joseph:qdp73}]
Boundary value problem (\ref{gelfand}) has positive radial
solutions $u$ on the unit ball which depend on $n$ and $\lambda$
in the following manner.
\begin{enumerate}
\item For $n=1,2$, there exists $\lambda^* >0$ such that
\begin{enumerate}
\item for $0< \lambda < \lambda^*$ there are two positive
solutions,
\item for $\lambda =\lambda^*$ there is a unique solution, and
\item for $\lambda > \lambda^*$ there are no solutions.
\end{enumerate}
\item For $3\leq n \leq 9$, let $\overline{\lambda}=2(n-2)$; then
there exist positive constants $\lambda_*$, $\lambda^*$ with
$0< \lambda_* < \overline{\lambda} < \lambda^*$, such that
\begin{enumerate}
\item for $\lambda = \lambda^*$ there is a unique solution,
\item for $\lambda > \lambda^*$ there are no solutions,
\item for $\lambda = \overline{\lambda}$ there is a countably infinite number
of solutions,
\item for $\lambda \in (\lambda_*,\lambda^*)$, $\lambda \neq
\overline{\lambda}$, there is a finite number of solutions,
\item for $\lambda < \lambda_*$ there is a unique solution.
\end{enumerate}
\item For $n\geq 10$, let $\lambda^* = 2(n-2)$ then
\begin{enumerate}
\item for $\lambda \geq \lambda^*$ there are no solutions,
\item for $\lambda \in (0,\lambda^*)$ there is a unique solution.
\end{enumerate}
\end{enumerate}
\end{thrm}

\chapter{Quadratic nonlinearities}\label{quad}
%
% Don't use \fixchapterheading here. Chapter is followed by a
% paragraph, not a heading.
%
In this chapter we derive results for the quadratic equation.

\section{Derivation of the quadratic formula}
A quadratic equation is one of the form
\begin{equation}\label{quadratic}
ax^2 + bx + c = 0
\end{equation}
where $a,b,c$ are known constants and $x$ is the unknown.
The results are summarized in Table \ref{pde.tab1} and Table
\ref{pde.tab2} below.

%
%
\section{Application of the quadratic formula}
%
%

If the differential operator generates a nonnegative form, then an
inequality is based on the following considerations. See
Figure \ref{gelfand.fig1} for $n=1,2$,
Figure \ref{gelfand.fig2} for $3\leq n \leq 9$
and
Figure \ref{gelfand.fig3} for $n\geq 10$.

% Example of a table:
% Table caption can be selected as paragraph style or centered style
% (for an inverted pyramid title). Use \oldstylecaptiontrue (paragraph)
% or \oldstylecaptionfalse (centered) to select the style.
%
\begin{table}[b]
\centering
\caption{\label{timing1} PDE solve times, $15^3+1$
equations.\label{pde.tab1}}
\plusline
\begin{tabular}{||l|l|l|l|l|l||}\hline
Precond. & Time & Nonlinear & Krylov
& Function & Precond. \\
 & & Iterations & Iterations & calls & solves \\ \hline
None & 1260.9u & 3 & 26 & 30 & 0  \\
 &(21:09) & & & &  \\ \hline
FFT  & 983.4u & 2  & 5  & 8  & 7 \\
&(16:31) & & & & \\ \hline
MILU & 629.7u & 3  & 11 & 15 & 14 \\
& (10:36) & & & & \\ \hline
\end{tabular}
\end{table}
\clearpage

\begin{table}[t]
\caption{Convergence properties of RQI.\label{pde.tab2}}
\centering
\plusline\small
\begin{tabular}{l|lll} \hline
Object & Normal Matrices & Diagonalizable Matrices
& Defective Matrices \\ \hline
$\rho$ & Stationary at ev's. &
Stationary at ev's. &
Stationary at ev's. \\
$\| r_k\|$ & $\to 0$ as $k\to\infty$. &
Can oscillate. &
Can oscillate. \\
$\rho_k$ & Converges. &
Unknown. &
Unknown. \\
Convergence to & is cubic. & is quadratic. & is linear. \\
eigensets & & & \\
\hline
\end{tabular}\normalsize
\end{table}

% Figures in LaTeX will go on the bottom of the same page, or the top of
% the next page, but never before the first reference. All figures must
% be referenced. The syntax is below. See uuguide for control.
%
%       \begin{figure}[x]    % x = b, t, h, p
%       ...
%       \caption{My title.}  % Captions are below the figure!
%       \end{figure}
%
%
% These graphs were created by gnuplot. For simple graphs this is a
% great utility.  For example, if you want a sin curve in your thesis
% try the following:
%
% (terminal window): gnuplot
% (in gnuplot):
%                 set terminal latex
%                 set output "foo.tex"
%                 plot sin(x)
%                 quit
%
\begin{figure}[b]       % Place it on the bottom of page
\centering              % Put \label{} into \caption.
\inputpicture{fig1.tex}
\caption{Gelfand equation on the ball, $n=1,2$.
\label{gelfand.fig1}}    % Use \ref{gelfand.fig1} for references
\end{figure}


\begin{figure}[p]       % Likely it will go on the top of the page
\centering
\inputpicture{fig2.tex}
\caption{Gelfand equation on the ball, $3\leq n \leq 9$.
\label{gelfand.fig2}}    % If not, then change [t] to [p]
\end{figure}

\begin{figure}[p] % Likely it will go on the top of the next page
                  % If not, then change [h] to [p]
\centering
\inputpicture{fig3.tex}
\caption{Gelfand equation on the ball, $n\geq 10$.
\label{gelfand.fig3}}
\end{figure}
\clearpage % dump figures where they below

\chapter{Systems}\label{systems}
\fixchapterheading
\section{Diagrams made with diagram.sty}
% \captionstyleparagraph

%% The full documentation is in the file: diagram.sty

An example diagram appears below in Figure \ref{diagram.fig1}. This is
typical of what can made with the diagram package.

\begin{figure}[b]               % Place at bottom of this page
$$
\begin{diagram}
\node{U} \arrow{e,t}{i_1} \arrow{s}
\node{X} \arrow{s,r}{\pi} \\
\node{Y-\partial Q} \arrow{e,t}{j_1} \node{Y}
\end{diagram}
$$
\caption{Diagram example\label{diagram.fig1}}
\end{figure}

\section{Sample diagrams from diagram.tex}

Example diagrams reproduced here were taken from various sources.
Compare the three diagrams of increasing sizes in
Figure \ref{file.fig1}, Figure \ref{file.fig2}, Figure \ref{file.fig3}
with the three diagrams in Figure \ref{file.fig4}, Figure
\ref{file.fig5},
Figure \ref{file.fig6}.


\begin{figure}[b]
$$
\setlength{\dgARROWLENGTH}{3.0em}
\begin{diagram}[\strut A]
\node{A} \arrow{e} \arrow{s} \arrow{se} \node{B} \arrow{s} \\
\node{C} \arrow{e}                      \node{D}
\end{diagram}
$$
\caption{Base diagram, Arrowlength = 3.0em
\label{file.fig1}}
\end{figure}

\begin{figure}[p]
$$
\setlength{\dgARROWLENGTH}{6.0em}
\begin{diagram}[\strut A]
\node{A} \arrow{e} \arrow{s} \arrow{se} \node{B} \arrow{s} \\
\node{C} \arrow{e}                      \node{D}
\end{diagram}
$$
\caption{Same as Figure \protect\ref{file.fig1}, but Arrowlength = 6.0em
\label{file.fig2}}
\end{figure}

\begin{figure}[p]
$$
\setlength{\dgARROWLENGTH}{12.0em}
\begin{diagram}[\strut A]
\node{A} \arrow{e} \arrow{s} \arrow{se} \node{B} \arrow{s} \\
\node{C} \arrow{e}                      \node{D}
\end{diagram}
$$
\caption{Same as Figure \protect\ref{file.fig1}, but Arrowlength =
12.0em \label{file.fig3}}
\end{figure}

\begin{figure}[p]
$$
\setlength{\dgARROWLENGTH}{3.0em}
\begin{diagram}[\strut A]
\node{A} \arrow{e} \arrow{s} \arrow{se} \node{B} \arrow{s} \\
\node{C} \arrow{e}                      \node{D}
\end{diagram}
$$
\caption{Base figure, same as Figure \protect\ref{file.fig1}.
\label{file.fig4}}
\end{figure}
\clearpage % make page of floats

\begin{figure}[t]
$$
\setlength{\dgARROWLENGTH}{3.0em}
\begin{diagram}[\strut\hspace{6.0em}]
\node{A} \arrow{e} \arrow{s} \arrow{se} \node{B} \arrow{s} \\
\node{C} \arrow{e}                      \node{D}
\end{diagram}
$$
\caption{Same as Figure \protect\ref{file.fig4}, but Bignode = strut
hspace 6.0em. \label{file.fig5}}
\end{figure}

\begin{figure}[t]
$$
\setlength{\dgARROWLENGTH}{3.0em}
\begin{diagram}[\strut\hspace{12.0em}]
\node{A} \arrow{e} \arrow{s} \arrow{se} \node{B} \arrow{s} \\
\node{C} \arrow{e}                      \node{D}
\end{diagram}
$$
\caption{Same as Figure \protect\ref{file.fig1}, but Bignode = strut
hspace 12.0em \label{file.fig6}}
\end{figure}

Below we show diagrams from the manual with a few modifications. The
first in Figure \ref{file.fig7} is essentially as it appears in the
manual, whereas the second, Figure \ref{file.fig8} has been
rescaled to a larger size.
\begin{figure}[p]
$$
\begin{diagram}[B^*]
\node{A} \arrow{e,t}{a} \arrow{s,l}{c} \arrow{ese,b,1}{u}
   \node{B^*} \arrow{e,t}{b^*}
      \node{C} \arrow{s,r}{d} \arrow{wsw,b,1}{v} \\
\node{D} \arrow[2]{e,b}{e}
   \node[2]{E}
\end{diagram}
$$
\caption{First diagram from manual
\label{file.fig7}}
\end{figure}

\begin{figure}[p]
$$
\setlength{\dgARROWLENGTH}{.75em}
\begin{diagram}[B^*]
\node{A} \arrow[2]{e,t}{a} \arrow[2]{s,l}{c} \arrow[2]{ese,b,1}{u}
   \node[2]{B^*} \arrow[2]{e,t}{b^*}
      \node[2]{C} \arrow[2]{s,r}{d} \arrow{wsw,b,-}{v}
\\
        \node[3]{} \arrow{wsw}
\\
\node{D} \arrow[4]{e,b}{e}
   \node[4]{E}
\end{diagram}
$$
\caption{First diagram from manual, rescaled.\label{file.fig8}}
\end{figure}

Below are several diagrams created by Bill Richter. The first, Figure
\ref{file.fig9} is modified slightly to produce Figure
\ref{file.fig10}. Both use fractur fonts. The last one, Figure
\ref{file.fig11}, is a complicated example illustrating the limits of
what can be done with diagrams.

The diagram below in Figure \ref{file.fig12}, the last of our series of
illustrations, is by Anders Thorup (\verb"thorup@math.ku.dk"),
originally done with a package developed by himself and Steven Kleiman
(\verb"kleiman@math.mit.edu"):


%%%%
%%%% If you're missing fractur fonts, then comment out these next 5
%%%% lines and type instead;
%%%% \let\frak=\bf
%%%%
\font\tenfrak=eufm10 scaled \magstep1
\font\sevenfrak=eufm7 scaled \magstep1
\font\fivefrak=eufm5 scaled \magstep1
\newfam\frakfam \def\frak{\fam\frakfam\tenfrak} \textfont\frakfam=\tenfrak
\scriptfont\frakfam=\sevenfrak  \scriptscriptfont\frakfam=\fivefrak
%%%%
%%%%
\def\a{ \alpha }
\def\d{ \delta }
\def\s{ \sigma }
\def\l{ \lambda }
\def\p{ \partial }
\def\st{{\tilde\s}}
\def\O{ \Omega }
\def\S{\Sigma}
\def\Z{{   \Bbb Z }}
\def\@{ \otimes }
\def\^{ \wedge }
\def\({ \left( }
\def\){ \right) }
\def\K#1{{ K\(\Z/2,#1\) }}
\def\KZ#1{{K\(\Z/4,#1\) }}
\def\id{ \mathop{id}\nolimits }
\def\h{ {\frak h} }
\def\e{ {\frak e} }
\def\G{ G }
\def\pinch{{ \mathop{{\rm pinch}} }}
\def\tuber{{ \bar\tau }}
\begin{figure}[p]
$$
\setlength{\dgARROWLENGTH}{1.5em}
\begin{diagram}[ \KZ{8n-1}  ]
\node[4]{ \K{8n+1} } \\
\node[2]{ \KZ{8n-1}  } \arrow{e} \arrow{ene,t}{Sq^2}
   \node{E} \arrow{ne,b}{\Theta} \arrow{s,l}{\pi} \\
\node{ \S\O X \^ \O X  } \arrow{e,t}{H_\mu} \arrow{ne,t}{\s(\a\@\a)}
   \node{ \Sigma \O X } \arrow{e,t}{\sigma} \arrow{ne,t}{\st}
       \node{ X }  \arrow{e,t}{\a^2}
           \node{ \KZ{8n}. }
\end{diagram}
$$
\caption{Bill Richter, first diagram\label{file.fig9}}
\end{figure}

\begin{figure}[p]
$$
\setlength{\dgARROWLENGTH}{-2.75em}
\begin{diagram}[ \O^2 \( \S A \^ \S A \) ]
\node[3]{\O\S A} \arrow[2]{e,t}{\l_2}
  \node[2]{\O^2 \( \S A \^ \S A \)}
\\
\node[4]{\#}
\\
% Note: the next two lines are like
% \node{\O B}  \arrow[2]{e,t,1}{\d}     \arrow[2]{ne,t}{\O\(\p\)}
% but put a gap in first arrow to make room for crossing arrow
\node{\O B}  \arrow{e,t,-}{\d}  \arrow[2]{ne,t}{\O\(\p\)}
  \node{} \arrow{e}
    \node{F} \arrow[2]{e,t}{\h}  \arrow[2]{s,r}{\pi} \arrow[2]{n,r}{J}
      \node[2]{\O^2 \( B \^ \S A \)} \arrow[2]{n,r}{\O^2\(\p\^\id\)}
\\
\\
\node{A} \arrow[2]{ne,t}{\e} \arrow[2]{e,t}{f}  \arrow[2]{nne,t,1}{E}
  \node[2]{X}  \arrow[2]{e,t}{h}
    \node[2]{B.}
\end{diagram}
$$
\caption{Bill Richter, second diagram\label{file.fig10}}
\end{figure}
\clearpage % Expunge all figures


\begin{figure}[p]
$$
\setlength{\dgARROWLENGTH}{-3.9em}
\begin{diagram}[ J\(S^4\^S^4\) ]
\node[9]{\O S^5}
\\
\\
\\
\node[8]{\beth}
\\
\node{\O\( M^5_{2\i}\)} \arrow[4]{e,t}{\O\(\pinch\)}
        \node[4]{\O S^5} \arrow[2]{e,t,-}{\d}
                                \arrow[4]{ne,t}{\O\(2\i\)}
                \node[2]{} \arrow[2]{e}
                        \node[2]{\G}    \arrow[4]{e,t}{\h_2}
                                        \arrow[2]{s,r,-}{\pi} \arrow[4]{n,r}{J}
                                \node[4]{J\(S^4\^S^4\)}
\\
\\
\node[3]{J_2\( M^4_{2\i}\)} \arrow[3]{e,t,3,-}{\d_2}
                                \arrow[2]{ne,t}{\i}
        \node[3]{}      \arrow{e}
                \node{\G_2} \arrow[4]{e,t,3}{\h_2}
                                \arrow[2]{ne,t}{\i}
                        \node[2]{}      \arrow[2]{s}
                                \node[2]{S^8} \arrow[2]{ne,b}{E}
\\
\node[4]{\aleph}
\\
\node{M^{12}_{2\i}} \arrow[4]{e,t}{\tuber} \arrow[2]{ne,t}{\tau}
        \node[4]{S^4} \arrow[4]{e,t}{\i}
                                \arrow[2]{ne,t}{\e}
                                        \arrow[4]{nne,t,3}{E}
                \node[4]{M^5_{2\i}} \arrow[4]{e,t}{\pinch}
                        \node[4]{S^5}
\end{diagram}
$$
\caption{Bill Richter, third diagram\label{file.fig11}}
\end{figure}

\begin{figure}[p]
$$
\setlength{\dgARROWLENGTH}{-6em}
\begin{diagram}[H^k(B_G\times N;Q)=H^k_G(N;Q)]
\node{H^k(B_G\times N;Q)=H^k_G(N;Q)}
      \arrow[2]{e,t}{f^*_j} \arrow[2]{s,l}{p^*} \arrow{se,t}{\tilde f^*}
   \node[2]{H^k_G(F_j;Q)}
      \arrow[2]{s,r}{q^*_j} \\
\node[2]{H^k_G(M;Q)}
      \arrow{ne,t}{i^*_j} \arrow[2]{s,l,1}{i^*} \\
\node{H^k(N;Q)}
      \arrow{e,t,-}{\tilde f^*_j=f^*_j} \arrow{se,b}{\tilde f^*=f^*}
   \node{}
      \arrow{e}
   \node{H^k(F_j;Q)} \\
\node[2]{H^k(M;Q)}
      \arrow{ne,b}{i^*_j}
\end{diagram}
$$
\caption{Anders Thorup diagram\label{file.fig12}}
\end{figure}
\clearpage

\numberofappendices=1
\appendix
\chapter{Classical identities}\label{appendix}
\fixchapterheading
\section*{Rellich's identity}\label{rellich.section}
\setcounter{thrm}{0}
%
%

Standard developments of Pohozaev's identity used an identity by
Rellich~\cite{rellich:der40}, reproduced here.

\begin{lem}[Rellich]
Given $L$ in divergence form and $a,d$ defined above, $u\in C^2
(\Omega )$, we have
\begin{equation}\label{rellich}
\int_{\Omega}(-Lu)\nabla u\cdot (x-\overline{x})\, dx
= (1-\frac{n}{2}) \int_{\Omega} a(\nabla u,\nabla u) \, dx
-
\frac{1}{2} \int_{\Omega}
d(\nabla u, \nabla u) \, dx
\end{equation}
$$
+
\frac{1}{2} \int_{\partial\Omega} a(\nabla u,\nabla u)(x-\overline{x})
\cdot \nu  \, dS
-
\int_{\partial\Omega}
a(\nabla u,\nu )\nabla u\cdot (x-\overline{x}) \, dS.
$$
\end{lem}
{\bf Proof:}\\
There is no loss in generality to take $\overline{x} = 0$. First
rewrite $L$:
$$Lu = \frac{1}{2}\left[ \sum_{i}\sum_{j}
\frac{\partial}{\partial x_i}
\left( a_{ij} \frac{\partial u}{\partial x_j} \right) +
\sum_{i}\sum_{j}
\frac{\partial}{\partial x_i}
\left( a_{ij} \frac{\partial u}{\partial x_j} \right)
\right]$$
Switching the order of summation on the second term and relabeling
subscripts, $j \rightarrow i$ and $i \rightarrow j$, then using the fact
that $a_{ij}(x)$ is a symmetric matrix,
gives the symmetric form needed to derive Rellich's identity.
\begin{equation}
Lu = \frac{1}{2} \sum_{i,j}\left[
\frac{\partial}{\partial x_i}
\left( a_{ij} \frac{\partial u}{\partial x_j} \right) +
\frac{\partial}{\partial x_j}
\left( a_{ij} \frac{\partial u}{\partial x_i} \right)
\right].
\end{equation}

Multiplying $-Lu$ by $\frac{\partial u}{\partial x_k} x_k$ and integrating
over $\Omega$, yields
$$\int_{\Omega}(-Lu)\frac{\partial u}{\partial x_k} x_k \, dx=
-\frac{1}{2} \int_{\Omega}
\sum_{i,j}\left[
\frac{\partial}{\partial x_i}
\left( a_{ij} \frac{\partial u}{\partial x_j} \right) +
\frac{\partial}{\partial x_j}
\left( a_{ij} \frac{\partial u}{\partial x_i} \right)
\right]
\frac{\partial u}{\partial x_k} x_k \, dx$$
Integrating by parts (for integral theorems see~\cite[p. 20]
{zeidler:nfa88IIa})
gives
$$= \frac{1}{2} \int_{\Omega}
\sum_{i,j} a_{ij} \left[
\frac{\partial u}{\partial x_j}
\frac{\partial^2 u}{\partial x_k\partial x_i} +
\frac{\partial u}{\partial x_i}
\frac{\partial^2 u}{\partial x_k\partial x_j}
\right] x_k \, dx
$$
$$
+
\frac{1}{2} \int_{\Omega}
\sum_{i,j} a_{ij} \left[
\frac{\partial u}{\partial x_j} \delta_{ik} +
\frac{\partial u}{\partial x_i} \delta_{jk}
\right] \frac{\partial u}{\partial x_k} \, dx
$$
$$- \frac{1}{2} \int_{\partial\Omega}
\sum_{i,j} a_{ij} \left[
\frac{\partial u}{\partial x_j} \nu_{i} +
\frac{\partial u}{\partial x_i} \nu_{j}
\right] \frac{\partial u}{\partial x_k} x_k \, dx
$$
= $I_1 + I_2 + I_3$, where the unit normal vector is $\nu$.
One may rewrite $I_1$ as
$$I_1 = \frac{1}{2} \int_{\Omega}
\sum_{i,j} a_{ij} \frac{\partial}{\partial x_k}\left(
\frac{\partial u}{\partial x_i}
\frac{\partial u}{\partial x_j}
\right) x_k \, dx
$$
Integrating the first term by parts again yields
$$I_1 = -\frac{1}{2} \int_{\Omega}
\sum_{i,j} a_{ij} \left(
\frac{\partial u}{\partial x_i}
\frac{\partial u}{\partial x_j}
\right) \, dx
+
\frac{1}{2} \int_{\partial\Omega}
\sum_{i,j} a_{ij} \left(
\frac{\partial u}{\partial x_i}
\frac{\partial u}{\partial x_j}
\right) x_k \nu_k \, dS
$$
$$
-
\frac{1}{2} \int_{\Omega}
\sum_{i,j} \left(
\frac{\partial u}{\partial x_i}
\frac{\partial u}{\partial x_j}
\right) x_k \frac{\partial a_{ij}}{\partial x_k}\, dx.
$$
Summing over $k$ gives
$$\int_{\Omega}(-Lu)(\nabla u\cdot x)\, dx =
-\frac{n}{2} \int_{\Omega}
\sum_{i,j} a_{ij} \left(
\frac{\partial u}{\partial x_i}
\frac{\partial u}{\partial x_j}
\right) \, dx
$$
$$
+
\frac{1}{2} \int_{\partial\Omega}
\sum_{i,j} a_{ij} \left(
\frac{\partial u}{\partial x_i}
\frac{\partial u}{\partial x_j}
\right) (x\cdot \nu ) \, dS
-
\frac{1}{2} \int_{\Omega}
\sum_{i,j} \left(
\frac{\partial u}{\partial x_i}
\frac{\partial u}{\partial x_j}
\right) (x\cdot  \nabla a_{ij}) \, dx
$$
$$
+
\frac{1}{2} \int_{\Omega}
\sum_{i,j,k} a_{ij} \left[
\frac{\partial u}{\partial x_j}
\frac{\partial u}{\partial x_k} \delta_{ik} +
\frac{\partial u}{\partial x_i}
\frac{\partial u}{\partial x_k} \delta_{jk}
\right] \, dx
$$
$$- \frac{1}{2} \int_{\partial\Omega}
\sum_{i,j} a_{ij} \left[
\frac{\partial u}{\partial x_j} \nu_{i} +
\frac{\partial u}{\partial x_i} \nu_{j}
\right] (\nabla u\cdot x) \, dS.
$$
Combining the first and fourth term on the right-hand side
simplifies the expression
$$\int_{\Omega}(-Lu)(\nabla u\cdot x)\, dx
=
(1-\frac{n}{2}) \int_{\Omega}
\sum_{i,j} a_{ij} \left(
\frac{\partial u}{\partial x_i}
\frac{\partial u}{\partial x_j}
\right) \, dx
$$
$$
+
\frac{1}{2} \int_{\partial\Omega}
\sum_{i,j} a_{ij} \left(
\frac{\partial u}{\partial x_i}
\frac{\partial u}{\partial x_j}
\right) (x\cdot \nu ) \, dS
-
\frac{1}{2} \int_{\Omega}
\sum_{i,j} \left(
\frac{\partial u}{\partial x_i}
\frac{\partial u}{\partial x_j}
\right) (x\cdot  \nabla a_{ij}) \, dx
$$
$$
-
\frac{1}{2} \int_{\partial\Omega}
\sum_{i,j} a_{ij} \left[
\frac{\partial u}{\partial x_j} \nu_{i} +
\frac{\partial u}{\partial x_i} \nu_{j}
\right] (\nabla u\cdot x) \, dS.
$$
Using the notation defined above, the result follows.


%
%
%
\section*{Fortran code}\label{code}
%
%

%% The following was constructed by a very handy program called
%% tgrind.  tgrind is a filter to convert C or fortran files into
%% formatted tex.  Starting with a fortran subroutine rhs.f:
%%
%% tgrind -lf -f >rhs.tex rhs.f
%%
%% This creates the tex file rhs.tex.  This may be directly included
%% in a latex document via the special command \tgrind (latex command
%% here):
%%
%% \begin{singlespace}
%% \begin{small}
%% \tgrind{rhs.tex}
%% \end{small}
%% \end{singlespace}
%%
%% Otherwise, the file rhs.tex needs to be edited (rhs_mod.tex)
%% to be included into a latex document (it is a stand-alone
%% tex file).  The line with the \File command (top) needs to be
%% removed or commented out: \File{rhs.f},{14:32},{Jul  5 1993} and
%% the \end command at the bottom also needs to be commented out.  The
%% file rhs.tex can then be included into the document:
%%
%% \begin{singlespace}
%% \begin{small}
%% \input tgrindmac
%\File{rhs.f},{14:32},{Jul  5 1993}
\L{\LB{}}
\L{\LB{      \K{subroutine} rhs(neq,v,rhsf)}}
\L{\LB{      \K{save}}}
\L{\LB{c}}
\L{\LB{c This \K{subroutine} computes the \K{function} values. Inputs are neq and }}
\L{\LB{c v, and on output the values of f are stored in the array of rhsf}}
\L{\LB{c}}
\L{\LB{      \K{include} \S{}\'parabolic.inc\'\SE{}}}
\L{\LB{}}
\L{\LB{      \K{integer} neq}}
\L{\LB{      \K{integer} i}}
\L{\LB{      \K{integer} j}}
\L{\LB{      \K{integer} k}}
\L{\LB{      \K{integer} ind}}
\L{\LB{      \K{integer} inde}}
\L{\LB{      \K{integer} indw}}
\L{\LB{      \K{integer} indn}}
\L{\LB{      \K{integer} inds}}
\L{\LB{      \K{integer} ind0}}
\L{\LB{      \K{integer} ind1}}
\L{\LB{      \K{integer} ind2}}
\L{\LB{}}
\L{\LB{      \K{double} \K{precision} v(neq)}}
\L{\LB{      \K{double} \K{precision} rhsf(neq)}}
\L{\LB{      \K{double} \K{precision} u(nv)}}
\L{\LB{      \K{double} \K{precision} diff}}
\L{\LB{      \K{double} \K{precision} diffn}}
\L{\LB{      \K{double} \K{precision} diffxn}}
\L{\LB{      \K{double} \K{precision} diffyn}}
\L{\LB{      \K{double} \K{precision} nl}}
\L{\LB{}}
\L{\LB{c      \K{write}(*,*)\S{}\'funct begin\'\SE{}}}
\L{\LB{}}
\L{\LB{c}}
\L{\LB{c     Compute F for the local dynamics, written as  F(u)= \-du\/dt + f(u)}}
\L{\LB{c     }}
\L{\LB{c}}
\L{\LB{c the system parameters}}
\L{\LB{c}}
\L{\LB{c      p1              ! \K{parameter} F}}
\L{\LB{c      p2              ! \K{parameter} k}}
\L{\LB{}}
\L{\LB{      \K{do} j = 1, ny }}
\L{\LB{         \K{do} i = 1, nx}}
\L{\LB{c}}
\L{\LB{c set up index}}
\L{\LB{c}}
\L{\LB{            ind = (i\-1)*nv + (j\-1)*meq}}
\L{\LB{c}}
\L{\LB{c Extract the jth component at current time}}
\L{\LB{c}}
\L{\LB{            nl = v(1+ind)*v(2+ind)*v(2+ind)}}
\L{\LB{}}
\L{\LB{            rhsf(1+ind) =  (\- nl + p1*(1.0d0 \- v(1+ind)))*local}}
\L{\LB{            rhsf(2+ind) =  (  nl \- (p1+p2)*v(2+ind))*local}}
\L{\LB{}}
\L{\LB{         \K{end} \K{do}}}
\L{\LB{      \K{end} \K{do}}}
\L{\LB{}}
\L{\LB{c \-\-\-\-\-\-\-\-\-\-\-\-\-\-\-\-\-\-\-\-\-\-\-\-\-\-\-\-\-\-\-\-\-\-\-\-\-\-\-\-}}
\L{\LB{c}}
\L{\LB{c     add diffusion for all species (zero diffusion }}
\L{\LB{c     coefficient takes care of those that \K{do} not diffuse). }}
\L{\LB{c }}
\L{\LB{c
\-\-\-\-\-\-\-\-\-\-\-\-\-\-\-\-\-\-\-\-\-\-\-\-\-\-\-\-\-\-\-\-\-\-\-\-\-\-\-
}}
\L{\LB{}}
\L{\LB{      \K{do} j = 1, ny}}
\L{\LB{         \K{do} i = 1, nx}}
\L{\LB{}}
\L{\LB{c indexing}}
\L{\LB{c}}
\L{\LB{            ind0 = (i\-1)*nv + (j\-1)*meq   ! point}}
\L{\LB{            indw = (i\-2)*nv + (j\-1)*meq   ! west point}}
\L{\LB{            inde = (i)*nv + (j\-1)*meq     ! east point}}
\L{\LB{            indn = (i\-1)*nv + (j)*meq     ! north point}}
\L{\LB{            inds = (i\-1)*nv + (j\-2)*meq   ! south point}}
\L{\LB{}}
\L{\LB{            \K{if}(i.eq.1) indw = (nx\-1)*nv + (j\-1)*meq}}
\L{\LB{            \K{if}(i.eq.nx) inde = (j\-1)*meq }}
\L{\LB{            \K{if}(j.eq.1) inds = (i\-1)*nv + (ny\-1)*meq}}
\L{\LB{            \K{if}(j.eq.ny) indn = (i\-1)*nv}}
\L{\LB{}}
\L{\LB{            \K{do} k = 1, 2}}
\L{\LB{}}
\L{\LB{c}}
\L{\LB{c First compute the contribution within a row at the current time}}
\L{\LB{c and at the preceding time. }}
\L{\LB{c}}
\L{\LB{               ind = k + ind0}}
\L{\LB{               ind1 = k + indw}}
\L{\LB{               ind2 = k + inde}}
\L{\LB{}}
\L{\LB{               diffxn = v(ind1) \- 2.0d0*v(ind) + v(ind2)}}
\L{\LB{}}
\L{\LB{c}}
\L{\LB{c Compute the contribution from the columns}}
\L{\LB{c}}
\L{\LB{               ind1 = k + indn}}
\L{\LB{               ind2 = k + inds}}
\L{\LB{}}
\L{\LB{               diffyn = v(ind1) \- 2.0d0*v(ind) + v(ind2)}}
\L{\LB{}}
\L{\LB{c}}
\L{\LB{c Multiply by  other factors and sum}}
\L{\LB{c}}
\L{\LB{               diff = d(k)*hxx*(diffxn + diffyn)*diffus}}
\L{\LB{}}
\L{\LB{               rhsf(ind) = rhsf(ind) + diff}}
\L{\LB{               }}
\L{\LB{}}
\L{\LB{            \K{end} \K{do}}}
\L{\LB{         \K{end} \K{do}}}
\L{\LB{      \K{end} \K{do}}}
\L{\LB{}}
\L{\LB{       }}
\L{\LB{      \K{return}}}
\L{\LB{      \K{end}}}
\L{\LB{}}
\L{\LB{}}
\L{\LB{}}
\vfill\eject
%\end

%% \end{small}
%% \end{singlespace}
%%
%% In either case, the \File command probably will need removing
%% because it places the page numbers differently than the normal
%% thesis style. Using both gives:

 \begin{singlespace}
 \begin{small}
% \let\end\relax \def\File#1,#2,#3{}
 \tgrind{rhs_mod.tex}
 \end{small}
 \end{singlespace}

\bibliographystyle{siam}
\bibliography{thesis}
\end{document}
\end{verbatim}

\noindent {\bf Eyre Thesis}. The following is the contents of
\verb|thesis.tex| for the Masters Thesis of David Eyre, March, 1990. It
is a numerical analysis thesis with a large section of computer code.
David produced the thesis on a SUN workstation and also on the VAX 8600
mainframe computer. It contains no figures. The thesis was first typed
in ARTICLE style as several separate documents and then re--edited into
the thesis format. It is an unusual example in that it contains no list
of figures, no list of tables and only a single appendix.

In the actual thesis produced for the library, the computer utility {\tt
lptops} was used to generate the thesis pages for the computer code,
complete with page numbering. The file {\tt code.tex} simply inserts the
correct page number so that the bibliography is numbered properly.

\begin{verbatim}
\documentstyle[12pt]{uuthesis}
\thesistype{thesis}
\title{EXACT ANALYSIS OF GENERAL SPARSE \\
~ \\ RATIONAL LINEAR SYSTEMS}
\author{David Jay Eyre}
\degree{Master of Science}
\department{Department of Mathematics}
\graduatedean{B. Gale Dick}
\submitdate{March 1990}
\copyrightyear{1990}
\committeechair{Peter Alfeld}
\firstreader{Paul C. Fife}
\secondreader{Don H. Tucker}
\departmentchair{Klaus Schmitt}
\dedication{\mbox{For Cas}}
\chairtitle{Professor}
\threelevels
\newtheorem{th}{Theorem}
\newtheorem{lm}[th]{Lemma}
\newtheorem{co}[th]{Corollary}
\newtheorem{df}[th]{Definition}
\begin{document}
\requiredfrontmatter{abstract}
\preface{acknowledge}{ACKNOWLEDGMENTS}
\maintext
\input{tbody}
\chapter{Classical identities}\label{appendix}
\fixchapterheading
\section*{Rellich's identity}\label{rellich.section}
\setcounter{thrm}{0}
%
%

Standard developments of Pohozaev's identity used an identity by
Rellich~\cite{rellich:der40}, reproduced here.

\begin{lem}[Rellich]
Given $L$ in divergence form and $a,d$ defined above, $u\in C^2
(\Omega )$, we have
\begin{equation}\label{rellich}
\int_{\Omega}(-Lu)\nabla u\cdot (x-\overline{x})\, dx
= (1-\frac{n}{2}) \int_{\Omega} a(\nabla u,\nabla u) \, dx
-
\frac{1}{2} \int_{\Omega}
d(\nabla u, \nabla u) \, dx
\end{equation}
$$
+
\frac{1}{2} \int_{\partial\Omega} a(\nabla u,\nabla u)(x-\overline{x})
\cdot \nu  \, dS
-
\int_{\partial\Omega}
a(\nabla u,\nu )\nabla u\cdot (x-\overline{x}) \, dS.
$$
\end{lem}
{\bf Proof:}\\
There is no loss in generality to take $\overline{x} = 0$. First
rewrite $L$:
$$Lu = \frac{1}{2}\left[ \sum_{i}\sum_{j}
\frac{\partial}{\partial x_i}
\left( a_{ij} \frac{\partial u}{\partial x_j} \right) +
\sum_{i}\sum_{j}
\frac{\partial}{\partial x_i}
\left( a_{ij} \frac{\partial u}{\partial x_j} \right)
\right]$$
Switching the order of summation on the second term and relabeling
subscripts, $j \rightarrow i$ and $i \rightarrow j$, then using the fact
that $a_{ij}(x)$ is a symmetric matrix,
gives the symmetric form needed to derive Rellich's identity.
\begin{equation}
Lu = \frac{1}{2} \sum_{i,j}\left[
\frac{\partial}{\partial x_i}
\left( a_{ij} \frac{\partial u}{\partial x_j} \right) +
\frac{\partial}{\partial x_j}
\left( a_{ij} \frac{\partial u}{\partial x_i} \right)
\right].
\end{equation}

Multiplying $-Lu$ by $\frac{\partial u}{\partial x_k} x_k$ and integrating
over $\Omega$, yields
$$\int_{\Omega}(-Lu)\frac{\partial u}{\partial x_k} x_k \, dx=
-\frac{1}{2} \int_{\Omega}
\sum_{i,j}\left[
\frac{\partial}{\partial x_i}
\left( a_{ij} \frac{\partial u}{\partial x_j} \right) +
\frac{\partial}{\partial x_j}
\left( a_{ij} \frac{\partial u}{\partial x_i} \right)
\right]
\frac{\partial u}{\partial x_k} x_k \, dx$$
Integrating by parts (for integral theorems see~\cite[p. 20]
{zeidler:nfa88IIa})
gives
$$= \frac{1}{2} \int_{\Omega}
\sum_{i,j} a_{ij} \left[
\frac{\partial u}{\partial x_j}
\frac{\partial^2 u}{\partial x_k\partial x_i} +
\frac{\partial u}{\partial x_i}
\frac{\partial^2 u}{\partial x_k\partial x_j}
\right] x_k \, dx
$$
$$
+
\frac{1}{2} \int_{\Omega}
\sum_{i,j} a_{ij} \left[
\frac{\partial u}{\partial x_j} \delta_{ik} +
\frac{\partial u}{\partial x_i} \delta_{jk}
\right] \frac{\partial u}{\partial x_k} \, dx
$$
$$- \frac{1}{2} \int_{\partial\Omega}
\sum_{i,j} a_{ij} \left[
\frac{\partial u}{\partial x_j} \nu_{i} +
\frac{\partial u}{\partial x_i} \nu_{j}
\right] \frac{\partial u}{\partial x_k} x_k \, dx
$$
= $I_1 + I_2 + I_3$, where the unit normal vector is $\nu$.
One may rewrite $I_1$ as
$$I_1 = \frac{1}{2} \int_{\Omega}
\sum_{i,j} a_{ij} \frac{\partial}{\partial x_k}\left(
\frac{\partial u}{\partial x_i}
\frac{\partial u}{\partial x_j}
\right) x_k \, dx
$$
Integrating the first term by parts again yields
$$I_1 = -\frac{1}{2} \int_{\Omega}
\sum_{i,j} a_{ij} \left(
\frac{\partial u}{\partial x_i}
\frac{\partial u}{\partial x_j}
\right) \, dx
+
\frac{1}{2} \int_{\partial\Omega}
\sum_{i,j} a_{ij} \left(
\frac{\partial u}{\partial x_i}
\frac{\partial u}{\partial x_j}
\right) x_k \nu_k \, dS
$$
$$
-
\frac{1}{2} \int_{\Omega}
\sum_{i,j} \left(
\frac{\partial u}{\partial x_i}
\frac{\partial u}{\partial x_j}
\right) x_k \frac{\partial a_{ij}}{\partial x_k}\, dx.
$$
Summing over $k$ gives
$$\int_{\Omega}(-Lu)(\nabla u\cdot x)\, dx =
-\frac{n}{2} \int_{\Omega}
\sum_{i,j} a_{ij} \left(
\frac{\partial u}{\partial x_i}
\frac{\partial u}{\partial x_j}
\right) \, dx
$$
$$
+
\frac{1}{2} \int_{\partial\Omega}
\sum_{i,j} a_{ij} \left(
\frac{\partial u}{\partial x_i}
\frac{\partial u}{\partial x_j}
\right) (x\cdot \nu ) \, dS
-
\frac{1}{2} \int_{\Omega}
\sum_{i,j} \left(
\frac{\partial u}{\partial x_i}
\frac{\partial u}{\partial x_j}
\right) (x\cdot  \nabla a_{ij}) \, dx
$$
$$
+
\frac{1}{2} \int_{\Omega}
\sum_{i,j,k} a_{ij} \left[
\frac{\partial u}{\partial x_j}
\frac{\partial u}{\partial x_k} \delta_{ik} +
\frac{\partial u}{\partial x_i}
\frac{\partial u}{\partial x_k} \delta_{jk}
\right] \, dx
$$
$$- \frac{1}{2} \int_{\partial\Omega}
\sum_{i,j} a_{ij} \left[
\frac{\partial u}{\partial x_j} \nu_{i} +
\frac{\partial u}{\partial x_i} \nu_{j}
\right] (\nabla u\cdot x) \, dS.
$$
Combining the first and fourth term on the right-hand side
simplifies the expression
$$\int_{\Omega}(-Lu)(\nabla u\cdot x)\, dx
=
(1-\frac{n}{2}) \int_{\Omega}
\sum_{i,j} a_{ij} \left(
\frac{\partial u}{\partial x_i}
\frac{\partial u}{\partial x_j}
\right) \, dx
$$
$$
+
\frac{1}{2} \int_{\partial\Omega}
\sum_{i,j} a_{ij} \left(
\frac{\partial u}{\partial x_i}
\frac{\partial u}{\partial x_j}
\right) (x\cdot \nu ) \, dS
-
\frac{1}{2} \int_{\Omega}
\sum_{i,j} \left(
\frac{\partial u}{\partial x_i}
\frac{\partial u}{\partial x_j}
\right) (x\cdot  \nabla a_{ij}) \, dx
$$
$$
-
\frac{1}{2} \int_{\partial\Omega}
\sum_{i,j} a_{ij} \left[
\frac{\partial u}{\partial x_j} \nu_{i} +
\frac{\partial u}{\partial x_i} \nu_{j}
\right] (\nabla u\cdot x) \, dS.
$$
Using the notation defined above, the result follows.


%
%
%
\section*{Fortran code}\label{code}
%
%

%% The following was constructed by a very handy program called
%% tgrind.  tgrind is a filter to convert C or fortran files into
%% formatted tex.  Starting with a fortran subroutine rhs.f:
%%
%% tgrind -lf -f >rhs.tex rhs.f
%%
%% This creates the tex file rhs.tex.  This may be directly included
%% in a latex document via the special command \tgrind (latex command
%% here):
%%
%% \begin{singlespace}
%% \begin{small}
%% \tgrind{rhs.tex}
%% \end{small}
%% \end{singlespace}
%%
%% Otherwise, the file rhs.tex needs to be edited (rhs_mod.tex)
%% to be included into a latex document (it is a stand-alone
%% tex file).  The line with the \File command (top) needs to be
%% removed or commented out: \File{rhs.f},{14:32},{Jul  5 1993} and
%% the \end command at the bottom also needs to be commented out.  The
%% file rhs.tex can then be included into the document:
%%
%% \begin{singlespace}
%% \begin{small}
%% \input tgrindmac
%\File{rhs.f},{14:32},{Jul  5 1993}
\L{\LB{}}
\L{\LB{      \K{subroutine} rhs(neq,v,rhsf)}}
\L{\LB{      \K{save}}}
\L{\LB{c}}
\L{\LB{c This \K{subroutine} computes the \K{function} values. Inputs are neq and }}
\L{\LB{c v, and on output the values of f are stored in the array of rhsf}}
\L{\LB{c}}
\L{\LB{      \K{include} \S{}\'parabolic.inc\'\SE{}}}
\L{\LB{}}
\L{\LB{      \K{integer} neq}}
\L{\LB{      \K{integer} i}}
\L{\LB{      \K{integer} j}}
\L{\LB{      \K{integer} k}}
\L{\LB{      \K{integer} ind}}
\L{\LB{      \K{integer} inde}}
\L{\LB{      \K{integer} indw}}
\L{\LB{      \K{integer} indn}}
\L{\LB{      \K{integer} inds}}
\L{\LB{      \K{integer} ind0}}
\L{\LB{      \K{integer} ind1}}
\L{\LB{      \K{integer} ind2}}
\L{\LB{}}
\L{\LB{      \K{double} \K{precision} v(neq)}}
\L{\LB{      \K{double} \K{precision} rhsf(neq)}}
\L{\LB{      \K{double} \K{precision} u(nv)}}
\L{\LB{      \K{double} \K{precision} diff}}
\L{\LB{      \K{double} \K{precision} diffn}}
\L{\LB{      \K{double} \K{precision} diffxn}}
\L{\LB{      \K{double} \K{precision} diffyn}}
\L{\LB{      \K{double} \K{precision} nl}}
\L{\LB{}}
\L{\LB{c      \K{write}(*,*)\S{}\'funct begin\'\SE{}}}
\L{\LB{}}
\L{\LB{c}}
\L{\LB{c     Compute F for the local dynamics, written as  F(u)= \-du\/dt + f(u)}}
\L{\LB{c     }}
\L{\LB{c}}
\L{\LB{c the system parameters}}
\L{\LB{c}}
\L{\LB{c      p1              ! \K{parameter} F}}
\L{\LB{c      p2              ! \K{parameter} k}}
\L{\LB{}}
\L{\LB{      \K{do} j = 1, ny }}
\L{\LB{         \K{do} i = 1, nx}}
\L{\LB{c}}
\L{\LB{c set up index}}
\L{\LB{c}}
\L{\LB{            ind = (i\-1)*nv + (j\-1)*meq}}
\L{\LB{c}}
\L{\LB{c Extract the jth component at current time}}
\L{\LB{c}}
\L{\LB{            nl = v(1+ind)*v(2+ind)*v(2+ind)}}
\L{\LB{}}
\L{\LB{            rhsf(1+ind) =  (\- nl + p1*(1.0d0 \- v(1+ind)))*local}}
\L{\LB{            rhsf(2+ind) =  (  nl \- (p1+p2)*v(2+ind))*local}}
\L{\LB{}}
\L{\LB{         \K{end} \K{do}}}
\L{\LB{      \K{end} \K{do}}}
\L{\LB{}}
\L{\LB{c \-\-\-\-\-\-\-\-\-\-\-\-\-\-\-\-\-\-\-\-\-\-\-\-\-\-\-\-\-\-\-\-\-\-\-\-\-\-\-\-}}
\L{\LB{c}}
\L{\LB{c     add diffusion for all species (zero diffusion }}
\L{\LB{c     coefficient takes care of those that \K{do} not diffuse). }}
\L{\LB{c }}
\L{\LB{c
\-\-\-\-\-\-\-\-\-\-\-\-\-\-\-\-\-\-\-\-\-\-\-\-\-\-\-\-\-\-\-\-\-\-\-\-\-\-\-
}}
\L{\LB{}}
\L{\LB{      \K{do} j = 1, ny}}
\L{\LB{         \K{do} i = 1, nx}}
\L{\LB{}}
\L{\LB{c indexing}}
\L{\LB{c}}
\L{\LB{            ind0 = (i\-1)*nv + (j\-1)*meq   ! point}}
\L{\LB{            indw = (i\-2)*nv + (j\-1)*meq   ! west point}}
\L{\LB{            inde = (i)*nv + (j\-1)*meq     ! east point}}
\L{\LB{            indn = (i\-1)*nv + (j)*meq     ! north point}}
\L{\LB{            inds = (i\-1)*nv + (j\-2)*meq   ! south point}}
\L{\LB{}}
\L{\LB{            \K{if}(i.eq.1) indw = (nx\-1)*nv + (j\-1)*meq}}
\L{\LB{            \K{if}(i.eq.nx) inde = (j\-1)*meq }}
\L{\LB{            \K{if}(j.eq.1) inds = (i\-1)*nv + (ny\-1)*meq}}
\L{\LB{            \K{if}(j.eq.ny) indn = (i\-1)*nv}}
\L{\LB{}}
\L{\LB{            \K{do} k = 1, 2}}
\L{\LB{}}
\L{\LB{c}}
\L{\LB{c First compute the contribution within a row at the current time}}
\L{\LB{c and at the preceding time. }}
\L{\LB{c}}
\L{\LB{               ind = k + ind0}}
\L{\LB{               ind1 = k + indw}}
\L{\LB{               ind2 = k + inde}}
\L{\LB{}}
\L{\LB{               diffxn = v(ind1) \- 2.0d0*v(ind) + v(ind2)}}
\L{\LB{}}
\L{\LB{c}}
\L{\LB{c Compute the contribution from the columns}}
\L{\LB{c}}
\L{\LB{               ind1 = k + indn}}
\L{\LB{               ind2 = k + inds}}
\L{\LB{}}
\L{\LB{               diffyn = v(ind1) \- 2.0d0*v(ind) + v(ind2)}}
\L{\LB{}}
\L{\LB{c}}
\L{\LB{c Multiply by  other factors and sum}}
\L{\LB{c}}
\L{\LB{               diff = d(k)*hxx*(diffxn + diffyn)*diffus}}
\L{\LB{}}
\L{\LB{               rhsf(ind) = rhsf(ind) + diff}}
\L{\LB{               }}
\L{\LB{}}
\L{\LB{            \K{end} \K{do}}}
\L{\LB{         \K{end} \K{do}}}
\L{\LB{      \K{end} \K{do}}}
\L{\LB{}}
\L{\LB{       }}
\L{\LB{      \K{return}}}
\L{\LB{      \K{end}}}
\L{\LB{}}
\L{\LB{}}
\L{\LB{}}
\vfill\eject
%\end

%% \end{small}
%% \end{singlespace}
%%
%% In either case, the \File command probably will need removing
%% because it places the page numbers differently than the normal
%% thesis style. Using both gives:

 \begin{singlespace}
 \begin{small}
% \let\end\relax \def\File#1,#2,#3{}
 \tgrind{rhs_mod.tex}
 \end{small}
 \end{singlespace}

\input{code}
\input{biblio}
\end{document}
\end{verbatim}

\noindent {\bf Joyce Thesis}. The following is the contents of
\verb|thesis.tex| for the Ph.D. thesis of Paul Joyce, June, 1988. It is
a mathematics thesis that does not use automatic features of the {\tt
uuthesis} style. This particular thesis was typeset using \TeX{}tures on
a MacIntosh SE computer with output from a Laserwriter printer. Figures
were automatically integrated at print time (no paste-up). Note in
particular the ordering of control sequences and the placement of
definitions in the preamble.

The Joyce thesis, done on a MacIntosh computer
using Textures,
is a large thesis consisting of the following computer source files:

\begin{center}
\begin{tabular}{|l|l|}
\hline
File Name               & Description \\
\hline
THESIS.TEX              & main control file \\
ABSTRACT.TEX         & abstract \\
ACK.TEX              & Acknowledgments \\
THESIS.GRAPHICS         & figures \\
CH1.TEX            & chapter 1 \\
CH2.TEX            & chapter 2 \\
CH3.TEX            & chapter 3 \\
CH4B.TEX            & chapter 4 \\
CH5.TEX            & chapter 5 \\
CH6.TEX            & chapter 6 \\
CH7.TEX            & chapter 7 \\
CH8.TEX            & chapter 8 \\
CH9.TEX            & chapter 9 \\
THESIS.BIB          & bibliography \\
\hline
\end{tabular}
\end{center}
The file THESIS.TEX is reproduced below, so that you can see how to
structure the writing of the thesis. Paul wrote the key chapters first,
then the figures, bibliography and introduction.


\begin{verbatim}
\documentstyle{uuthesis}
\includeonly{ch7,ch8,ch9} % Debug only chapter 7, 8, 9
\title{Age--Ordered Distributions For Population Genetics
Models}
\author{Paul Joseph Joyce}
\thesistype{dissertation}
\graduatedean{B. Gale Dick}
\department{Department of Mathematics}
\degree{Doctor of Philosophy}
\departmentchair{T. Benny Rushing}
\committeechair{Simon Tavar\'e}
\firstreader{Robert M. Brooks}
\secondreader{Stewart N. Ethier}
\thirdreader{James P. Keener}
\fourthreader{Klaus Schmitt}
\chairtitle{Professor}
\submitdate{June 1988}
\copyrightyear{1988}
\dedication{This thesis is dedicated to my parents
Dorothy A. and Thomas F. Joyce.}
\threelevels
\newtheorem{lemma}[theorem]{Lemma}
\begin{document}
\frontmatterformat
\titlepage
\copyrightpage
\committeeapproval
\readingapproval
\preface{abstract}{ABSTRACT}
\dedicationpage
\tableofcontents
\listoffigures
\preface{ack}{ACKNOWLEDGMENTS}
\maintext
\include{ch1}
\include{ch2}
\include{ch3}
\include{ch4b}
\include{ch5}
\include{ch6}
\include{ch7}
\include{ch8}
\include{ch9}
\bibliographystyle{acm}
\end{document}
\end{verbatim}

Below is a sample \LaTeX{}2e source.


\begin{verbatim}
% -*-latex-*-
% Sample thesis source for uuthesis format 1998
% Tested with LaTeX 2.09 and LaTeX2e
% Set up for \documentclass{} constructions of LaTeX2e
%
\documentclass[Chicago]{uuthesis2e}
  %\documentstyle{uuthesis1-8}
  %\documentstyle[honors]{uuthesis1-8}% Use for an honors thesis
  %\documentstyle[12pt]{uuthesis1-8}
  %\documentstyle[11pt]{uuthesis1-8}
  %\doublespacedheadings% single-spaced is the default
\fourlevels
\title{THEORETICAL STUDIES ON SECOND \\ ORDER DYNAMICAL SYSTEMS}
\author{John Andrew Smith}
\thesistype{dissertation}
  %\thesistype{Honors Thesis}%       Use these for an honors thesis
  %\honorssupervisor{Hans G.~Othmer}
  %\honorsadvisor{John R.~Nelson}
  %\honorsdirector{Richard J.~Cummings}
  %\honorsdepartment{Mathematics}
\graduatedean{Ann W. Hart}
\department{Department of Mathematics}
\degree{Doctor of Philosophy}
\departmentchair{Paul C. Fife}
\committeechair{Hans G. Othmer}
\firstreader{Frederick R. Adler}
\secondreader{Aaron L. Fogelson}
\thirdreader{James P. Keener}
\fourthreader{Mark A. Lewis}
\chairtitle{Professor}
\submitdate{December 1993}
\copyrightyear{1993}
\dedication{To the memory of my grandfather}
\def\mainheadingwidth{4.25in}
\def\BR{\protect\\}
\def\abstracttext{
  \hskip\parindent
  Signal relay and adaptation in response to cAMP stimuli in the
  cellular slime mold {\it Dictyostelium discoideum} are a model system
  for the study of signal transduction.  Calcium dynamics in different
  cell types, especially in deutosome eggs and cardiac cells, have
  attracted a lot of interest lately.  This thesis attempts to develop
  some general theories about second messenger dynamics.
}
\def\acknowledgementtext{
  \hskip\parindent
I wish to acknowledge joint work with coauthor Hans G. Othmer, part of
which appears with minor modification as Chapter 2 and Chapter 4. These
materials are included with permission of the publishers.
\par
Professor Kenneth W.\ Spitzer and Mr.\ Timothy J.\ Lewis provided me
with much information and fruitful discussions during the development of
the models presented in Chapter 6.  I also want to express my deep
appreciation for the patient help received from Dr.\ Nelson H.\ F.\
Beebe and Mr.\ Pieter J.\ Bowman on my computational work and the
preparation of the manuscript.
\par
The research in this thesis was supported in part under NIH
Grant \#GM123456.
}
\def\FIG#1#2#3{
\begin{figure}[#2]\begin{center}
\def\Xsize{3}\def\Ysize{1.75}\def\Zdim{1.7in}
\fbox{\setlength{\unitlength}{1in}
\begin{picture}(\Xsize,\Ysize)\vspace*{\Zdim}
\end{picture}}
\caption{#1\label{#3}}\end{center}\end{figure}
}

\long\def\TAB#1#2#3{
\begin{table}[#2]
\begin{center}
\caption{#1\label{#3}}
\bigskip
\protect
\begin{tabular}{|l|l|}
\hline
\bf Control name    &      \bf Point size       \\
\hline
\tt tiny            &        6pt                    \\
\tt scriptsize      &        8pt                    \\
\tt footnotesize    &        9.5pt                  \\
\tt normalsize      &        12pt                   \\
\tt large           &        14pt                   \\
\tt Large           &        18pt                   \\
\tt LARGE           &        22pt                   \\
\tt huge            &        25pt                   \\
\tt Huge            &        30pt                   \\
\hline
\end{tabular}
\end{center}
\end{table}
}
\def\TEXT{
Once upon a time, in a distant galaxy called \"O\"o\c c, there lived a
computer named R.~J. Drofnats. Mr.~Drofnats---or ``R. J.,'' as he
preferred to be called--- was happiest when he was at work typesetting
beautiful documents. Once upon a time, in a distant galaxy called
\"O\"o\c c, there lived a computer named R.~J. Drofnats.
Mr.~Drofnats---or ``R. J.,'' as he preferred to be called.
}
\begin{document}
\frontmatterformat
\titlepage
\copyrightpage
\committeeapproval
\readingapproval
\prefacesection{Abstract}\abstracttext
\dedicationpage
\tableofcontents
\listoffigures
\listoftables
\prefacesection{Acknowledgements}\acknowledgementtext
\maintext
\def\mainheadingwidth{4.5in}
\part[Planet Earth and Other Planets of the\BR Solar System]
{Planet Earth and Other Planets\BR of the Solar System}
\def\mainheadingwidth{4.25in}

\chapter{Planets and the Sun}\TEXT{}

\section{Mercury, Venus, Mars and the Earth}\TEXT{}

\section[Jupiter, Saturn, Uranus, Neptune and Pluto]
        {Jupiter, Saturn, Uranus,\BR Neptune and Pluto}
\TEXT{} See Figure \ref{fig1}.\par
\FIG{Planetary Configurations.}{t}{fig1}\par\TEXT{}

\section{Rising and Setting of Planets}\TEXT{}\par\TEXT{}

%\vbox{
\subsection{Calculation of Risetimes}\TEXT{}
%}

\subsection{The Northern and Southern Lights}\TEXT{} See Figure \ref{fig2}.
\par\FIG{Aurora Borealis and Aurora Australis.}{b}{fig2}\par\TEXT{}

\subsubsection{Eclipses of the Sun and Moon}\TEXT{}

\subsection{Constellations}
\subsubsection{Andromeda and Aquarius}

%\subsubsubsection{The Chained Maiden}\TEXT{}
%\subsubsubsection{The Water Bearer}\TEXT{}
\TEXT{}

\paragraph{Abbreviations for some constellations.}\TEXT{}

\section[The Sun as a Controlling Body of Our Solar System]
        {The Sun as a Controlling Body\BR of Our Solar System}
\subsection{Spheres and Corona}\TEXT{} See Table \ref{tab1}.\par
\TAB{Moon's Perigee and Apogee.}{b}{tab1}\par\TEXT{}\par
\TEXT{} See Figure \ref{fig3}.\par\TEXT{}\par\TEXT{}
\FIG{Photosphere and Chromosphere.}{b}{fig3}
\clearpage

\chapter[The Earth: Size, Computation of Time and\BR Seasons]
        {The Earth: Size, Computation\BR of Time and Seasons}
\fixchapterheading
\section{Size and Dimensions}\TEXT{}

\section{Atmosphere of the Earth}\TEXT{} See Table \ref{tab2}.\par

\section{Computation of Time}\TEXT{}

\section{The Zones and Seasons}\TEXT{}

\TAB{Chronological Cycles.}{t}{tab2}\par\TEXT{}\par
\clearpage

\numberofappendices=1
\appendix
\chapter{Morning and Evening Stars}

\fixchapterheading
\section{Morning Stars}\TEXT{} See Table \ref{tab3}.\par
\TAB{Data on Morning Stars.}{b}{tab3}\par\TEXT{}\par
\TEXT{}\par\TEXT{}

\section{Evening Stars}\TEXT{} See Table \ref{tab4}.\par
\TAB{Data on Evening Stars.}{b}{tab4}\par\TEXT{}\par
\TEXT{}

\typeout{Using bibtex SIAM style bibliography}
\begin{thebibliography}{10000}

\bibitem{ado90}
{\sc J.~C. Alexander, E.~J. Doedel, and H.~G. Othmer}, {\em On the resonance
  structure in a forced excitable system}, SIAM J. Appl. Math., 50 (1990),
  pp.~1373--1418.

\bibitem{cas75}
{\sc R.~Casten, H.~Cohen, and P.~Lagerstorm}, {\em Perturbation analysis of an
  approximation to the {H}odgkin-{H}uxley theory}, Quart. Appl. Math., 32
  (1975), pp.~365--402.

\bibitem{cle55}
{\sc E.~A. Coddington and N.~Levinson}, {\em Theory of Ordinary Differential
  Equations}, McGraw-Hill, New York, 1955.

\bibitem{con84}
{\sc C.~Conley}, {\em On travelling wave solutions of nonlinear diffusion
  equations}, Ind. Univ. Math. J., 33 (1984), pp.~319--343.

\bibitem{fif76}
{\sc P.~C. Fife}, {\em Singular perturbation and wave front techniques in
  reaction-diffusion problems}, in Proc. AMS-SIAM Symposium on Asymptotic
  Methods and Singular Perturbations, New York, 1976, Am. Math. Soc.

\bibitem{tur52}
{\sc A.~M. Turing}, {\em The chemical basis of morphogenesis}, Phil. Trans. R.
  Soc. Lond. B, 237 (1952), pp.~37--72.

\end{thebibliography}

\end{document}

\end{verbatim}

\end{document}
