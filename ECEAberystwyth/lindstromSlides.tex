\documentclass[final]{beamer}
%% \title{Energy and Growth, History and Dynamics}
%\title{Economic development with unlimited supplies of energy:
%\\The English Industrial Revolution and modern economic growth}

\title{Non-Linear ECE Electromagnetism} %for 2103 EEA NYC
\author{Douglas W. Lindstrom}

%\date{Drafts May 2012,}
\date{AIAS July 2013}
\usepackage[latin1]{inputenc}
%\usepackage[english]{babel}
\usepackage{amsmath}
\usepackage{amsfonts}
\usepackage{txfonts}
\usepackage{amssymb}
\usepackage{pgfpages}
\usepackage{booktabs}
\usepackage{longtable}

%\usetheme{Darmstadt}
%\usecolortheme{beaver}

\usepackage{chngpage}
%\usepackage{pdfpages}
\usepackage{graphicx}
%\usepackage[lofdepth,lotdepth,position=bottom]{subfig}
\usepackage{caption}
%\usepackage{draftwatermark}

\usepackage{verbatim}
%\usepackage{underscore}
%\linespread{1.9}	% remove for single, 1.3 for 1.5 and 1.6 for 2.0. use this setting for print editing

\usepackage{glossaries}

\graphicspath{{../images/}}

%\textwidth{7.5in}
%\addtolength{\textwidth}{1.0in} 
%\addtolength{\oddsidemargin}{-0.5in} 
%\addtolength{\evensidemargin}{-0.5in} 
%\addtolength{\textheight}{1.25in}
%\addtolength{\topmargin}{-0.75in}

\setbeamertemplate{navigation symbols}{} %no nav symbols

\usepackage{hyperref}

%\makeglossaries

%\loadglsentries{glossary.tex}

%\setcounter{secnumdepth}{4}%to number paragraphs so can ref them?

\begin{document}
\maketitle

\begin{frame}
\frametitle{Need for Non-Linearity}
%\begin{flushleft}
%$$
\begin{flalign*}
&\bold{E}^a = - \underline{\nabla} \phi^a - \frac{\partial \bold{A}^a}{\partial t} - \omega^a_{0b}\bold{A}^b + \pmb{\omega}^a_b \phi^b\\
&\bold{B}^a = \underline{\nabla} \times \bold{A}^a - \pmb{\omega}^a_b \times \bold{A}^b\\
\end{flalign*}
\begin{flalign*}
&\underline{\nabla} \cdot \bold{B}^a = 0\\
&\underline{\nabla} \times \bold{E}^a + \frac{\partial \bold{B}^a}{\partial t}= 0\\
\end{flalign*}
\end{frame}

\begin{frame}
\frametitle{Need for Non-Linearity}
From Gauss's Law
\begin{flalign*}
&\underline{\nabla} \cdot (\pmb{\omega}^a_b \times \bold{A}^b) = 0\\
&\pmb{\omega}^a_b \times \bold{A}^b = \underline{\nabla} \times \bold{F}^a\\
\end{flalign*}
Note that if
\begin{flalign*}
F^a = - A^a
\end{flalign*}
This is the ``Lindstrom Constraint'' for magnetic antisymmetry.
\end{frame}

\begin{frame}
\frametitle{Need for Non-Linearity}
Substituting these into the Faraday Equation
\begin{flalign*}
&\underline{\nabla} \times \left(- \omega^a_{0b}\bold{A}^b + \pmb{\omega}^a_b \phi^b - \frac{\partial \bold{F}^a}{\partial{t}} \right)=0\\
&- \omega^a_{0b}\bold{A}^b + \pmb{\omega}^a_b \phi^b - \frac{\partial \bold{F}^a}{\partial{t}}  = \underline{\nabla} \psi^a\\
\end{flalign*}
Write
\begin{flalign*}
&\Phi^a = \phi^a - \psi^a \quad \quad \quad \bold{\mathcal{A}}^a = \bold{A}^a - \bold{F}^a\\
\end{flalign*}
Then a Maxwell-Heaviside theory emerges, i.e.
\begin{flalign*}
&\bold{E}^a = -\underline{\nabla}\Phi^a - \frac{\partial \bold{\mathcal{A}}^a}{\partial t} \quad \quad \quad \bold{B}^a = \underline{\nabla} \times \bold{\mathcal{A}}^a\\
\end{flalign*}

\end{frame}

\begin{frame}
\frametitle{Non-Linear Field Equations}
From the first Bianchi Identity
\begin{flalign*}
&\partial_{\mu} \widetilde{T}^{\; a \mu \nu} + \omega^a_{\;\; \mu b} \widetilde{T}^{\; b \mu \nu} = \widetilde{R}^{\;\; a \mu \nu}_\mu\\
&\partial_{\mu} T^{\; a \mu \nu} + \omega^a_{\;\; \mu b} T^{\; b \mu \nu} = R^{\;\; a \mu \nu}_\mu\\
\end{flalign*}
Until now
\begin{flalign*}
&j^{a \nu}_H  = \widetilde{R}^{\;\; a \mu \nu}_\mu - \omega^a_{\;\; \mu b} \widetilde{T}^{\; a \mu \nu} \approx 0\\
&j^{a \nu}_I = R^{\;\; a \mu \nu}_\mu - \omega^a_{\;\; \mu b} T^{\; a \mu \nu} \\
\end{flalign*}
Giving
\begin{flalign*}
&\partial_{\mu} \widetilde{T}^{\; a \mu \nu} \approx 0 \quad \quad \quad \partial_{\mu} T^{\; a \mu \nu} = j^{a \nu}_I \\
\end{flalign*}

\end{frame}

\begin{frame}
\frametitle{Non-Linear Field Equations}
If we redefine the 4-current densities as
\begin{flalign*}
&j^{a \nu}_I = R^{\;\; a \mu \nu}_\mu \\
&j^{a \nu}_H  = \widetilde{R}^{\;\; a \mu \nu}_\mu \approx 0\\
\end{flalign*}
Then new non-linear field equations emerge
\begin{flalign*}
&\partial_{\mu} T^{\; a \mu \nu} + \omega^a_{\;\; \mu b} T^{\; b \mu \nu} = R^{\;\; a \mu \nu}_\mu = j^{a \nu}_I\\
&\partial_{\mu} \widetilde{T}^{\; a \mu \nu} + \omega^a_{\;\; \mu b} \widetilde{T}^{\; b \mu \nu} = \widetilde{R}^{\;\; a \mu \nu}_\mu = j^{a \nu}_H \approx 0\\
\end{flalign*}
\end{frame}

\begin{frame}
\frametitle{Non-Linear Field Equations}
In vector notation
\begin{flalign*}
&\underline{\nabla} \cdot \bold{B}^a - \pmb{\omega}^a_b \cdot \bold{B}^b = 0\\
&\underline{\nabla} \times \bold{E}^a + \frac{\partial \bold{B}^a}{\partial	t} + \omega^a_{0b}\bold{B}^b - \pmb{\omega}^a_b \times \bold{E}^b = 0\\
&\underline{\nabla} \cdot \bold{D}^a - \pmb{\omega}^a_b \cdot \bold{D}^b = \rho^a\\
&\underline{\nabla} \times \bold{H}^a + \frac{\partial \bold{D}^a}{\partial	t} + \omega^a_{0b}\bold{D}^b - \pmb{\omega}^a_b \times \bold{H}^b = \bold{J}^a\\
\end{flalign*}

\end{frame}

\end{document}