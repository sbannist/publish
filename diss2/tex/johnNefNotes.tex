\documentclass[11pt]{article}
%% \title{Energy and Growth, History and Dynamics}
%\title{Economic development with unlimited supplies of energy:
%\\The English Industrial Revolution and modern economic growth}

\title{John Nef notes} %for 2103 EEA NYC
\author{Stephen C. Bannister\\
	Department of Economics\\
	University of Utah\\
	Salt Lake City, Utah 84112\\
	USA\\
	\href{mailto:steve.bannister@econ.utah.edu}{steve.bannister@econ.utah.edu}\\
	}

%\date{Drafts May 2012,}
\date{}
\usepackage[latin1]{inputenc}
%\usepackage[english]{babel}
\usepackage{amsmath}
\usepackage{amsfonts}
\usepackage{txfonts}
\usepackage{amssymb}
\usepackage{pgfpages}
\usepackage{booktabs}
\usepackage{longtable}

\usepackage{chngpage}
%\usepackage{pdfpages}
\usepackage{graphicx}
\usepackage[lofdepth,lotdepth,position=bottom]{subfig}
\usepackage{caption}
%\usepackage{draftwatermark}

\usepackage{verbatim}
%\usepackage{underscore}
\linespread{1.9}	% remove for single, 1.3 for 1.5 and 1.6 for 2.0. use this setting for print editing

\usepackage{glossaries}

\graphicspath{{../images/}}

%\textwidth{7.5in}
\addtolength{\textwidth}{1.0in} 
\addtolength{\oddsidemargin}{-0.5in} 
\addtolength{\evensidemargin}{-0.5in} 
\addtolength{\textheight}{1.25in}
\addtolength{\topmargin}{-0.75in}

\usepackage{tocloft}

\usepackage{hyperref}

\makeglossaries

\loadglsentries{glossary.tex}

%\setcounter{secnumdepth}{4}%to number paragraphs so can ref them?

\begin{document}

%\SetWatermarkLightness{0.93}
%\SetWatermarkScale{1}

	\maketitle
	\nocite{*}

John Nef provides a richly detailed history of the early English Industrial Revolution; the central story is the rise of the coal industry; this had many important growth and institutional effects.

%wool demand, enclosures, reduced forest, increase pop, rising wood price, subs coal for wood 1550-1750, deeper mines/more capital, transport infrastructure/more capital. coal/lime/3x rise ag prod (237), coal ships/better navy ships/seamen, 17th c iron/copper/brass imports from lower fuel cost countrie (232), subbed coal as foreign wood prices rose, 

The essences of Nef's history start in Elizabethan England with general deforestation; both sheep and people are the protagonists. Population was rising, which put increasing pressure on increasingly scarce wood resources consumed in both domestic and industrial heating. Wool demand rose, due to increased export demand (the ``new draperies'') and increase domestic demand. More and more valuable sheep led to the enclosure movement, which further reduced available land for wood.

\end{document} \section{End}

\section{notes}
\subsection{2013 EHA notes}
Nef, Habakuk\\
Agriculture transition\\

\subsection{Thirsk 05/13}
ISI, government encouraged, sponsored\\
demand?\\

\subsection{committee notes}
Tim -- remove personal references e.g. Marx  %done 10/13/11
Tim -- quantitative measure of unlimited energy % so lewis talked about labour with at zero marginal product. my idea is that, relative to any labor supply, the amount of mineral energy was essentially unlimited for the initial developing economies. another way of thinking of it is the human energy chart. an interesting question is whether pure energy has diminish marginal product, I think not, thus that you have to look at specific sources and their physical requirements to determine the amount of diminishing marginal product, and in any case it is less than human labour.  10/13/11. %done 11/14/11.

Richard -- other metric approaches. %Why, since looking at graphs of gdp and energy consumption per capita presents such a clear relationship, do I consider a more detailed statistical look? I am interested in looking inside the dynamics that are supported by the long period time series. What this examination should tell is whether the the prime dynamic drivers, so the leading-in-time variables, changed places in the short run and the long run in the economies I study. While I purposefully avoid the term causality at this point, depending on the strength of the results, I may find causality in the data. What I want to understand, beyond my hypothesis of centrality of mineral energy consumption as the defining invention of the Industrial Revolution, is what implications this has for modern development and for sustained per capita economic development in an age of potentially emission constrained economies. The time series methodology I propose has the ability to do this. And it has the capability of incorporating important time related events that enter the time series as discontinuties. Both of these capabilities are core to my research. 10/14/11. %done 11/14/11.

11/17
Rudi - 5 major items. mainly tighten. He thinks if I do a 2-3 page abstract of intro section, Lance might be interested.

You've done already a lot of work. On a lot of this I am no expert (neither history nor metrics). So bear with me.
Let me jot down just a few quick notes:


\begin{itemize}
\item 1) Presentation: Put hypothesis first; don't make the reader wait or search. In this draft, and in a later paper, and in your presentation, the juxtaposition of the various explanations should come much earlier. I.e.: On the first page (!!), there should be a paragraph saying that "the industrial revolution in England is commonly explained by (1) cultural exceptionalism, wihch essentially means ... , (2) ... , (3) energy, .... (4) thermodynamics. My hypothesis is that (x) matters more than previously believed, and my dissertation tests the evidence for that. The anecdotal evidence includes the lack of Dutch industrialization ... etc. " 

\item 2) Hypothesis: Sharpen it. How distinct is your hypothesis really from culture/institution-driven explanations? You do argue that all, even cultural exceptionalists, place emphasis on coal. It might be reasonable to say that the explanation must be found in a combination of these factors -- and your contribution is to "shift the weights." Is that about right? It could be fleshed out more, sharpened at the edges.

\item 3) Metrics: Institutional variables. I didn't go through the details. Do you use some proxies for institutional development, or cultural factors? Should that be there, if you want to compare the influence of those variables to energy? 

\item 4) Contributions: Limit yourself. Section 2.1.2 is too broad, and contains too much, I would say. It is not clear what you mean by 5., and, if clarified, it is by itself possibly a whole dissertation. I would drop it for now. 3. and 4. should go together. Arguably, 4. might be post-dissertation research. You want a classic 3 goals (and papers) in here: (1) Literature review, which you've done in part;  (2) Explanation and re-evaluation of the industrial revolution in England in the language of economic history; (3) Explanation and re-evaluation of the industrial revolution in England in the language of econometrics. 

\item 5) The link to growth theory: Rather than 5., you might want to consider "old" growth theories. You make reference to Lewis. So combine Lewis (or Marx and Kalecki, really) with Malthus, and see where that takes you. Lewis warns of the turning point, when surplus labor is exhausted, and real wages must rise. If energy is the real labor, then we might be there. What does that mean for future growth? That's an interesting question, and you can fruitfully feed future research --- but I would try to make these three papers quite focused on what you have in mind now. 

\end{itemize}


thinks I could do england vs holland and nail that. would need dutch energy series.

Tim - wants me to change graphs, support his method, has many notes, likes my lit review.

\begin{itemize}
\item 
\end{itemize}

\begin{itemize}
\item Why time series
\item Various ts approaches
\item univariate arima
\item static
\item ardl
\item var
\item cointegrated var

\end{itemize}

may use the Diebold 98 article for organizing, or Dougherty

\subsection{sequence for bibliography}

pdflatex
bibtex on .bib F11
pdflatex
pdflatex
bibtex on .tex  F11
pdflatex
pdflatex

sequence for glossary

    Build LaTeX document, this will generate the files used by makeglossaries
    Run the indexing/sorting, the recommended way is to use makeglossaries (a script that runs xindy or makeindex depending on options in the document with correct encoding and language settings):

    makeglossaries <myDocument>

    Build LaTeX document again to get document with glossary entries


\begin{comment}
% save just for reference...covered by table and annotated bib
		\subsection{Data Sources}
		I anticipate using a wide variety of sources. Should I just do a bibliography here?
			\subsubsection{English historical-institutional sources}
			\begin{itemize}
			\item Wrigley
			\item van Zanden
			\item for reference, list in Wrigley discussion
			\item Landes
			\item Crafts
			\item Temin
			\item Pomeranz
			\item Allen
			\item Mokyr
			\item Goldstone
			\item McCloskey
			\item de Vries
			\item Jevons
			\item Clark
			\item TS Ashton
			\end{itemize}
			\subsubsection{English long-period time series sources}
			\begin{itemize}
			\item Mitchell
			\item Snooks
			\item Officer
			\item Fouquet
			\item Warde
			\item Maddison
			\end{itemize}
			\subsubsection{U.S. historical-institutional sources}
			\begin{itemize}
			\item Vaclav Smil
			\item Daly and Townsend
			\item Lewis
			\item North
			\end{itemize}
			\subsubsection{U.S. long-period time series sources}
			\begin{itemize}
			\item Historical Statistics of the United States 1780-1945
			\item Historical Statistics of the United States Millennial Edition Online
			\item Historical, Demographic, Economic, and Social Data: The United States, 1790-2002
			\item Gordon Whitney
			\item Kindleberger
			\end{itemize}
			\subsubsection{Chinese historical-institutional sources}
			\begin{itemize}
			\item Xu, Dixin, and Zhengming Wu, eds. 2000. Chinese Capitalism, 1522-1840. New York: St. Martin's Press.
			\item Hsien-Chun Wang. 2009. ``Discovering Steam Power in China, 1840s-1860s." Technology and Culture 51(1): 31-54.
			\item ``Pomeranz, K.: The Great Divergence: China, Europe, and the Making of the Modern World Economy."
			\item Andre Gunder Frank
			\item Janet Abu-Lughod
			\item Jack Goldstone
			\end{itemize}
			\subsubsection{Chinese long-period time series sources}(May not be available)
			\begin{itemize}
			\item Sinton, Jonathan E. 2001. ``Accuracy and reliability of China's energy statistics." China Economic Review 12(4): 373-383.
			\item Chinese Statistical Yearbook
			\end{itemize}
			\subsubsection{World long-period time series sources}
			\begin{itemize}
			\item Payne - a literature review of recent energy-gdp empirical studies
			\item US DOE EIA
			\item OECD
			\item UN
			\item Katarina Juselius. 2010. On the Role of Theory and Evidence in Macroeconomics. University of Copenhagen. Department of Economics.
			\item Epic of Gilgamesh. The tablets VII and  XI story of the flood. BC 2700. As a hook on which to illustrate the length of time between the Neolithic and Industrial revolutions. One of very earliest histories. So we know about half the 10K years, and there is essentially nothing economically that happened.
			\item Michael Kremer, 1993. Population Growth and Technological Change: One Million B.C. to 1990.
			\item Johansen et al. 2000 Cointegration analysis in the presence of structural breaks
			\item Foley 1996 Statistical Equilibrium Models in Economics
			\end{itemize}
\end{comment}			

\begin{comment}
%leave this out as the glossary is linked		
		\subsection{Definition of Terms}
		\begin{itemize}
		\item \gls{organic}
		\item \gls{mineral}
		\item \gls{neorev}
		\item \gls{insolation}
		\item \gls{areal}
		\item \gls{punctiform}
		\item \gls{indrev}
		\item \gls{early}
		\item \gls{high}
		\item \gls{earmodern}
		\item \gls{modern}
		\item \gls{energy}
		\item \gls{growth}
		\end{itemize}
\end{comment}		

in the full writeup, I need to highlight the ironic effloressence description of jack goldstone.

data sources -- maddison chinese stats 1998 oecd

data sources -- mitchell on american stats

\subsection{Methodology notes}

Cleaning time series data\\
http://www.r-bloggers.com/cleaning-time-series-and-other-data-streams/
\\
package::pracma, method::outlierMAD. Instances of 'moving window data cleaning'\\