\newglossaryentry{neorev} {name = Neolithic Revolution,
description={The Neolithic Revolution is the first agricultural revolution -- the transition from hunting and gathering to agriculture and settlement. Archaeological data indicate that various forms of domestication of plants and animals arose independently in six separate locales worldwide ca. 10,000 -- 7000 years BP.
}}

\newglossaryentry{coint} {name = cointegration, description = {If two or more series are individually integrated (in the time series sense) but some linear combination of them has a lower order of integration, then the series are said to be cointegrated. A common example is where the individual series are first-order integrated (I(1)) but some (cointegrating) vector of coefficients exists to form a stationary linear combination of them.}}

\newglossaryentry{var} {name = Vector Autoregressive models (VAR), description={Vector autoregression (VAR) is an econometric model used to capture the evolution and the interdependencies between multiple time series, generalizing the univariate AR models. All the variables in a VAR are treated symmetrically by including for each variable an equation explaining its evolution based on its own lags and the lags of all the other variables in the model.}}

\newglossaryentry{cvar} {name = Cointegrated Vector Autogregressive models (CVAR), description ={The CVAR approach is related to Haavelmo's famous ``Probability Approach in Econometric'' (1944). It insists on careful stochastic specification as a necessary groundwork for econometric inference and the testing of economic theories. In time-series data, the probability approach requires careful specification of the integration and cointegration properties of variables in systems of equations.}}

\newglossaryentry{surplus} {name = surplus value, description ={Surplus value is a concept used famously by Karl Marx in his critique of political economy, although he did not himself invent the concept. It refers roughly to that part of the new value created by production which is claimed by enterprises as "generic gross profit."}}

\newglossaryentry{copenhagen} {name=Copenhagen school of time-series econometrics, description = {This Full Information Maximum Likelihood analysis of the cointegrated vector equlibrium (or error) correction model has been developed over the last twenty years by S�ren Johansen of the University of Copenhagen, in close cooperation with Katarina Juselius.}}

\newglossaryentry{organic} {name= Organic economy, description = {As described primarily by E. A. Wrigley, the epoch of the organic economy is defined by energy primarily provided by insolation, and thus imposes a limit to economic growth.}}

\newglossaryentry{mineral} {name= Mineral economy, description = {Again, as described by E. A. Wrigley, an economy in which the primary energy source is not organic, but is mineral (such as coal), and is therefore no longer subject to the growth limitations of an organic economy.}}

\newglossaryentry{insolation} {name= insolation, description = {Insolation is a measure of solar radiation energy received on a given surface area in a given time. Ignoring clouds, the average insolation for the Earth is approximately 250 watts per square meter ($6 \times kW \times h / m^2$)/day), taking into account the lower radiation intensity in early morning and evening, and its near-absence at night.}}

\newglossaryentry{areal} {name= areal, description = {A farming area.}}

\newglossaryentry{punctiform} {name= punctiform, description = {Having the form of a point.}}

\newglossaryentry{early} {name= Early Middle Ages, description = {European, AD 500 -- 1000.}}

\newglossaryentry{high} {name= High Middle Ages, description = {(Feudalism) European military expansion, 1000 -- 1450 CE}}

\newglossaryentry{earmodern} {name= Early Modern Period, description = {Europe, 16th century -- 18th century }}

\newglossaryentry{modern} {name= Modern Era, description = {Europe, 18th century -- 20th century
}}

\newglossaryentry{energy} {name= energy consumption, description = {In 2008, total worldwide energy consumption was 474 exajoules ($474 \times 10^{18}
 J=132,000 \text{ TWh}$). This is equivalent to an average annual power consumption rate of 15 terawatts ($1.504 \times 10^{13}$ W) }}

\newglossaryentry{growth} {name= modern economic growth, description = {Modern Economic Growth (MEG) is the term applied by Simon Kuznets (1966)\cite{} to describe the economic epoch of the last 250 years, distinguished by the pervasive application of science-based technology to production. An economic epoch is a relatively long period (over a century) with distinctive characteristics that give it unity and differentiate it from other epochs (1966, p. 2). The principal quantitative characteristics commonly observed in the growth of the presently developed countries are: high rates of growth of per capita product, of population, and of factor productivity, and a high rate of structural transformation. Major aspects of structural change include the shift away from agriculture, increase in the scale of productive units, shifts in organization and in the status of labor, and shifts in the structure of consumption.}}

\newglossaryentry{indrev} {name= Industrial Revolution, description = {E. A. Wrigley defines the Industrial Revolution as the transition, in $17^{th}$ through $19^{th}$ century England, from an advanced organic economy to a mineral economy}}

\newglossaryentry{lizIreign} {name= Queen Elizabeth I, description = {Queen regnant of England and Queen regnant of Ireland from 17 November 1558 until her death on 24 March 1603}}

\newglossaryentry{vicreign} {name=  Queen Victoria, description = {Monarch of the United Kingdom of Great Britain and Ireland from 20 June 1837 until her death on 22 January 1901)}}

\newglossaryentry{bio} {name= bio-cultural subsistence, description = {From Brad De Long, the level of subsistence as it evolves through cultural generations.}}

\newglossaryentry{ecm} {name= equilibrium (or error) correction model (ECM) , description = {The statistical technique appropriate for time series macro economic models if two or more of the information set variables are cointegrated.}}

\newglossaryentry{arima} {name= Auto Regressive Integrated Moving Average, description = {an extension of Box Cox time series modeling which identifies three components of a time series: the autoregressive autocorrelation coefficients, the level of integration to yield difference stationarity, and moving average autocorrelation coefficients}}

\newglossaryentry{pk} {name= post-Keynesian , description = {a macroeconomic school follwoing the work of Nicholas Kaldor typically embracing demand led output, conflictual distribution, and endogenous money}}

\newglossaryentry{verdoorn} {name=Verdoornian , description = {an empirically observed regularity from Petrus Verdoorn that increased output increases productivity by a factor of about .5}}

\newglossaryentry{mtoe} {name=Million Tonne of Oil Equivalent (MTOE), description = {one \textit{toe} is a unit of energy: the amount of energy released by burning one tonne of crude oil, approximately 42 GJ (as different crude oils have different calorific values), so MTOE is $1 x 10^6$ times the measures in this table:\\
\begin{tabular}{rll}
1 toe &=& 11,630 kilowatt hours\\
1 toe &=& 41.87 gigajoules\\
1 toe &=& 39,683,205.411 BTU\\
1 toe &=& 7.11, 7.33, or 7.4 barrel of oil equivalent (boe)\\
\end{tabular}\\
}
}

\newglossaryentry{gkusd} {name=1990 Geary-Khamis international dollars , description = {A measure of purchasing power parity (PPP) stated in equivalent 1990 USD, commonly used by Angus Maddison in his cross-country GDP comparisons}}

\newglossaryentry{} {name= , description = {}}